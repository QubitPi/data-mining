\lecture{sumrep}{sumrep}

\date{Chapter 9: Summarizing Itemsets}

\begin{frame}
\titlepage
\end{frame}


\newcommand{\alggenmax}{\textsc{GenMax}\xspace}
\newcommand{\algcharm}{\textsc{Charm}\xspace}
\newcommand{\algndi}{\textsc{NDI}\xspace}
\newcommand{\algcomputebounds}{\textsc{ComputeBounds}\xspace}

\begin{frame}{An Example Database}
\begin{footnotesize}
\begin{minipage}{1.5in}
\begin{center}
Transaction database \\[0.5cm]

  \centering
\begin{tabular}{|c|c|}
  \hline
  Tid & Itemset\\
\hline
1 & $\mathit{ABDE}$\\
2 & $\mathit{BCE}$\\
3 & $\mathit{ABDE}$\\
4 & $\mathit{ABCE}$\\
5 & $\mathit{ABCDE}$\\
6 & $\mathit{BCD}$\\
\hline
\end{tabular}
\end{center}
\end{minipage}
\hspace{0.1in} 
\begin{minipage}{6cm}
\begin{center}
Frequent itemsets ($\minsup = 3$) \\[0.5cm]

\begin{tabular}{|c|c|}
\hline
$\supp$ & Itemsets\\
\hline
6 & $B$\\
5 & $E, \mathit{BE}$\\
4 & $A, C, D, \mathit{AB}, \mathit{AE}, \mathit{BC}, \mathit{BD}, \mathit{ABE}$\\
3 &  $\mathit{AD}, \mathit{CE}, \mathit{DE}, \mathit{ABD}, \mathit{ADE}, \mathit{BCE}, \mathit{BDE}, \mathit{ABDE}$\\
\hline
\end{tabular}
\end{center}
\end{minipage}
\end{footnotesize}
\end{frame}


\begin{frame}{Maximal Frequent Itemsets}

Given a binary database $\bD \subseteq \cT \times \cI$,
over the tids $\cT$ and items $\cI$, let $\cF$
denote the set of all frequent itemsets, that is,
\begin{align*}
\tcbhighmath{
\cF = \bigl\{X \mid X \subseteq \cI \mbox{ and } \supp(X) \ge \minsup
\bigr\}
}
\end{align*}

\medskip
A frequent itemset $X \in \cF$ is called {\em maximal}
\index{maximal itemsets}
\index{itemsets!maximal}
if it has no
frequent supersets. Let $\cM$ be the set of all maximal frequent
itemsets, given as
\begin{align*}
\tcbhighmath{
\cM = \bigl\{X \mid X \in \cF \mbox{ and } \not\exists Y \supset X, \mbox{
  such that } Y \in \cF \bigr\}
}
\end{align*}

\medskip
The set $\cM$ is a condensed representation of the set of all frequent
itemset $\cF$, because we can determine whether any itemset $X$ is
frequent or not using $\cM$. If there exists a maximal itemset $Z$ such
that $X \subseteq Z$, then $X$ must be frequent; otherwise $X$ cannot be
frequent. 
\end{frame}



\begin{frame}{Closed Frequent Itemsets}
  \small
Given $T \subseteq \cT$, and $X
\subseteq \cI$, define
\begin{align*}
  \vT(X) & = \{ t \in \cT \mid\; t \text{ contains } X\}\\
  \vI(T) & = \{ x \in \cI \mid\; \forall t\in T, \;t \mbox{ contains } x
  \}\\
   \vc(X) & =  \vI \circ \vT (X) = \vI ( \vT (X))
\end{align*}

\medskip
The function $\vc$ is a {\em closure operator} 
and an itemset $X$ is called {\em closed} if $\vc(X) = X$.
It follows that $\vT(\vc(X))=\vT(X)$. The
set of all closed frequent itemsets is thus def\/{i}ned as
\begin{align*}
\tcbhighmath{
\cC = \bigl\{X \mid X \in  \cF \text{ and } \not\exists Y \supset X
\text{  such that } \supp(X) = \supp(Y)\bigr\}
}
\end{align*}
$X$ is closed if all supersets of $X$ have strictly less
support, that is, $\supp(X) > \supp(Y)$, for all $Y \supset X$.

\medskip
The set of all closed frequent itemsets $\cC$ is a condensed
representation, as we can determine whether an itemset $X$ is
frequent, as well as the exact support of $X$ using $\cC$ alone.
\end{frame}


\begin{frame}{Minimal Generators}
A frequent itemset $X$ is a {\em minimal generator} if
it has no subsets with the same support:
\begin{align*}
\tcbhighmath{
\cG = \bigl\{X \mid X \in  \cF \text{ and } \not\exists Y \subset X,
\text{ such that } \supp(X) = \supp(Y)\bigr\}
}
\end{align*}

  \medskip
In other words, all subsets of $X$
have strictly higher support, that is,
  $\supp(X) < \supp(Y)$, for all $Y \subset X$.

  \medskip
Given an equivalence class of itemsets that have the
same tidset, a closed itemset is the unique maximum element of the class,
whereas the minimal generators are the minimal elements of the class.
\end{frame}


\begin{frame}{\large Frequent Itemsets: Closed, Minimal Generators and Maximal}
\scalebox{0.8}{
 \def\nA{\Tr{
	\begin{tabular}{|c|}
		\hline
		\psframebox{$A$}\\
		1345\\
		\hline
	\end{tabular}}}
\def\nB{\TR{
	\begin{tabular}{|c|}
		\hline
		\psframebox[fillcolor=lightgray,fillstyle=solid]{$B$}\\
		123456\\
		\hline
	\end{tabular}}}
\def\nC{\TR{
	\begin{tabular}{|c|}
		\hline
		\psframebox{$C$}\\
		2456\\
		\hline
	\end{tabular}}}
\def\nD{\TR{
	\begin{tabular}{|c|}
		\hline
		\psframebox{$D$}\\
		1356\\
		\hline
	\end{tabular}}}
\def\nE{\TR{
	\begin{tabular}{|c|}
		\hline
		\psframebox{$E$}\\
		12345\\
		\hline
	\end{tabular}}}
\def\nAB{\TR{
	\begin{tabular}{|c|}
		\hline
		$AB$\\
		1345\\
		\hline
	\end{tabular}}}
\def\nAD{\TR{
	\begin{tabular}{|c|}
		\hline
		\psframebox{$AD$}\\
		135\\
		\hline
	\end{tabular}}}
\def\nAE{\TR{
	\begin{tabular}{|c|}
		\hline
		$AE$\\
		1345\\
		\hline
	\end{tabular}}}
\def\nBC{\TR{
	\begin{tabular}{|c|}
		\hline
		\psframebox[fillcolor=lightgray,fillstyle=solid]{$BC$}\\
		2456\\
		\hline
	\end{tabular}}}
\def\nBD{\TR{
	\begin{tabular}{|c|}
		\hline
		\psframebox[fillcolor=lightgray,fillstyle=solid]{$BD$}\\
		1356\\
		\hline
	\end{tabular}}}
\def\nBE{\TR{
	\begin{tabular}{|c|}
		\hline
		\psframebox[fillcolor=lightgray,fillstyle=solid]{$BE$}\\
		12345\\
		\hline
	\end{tabular}}}
\def\nCE{\TR{
	\begin{tabular}{|c|}
		\hline
		\psframebox{$CE$}\\
		245\\
		\hline
	\end{tabular}}}
\def\nDE{\TR{
	\begin{tabular}{|c|}
		\hline
		\psframebox{$DE$}\\
		135\\
		\hline
	\end{tabular}}}
\def\nABE{\TR{
	\begin{tabular}{|c|}
		\hline
		\psframebox[fillcolor=lightgray,fillstyle=solid]{$ABE$}\\
		1345\\
		\hline
	\end{tabular}}}
\def\nABD{\TR{
	\begin{tabular}{|c|}
		\hline
		$ABD$\\
		135\\
		\hline
	\end{tabular}}}
\def\nADE{\TR{
	\begin{tabular}{|c|}
		\hline
		$ADE$\\
		135\\
		\hline
	\end{tabular}}}
\def\nBDE{\TR{
	\begin{tabular}{|c|}
		\hline
		$BDE$\\
		135\\
		\hline
	\end{tabular}}}
\def\nBCE{\TR*{
	\begin{tabular}{|c|}
		\hline
		\psframebox[doubleline=true,
		fillcolor=lightgray,fillstyle=solid]{$BCE$}\\
		245\\
		\hline
	\end{tabular}}}
\def\nABDE{\TR*{
	\begin{tabular}{|c|}
		\hline
		\psframebox[doubleline=true,
		fillcolor=lightgray,fillstyle=solid]{$ABDE$}\\
		135\\
		\hline
	\end{tabular}}}
  \centerline{
  \scalebox{0.6}{%
 \psset{unit=0.75in}
  \psmatrix[colsep=0.5in,rowsep=1in]
  &&[name=empty]$\emptyset$\psspan{3}\\\empty
  &[name=A]\nA & [name=B]\nB 
   & [name=D]\nD 
  & [name=E]\nE
  & [name=C]\nC\\\empty
  [name=AD]\nAD & [name=DE]\nDE & 
  [name=AB]\nAB  &  [name=AE]\nAE 
  & [name=BD]\nBD & [name=BE]\nBE& [name=BC]\nBC  & 
  [name=CE]\nCE\\\empty
   & [name=ABD]\nABD & [name=ADE]\nADE & [name=BDE]\nBDE 
   & [name=ABE]\nABE &[mnode=none]&
   [name=BCE]\nBCE\psspan{1}\\\empty
   &&[name=ABDE]\nABDE\psspan{2}
  \endpsmatrix
  \psset{linewidth=2pt}
  \pscurve(-6.75,-0.5)(-6.1,-0.4)(-5.9,0)(-5.5,2)(-8.1,3)%
  (-8.5,4)(-9.5,4.4)(-10.9,4)%
  (-10.5,3)(-9.5,1)(-8.1,-0.2)(-6.75,-0.5)
  \pscurve(-4.5,1.4)(-3.7,1.6)(-4.5,3)(-5.1,3.4)(-5.5,4)%
  (-8.1,5)(-8.5,6)(-9,6.2)%
  (-9.5,6)(-9.1,4.8)(-8.3,4.4)%
  (-7.9,3.4)(-5.5,2.6)(-5.1,1.6)(-4.5,1.4)
  \ncbox[linearc=0.3,boxsize=0.5,nodesep=0.1in]{B}{B}
  \ncbox[linearc=0.5,boxsize=0.5,nodesep=0.1in]{D}{BD}
  \ncbox[linearc=0.5,boxsize=0.5,nodesep=0.1in]{E}{BE}
  \ncbox[linearc=0.5,boxsize=0.5,nodesep=0.1in]{C}{BC}
  \ncbox[linearc=0.5,boxsize=0.5,nodesep=0.1in]{CE}{BCE}
   \psset{linewidth=1pt,linecolor=gray}
   \ncline{ABDE}{ABD}
   \ncline{ABDE}{ABE}
   \ncline{ABDE}{ADE}
   \ncline{ABDE}{BDE}
   %
   \ncline{ABD}{AB}
   \ncline{ABD}{AD}
   \ncline{ABD}{BD}
   %
   \ncline{ABE}{AB}
   \ncline{ABE}{AE}
   \ncline{ABE}{BE}
   %
   \ncline{ADE}{AD}
   \ncline{ADE}{AE}
   \ncline{ADE}{DE}
   %
   \ncline{BCE}{BC}
   \ncline{BCE}{BE}
   \ncline{BCE}{CE}
   %
   \ncline{BDE}{BD}
   \ncline{BDE}{BE}
   \ncline{BDE}{DE}
   %
   \ncline{AB}{A}
   \ncline{AB}{B}
   %
   \ncline{AD}{A}
   \ncline{AD}{D}
   %
   \ncline{AE}{A}
   \ncline{AE}{E}
   %
   \ncline{BC}{B}
   \ncline{BC}{C}
   %
   \ncline{BD}{B}
   \ncline{BD}{D}
   %
   \ncline{BE}{B}
   \ncline{BE}{E}
   %
   \ncline{CE}{C}
   \ncline{CE}{E}
   %
   \ncline{DE}{D}
   \ncline{DE}{E}
   %
   \ncline{A}{empty}
   \ncline{B}{empty}
   \ncline{C}{empty}
   \ncline{D}{empty}
   \ncline{E}{empty}
  }}%scalebox
}

\small\medskip
Itemsets boxed and shaded are closed, double boxed are
maximal, and those boxed are minimal generators
\end{frame}

\ifdefined\wox \begin{frame} \titlepage \end{frame} \fi

\begin{frame}{Mining Maximal Frequent Itemsets: GenMax Algorithm}
Mining maximal itemsets requires additional steps beyond
simply determining the frequent itemsets.
Assuming that the set of maximal frequent itemsets is initially
empty, that is, $\cM = \emptyset$, each time we generate a new
frequent itemset $X$, we have to perform the following maximality checks
\begin{itemize}
\item {\bf Subset Check:} $\not \exists  Y \in {\cM}$, such that $X
  \subset Y$. If such a $Y$ exists, then clearly $X$ is not maximal.
  Otherwise, we add $X$ to $\cM$, as a potentially maximal itemset.

\item {\bf Superset Check:} $\not \exists  Y \in {\cM}$, such that $Y
  \subset X$. If such a $Y$ exists, then $Y$ cannot be maximal, and we
  have to remove it from $\cM$.
\end{itemize}
\end{frame}


\begin{frame}{GenMax Algorithm: Maximal Itemsets}
  \small
  GenMax is based on dEclat, i.e., it uses diffset intersections for
  support computation.
The initial call takes as input the set of
frequent items
along with their tidsets, $\tup{i,\vT(i)}$, and the initially
empty set of maximal itemsets, $\cM$. Given a set of itemset--tidset
pairs, called IT-pairs, of the form
$\tup{X, \vT(X)}$, the recursive GenMax method works as follows.

\medskip
If the union of all the itemsets, $Y = \bigcup X_i$,
is already subsumed by (or contained in) some maximal pattern $Z
\in \cM$, then no maximal itemset can be generated from the
current branch, and it is pruned.
Otherwise,
we intersect each IT-pair
$\tup{X_i,\vT(X_i)}$ with all the other IT-pairs
$\tup{X_{j},\vT(X_{j})}$, with $j > i$, to
generate new candidates $X_{ij}$, which are added to the IT-pair
set $P_i$. 

\medskip
If $P_i$ is not empty, a recursive call to GenMax is made
to f\/{i}nd other potentially frequent extensions of
$X_i$. On the other hand, if $P_i$ is empty, it
means that $X_i$ cannot be extended, and it is potentially
maximal. In this case, we add $X_i$ to the set $\cM$, provided
that $X_i$ is not contained in any previously added maximal set
$Z \in \cM$.
\end{frame}

\begin{frame}[fragile]{GenMax Algorithm}
\begin{tightalgo}[H]{\textwidth-18pt}
\SetKwInOut{AlgorithmA}{\alggenmax($P$, $\minsup$, $\cM$)}
\tcp{Initial Call: $\cM \assign \emptyset$,
$P \assign \bigl\{ \tup{i,\vT(i)} \mid i \in
\cI, \supp(i)\ge \minsup \bigr\}$}
\AlgorithmA{}
$Y \assign \bigcup X_i$ \nllabel{alg:fpm:sumrep:alggenmax:Us}\;
\If{$\exists Z \in \cM, \text{ such that } Y \subseteq Z$}{
\Return \nllabel{alg:fpm:sumrep:alggenmax:Ue} \tcp{prune entire branch}
}
\ForEach{$\tup{X_i, \vT(X_i)} \in P$}{
  $P_i \assign \emptyset$\;
  \ForEach{$\tup{X_{j},\vT(X_{j})} \in P$, with $j > i$}{
    \nllabel{alg:fpm:sumrep:alggenmax:Ps}
    $X_{ij} \assign X_i \cup X_{j}$\;
    $\vT(X_{ij}) = \vT(X_i)\; \cap\; \vT(X_{j})$\;
    \lIf{$\supp(X_{ij}) \geq \minsup$}{
      $P_i \assign P_i \cup \{ \tup{X_{ij}, \vT(X_{ij})} \}$
      \nllabel{alg:fpm:sumrep:alggenmax:Pe}
    }
  }
  \lIf{$P_i \ne \emptyset$}{
    \alggenmax($P_i$, $\minsup$, $\cM$)
  }
  \ElseIf{$\not\exists Z \in \cM, X_i \subseteq Z$}{
  $\cM = \cM \cup X_i\quad$ \nllabel{alg:fpm:sumrep:alggenmax:addM}
  \tcp{add $X_i$ to maximal set}}
}
\end{tightalgo}
\end{frame}


  \def\ON#1{\ovalnode[fillstyle=solid,fillcolor=lightgray!50]{#1}{#1}}
  \def\nr{\Tr{
    \begin{tabular}{|c|c|c|c|c|}
        \hline
        $A$ & $B$ & $C$ & $D$ & \st{$E$}\\
        \hline
        1345 & 123456 & 2456 & 1356 & 12345\\
        \hline
    \end{tabular}}}
  \def\nA{\Tr{
    \begin{tabular}{|c|c|c|}
        \hline
        $AB$ & $AD$ & \st{$AE$}\\
        \hline
        1345 & 135 & 1345\\
        \hline
    \end{tabular}}}
  \def\nAB{\Tr{
    \begin{tabular}{|c|c|}
        \hline
        $ABD$ & \st{$ABE$}\\
        \hline
        135 & 1345\\
        \hline
    \end{tabular}}}
  \def\nABD{\Tr{
    \begin{tabular}{|c|}
        \hline
        \ON{$ABDE$}\\
        \hline
        135\\
        \hline
    \end{tabular}}}
  \def\nAD{\Tr{
    \begin{tabular}{|c|}
        \hline
        \st{$ADE$}\\
        \hline
        135\\
        \hline
    \end{tabular}}}
  \def\nB{\Tr{
    \begin{tabular}{|c|c|c|}
        \hline
        $BC$ & $BD$ & \st{$BE$}\\
        \hline
        2456 & 1356 & 12345\\
        \hline
    \end{tabular}}}
  \def\nBC{\Tr{
    \begin{tabular}{|c|}
        \hline
        \ON{$BCE$}\\
        \hline
        245\\
        \hline
    \end{tabular}}}
  \def\nBD{\Tr{
    \begin{tabular}{|c|}
        \hline
        \st{$BDE$}\\
        \hline
        135\\
        \hline
    \end{tabular}}}
  \def\nC{\Tr{
    \begin{tabular}{|c|}
        \hline
        \st{$CE$}\\
        \hline
        245\\
        \hline
    \end{tabular}}}
  \def\nD{\Tr{
    \begin{tabular}{|c|}
        \hline
        \st{$DE$}\\
        \hline
        135\\
        \hline
    \end{tabular}}}
\begin{frame}{Mining Maximal Frequent Itemsets}
  \centering
  \psset{unit=1in}
  \scalebox{0.7}{
  \psset{tpos=0.5,treesep=0.15,levelsep=*0.75,radius=0.05}
  \pstree[]{\nr}{
    \pstree[]{\nA\ncput*{$P_A$}}{
    \pstree[]{\nAB\ncput*{$P_{AB}$}}{\nABD\ncput*{$P_{ABD}$}}
    \nAD\ncput*{$P_{AD}$}
    }
    \pstree[]{\nB\ncput*{$P_B$}}{
    \nBC\ncput*{$P_{BC}$}
    \nBD\ncput*{$P_{BD}$}
    }
    \nC\ncput*{$P_C$}
    \nD\ncput*{$P_D$}
  }
  }
\end{frame}

\ifdefined\wox \begin{frame} \titlepage \end{frame} \fi

\begin{frame}{Mining Closed Frequent Itemsets: Charm Algorithm}
Mining closed frequent itemsets requires that we perform closure
checks, that is, whether $X = \vc(X)$. Direct closure checking
can be very expensive.

Given a collection of IT-pairs $\{\tup{X_i, \vT(X_i)}\}$, Charm
uses the following three properties:

\begin{enumerate}[Property (1)]
\item If $\vT(X_i) = \vT(X_{j})$, then $\vc(X_i)=\vc(X_{j})=\vc(X_i \cup
  X_{j})$, which implies that we can replace every occurrence of $X_i$
  with $X_i \cup X_{j}$ and prune the branch under $X_{j}$ because its
  closure is identical to the closure of $X_i \cup X_{j}$.

\item If $\vT(X_i) \subset \vT(X_{j})$, then $\vc(X_i) \neq \vc(X_{j})$ but
  $\vc(X_i)=\vc(X_i \cup X_{j})$, which means that we can replace every
  occurrence of $X_i$ with $X_i \cup X_{j}$, but we cannot prune $X_{j}$
  because it generates a different closure. Note that if $\vT(X_i) \supset
  \vT(X_{j})$ then we simply interchange the role of $X_i$ and $X_{j}$.

\item If $\vT(X_i) \neq \vT(X_{j})$, then $\vc(X_i) \neq \vc(X_{j}) \neq
  \vc(X_i \cup X_{j})$. In this case we cannot remove either $X_i$ or
  $X_{j}$, as each of them generates a different closure.
\end{enumerate}
\end{frame}

\begin{frame}{Charm Algorithm: Closed Itemsets}
\begin{footnotesize}
\begin{tightalgo}[H]{\textwidth-18pt}
\SetKwInOut{AlgorithmA}{\algcharm\ ($P$, $\minsup$, $\cC$)}
\tcp{Initial Call: $\cC \assign \emptyset$,
$P \assign \bigl\{ \tup{i,\vT(i)} : i \in \cI,
sup(i) \ge \minsup\bigr\}$}
\AlgorithmA{}
Sort $P$ in increasing order of support (i.e., by increasing $|\vT(X_i)|$)\;
\ForEach{$\tup{X_i, \vT(X_i)} \in P$}{
  $P_i \assign \emptyset$\;
  \ForEach{$\tup{X_{j},\vT(X_{j})} \in P$, with $j > i$}{
    $X_{ij} = X_i \cup X_{j}$\;
    $\vT(X_{ij}) = \vT(X_i)\; \cap\; \vT(X_{j})$\;
    \If{$\supp(X_{ij}) \geq \minsup$}{
      \eIf(\tcp*[h]{Property 1}
        \nllabel{alg:fpm:sumrep:algcharm:P1}){$\vT(X_i) = \vT(X_{j})$ }{
            Replace $X_i$ with  $X_{ij}$ in $P$ and $P_i$\;
            Remove $\tup{X_{j},\vT(X_{j})}$ from
            $P$
      }{
        \eIf(\tcp*[h]{Property 2}
        \nllabel{alg:fpm:sumrep:algcharm:P2}){%
        $\vT(X_i) \subset \vT(X_{j})$}{
            Replace $X_i$ with  $X_{ij}$ in $P$ and
            $P_i$
        }(~\tcp*[h]{Property 3}
          \nllabel{alg:fpm:sumrep:algcharm:P3}){
            $P_i \assign P_i \cup \bigl\{ \tup{X_{ij}, \vT(X_{ij})}
            \bigr\}$
        }
      }
    }
  }
  \lIf{$P_i \ne \emptyset$}{
    \algcharm($P_i$, $\minsup$, $\cC$)
  }
  \If{$\not\exists Z \in \cC, \text{ such that }
  X_i \subseteq Z \text{ and } \vT(X_i) = \vT(Z)$}{
        $\cC = \cC \cup X_i\quad$ \tcp{Add $X_i$ to closed
        set}\nllabel{alg:fpm:sumrep:algcharm:addtoC}
  }
}
\end{tightalgo}
\end{footnotesize}
\end{frame}

  
\def\ON#1{\ovalnode[fillstyle=solid,fillcolor=lightgray!50]{#1}{#1}}
  \vspace{0.2in}
\label{fig:fpm:sumrep:charmA}
  \def\nrA{\TR{
    \begin{tabular}{|c|}
        \hline
        \st{$A$} \st{$AE$} \ON{$AEB$}\\
        \hline
        1345\\
        \hline
    \end{tabular}}}
  \def\nrC{\TR{
    \begin{tabular}{|c|}
        \hline
        $C$\\
        \hline
        2456\\
        \hline
    \end{tabular}}}
  \def\nrD{\TR{
    \begin{tabular}{|c|}
        \hline
        $D$\\
        \hline
        1356\\
        \hline
    \end{tabular}}}
  \def\nrE{\TR{
    \begin{tabular}{|c|}
        \hline
        $E$\\
        \hline
        12345\\
        \hline
    \end{tabular}}}
  \def\nrB{\TR{
    \begin{tabular}{|c|}
        \hline
        $B$\\
        \hline
        123456\\
        \hline
    \end{tabular}}}
    \def\nPA{\TR{
        \begin{tabular}{|c|}
            \hline
            \st{$AD$} \st{$ADE$} \ON{$ADEB$}\\
            \hline
            135\\
            \hline
        \end{tabular}}}
\begin{frame}{Mining Frequent Closed Itemsets: Charm}
\framesubtitle{Process $A$}
  \psset{unit=1in}
  \psmatrix[rowsep=0.75in,colsep=0in]
  [name=A]\nrA & \nrC & \nrD & \nrE & [name=B]\nrB\\
  [name=PA]\nPA
  \endpsmatrix
  \ncline{A}{PA}\ncput*{$P_A$}
  \ncbox[nodesep=2pt,boxsize=0.3]{A}{B}
\end{frame}



  \def\nr{\Tr{
    \begin{tabular}{|c|c|c|c|c|}
        \hline
        \st{$A$} \st{$AE$} \ON{$AEB$} & \st{$C$} \ON{$CB$} & \st{$D$}
        \ON{$DB$}
        & \st{$E$} \ON{$EB$} & \ON{$B$}\\
        \hline
        1345 & 2456 & 1356 & 12345 & 123456\\
        \hline
    \end{tabular}}}
    \def\nA{\Tr{
        \begin{tabular}{|c|}
            \hline
            \st{$AD$} \st{$ADE$} \ON{$ADEB$}\\
            \hline
            135\\
            \hline
        \end{tabular}}}
  \def\nC{\Tr{
    \begin{tabular}{|c|}
            \hline
            \st{$CE$} \ON{$CEB$}\\
            \hline
            245\\
            \hline
        \end{tabular}}}
  \def\nD{\Tr{
    \begin{tabular}{|c|}
            \hline
            \st{$DE$} $DEB$\\
            \hline
            135\\
            \hline
        \end{tabular}}}
\begin{frame}{Mining Frequent Closed Itemsets: Charm}
  \psset{unit=1in}
  \psset{tpos=0.5,treesep=0.15,levelsep=*0.75,radius=0.05}
  \pstree[]{\nr}{
    \nA\ncput*{$P_A$}
    \nC\ncput*{$P_C$}
    \nD\ncput*{$P_D$}
    }%}}
\end{frame}

\ifdefined\wox \begin{frame} \titlepage \end{frame} \fi

\begin{frame}{Nonderivable Itemsets}
\small

An itemset is called {\em nonderivable} if its support cannot be deduced
from the supports of its subsets.  The set of all frequent nonderivable
itemsets is a summary or condensed representation of the set of all
frequent itemsets.  Further, it is lossless with respect to support,
that is, the exact support of all other frequent itemsets can be deduced
from~it.

\medskip{\bf Generalized Itemsets:} Let $X$ be a $k$-itemset, that is,
$X=\{x_1,x_2,\ldots,x_k\}$.  The $k$ tidsets  $\vT(x_i)$ for each item
$x_i\in X$ induce a partitioning of the set of all tids into $2^k$
regions, where each partition contains the tids for some subset of items
$Y\subseteq X$, but for none of the remaining items $Z=X\setminus Y$. 

\medskip Each partition is therefore the tidset of a {\em generalized
itemset} $Y\ol{Z}$, where $Y$ consists of regular items and $Z$ consists
of negated items. 

\medskip
Def\/{i}ne the support of a generalized itemset $Y\ol{Z}$ as the number
of transactions that contain all items in $Y$ but no item in $Z$:
\begin{align*}
\tcbhighmath{
  \supp(Y\ol{Z}) = \bigl| \{ t \in \cT \mid Y \subseteq \vI(t)
  \text{ and } Z \cap \vI(t) = \emptyset\} \bigr|
}
\end{align*}
\end{frame}


\begin{frame}{Tidset Partitioning Induced by $\vT({\it A})$,$\vT({\it C})$, and $\vT({\it D})$} 

\begin{tabular}{TT}
{
\begin{tabular}{|c|c|}
  \hline
  Tid & Itemset\\
\hline
1 & $\mathit{ABDE}$\\
2 & $\mathit{BCE}$\\
3 & $\mathit{ABDE}$\\
4 & $\mathit{ABCE}$\\
5 & $\mathit{ABCDE}$\\
6 & $\mathit{BCD}$\\
\hline
\end{tabular} 
} & { 
\small
\psset{unit=1cm}
\scalebox{0.75}{
\begin{pspicture}(11,9.5)
  \psframe[](0.5,0)(10.5,9.5)
  \pscircle[linewidth=2pt](4,6){3}
  \pscircle[linewidth=2pt](7,6){3}
  \pscircle[linewidth=2pt](5.5,3.5){3}
  \uput[d](2.5,7){$\vT(A\ol{C}\ol{D})=\emptyset$}
  \uput[d](8.5,7){$\vT(\ol{A}C\ol{D})=2$}
  \uput[d](5.5,2.5){$\vT(\ol{A}\ol{C}D)=\emptyset$}
  \uput[d](5.5,7.5){$\vT(\mathit{AC}\ol{D})=4$}
  \uput[d](3.75,4.15){$\vT(A\ol{C}D)=13$}
  \uput[d](7.25,4.15){$\vT(\ol{A}\mathit{CD})=6$}
  \uput[d](5.5,5.5){$\vT(\mathit{ACD})=5$}
  \uput[d](2,1.5){$\vT(\ol{A}\ol{C}\ol{D})=\emptyset$}
  \uput[ur](1,8){$\vT(A)$}
  \uput[ur](9.15,8){$\vT(C)$}
  \uput[d](5.5,0.6){$\vT(D)$}
\end{pspicture}
}
} \\
\end{tabular}
\end{frame}


\begin{frame}{Inclusion--Exclusion Principle: Support Bounds}


Let $Y\ol{Z}$ be a generalized itemset, and let $X = Y \cup Z = \mathit{YZ}$.
The inclusion--exclusion principle allows one to directly compute the
support of $Y\ol{Z}$ as a combination of the supports for all itemsets
$W$, such that $Y \subseteq W \subseteq X$:

\medskip

\begin{empheq}[box=\tcbhighmath]{align*}
\begin{split}
  \supp(Y\ol{Z}) & = \sum_{Y \subseteq W \subseteq X} -1^{\card{W\setminus Y}}\cdot \supp(W)
\end{split}
\end{empheq}

%\medskip
%From the $2^k$
%possible subsets $Y\subseteq X$, we derive $2^{k-1}$ lower bounds and
%$2^{k-1}$ upper bounds  for $\supp(X)$, obtained after
%setting $\supp(Y\ol{Z}) \ge 0$
%%\begin{empheq}[box=\tcbhighmath]{align*}
%%\begin{split}
%\begin{align*}
%\text{{\bf Upper Bounds }} & (|X\setminus Y| \mbox{ is odd}): & &  
%\supp(X) & \leq\sum_{Y \subseteq W \subset X} -1^{\lB(|X\setminus Y|+1\rB)} \supp(W)\\[2pt]
%\text{{\bf Lower Bounds }} & (|X\setminus Y| \mbox{ is even}): & & 
%\supp(X) & \geq\sum_{Y \subseteq W \subset X} -1^{\lB(|X\setminus Y|+1\rB)} \supp(W) \\
%%\end{split}
%%\end{empheq}
%\end{align*}
\end{frame}



\begin{frame}{Inclusion--Exclusion for Support}
Consider the generalized itemset
$\ol{A}C\ol{D} = C\ol{\mathit{AD}}$, where $Y=C$, $Z=\mathit{AD}$ and $X=\mathit{YZ}=\mathit{ACD}$.
In the Venn diagram,
we start with all the tids in $\vT(C)$,
and remove the tids contained in $\vT(AC)$ and $\vT(CD)$.
However, we
realize that in terms of support this removes $\supp(\mathit{ACD})$ twice, so we need to add
it back. In other words, the support of $C\ol{\mathit{AD}}$ is given as
\begin{align*}
\supp(C\ol{\mathit{AD}}) &= \supp(C) - \supp(\mathit{AC}) - \supp(\mathit{CD}) + \supp(\mathit{ACD})\\[2pt]
  & = 4-2-2+1=1
\end{align*}
But, this is precisely what the inclusion--exclusion formula gives:
\begin{align*}
\begin{array}{@{}l@{\,\,}c@{\,}ll@{}}
\supp(C\ol{\mathit{AD}}) &= &~ (-1)^0 \;\supp(C)+ &  {W = C,  |W\setminus Y| = 0}\\[2pt]
                &  &~  (-1)^1 \;\supp(\mathit{AC})+ &  {W = \mathit{AC}, |W\setminus Y| = 1}\\[2pt]
                &  &~  (-1)^1 \;\supp(CD)+ &  {W = \mathit{CD}, |W\setminus Y| = 1}\\[2pt]
                &  &~   (-1)^2 \;\supp(\mathit{ACD})& {W = \mathit{ACD}, |W\setminus Y| = 2}\\[2pt]
                &= &~ \supp(C) - \supp(\mathit{AC}) -  & \\[2pt]
                &  &~ \supp(\mathit{CD}) + \supp(\mathit{ACD}) & \\
\end{array}
\end{align*}
\end{frame}


\begin{frame}{Support Bounds}

Because the support of any (generalized)
itemset must be non-negative, we can derive a bound on the support of
$X$ from each of the $2^k$ generalized itemsets by setting
$\supp(Y\ol{Z}) \ge 0$. 

Thus, from the $2^k$
possible subsets $Y\subseteq X$, we derive $2^{k-1}$ lower bounds and
$2^{k-1}$ upper bounds  for $\supp(X)$, obtained after
setting $\supp(Y\ol{Z}) \ge 0$:

\begin{empheq}[box=\tcbhighmath]{align}
\text{{\bf Upper Bounds} ($|X\setminus Y|$ is odd):} & &
\supp(X) & \leq\sum_{Y \subseteq W \subset X} -1^{\lB(|X\setminus
W|+1\rB)} \supp(W) \label{eq:fpm:sumrep:IEPU}\\[2pt]
\text{{\bf Lower Bounds} ($|X\setminus Y|$ is even):} & &
\supp(X) & \geq\sum_{Y \subseteq W \subset X} -1^{\lB(|X\setminus
W|+1\rB)} \supp(W) \label{eq:fpm:sumrep:IEPL}
\end{empheq}

\end{frame}

\begin{frame}{Support Bounds for Subsets}
  \begin{center}
    \scalebox{0.85}{\hskip10pt\begin{minipage}{5in}
  \begin{pspicture}(12.75,6.5)
  \psframe[](-0.5,-0.5)(12.75,6.5)
  \psset{mnode=oval}
  \psmatrix[colsep=1cm,rowsep=0.75cm]
  & [mnode=none]subset lattice & & & &[mnode=none]\\
  & $\mathit{ACD}$ & & [mnode=none]sign & [mnode=none]inequality
  & [mnode=none]level\\
  $\mathit{AC}$ & $\mathit{AD}$ & $\mathit{CD}$ & [mnode=none]1 & [mnode=none]$\le$ &
  [mnode=none]$1$\\
  $A$ & $C$ & $D$ & [mnode=none]$-$1 & [mnode=none]$\ge$ & [mnode=none]$2$\\
  & $\emptyset$ & & [mnode=none]1 & [mnode=none]$\le$  & [mnode=none]$3$
  \ncline{2,2}{3,1}
  \ncline{2,2}{3,2}
  \ncline{2,2}{3,3}
  \ncline{3,1}{4,1}
  \ncline{3,1}{4,2}
  \ncline{3,2}{4,1}
  \ncline{3,2}{4,3}
  \ncline{3,3}{4,2}
  \ncline{3,3}{4,3}
  \ncline{5,2}{4,1}
  \ncline{5,2}{4,2}
  \ncline{5,2}{4,3}
  \endpsmatrix
  \end{pspicture}
  \end{minipage}}
  \end{center}
\end{frame}


\begin{frame}{Support Bounds}
\framesubtitle{Example}

 For example, if $Y=C$, then the inclusion-exclusion principle
  gives us
  \begin{align*}
\supp(C\ol{\mathit{AD}}) = \supp(C) - \supp(\mathit{AC}) - \supp(\mathit{CD}) + \supp(\mathit{ACD})
\end{align*}
Setting $\supp(C\ol{\mathit{AD}}) \ge 0$, we obtain
  \begin{align*}
\supp(\mathit{ACD}) \ge  -\supp(C) + \supp(\mathit{AC}) + \supp (\mathit{CD})
\end{align*}

\medskip

From each of the partitions, we
get one bound, and out of the eight possible regions, exactly four give
upper bounds and the other four give lower bounds for the support of $\mathit{ACD}$:
\begin{align*}
\begin{array}{lll}
    \supp(\mathit{ACD}) & \geq 0                   & \mbox{ when } Y=\mathit{ACD}\\
    &\leq \supp(\mathit{AC})                     & \mbox{ when } Y=\mathit{AC}\\
    &\leq \supp(\mathit{AD})                  & \mbox{ when } Y=\mathit{AD}\\
    &\leq \supp(\mathit{CD})                     & \mbox{ when } Y=\mathit{CD}\\
    &\geq \supp(\mathit{AC}) +\supp(\mathit{AD}) - \supp(A)   & \mbox{ when } Y=A\\
    &\geq \supp(\mathit{AC}) + \supp(\mathit{CD}) - \supp(C)  & \mbox{ when } Y=C\\
    &\geq \supp(\mathit{AD}) +\supp(\mathit{CD}) - \supp(D)   & \mbox{ when } Y=D\\
    &\leq \supp(\mathit{AC}) + \supp(\mathit{AD}) + \supp(\mathit{CD})- \\
    & \quad \supp(A) - \supp(C) - \supp(D)
    + \supp(\emptyset) &\mbox{ when } Y=\emptyset\\
\end{array}
\end{align*}

\end{frame}



\begin{frame}{Nonderivable Itemsets}

  \small
  Given an itemset $X$, and $Y \subseteq X$, let $\mathit{IE}(Y)$ denote the
summation
\begin{align*}
  \mathit{IE}(Y) = \sum_{Y \subseteq W \subset X}\, -1^{\lB(|X\setminus
Y|+1\rB)} \cdot \supp(W)
\end{align*}

\medskip
Then, the sets of all upper and lower bounds for
$\supp(X)$ are given as
\begin{empheq}[box=\tcbhighmath]{align*}
\begin{split}
  \mathit{UB}(X) & = \Bigl\{\mathit{IE}(Y) \big|\; Y \subseteq X,\; \card{X\setminus Y}
  \text{ is odd} \Bigr\}\\
  \mathit{LB}(X) & = \Bigl\{\mathit{IE}(Y) \big|\; Y \subseteq X,\; \card{X\setminus Y}
  \text{ is even} \Bigr\}
\end{split}
\end{empheq}

\bigskip
An itemset $X$ is called {\em nonderivable} if $\max\{\mathit{LB}(X)\} \ne
\min\{\mathit{UB}(X)\}$,
%which implies that the support of $X$ cannot be derived from the
%support values of its subsets;
and we know only the range of possible values, that is,
$$\supp(X) \in \Bigl[  \max\{\mathit{LB}(X)\}, \min\{\mathit{UB}(X)\} \Bigr]$$
$X$ is derivable if
$\supp(X) = \max\{\mathit{LB}(X)\} = \min\{\mathit{UB}(X)\}$
%because in this case $\supp(X)$
%can be derived exactly using the supports of its subsets.
Thus, the set of all frequent nonderivable itemsets is given as
\begin{align*}
  \cN = \bigl\{X \in \cF \mid \max\{\mathit{LB}(X)\} \ne \min\{\mathit{UB}(X)\}\bigr\}
\end{align*}
where $\cF$ is the set of all frequent itemsets.
\end{frame}

%

\begin{frame}{Nonderivable Itemsets: Example}
  \small
Consider the support bound formulas for
$\supp(\mathit{ACD})$. The lower bounds are
\begin{align*}
 \supp(\mathit{ACD}) & \geq 0 \\
&\geq \supp(\mathit{AC}) +\supp(\mathit{AD}) - \supp(A) = 2+3-4 = 1\\
&\geq \supp(\mathit{AC}) + \supp(\mathit{CD}) - \supp(C) = 2+2-4 = 0\\
&\geq \supp(\mathit{AD}) +\supp(\mathit{CD}) - \supp(D) = 3+2-4 = 0
\end{align*}
and the upper bounds are
\begin{align*}
 \supp(\mathit{ACD}) & \leq \supp(\mathit{AC})=2\\
            &\leq \supp(\mathit{AD})=3\\
            &\leq \supp(\mathit{CD})=2\\
            &\leq \supp(\mathit{AC}) + \supp(\mathit{AD}) + \supp(\mathit{CD}) - \supp(A)
            - \supp(C) -\\
            &\quad \supp(D)
            + \supp(\emptyset) = 2+3+2-4-4-4+6 = 1
\end{align*}

\medskip
Thus, we have
\begin{align*}
  \mathit{LB}(\mathit{ACD}) & = \{ 0,1\} & \max \{ \mathit{LB}(\mathit{ACD}) \} & = 1\\
  \mathit{UB}(\mathit{ACD}) & = \{1,2,3\} & \min \{ \mathit{UB}(\mathit{ACD}) \} & = 1
\end{align*}
Because $\max\{\mathit{LB}(\mathit{ACD})\} = \min\{\mathit{UB}(\mathit{ACD})\}$ we conclude that $\mathit{ACD}$ is
derivable.

\end{frame}

