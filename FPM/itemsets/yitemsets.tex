\lecture{itemsets}{itemsets}

\date{Chapter 8: Itemset Mining}

\begin{frame}
\titlepage
\end{frame}

\newcommand{\algbf}{\textsc{BruteForce}\xspace}
\newcommand{\algcomputesupport}{\textsc{ComputeSupport}\xspace}
\newcommand{\algApriori}{\textsc{Apriori}\xspace}
\newcommand{\algextprefix}{\textsc{ExtendPrefixTree}\xspace}
\newcommand{\algeclat}{\textsc{Eclat}\xspace}
\newcommand{\algdeclat}{\textsc{dEclat}\xspace}
\newcommand{\algfpgrowth}{\textsc{FPGrowth}\xspace}
\newcommand{\algrulemining}{\textsc{AssociationRules}\xspace}


\begin{frame}{Frequent Itemset Mining}

In many applications one is interested in how often two or more
objects of interest co-occur, the so-called {\em itemsets}. 

\medskip
The prototypical application was {\it market basket
analysis}, that is, to mine the sets of items that are frequently
bought together at a supermarket by analyzing the customer
shopping carts (the so-called ``market baskets''). 

\medskip
Frequent itemset mining is a basic exploratory mining task, since the
since the basic operation is to find the co-occurrence, which gives an
estimate for the joint probability mass function.

\medskip
Once we mine the frequent sets, they allow us to extract {\em
association rules} among the itemsets, where we make some
statement about how likely are two sets of items to co-occur or
to conditionally occur. 

\end{frame}


\begin{frame}{Frequent Itemsets: Terminology}

{\bf Itemsets:}  
Let $\cI = \{x_1, x_2, \ldots, x_m\}$ be a set of elements called
{\em items}.  A set $X \subseteq \cI$ is called an {\em itemset}.
An itemset of cardinality
(or size) $k$ is called a $k$-itemset. Further, we denote by $\Ik$
the set of all $k$-itemsets, that is, subsets of $\cI$ with size
$k$.  

\medskip{\bf Tidsets:}
Let $\cT = \{t_1, t_2, \ldots, t_n\}$ be another set of
elements called transaction identif\/{i}ers or {\em tids}.  A set $T
\subseteq \cT$ is called a {\em tidset}. 
Itemsets and tidsets are kept
sorted in lexicographic order.


\medskip{\bf Transactions:}
A {\em transaction}
is a tuple of the form $\tup{t,X}$, where $t
\in \cT$ is a unique transaction identif\/{i}er, and $X$ is an
itemset.  

\medskip{\bf Database:}
A binary database $\bD$ is a binary relation on the set of tids
\index{binary database} and items, that is, $\bD \subseteq \cT
\times \cI$. We say that tid $t \in \cT$ {\em contains} item $x
\in \cI$ iff $(t,x) \in \bD$.  In other words, $(t,x) \in \bD$ iff
$x \in X$ in the tuple $\tup{t,X}$. We say that tid $t$ {\em
contains} itemset $X=\{x_1,x_2,\ldots,x_k\}$ iff $(t,x_i) \in \bD$
for all $i=1,2,\ldots,k$.

\end{frame}


\begin{frame}{Database Representation}
Let $\pow{X}$ denote the powerset of $X$, that is,
the set of all subsets of $X$.  Let $\vI: \pow{\cT} \to
\pow{\cI}$ be a function, def\/{i}ned as follows:
\begin{align*}
\vI(T) = \{ x \mid\; \forall t\in T, \;t \mbox{ contains } x \}
\end{align*}
where $T \subseteq \cT$, and $\vI(T)$
is the set of items that are common to {\em
all} the transactions in the tidset $T$. In particular, $\vI(t)$ is the
set of items contained in tid $t \in \cT$. 

\medskip
Let $\vT: \pow{\cI} \to \pow{\cT}$ be a function, def\/{i}ned as
follows:
\begin{align}
  \vT(X) = \{ t \mid\; t \in \cT \text{ and $t$ contains } X\}
\label{eq:fpm:iset:t}
\end{align}
where $X \subseteq \cI$, and $\vT(X)$ is the set of
tids that contain {\em
all} the items in the itemset $X$. In particular, $\vT(x)$ is the
set of tids that contain the single item $x \in \cI$.

\medskip
The binary database $\bD$ can be represented as a
{\em horizontal} or 
{\em transaction database} consisting of tuples of the form
$\tup{t,\vI(t)}$, with $t \in \cT$. 

\medskip
The binary database $\bD$ can also be represented as a
{\em vertical} or {\em tidset database} containing a collection of tuples of the form
$\tup{x, \vT(x)}$, with $x\in \cI$.  
\end{frame}


\begin{frame}{Binary Database: Transaction and Vertical Format}
  
  \begin{tabular}{TTT}
	\small
      \begin{tabular}{|l||c|c|c|c|c|} \hline
        $\bD$ & $A$ & $B$ & $C$ & $D$ & $E$\\ \hline
        1 & 1 & 1 & 0 & 1 & 1\\ \hline
    2 & 0 & 1 & 1 & 0 & 1\\ \hline
    3 & 1 & 1 & 0 & 1 & 1\\ \hline
    4 & 1 & 1 & 1 & 0 & 1\\ \hline
    5 & 1 & 1 & 1 & 1 & 1\\ \hline
    6 & 0 & 1 & 1 & 1 & 0\\ \hline
      \end{tabular}
  &
      \begin{tabular}{|c|c|} \hline
        $t$ & $\vI(t)$\\ \hline
    1 & $\mathit{ABDE}$\\ \hline
    2 & $\mathit{BCE}$\\ \hline
    3 & $\mathit{ABDE}$\\ \hline
    4 & $\mathit{ABCE}$\\ \hline
    5 & $\mathit{ABCDE}$\\ \hline
    6 & $\mathit{BCD}$\\     \hline
      \end{tabular}
	&
      \begin{tabular}{|c|c|c|c|c|}
    \hline
	\multicolumn{5}{|c|}{$\vT(x)$}\\
	\hline
     $A$ & $B$ & $C$ & $D$ & $E$\\
    \hline
    1 & 1 & 2 & 1 & 1\\
    3 & 2 & 4 & 3 & 2\\
    4 & 3 & 5 & 5 & 3\\
    5 & 4 & 6 & 6 & 4\\
	& 5 &   &   & 5\\
     & 6 &   &   &  \\
    \hline
    \end{tabular}\\
	Binary Database & Transaction Database & Vertical Database
  \end{tabular}

  \bigskip
  This dataset $\bD$ has 5 items, $\cI = \{A,B,C,D,E\}$ and 6 tids
  $\cT = \{1,2,3,4,5,6\}$. 
  
  \medskip
  The 
  the f\/{i}rst
transaction is $\tup{1, \{A,B,D,E\}}$, where we omit item $C$ since
$(1,C) \not\in \bD$.  Henceforth, for
convenience, we drop the set notation for itemsets and
tidsets. Thus, we write $\tup{1,
\{A,B,D,E\}}$ as $\tup{1,\mathit{ABDE}}$.
\end{frame}

\begin{frame}{Support and Frequent Itemsets}
The {\em support}
of an itemset $X$ in a dataset $\bD$, denoted
$\supp(X)$, is the number of transactions in $\bD$ that
contain $X$:
\begin{align*}
\tcbhighmath{
\supp(X) = \bigl|\{ t \mid \; \tup{t,\vI(t)} \in \bD \mbox{ and
} X \subseteq \vI(t) \}\bigr| = \card{\vT(X)}
}
\end{align*}

\medskip
The {\em relative support}
of $X$ is the fraction of
transactions that contain $X$:
\begin{align*}
\tcbhighmath{
\rsupp(X) = \frac{\supp(X)}{\card{\bD}}
}
\end{align*}
It is an estimate of the {\em joint
  probability} of
the items comprising $X$.

\medskip
An itemset $X$ is said to be {\em frequent}
\index{frequent itemset}
in $\bD$ if $\supp(X) \geq
\minsup$, where $\minsup$ is a user def\/{i}ned {\em minimum support
threshold}.

\medskip
The set $\cF$ to denotes the
set of all frequent itemsets, and $\Fk{k}$ denotes the set of
frequent $k$-itemsets.

\end{frame}

\begin{frame}{Frequent Itemsets}
  \framesubtitle{Minimum support: $\minsup=3$}
\begin{center}
  \begin{tabular}{cc}
      \begin{tabular}{|c|c|} \hline
        $t$ & $\vI(t)$\\ \hline
    1 & $\mathit{ABDE}$\\ \hline
    2 & $\mathit{BCE}$\\ \hline
    3 & $\mathit{ABDE}$\\ \hline
    4 & $\mathit{ABCE}$\\ \hline
    5 & $\mathit{ABCDE}$\\ \hline
    6 & $\mathit{BCD}$\\     \hline
      \end{tabular}
&
\begin{tabular}{|c|c|}
\hline
$\supp$ & itemsets\\
\hline
6 & $B$\\
5 & $E, \mathit{BE}$\\
4 & $A, C, D, \mathit{AB}, \mathit{AE}, \mathit{BC}, \mathit{BD}, \mathit{ABE}$\\
3 &  $\mathit{AD}, \mathit{CE}, \mathit{DE}, \mathit{ABD}, \mathit{ADE}, \mathit{BCE}, \mathit{BDE}, \mathit{ABDE}$\\
\hline
\end{tabular}\\
Transaction Database & Frequent Itemsets
\end{tabular}
\end{center}

\small
\bigskip
The 19 frequent itemsets shown in the table comprise the set $\cF$.
    The sets of all frequent $k$-itemsets are
    \begin{align*}
    \Fk{1} & = \{A,B,C,D,E\}\\[-3pt]
    \Fk{2} & = \{\mathit{AB}, \mathit{AD}, \mathit{AE}, \mathit{BC}, \mathit{BD}, \mathit{BE}, \mathit{CE}, \mathit{DE}\}\\[-3pt]
    \Fk{3} & = \{\mathit{ABD}, \mathit{ABE}, \mathit{ADE}, \mathit{BCE}, \mathit{BDE}\}\\[-3pt]
    \Fk{4} & = \{\mathit{ABDE}\}
    \end{align*}


\end{frame}



\begin{frame}{Itemset Mining Algorithms: Brute Force}
\small
The brute-force
algorithm enumerates all the
possible itemsets $X \subseteq \cI$, and for each such subset determines
its support in the input dataset $\bD$. The method comprises two main
steps: (1) candidate generation and (2) support
computation.

\medskip{\bf Candidate Generation:}
This step generates all the subsets of
$\cI$, which are called {\em candidates}, as each itemset is
potentially a candidate frequent pattern. The candidate itemset search
space is clearly exponential because there are $2^{\card{\cI}}$
potentially frequent itemsets.   


\medskip{\bf Support Computation:}
This step computes the support of each
candidate pattern $X$ and determines if it is frequent.
For each transaction
$\tup{t, \vI(t)}$ in the database, we determine if $X$ is a
subset of $\vI(t)$. If so, we increment the support of $X$.


\medskip{\bf Computational Complexity:}
Support computation takes time $O(|\cI| \cdot | \bD |)$ in the worst
case, and because there are $O(2^{\card{\cI}})$ possible candidates,
the computational complexity of the brute-force method is
$O(\card{\cI}\cdot\card{\bD}\cdot2^{\card{\cI}})$.

\end{frame}


\begin{frame}[fragile]{Brute Force Algorithm}
\begin{tightalgo}[H]{\textwidth-18pt}
\SetKwInOut{AlgorithmBF}{\algbf ($\bD$, $\cI$, $\minsup$)}
\AlgorithmBF{}
$\cF \assign \emptyset$ \tcp{set of frequent itemsets}\;
\ForEach{$X \subseteq \cI$}{
  $\supp(X)$ $\assign$ \algcomputesupport ($X, \bD$)\;
  \If{$\supp(X) \geq \minsup$}{
  $\cF \assign \cF \cup \bigl\{(X, \supp(X))\bigr\}$\;
  }
}
\Return{$\cF$}
\end{tightalgo}
\begin{tightalgo}[H]{\textwidth-18pt}
\SetKwInput{AlgorithmCS}{\algcomputesupport ($X, \bD$)}
\AlgorithmCS{}
$\supp(X) \assign 0$\;
\ForEach{$\tup{t, \vI(t)} \in \bD$}{
  \If{$X \subseteq \vI(t)$}{
    $\supp(X) \assign \supp(X)+1$\;
  }
}
\Return{$\supp(X)$}
\end{tightalgo}
\end{frame}

\ifdefined\wox \begin{frame} \titlepage \end{frame} \fi

\begin{frame}{Itemset lattice and pref\/{i}x-based search tree}

  \begin{columns}
	\column{0.4\textwidth}
  \small
Itemset search space is a
lattice where any two itemsets $X$
and $Y$ are connected by a link iff $X$ is an {\em immediate subset}
of $Y$, that is, $X \subseteq Y$ and $\card{X} = \card{Y}-1$.  

\bigskip
Frequent itemsets can
enumerated using either a BFS or DFS
search on the {\em pref\/{i}x tree},
where two
itemsets $X,Y$ are connected by a link iff $X$ is an immediate
subset and pref\/{i}x of $Y$. 
This allows one to enumerate itemsets
starting with an empty set, and adding one more item at a time.

\column{0.6\textwidth}  
\centerline{
  \scalebox{0.45}{%
  \psset{mnode=oval}
  \psmatrix[colsep=0.25cm,rowsep=2.5cm]
  &&&&[name=empty]$\emptyset$\psspan{2}\\\empty
   &&[name=A]$A$ & [name=B]$B$ & [name=C]$C$\psspan{2} & [name=D]$D$ &
   [name=E]$E$\\\empty
   [name=AB]$\mathit{AB}$ & [name=AC]$\mathit{AC}$ & [name=AD]$\mathit{AD}$ & [name=AE]$\mathit{AE}$
   & [name=BC]$\mathit{BC}$ & [name=BD]$\mathit{BD}$& [name=BE]$\mathit{BE}$ & [name=CD]$\mathit{CD}$
   & [name=CE]$\mathit{CE}$ & [name=DE]$\mathit{DE}$\\\empty
   [name=ABC]$\mathit{ABC}$ & [name=ABD]$\mathit{ABD}$ & [name=ABE]$\mathit{ABE}$ & [name=ACD]$\mathit{ACD}$
   & [name=ACE]$\mathit{ACE}$ & [name=ADE]$\mathit{ADE}$ & [name=BCD]$\mathit{BCD}$ & [name=BCE]$\mathit{BCE}$
   & [name=BDE]$\mathit{BDE}$ & [name=CDE]$\mathit{CDE}$\\\empty
   &&[name=ABCD]$\mathit{ABCD}$ & [name=ABCE]$\mathit{ABCE}$ & [name=ABDE]$\mathit{ABDE}$\psspan{2}
      & [name=ACDE]$\mathit{ACDE}$ & [name=BCDE]$\mathit{BCDE}$\\\empty
      &&&&[name=ABCDE]$\mathit{ABCDE}$\psspan{2}
   \ncline[linewidth=3pt]{ABCDE}{ABCD}
   \ncline{ABCDE}{ABCE}
   \ncline{ABCDE}{ABDE}
   \ncline{ABCDE}{ACDE}
   \ncline{ABCDE}{BCDE}
   %
   \ncline[linewidth=3pt]{ABCD}{ABC}
   \ncline{ABCD}{ABD}
   \ncline{ABCD}{ACD}
   \ncline{ABCD}{BCD}
   %
   \ncline[linewidth=3pt]{ABCE}{ABC}
   \ncline{ABCE}{ABE}
   \ncline{ABCE}{ACE}
   \ncline{ABCE}{BCE}
   %
   \ncline[linewidth=3pt]{ABDE}{ABD}
   \ncline{ABDE}{ABE}
   \ncline{ABDE}{ADE}
   \ncline{ABDE}{BDE}
   %
   \ncline[linewidth=3pt]{ACDE}{ACD}
   \ncline{ACDE}{ACE}
   \ncline{ACDE}{ADE}
   \ncline{ACDE}{CDE}
   %
   \ncline[linewidth=3pt]{BCDE}{BCD}
   \ncline{BCDE}{BCE}
   \ncline{BCDE}{BDE}
   \ncline{BCDE}{CDE}
   %
   \ncline[linewidth=3pt]{ABC}{AB}
   \ncline{ABC}{AC}
   \ncline{ABC}{BC}
   %
   \ncline[linewidth=3pt]{ABD}{AB}
   \ncline{ABD}{AD}
   \ncline{ABD}{BD}
   %
   \ncline[linewidth=3pt]{ABE}{AB}
   \ncline{ABE}{AE}
   \ncline{ABE}{BE}
   %
   \ncline[linewidth=3pt]{ACD}{AC}
   \ncline{ACD}{AD}
   \ncline{ACD}{CD}
   %
   \ncline[linewidth=3pt]{ACE}{AC}
   \ncline{ACE}{AE}
   \ncline{ACE}{CE}
   %
   \ncline[linewidth=3pt]{ADE}{AD}
   \ncline{ADE}{AE}
   \ncline{ADE}{DE}
   %
   \ncline[linewidth=3pt]{BCD}{BC}
   \ncline{BCD}{BD}
   \ncline{BCD}{CD}
   %
   \ncline[linewidth=3pt]{BCE}{BC}
   \ncline{BCE}{BE}
   \ncline{BCE}{CE}
   %
   \ncline[linewidth=3pt]{BDE}{BD}
   \ncline{BDE}{BE}
   \ncline{BDE}{DE}
   %
   \ncline[linewidth=3pt]{CDE}{CD}
   \ncline{CDE}{CE}
   \ncline{CDE}{DE}
   %
   \ncline[linewidth=3pt]{AB}{A}
   \ncline{AB}{B}
   %
   \ncline[linewidth=3pt]{AC}{A}
   \ncline{AC}{C}
   %
   \ncline[linewidth=3pt]{AD}{A}
   \ncline{AD}{D}
   %
   \ncline[linewidth=3pt]{AE}{A}
   \ncline{AE}{E}
   %
   \ncline[linewidth=3pt]{BC}{B}
   \ncline{BC}{C}
   %
   \ncline[linewidth=3pt]{BD}{B}
   \ncline{BD}{D}
   %
   \ncline[linewidth=3pt]{BE}{B}
   \ncline{BE}{E}
   %
   \ncline[linewidth=3pt]{CD}{C}
   \ncline{CD}{D}
   %
   \ncline[linewidth=3pt]{CE}{C}
   \ncline{CE}{E}
   %
   \ncline[linewidth=3pt]{DE}{D}
   \ncline{DE}{E}
   %
   \ncline[linewidth=3pt]{A}{empty}
   \ncline[linewidth=3pt]{B}{empty}
   \ncline[linewidth=3pt]{C}{empty}
   \ncline[linewidth=3pt]{D}{empty}
   \ncline[linewidth=3pt]{E}{empty}
  \endpsmatrix
  }}%scalebox
\end{columns}
\end{frame}


\begin{frame}{Level-wise Approach: Apriori Algorithm}
If $X \subseteq Y$, then
$\supp(X) \geq \supp(Y)$, which leads to the following two observations:
(1) if $X$ is frequent, then any subset $Y \subseteq X$ is also frequent,
and (2) if $X$ is not frequent, then any superset $Y \supseteq X$ cannot
be frequent. 

\medskip
The {\em Apriori algorithm} utilizes these two properties
to signif\/{i}cantly improve the brute-force approach.  It employs a
level-wise or breadth-f\/{i}rst exploration of the itemset search space, and
prunes all supersets of any infrequent candidate, as no superset of
an infrequent itemset can be frequent. It also avoids generating any
candidate that has an infrequent subset.

\medskip
In addition to improving the candidate generation step via itemset
pruning, the Apriori method also signif\/{i}cantly improves the I/O
complexity. Instead of counting the support for a single itemset, it
explores the pref\/{i}x tree in a breadth-f\/{i}rst manner, and computes the
support of all the valid candidates of size $k$ that comprise level $k$ in the
pref\/{i}x tree.
\end{frame}


\SetKwInput{AlgorithmA}{\algApriori ($\bD$, $\cI$, $\minsup$)}
\SetKwInput{AlgorithmB}{\algcomputesupport ($\Ck{k}, \bD$)}
\SetKwInput{AlgorithmC}{\algextprefix ($\Ck{k}$)}
\begin{frame}[fragile]{The Apriori Algorithm}
\begin{tightalgo}[H]{\textwidth-18pt}
\AlgorithmA{}
$\cF \assign \emptyset$\;
$\Ck{1} \assign \{\emptyset\}$ \tcp{Initial pref\/{i}x tree with single items}
\lForEach{$i \in \cI$}{
Add $i$ as child of $\emptyset$ in $\Ck{1}$ with $sup(i) \assign 0$
}
$k \assign 1$ \tcp{$k$ denotes the level}
\While{$\Ck{k} \not= \emptyset$}{
  \nllabel{alg:fpm:iset:apr:whilestart}
  \algcomputesupport ($\Ck{k}, \bD$)\;
  \ForEach{ leaf $X \in \Ck{k}$}{
    \lIf{$\supp(X) \ge \minsup$}{
      $\cF \assign \cF \cup \bigl\{(X, \supp(X))\bigr\}$
    }
    \lElse{
    remove $X$ from $\Ck{k}$\nllabel{alg:fpm:iset:apr:rem}
    }
  }
 $\Ck{k+1} \assign$ \algextprefix ($\Ck{k}$)
 \nllabel{alg:fpm:iset:apr:ext}\;
  $k \assign k+1$\;
\nllabel{alg:fpm:iset:apr:whileend}
}
\Return{$\Fk{k}$}
\end{tightalgo}
\end{frame}



\begin{frame}[fragile]{Apriori Algorithm}
\begin{tightalgo}[H]{\textwidth-18pt}
\AlgorithmB{}
\ForEach{$\tup{t, \vI(t)} \in \bD$}{
  \ForEach{$k$-subset $X \subseteq \vI(t)$}{
    \lIf{$X \in \Ck{k}$}{
      $\supp(X) \assign \supp(X)+1$
    }
  }
}
\end{tightalgo}

\begin{tightalgo}[H]{\textwidth-18pt}
\AlgorithmC{}
\ForEach{$ \text{leaf } X_a \in \Ck{k}$}{
\ForEach{$\text{leaf } X_b \in \textsc{sibling}(X_a),
    \text{such that } b > a$}{
    $X_{ab} \assign X_a \cup X_b$\;
    \tcp{prune if there are any infrequent subsets}
    \If{$X_{j} \in \Ck{k}, \textbf{ for all }
    X_{j} \subset X_{ab}, \text{ such that }
    \card{X_{j}} = \card{X_{ab}}-1$}{
     Add $X_{ab}$ as child of $X_a$ with
     $sup(X_{ab}) \assign 0$}
   }
   \If{no extensions from $X_a$}{remove $X_a$ and its ancestors with no extensions from $\Ck{k}$ }
}
\Return $\Ck{k}$\;
\end{tightalgo}
\end{frame}



\begin{frame}[fragile]{Itemset Mining: Apriori Algorithm}
  \framesubtitle{Infrequent itemsets in gray}
  \psset{unit=0.75in}
  %C1
  \begin{center}

  %\begin{tabular}[t]{TTT}
  \begin{tabular}[t]{ccc}
  $\bD$ & & $\Ck{1}$\\
  {
      \begin{tabular}{|c|c|} \hline
        $t$ & $\vI(t)$\\ \hline
    1 & $\mathit{ABDE}$\\ \hline
    2 & $\mathit{BCE}$\\ \hline
    3 & $\mathit{ABDE}$\\ \hline
    4 & $\mathit{ABCE}$\\ \hline
    5 & $\mathit{ABCDE}$\\ \hline
    6 & $\mathit{BCD}$\\     \hline
      \end{tabular}
  }
	  & \hspace{0.5in} &
    {
    \scalebox{0.55}{
    \psset{treesep=0.3,levelsep=*0.75}
    \pstree[]{\Toval{$\emptyset(6)$}}{
    \Toval{$A$(4)}
    \Toval{$B$(6)}
    \Toval{$C$(4)}
    \Toval{$D$(4)}
    \Toval{$E$(5)}
    }
    }
    } \\
  \end{tabular}
 % C2
  \centerline{\hspace{0.8in}$\Ck{2}$}

  \scalebox{0.55}{%
  \psset{treesep=0.125,levelsep=*0.75}
  \pstree[]{\Toval{$\emptyset(6)$}}{
    \pstree[]{\Toval{$A$(4)}}{
      \Toval{$\mathit{AB}$(4)}
      \Toval[fillstyle=solid,fillcolor=lightgray]{$\mathit{AC}$(2)}
      \Toval{$\mathit{AD}$(3)}
      \Toval{$\mathit{AE}$(4)}
    }
    \pstree[]{\Toval{$B$(6)}}{
      \Toval{$\mathit{BC}$(4)}
      \Toval{$\mathit{BD}$(4)}
      \Toval{$\mathit{BE}$(5)}
    }
    \pstree[]{\Toval{$C$(4)}}{
      \Toval[fillstyle=solid,fillcolor=lightgray]{$\mathit{CD}$(2)}
      \Toval{$\mathit{CE}$(3)}
    }
    \pstree[]{\Toval{$D$(4)}}{
      \Toval{$\mathit{DE}$(3)}
    %}
    }}
    }
\end{center}
\end{frame}

\begin{frame}{Itemset Mining: Apriori Algorithm}
  \psset{unit=0.75in}
 % C3
\begin{minipage}{5cm}
  \begin{center}
	\hspace{1.2in}$\Ck{3}$

  \scalebox{0.55}{%
  \psset{treesep=0.2,levelsep=*0.75}
  \pstree[]{\Toval{$\emptyset(6)$}}{
    \pstree[]{\Toval{$A$(4)}}{
      \pstree[]{\Toval{$\mathit{AB}$(4)}}{
        \Toval{$\mathit{ABD}$(3)}
        \Toval{$\mathit{ABE}$(4)}
      }
      \pstree[]{\Toval{$\mathit{AD}$(3)}}{
        \Toval{$\mathit{ADE}$(3)}
      }
    }
    \pstree[]{\Toval{$B$(6)}}{
      \pstree[]{\Toval{$\mathit{BC}$(4)}}{
        \Toval{$\mathit{BCE}$(3)}
      }
      \pstree[]{\Toval{$\mathit{BD}$(4)}}{
        \Toval{$\mathit{BDE}$(3)}
      }
    }
    }
    }
\end{center}
\end{minipage}
 % C4
\hspace{3cm}
\begin{minipage}{2cm}
\begin{center}
 $\Ck{4}$

  \scalebox{0.55}}
\end{center}
\end{minipage}
\end{frame}

\begin{frame}{Apriori Algorithm: Pref\/{i}x Search Tree and Pruning}
  \psset{unit=1in}
  \centerline{
  \scalebox{0.45}{%
  \psset{mnode=oval, dash=3pt 2pt}
  \psmatrix[colsep=0.15cm,rowsep=2.5cm]
  &&&&[name=empty]$\emptyset$\psspan{2}\\\empty
  &[name=A]$A$(4)\psspan{2} & [mnode=none]& [name=B]$B$(6)\psspan{2}
   & [name=C]$C$(4)\psspan{2}
   & [name=D]$D$(4) &
   [name=E]$E$(5)\\\empty
   [name=AB]$\mathit{AB}$(4)
   & [name=AC,fillcolor=lightgray,fillstyle=solid]$\mathit{AC}$(2)
   & [name=AD]$\mathit{AD}$(3)
   & [name=AE]$\mathit{AE}$(4)
   & [name=BC]$\mathit{BC}$(4) & [name=BD]$\mathit{BD}$(4)& [name=BE]$\mathit{BE}$(5)
   & [name=CD,fillcolor=lightgray,fillstyle=solid]$\mathit{CD}$(2)
   & [name=CE]$\mathit{CE}$(3) & [name=DE]$\mathit{DE}$(3)\\\empty
   [name=ABC,linestyle=dashed]$\mathit{ABC}$ & [name=ABD]$\mathit{ABD}$(3)
   & [name=ABE]$\mathit{ABE}$(4) & [name=ACD,linestyle=dashed]$\mathit{ACD}$
   & [name=ACE,linestyle=dashed]$\mathit{ACE}$ & [name=ADE]$\mathit{ADE}$(3)
   & [name=BCD,linestyle=dashed]$\mathit{BCD}$
   & [name=BCE]$\mathit{BCE}$(3)
   & [name=BDE]$\mathit{BDE}$(3) & [name=CDE,linestyle=dashed]$\mathit{CDE}$\\\empty
   [name=ABCD,linestyle=dashed]$\mathit{ABCD}$
   & [name=ABCE,linestyle=dashed]$\mathit{ABCE}$ &
   [name=ABDE]$\mathit{ABDE}$(3)
   & [name=ACDE,linestyle=dashed]$\mathit{ACDE}$ & [mnode=none]
   & [mnode=none]& [name=BCDE,linestyle=dashed]$\mathit{BCDE}$\\\empty
      [name=ABCDE,linestyle=dashed]$\mathit{ABCDE}$
  \endpsmatrix
   \ncline[linestyle=dashed]{ABCDE}{ABCD}
   %
   \ncline[linestyle=dashed]{ABCD}{ABC}
   %
   \ncline[linestyle=dashed]{ABCE}{ABC}
   %
   \ncline{ABDE}{ABD}
   %
   \ncline[linestyle=dashed]{ACDE}{ACD}
   %
   \ncline[linestyle=dashed]{BCDE}{BCD}
   %
   \ncline{ABC}{AB}
   %
   \ncline{ABD}{AB}
   %
   \ncline{ABE}{AB}
   %
   \ncline[linestyle=dashed]{ACD}{AC}
   %
   \ncline[linestyle=dashed]{ACE}{AC}
   %
   \ncline{ADE}{AD}
   %
   \ncline{BCD}{BC}
   %
   \ncline{BCE}{BC}
   %
   \ncline{BDE}{BD}
   %
   \ncline[linestyle=dashed]{CDE}{CD}
   %
   \ncline{AB}{A}
   %
   \ncline{AC}{A}
   %
   \ncline{AD}{A}
   %
   \ncline{AE}{A}
   %
   \ncline{BC}{B}
   %
   \ncline{BD}{B}
   %
   \ncline{BE}{B}
   %
   \ncline{CD}{C}
   %
   \ncline{CE}{C}
   %
   \ncline{DE}{D}
   %
   \ncline{A}{empty}
   \ncline{B}{empty}
   \ncline{C}{empty}
   \ncline{D}{empty}
   \ncline{E}{empty}
 %
    \nput[labelsep=17pt]{120}{A}{\large Level 1}
    \nput[labelsep=17pt]{120}{AB}{\large Level 2}
    \nput[labelsep=17pt]{120}{ABC}{\large Level 3}
    \nput[labelsep=17pt]{120}{ABCD}{\large Level 4}
    \nput[labelsep=17pt]{120}{ABCDE}{\large Level 5}
    \psset{linestyle=dotted}
    \ncbox[linearc=0.3,boxsize=0.3,nodesep=0.1]{A}{E}
    \ncbox[linearc=0.3,boxsize=0.3,nodesep=0.1]{AB}{DE}
    \ncbox[linearc=0.3,boxsize=0.3,nodesep=0.1]{ABC}{CDE}
    \ncbox[linearc=0.3,boxsize=0.3,nodesep=0.1]{ABCD}{BCDE}
    \ncbox[linearc=0.3,boxsize=0.3,nodesep=0.1]{ABCDE}{ABCDE}
    }
  }
\end{frame}


\begin{frame}{Apriori Algorithm: Details}
  \small 
  
  Let $\Ck{k}$ denote the pref\/{i}x tree comprising all the
  candidate $k$-itemsets.  

\medskip
Apriori begins by inserting the single items into an initially
empty pref\/{i}x tree to populate $\Ck{1}$.

\medskip   The
support for the current candidates is obtained via
\algcomputesupport procedure that generates $k$-subsets of each
transaction in the database $\bD$, and for each such subset it
increments the support of the corresponding candidate in $\Ck{k}$ if it
exists. Next, we remove any infrequent candidate.

\medskip
The leaves of the pref\/{i}x tree that survive comprise the set of
frequent $k$-itemsets $\Fk{k}$, which are used to generate the candidate
$(k+1)$-itemsets for the next level. The \algextprefix procedure employs
pref\/{i}x-based extension for candidate generation. Given two frequent
$k$-itemsets $X_a$ and $X_b$ with a common $k-1$ length pref\/{i}x, that
is, given two sibling leaf nodes with a common parent, we generate the
$(k+1)$-length candidate $X_{ab} = X_a \cup X_b$.  This candidate is
retained only if it has no infrequent subset. F{i}nally, if a
$k$-itemset $X_a$ has no extension, it is pruned from the pref\/{i}x
tree, and we recursively prune any of its ancestors with no $k$-itemset
extension, so that in $\Ck{k}$ all leaves are at level $k$.  

\medskip If new
candidates were added, the whole process is repeated for the next level.
This process continues until no new candidates are added.
\end{frame}


\ifdefined\wox \begin{frame} \titlepage \end{frame} \fi


\begin{frame}{Tidset Intersection Approach: Eclat Algorithm}

  The support counting step can be improved signif\/{i}cantly if we can
  index the database in such a way that it allows fast frequency
  computations.

\medskip
The Eclat algorithm leverages the tidsets directly for support
computation. The basic idea is that the support of a candidate itemset
can be computed by intersecting the tidsets of suitably chosen subsets.
In general, given $\vT(X)$ and $\vT(Y)$ for any two frequent itemsets
$X$ and $Y$, we have $$\vT(\mathit{XY}) = \vT(X) \cap \vT(Y)$$ The
support of candidate $\mathit{XY}$ is simply the cardinality of
$\vT(\mathit{XY})$, that is, $\supp(\mathit{XY}) =
\card{\vT(\mathit{XY})}$.  


\medskip
Eclat intersects the tidsets only if the
frequent itemsets share a common pref\/{i}x, and it traverses the
pref\/{i}x search tree in a DFS-like manner, processing a group of
itemsets that have the same pref\/{i}x, also called a {\em pref\/{i}x
equivalence class}.

\end{frame}


\begin{frame}[fragile]{Eclat Algorithm}
\begin{tightalgo}[H]{\textwidth-18pt}
\SetKwInOut{AlgorithmA}{\algeclat ($P$, $\minsup$, $\cF$)}
\tcp{Initial Call: $\cF \assign \emptyset,
P\assign \bigl\{\tup{i,\vT(i)} \mid i \in \cI,
\card{\vT(i)} \ge \minsup\bigr\}$}
\AlgorithmA{}
\ForEach{$\tup{X_a, \vT(X_a)} \in P$}{
  $\cF \assign \cF \cup \bigl\{(X_{a}, \supp(X_{a}))\bigr\}$\;
  $P_a \assign \emptyset$\;
  \ForEach{$\tup{X_b,\vT(X_b)} \in P$, with $X_b > X_a$}{
  $X_{ab} = X_a \cup X_b$\;
    $\vT(X_{ab}) = \vT(X_a) \cap \vT(X_b)$\;
    \If{$\supp(X_{ab}) \geq \minsup$}{
      $P_a \assign P_a \cup \bigl\{ \tup{X_{ab}, \vT(X_{ab})} \bigr\}$\;
    }
  }
  \lIf{$P_a \ne \emptyset$}{
    \algeclat($P_a$, $\minsup$, $\cF$)
  }
}
\end{tightalgo}
\end{frame}


\begin{frame}{Eclat Algorithm: Tidlist Intersections}
  \framesubtitle{Infrequent itemsets in gray}
    \psset{unit=1in}
  \centerline{
  \scalebox{0.6}{%
  \psset{treesep=0.1,levelsep=*0.75,treefit=loose}
  \def\cedge{\ncarc[angle=30]}
  \pstree[]{\TR[]{
    \begin{tabular}{|c|}\hline
    $\emptyset$\\\hline
    %123456\\\hline
    \end{tabular}}
  }{
  %A
      \pstree[treefit=tight]{
          \TR[]{
          \begin{tabular}{|c|}\hline
          $A$\\\hline
          1345\\\hline
          \end{tabular}}
      }{
      \pstree[]{
      \TR{
      \begin{tabular}{|c|}\hline
        $\mathit{AB}$\\\hline
        1345\\\hline
      \end{tabular}}
      }{
        \pstree[]{
        \TR{
      \begin{tabular}{|c|}\hline
        $\mathit{ABD}$\\\hline
        135\\\hline
      \end{tabular}}
        }{
          \TR{
      \begin{tabular}{|c|}\hline
        $\mathit{ABDE}$\\\hline
        135\\\hline
      \end{tabular}}
        }
        \TR{
      \begin{tabular}{|c|}\hline
        $\mathit{ABE}$\\\hline
        1345\\\hline
      \end{tabular}}
      }
      \TR*[fillstyle=solid,
          fillcolor=lightgray]{
      \begin{tabular}{|c|}\hline
        $\mathit{AC}$\\\hline
        45\\\hline
      \end{tabular}}
      %AD
      \pstree[]{
      \TR{
          \begin{tabular}{|c|}\hline
          $\mathit{AD}$\\\hline
          135\\\hline
          \end{tabular}}
      }{
        \TR{
          \begin{tabular}{|c|}\hline
          $\mathit{ADE}$\\\hline
          135\\\hline
          \end{tabular}}
      }
      \TR{
          \begin{tabular}{|c|}\hline
          $\mathit{AE}$\\\hline
          1345\\\hline
          \end{tabular}}
    }
    %B
    \pstree[treefit=tight]{
    \TR{
          \begin{tabular}{|c|}\hline
          $B$\\\hline
          123456\\\hline
          \end{tabular}}
      }{
      \pstree[]{
      \TR{
          \begin{tabular}{|c|}\hline
          $\mathit{BC}$\\\hline
          2456\\\hline
          \end{tabular}}
      }{
        \TR*[fillstyle=solid,
          fillcolor=lightgray]{
          \begin{tabular}{|c|}\hline
          $\mathit{BCD}$\\\hline
          56\\\hline
          \end{tabular}}
        \TR{
          \begin{tabular}{|c|}\hline
          $\mathit{BCE}$\\\hline
          245\\\hline
          \end{tabular}}
      }
      \pstree[]{
      \TR{
          \begin{tabular}{|c|}\hline
          $\mathit{BD}$\\\hline
          1356\\\hline
          \end{tabular}}
      }{
        \TR{
          \begin{tabular}{|c|}\hline
          $\mathit{BDE}$\\\hline
          135\\\hline
          \end{tabular}}
      }
      \TR{
          \begin{tabular}{|c|}\hline
          $\mathit{BE}$\\\hline
          12345\\\hline
          \end{tabular}}
    }
    %C
    \pstree[]{
    \TR{ \begin{tabular}{|c|}\hline
          $C$\\\hline
          2456\\\hline
          \end{tabular}}
      }{
      \TR*[fillstyle=solid,
          fillcolor=lightgray]{
          \begin{tabular}{|c|}\hline
          $\mathit{CD}$\\\hline
          56\\\hline
          \end{tabular}}
      \TR{
          \begin{tabular}{|c|}\hline
          $\mathit{CE}$\\\hline
          245\\\hline
          \end{tabular}}
    }
    %D
    \pstree[xbbl=-4pt]{
    \TR[]{
          \begin{tabular}{|c|}\hline
          $D$\\\hline
          1356\\\hline
          \end{tabular}}
          }{
      \TR{
          \begin{tabular}{|c|}\hline
          $\mathit{DE}$\\\hline
          135\\\hline
          \end{tabular}}
    }
    %E
    \tspace{0.15in}
    \TR[edge=\cedge]{\begin{tabular}{|c|}\hline
          $E$\\\hline
          12345\\\hline
          \end{tabular}}}
    }
  }
\end{frame}




\begin{frame}{Diffsets: Difference of Tidsets}

The Eclat algorithm can be
signif\/{i}cantly improved if we can shrink the size of the
intermediate tidsets. This can be achieved by keeping track of the
differences in the tidsets as opposed to the full tidsets.

\medskip
Let $X_a = \{x_1, \ldots, x_{k-1}, x_a\}$
and $X_b = \{x_1, \ldots, x_{k-1}, x_b\}$, so that
$X_{ab} = X_a \cup X_b = \{x_1, \ldots, x_{k-1}, x_a,
x_b \}$. 

\medskip
The {\em diffset} of $X_{ab}$ is the set of tids
that contain the pref\/{i}x $X_a$, but not
the item $X_b$
\begin{align*}
  \bd(X_{ab}) &= \vT(X_{a}) \setminus \vT(X_{ab}) =
    \vT(X_{a}) \setminus \vT(X_{b})
\end{align*}

\medskip
We can obtain an expression for $\bd(X_{ab})$ in terms of
$\bd(X_a)$ and $\bd(X_b)$ as follows:
\begin{align*}
  \bd(X_{ab}) 
    & = \bd(X_b) \setminus \bd(X_a)
\end{align*}
which means that we can replace all
intersection operations in Eclat with diffset operations. 

\end{frame}


\begin{frame}{Diffsets: Difference of Tidsets}

  Given $\cT = 123456$:
  \begin{align*}
	  \bd(A) & = \cT \setminus 1345 = 26 & \bd(B) & = \cT \setminus 123456 = \emptyset\\
	  \bd(C) & = \cT \setminus 2456 = 13 & \bd(D) & = \cT \setminus 1356 = 24\\
	  \bd(E) & = \cT \setminus 12345 = 6 & & \\
  \end{align*}

  For instance, the diffsets of $\mathit{AB}$ and $\mathit{AC}$ are given~as
  \begin{align*}
  \bd(\mathit{AB}) & = \bd(B) \setminus \bd(A)
  = \emptyset \setminus \{2,6\} =   \emptyset\\
  \bd(\mathit{AC}) & = \bd(C) \setminus \bd(A) = \{1,3\} \setminus \{2,6\} = 13
  \end{align*}
  and their support values are
  \begin{align*}
    \supp(\mathit{AB}) &= \supp(A) - \card{\bd(\mathit{AB})} = 4-0 = 4\\
    \supp(\mathit{AC}) &= \supp(A) - \card{\bd(\mathit{AC})} = 4-2 = 2
  \end{align*}

  The new pref\/{i}x equivalence class $P_A$ is:
  $$P_A = \bigl\{ \tup{\mathit{AB},\emptyset,4}, \tup{\mathit{AD}, 4, 3},
  \tup{\mathit{AE}, \emptyset, 4}\bigr\}$$

\end{frame}


\begin{frame}[fragile]{Algorithm dEclat}
\begin{tightalgo}[H]{\textwidth-18pt}
\SetKwInOut{AlgorithmA}{\algdeclat ($P$, $\minsup$, $\cF$)}
\tcp{Initial Call: $\cF \assign \emptyset$,
$P \assign \bigl\{\tup{i,\bd(i),\supp(i)} \mid i \in \cI,
\bd(i) = \cT \setminus \vT(i), \supp(i) \ge \minsup\bigr\}$}
\AlgorithmA{}
\ForEach{$\tup{X_a, \bd(X_a), \supp(X_a)} \in P$}{
  $\cF \assign \cF \cup \bigl\{(X_{a}, \supp(X_{a}))\bigr\}$\;
  $P_a \assign \emptyset$\;
  \ForEach{$\tup{X_b,\bd(X_b), \supp(X_b)} \in P$, with $X_b > X_a$}{
  $X_{ab} = X_a \cup X_b$\;
  $\bd(X_{ab}) = \bd(X_b) \setminus \bd(X_a)$
  \nllabel{alg:fpm:iset:declat:diff}\;
    $\supp(X_{ab}) = \supp(X_a) - \card{\bd(X_{ab})}$
    \nllabel{alg:fpm:iset:declat:dsup}\;
    \If{$\supp(X_{ab}) \geq \minsup$}{
      $P_a \assign P_a \cup \bigl\{ \tup{X_{ab}, \bd(X_{ab}),
      \supp(X_{ab})} \bigr\}$\;
    }
  }
  \lIf{$P_a \ne \emptyset$}{
    \algdeclat($P_a$, $\minsup$, $\cF$)
  }
}
\end{tightalgo}
\end{frame}


\begin{frame}{dEclat Algorithm: Diffsets}
  \framesubtitle{support shown within brackets; infrequent itemsets in
  gray}
    \psset{unit=1in}
  \centerline{
  \scalebox{0.55}{%
  \psset{treesep=0.15,levelsep=*0.75,treefit=loose}
  \def\cedge{\ncarc[angle=30]}
  \pstree[]{\TR[]{
    \begin{tabular}{|c|}\hline
    $\emptyset$\\\hline
    \end{tabular}}
  }{
  %A
      \pstree[treefit=tight]{
          \TR[]{
          \begin{tabular}{|c|}\hline
          $A$\\(4)\\\hline
          26\\\hline
          \end{tabular}}
      }{
      \pstree[]{
      \TR{
      \begin{tabular}{|c|}\hline
        $\mathit{AB}$\\(4)\\\hline
        $\emptyset$\\\hline
      \end{tabular}}
      }{
        \pstree[]{
        \TR{
      \begin{tabular}{|c|}\hline
        $\mathit{ABD}$\\(3)\\\hline
        4\\\hline
      \end{tabular}}
        }{
          \TR{
      \begin{tabular}{|c|}\hline
        $\mathit{ABDE}$\\(3)\\\hline
        $\emptyset$\\\hline
      \end{tabular}}
        }
        \TR{
      \begin{tabular}{|c|}\hline
        $\mathit{ABE}$\\(4)\\\hline
        $\emptyset$\\\hline
      \end{tabular}}
      }
      \TR*[fillstyle=solid,
          fillcolor=lightgray]{
      \begin{tabular}{|c|}\hline
        $\mathit{AC}$\\(2)\\\hline
        13\\\hline
      \end{tabular}}
      %AD
      \pstree[]{
      \TR{
          \begin{tabular}{|c|}\hline
          $\mathit{AD}$\\(3)\\\hline
          4\\\hline
          \end{tabular}}
      }{
        \TR{
          \begin{tabular}{|c|}\hline
          $\mathit{ADE}$\\(3)\\\hline
          $\emptyset$\\\hline
          \end{tabular}}
      }
      \TR{
          \begin{tabular}{|c|}\hline
          $\mathit{AE}$\\(4)\\\hline
          $\emptyset$\\\hline
          \end{tabular}}
    }
    %B
    \pstree[treefit=tight]{
    \TR{
          \begin{tabular}{|c|}\hline
          $B$\\(6)\\\hline
          $\emptyset$\\\hline
          \end{tabular}}
      }{
      \pstree[]{
      \TR{
          \begin{tabular}{|c|}\hline
          $\mathit{BC}$\\(4)\\\hline
          13\\\hline
          \end{tabular}}
      }{
        \TR*[fillstyle=solid,
          fillcolor=lightgray]{
          \begin{tabular}{|c|}\hline
          $\mathit{BCD}$\\(2)\\\hline
          24\\\hline
          \end{tabular}}
        \TR{
          \begin{tabular}{|c|}\hline
          $\mathit{BCE}$\\(3)\\\hline
          6\\\hline
          \end{tabular}}
      }
      \pstree[]{
      \TR{
          \begin{tabular}{|c|}\hline
          $\mathit{BD}$\\(4)\\\hline
          24\\\hline
          \end{tabular}}
      }{
        \TR{
          \begin{tabular}{|c|}\hline
          $\mathit{BDE}$\\(3)\\\hline
          6\\\hline
          \end{tabular}}
      }
      \TR{
          \begin{tabular}{|c|}\hline
          $\mathit{BE}$\\(5)\\\hline
          6\\\hline
          \end{tabular}}
    }
    %C
    \pstree[]{
    \TR{ \begin{tabular}{|c|}\hline
          $C$\\(4)\\\hline
          13\\\hline
          \end{tabular}}
      }{
      \TR*[fillstyle=solid,
          fillcolor=lightgray]{
          \begin{tabular}{|c|}\hline
          $\mathit{CD}$\\(2)\\\hline
          24\\\hline
          \end{tabular}}
      \TR{
          \begin{tabular}{|c|}\hline
          $\mathit{CE}$\\(3)\\\hline
          6\\\hline
          \end{tabular}}
    }
    %D
    \pstree[xbbl=-4pt]{
    \TR[]{
          \begin{tabular}{|c|}\hline
          $D$\\(4)\\\hline
          24\\\hline
          \end{tabular}}
          }{
      \TR{
          \begin{tabular}{|c|}\hline
          $\mathit{DE}$\\(3)\\\hline
          6\\\hline
          \end{tabular}}
    }
    %E
    \tspace{0.15in}
    \TR[edge=\cedge]{\begin{tabular}{|c|}\hline
          $E$\\(5)\\\hline
          6\\\hline
          \end{tabular}}}
    }
  }
\end{frame}

\ifdefined\wox \begin{frame} \titlepage \end{frame} \fi


\begin{frame}{Frequent Pattern Tree Approach: FPGrowth Algorithm}
The FPGrowth method indexes the database for fast support
computation via the use of an augmented pref\/{i}x tree called the
{\em frequent pattern tree} (FP-tree).  

\medskip
Each
node in the tree is labeled with a single item, and each child node
represents a different item. Each node also stores the support
information for the itemset comprising the items on the path from the
root to that node. 

\medskip
The FP-tree is constructed as follows.
Initially the tree contains as root the null item
$\emptyset$. Next, for each tuple
$\tup{t, X} \in \bD$, where $X=\vI(t)$,
we insert the itemset $X$ into the FP-tree,
incrementing the count of all nodes
along the path that represents $X$.

\medskip
If $X$ shares a pref\/{i}x with some previously inserted
transaction, then $X$ will follow the same path until the common
pref\/{i}x. For the remaining items in $X$, new nodes are created
under the common pref\/{i}x, with counts initialized to $1$. The
FP-tree is complete when all transactions have been inserted.
\end{frame}



\begin{frame}{Frequent Pattern Tree}
  The FP-tree is a pref\/{i}x compressed representation of
  $\bD$. For most compression items are sorted in descending order of
  support.

  \psset{unit=0.75in,treefit=loose}
  \def\bedge{\ncline[linewidth=3pt]}
  %
  \centerline{
  \begin{minipage}[t][][c]{2cm}
  \begin{footnotesize}
  \begin{tabular}{|l|}
  \hline
  Transactions \\ \hline
  BEAD \\ \hline
  BEC \\ \hline
  BEAD \\ \hline
  BEAC \\ \hline
  BEACD \\ \hline
  BCD \\ \hline
  \end{tabular}
  \end{footnotesize}
  \end{minipage}
  \hspace{0.2in}
  \psset{treesep=0.1,levelsep=*0.5}
  %\subfloat[$\tup{1,\mathit{BEAD}}$]{
    \label{fig:fpm:iset:APT1}
    \scalebox{0.4}
  \hspace{0.2in}
  %\subfloat[$\tup{2,\mathit{BEC}}$]{
    \label{fig:fpm:iset:APT2}
    \scalebox{0.4}
  \hspace{0.2in}
  %\subfloat[$\tup{3,\mathit{BEAD}}$]{
    \label{fig:fpm:iset:APT3}
    \scalebox{0.4}
  \vspace{0.2in}
  \centerline{
  %\subfloat[$\tup{4,\mathit{BEAC}}$]{
    \label{fig:fpm:iset:APT4}
    \scalebox{0.4}
  \hspace{0.2in}
  \psset{treesep=0.1,levelsep=*0.4}
  %\subfloat[$\tup{5,\mathit{BEACD}}$]{
    \label{fig:fpm:iset:APT5}
    \scalebox{0.4}
  }
  \hspace{0.2in}
  %\subfloat[$\tup{6,\mathit{BCD}}$]{
    \label{fig:fpm:iset:APT6}
    \scalebox{0.4}
  }
\end{frame}


\begin{frame}{FPGrowth Algorithm: Rationale}
Given an FP-tree $R$, projected FP-trees are built for each
frequent item $i$ in $R$ in increasing order of support in a recursive
manner.

\medskip
To project $R$ on item $i$, we f\/{i}nd
all the occurrences of $i$ in the tree, and for each occurrence,
we determine the corresponding path from the root to $i$.
The count of item $i$ on a given
path is recorded in $cnt(i)$
and the
path is inserted into the new projected tree $R_X$, where $X$ is
the itemset obtained by extending the pref\/{i}x $P$ with the item
$i$.  While inserting the path, the count of each node in $R_X$ along
the given path is incremented by the path count $cnt(i)$.


\medskip
The base case
for the recursion happens when the input FP-tree $R$ is a single
path. FP-trees that are paths are handled by enumerating all
itemsets that are subsets of the path, with the support of each
such itemset being given by the least frequent item in it.

\end{frame}


\begin{frame}[fragile]{FPGrowth Algorithm}
\begin{footnotesize}
\begin{tightalgo}[H]{\textwidth-18pt}
\SetKwInOut{AlgorithmA}{\algfpgrowth ($R$, $P$, $\cF$, $\minsup$)}
\tcp{Initial Call: $R \assign \text{FP-tree}(\bD)$,
    $P \assign \emptyset$, $\cF \assign \emptyset$ }
\AlgorithmA{}
Remove infrequent items from $R$\;
\eIf(\tcp*[h]{insert subsets of $R$ into $\cF$}){\textsc{IsPath}($R$)}{
    \nllabel{alg:fpm:iset:fp:pathS}
    \ForEach{$Y \subseteq R$}{
        $X \assign P \cup Y$\;
        $\supp(X) \assign \min_{x \in Y} \{cnt(x)\}$ \;
        $\cF \assign \cF \cup \bigl\{(X, \supp(X))\bigr\}$
        \nllabel{alg:fpm:iset:fp:pathE}\;
    }
}($\quad$ \tcp*[h]{process projected FP-trees for each frequent item $i$})
{
    \ForEach{$i \in R$ in increasing order of $\supp(i)$}{
        $X \assign P \cup \{i\}$\nllabel{alg:fpm:iset:fp:Pi} \;
        $\supp(X) \assign \supp(i)$ \tcp{sum of $cnt(i)$ for all nodes
        labeled $i$}\;
        $\cF \assign \cF \cup \bigl\{(X, \supp(X))\bigr\}$\;
        $R_X \assign \emptyset$ \tcp{projected FP-tree for $X$}\;
        \ForEach{$path \in \textsc{PathFromRoot}(i)$ \nllabel{alg:fpm:iset:fp:ptr}}{
            $cnt(i) \assign $ count of $i$ in $path$\;
            \nllabel{alg:fpm:iset:fp:pathsup}
            Insert $path$, excluding $i$,
            into FP-tree $R_X$ with count $cnt(i)$
            \nllabel{alg:fpm:iset:fp:atr}\;
        }
        \lIf{$R_X \neq \emptyset$}{
            \algfpgrowth ($R_X$, $X$, $\cF$, $\minsup$)
            \nllabel{alg:fpm:iset:fp:recurse}
        \nllabel{alg:fpm:iset:fp:condE}
        }
    }
}
\end{tightalgo}
\end{footnotesize}
\end{frame}


\begin{frame}{Projected Frequent Pattern Tree for {\it D}}
  \psset{unit=0.75in,treefit=loose}
  \def\bedge{\ncline[linewidth=3pt]}
  \begin{tabular}{TTTT}
  FP-Tree & Add $\mathit{BC},cnt=1$  &
  Add $\mathit{BEAC},cnt=1$ & Add $\mathit{BEA},cnt=2$\\
  \begin{minipage}{1in}
  \scalebox{0.5}{%
    \pstree[]{\Toval[edge=\bedge]{$\emptyset(6)$}}{
      \pstree[]{\Toval[edge=\bedge]{$B$(6)}}{
      \pstree[]{\Toval[edge=\bedge]{$C$(1)}}{
        \Toval[edge=\bedge]{$D$(1)}
       }
        \pstree[]{\Toval{$E$(5)}}{
          \pstree[]{\Toval{$A$(4)}}{
            \pstree{\Toval{$C$(2)}}{
              \Toval{$D$(1)}
            }
            \Toval{$D$(2)}
          }
          \Toval{$C$(1)}
       }
      }
    }
   }
 \end{minipage}
   &
   \begin{minipage}{0.5in}
    \scalebox{0.6}{%
    \psset{treesep=0.1,levelsep=*0.45}
    \pstree[xbbl=0.5in,xbbr=0.5in]{\Toval{$\emptyset(1)$}}{
      \pstree[]{\Toval{$B$(1)}}{
        \Toval{$C$(1)}
       }
      }
    }
  \end{minipage}
	&
  \begin{minipage}{0.4in}
    \scalebox{0.6}{%
    \psset{treesep=0.1,levelsep=*0.45}
    \pstree[xbbl=0.3in,xbbr=0.3in]{\Toval{$\emptyset(2)$}}{
      \pstree[]{\Toval{$B$(2)}}{
        \Toval{$C$(1)}
        \pstree[]{\Toval{$E$(1)}}{
          \pstree[]{\Toval{$A$(1)}}{
            \Toval{$C$(1)}
          }
       }
      }
    }
   }
 \end{minipage}
   & 
  \begin{minipage}{0.4in}
   \scalebox{0.6}{%
    \psset{treesep=0.1,levelsep=*0.45}
    \pstree[xbbl=0.3in,xbbr=0.3in]{\Toval{$\emptyset(4)$}}{
      \pstree[]{\Toval{$B$(4)}}{
        \Toval{$C$(1)}
        \pstree[]{\Toval{$E$(3)}}{
          \pstree[]{\Toval{$A$(3)}}{
            \Toval{$C$(1)}
          }
       }
      }
    }
   }
 \end{minipage} \\
\end{tabular}
\end{frame}


\begin{frame}{FPGrowth Algorithm: Frequent Pattern Tree Projection}
  \captionsetup[subfloat]{captionskip=0.1in}
  \psset{unit=0.75in,treefit=loose}
  \def\bedge{\ncline[linewidth=3pt]}
  \def\nedge{\ncdiag[angleA=-90,armA=0,armB=0.25,angleB=90,
        linewidth=2pt]}
  %
  \centerline{
  \psset{treesep=0.1,levelsep=*0.5}
    \scalebox{0.38}{%
    \psmatrix[colsep=0.5in,rowsep=1in]
    & \Rnode[vref=-1.5in]{R}{
      \TR[showbbox=true,xbbl=0.1in,xbbr=0.15in,xbbh=0.1in,xbbd=0.25in]{%
        \begin{tabular}{c}
        \pstree[]{\Toval{$\emptyset(6)$}}{
          \pstree[]{\Toval{$B$(6)}}{
            \pstree[]{\Toval{$C$(1)}}{
              \Toval{$D$(1)}
            }
            \pstree[]{\Toval{$E$(5)}}{
              \pstree[]{\Toval{$A$(4)}}{
                \pstree{\Toval{$C$(2)}}{
                  \Toval{$D$(1)}
                }
                \Toval{$D$(2)}
              }
              \Toval{$C$(1)}
            }
          }
        }\\
       \end{tabular}
      }
    }\\
    \Rnode{RD}{
      \TR[showbbox=true,xbbl=0.1in,xbbr=0.15in,xbbh=0.1in,xbbd=0.2in]{%
        \pstree{\Toval{$\emptyset(4)$}}{
          \pstree[]{\Toval{$B$(4)}}{
            \Toval{$C$(1)}
            \pstree[]{\Toval{$E$(3)}}{
              \pstree[]{\Toval{$A$(3)}}{
                \Toval{$C$(1)}
              }
            }
          }
        }
      }
    } &
    \Rnode{RC}{
      \TR[showbbox=true,xbbl=0.1in,xbbr=0.1in,xbbh=0.1in,xbbd=0.2in]{%
        \pstree{\Toval{$\emptyset(4)$}}{
          \pstree[]{\Toval{$B$(4)}}{
            \pstree[]{\Toval{$E$(3)}}{
              \Toval{$A$(2)}
            }
          }
        }
      }
    } &
    \Rnode{RA}{
      \TR[showbbox=true,xbbl=0.1in,xbbr=0.1in,xbbh=0.1in,xbbd=0.2in]{%
        \pstree{\Toval{$\emptyset(4)$}}{
          \pstree[]{\Toval{$B$(4)}}{
            \Toval{$E$(4)}
          }
        }
      }
    } &
    \Rnode{RE}{
      \TR[showbbox=true,xbbl=0.1in,xbbr=0.1in,xbbh=0.1in,xbbd=0.2in]{%
        \pstree{\Toval{$\emptyset(5)$}}{
          \Toval{$B$(5)}
        }
      }
    }
    \endpsmatrix
    %\psset{npos=0.75}
    \ncline{R}{RD}\ncput*{$R_D$}
    \ncline{R}{RC}\ncput*{$R_C$}
    \ncline{R}{RA}\ncput*{$R_A$}
    \ncline{R}{RE}\ncput*{$R_E$}
    } }
\end{frame}

\ifdefined\wox \begin{frame} \titlepage \end{frame} \fi

\begin{frame}{Association Rules}
An {\it association rule}
is an expression 
$$X \arule{s,c} Y$$
where $X$ and $Y$ are itemsets and they are disjoint, that is,
$X,Y \subseteq \cI$, and $X \cap Y = \emptyset$.
Let the itemset $X \cup Y$ be denoted as $\mathit{XY}$.

\smallskip
The
{\em support}
of the rule is the number of transactions in which
both $X$ and $Y$ co-occur as subsets:
\begin{align*}
  s = \supp(X \arule Y) = \card{\vT(\mathit{XY})} = \supp(\mathit{XY})
\end{align*}

\smallskip
The {\em relative support} of the rule is def\/{i}ned as the fraction of transactions
where $X$ and $Y$ co-occur, and it provides an estimate of the joint probability of $X$ and $Y$:
\begin{align*}
  \rsupp(X \arule Y) =  \frac{\supp(\mathit{XY})}{\card{\bD}} =
  P(X \wedge Y)
\end{align*}

\smallskip
The {\em conf\/{i}dence}
of a rule is the conditional probability
that a transaction contains $Y$ given that it contains $X$:
\begin{align*}
c = \conff(X \arule Y) = P(Y|X) =
\frac{P(X \wedge Y)}{P(X)} =
 \frac{\supp (\mathit{XY})}{\supp (X)}
\end{align*}

\end{frame}



\begin{frame}{Generating Association Rules}
Given a frequent itemset $Z \in \cF$,
we look at all proper subsets
$X\subset Z$ to compute rules of the form
\begin{align*}
  X \arule{s,c} Y, \text{ where } Y = Z\setminus X
\end{align*}
where $Z \setminus X  = Z - X$.

\medskip
The rule must be frequent
because
\begin{align*}
s = \supp(\mathit{XY}) = \supp(Z) \geq \minsup
\end{align*}

\medskip
We compute the conf\/{i}dence as follows:
\begin{align*}
\tcbhighmath{
c = \frac{\supp(X \cup Y)}{\supp(X)} = \frac{\supp(Z)}{\supp(X)}
}
\end{align*}
If $c \geq \minconf$, then the rule is a strong rule. On the other
hand, if $\conff(X\arule Y) < c$, then
$\conff(W \arule Z \setminus W) < c$ for all subsets $W \subset X$,
as $\supp(W) \ge \supp(X)$. We can thus
avoid checking subsets of $X$.
\end{frame}


\begin{frame}[fragile]{Association Rule Mining Algorithm}
\begin{tightalgo}[H]{\textwidth-18pt}
\SetKwInOut{AlgoRuleMining}{\algrulemining ($\cF$, $\minconf$)}
\AlgoRuleMining{}
\ForEach{$Z \in \cF$, such that $\card{Z} \ge 2$}{
  $\cA \gets \bigl\{X \mid X \subset Z, X \ne \emptyset \bigr\}$
  \nllabel{alg:fpm:iset:arule:initA}\;
  \While{$\cA \ne \emptyset$}{
    $X \gets \text{maximal element in } \cA$\;
    $\cA \gets \cA \setminus X$\tcp{remove $X$ from $\cA$}
    $c \gets \supp(Z)/\supp(X)$ \nllabel{alg:fpm:iset:arule:conf}\;
    \eIf{$c \ge \minconf$}{
      print $X \arule Y$, $\supp(Z)$, $c$\;
    }{
      $\cA \gets \cA \setminus
      \bigl\{W \mid W \subset X\bigr\}$ \;
      \nllabel{alg:fpm:iset:arule:removeW}
      \tcp{remove all subsets of $X$ from $\cA$}
    }
  }
}
\end{tightalgo}
\end{frame}
