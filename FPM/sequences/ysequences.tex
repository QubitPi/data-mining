\lecture{sequences}{sequences}

\date{Chapter 10: Sequence Mining}

\begin{frame}
\titlepage
\end{frame}

\newcommand{\alggsp}{\textsc{GSP}\xspace}
\newcommand{\algspade}{\textsc{Spade}\xspace}
\newcommand{\algSeqcomputesupport}{\textsc{ComputeSupport}\xspace}
\newcommand{\algSeqextprefix}{\textsc{ExtendPrefixTree}\xspace}
\newcommand{\algprefixspan}{\textsc{PrefixSpan}\xspace}
\newcommand{\algUkkonenSimple}{\textsc{NaiveUkkonen}\xspace}
\newcommand{\algUkkonen}{\textsc{Ukkonen}\xspace}


\begin{frame}{Sequence Mining: Terminology}
Let $\Sigma$ be the {\em alphabet}, a set of symbols.
A {\em sequence} or a {\em string} is def\/{i}ned
as an ordered list of symbols, and is written as $\bs =  s_{1}s_{2}
\dots s_{k}$, where $s_i \in \Sigma$ is a symbol at position $i$, also
denoted as $\bs[i]$. 
$|\bs| = k$ denotes the {\em length} of the
sequence.  

\medskip
The notation $\bs[i:j] = s_is_{i+1}\cdots
s_{j-1}s_{j}$ denotes the {\em substring} or sequence of consecutive
symbols in positions $i$ through $j$, where $j > i$.  

\medskip
Def\/{i}ne the {\em
pref\/{i}x} of a sequence $\bs$ as any substring of the form
$\bs[1:i]=s_1s_2\ldots s_i$, with $0 \le i \le n$.


\medskip
Def\/{i}ne the {\em suff\/{i}x}
of $\bs$ as any substring of the form $\bs[i:n] = s_is_{i+1}\ldots
s_{n}$, with $1 \le i \le n+1$. 

\medskip
$\bs[1:0]$ is the empty
pref\/{i}x, and $\bs[n+1:n]$ is the empty suff\/{i}x. Let $\Sigma^\star$ be the set of all possible sequences
that can be constructed using the symbols in $\Sigma$, including the
empty sequence $\emptyset$ (which has length zero).

\end{frame}



\begin{frame}{Sequence Mining: Terminology}

  Let $\bs = s_1s_2 \ldots s_{n}$ and $\br = r_1r_2 \ldots r_{m}$ be two
sequences over $\Sigma$.  We say that $\br$ is a {\em subsequence} of
$\bs$ denoted $\br \subseteq \bs$, if there exists a one-to-one mapping
$\phi: [1,m] \to [1,n]$, such that $\br[i] = \bs[\phi(i)]$ and for any
two positions $i,j$ in $\br$, $i < j \implies \phi(i) < \phi(j)$. In
If $\br \subseteq \bs$, we also say that $\bs$ {\em contains} $\br$.

\medskip
The sequence $\br$ is called a {\em consecutive subsequence} or
substring of $\bs$ provided $r_1r_2\ldots r_{m} = s_{j}s_{j+1}\ldots
s_{j+m-1}$, i.e., $\br[1:m] =\bs[j:j+m-1]$, with $1\le j \le n-m+1$.

\begin{block}

Let $\Sigma = \{A,C,G,T\}$, and let $\bs = \mathit{ACTGAACG}$.  

Then
$\br_1=\mathit{CGAAG}$ is a subsequence of $\bs$, and $\br_2=\mathit{CTGA}$ is a substring
of $\bs$.  The sequence $\br_3=\mathit{ACT}$ is a pref\/{i}x of $\bs$, and so is
$\br_4=\mathit{ACTGA}$, whereas $\br_5=\mathit{GAACG}$ is one of the suff\/{i}xes of $\bs$.
\end{block}%\pagebreak
\end{frame}
%

\begin{frame}{Frequent Sequences}
Given a database $\bD = \{\bs_1,\bs_2, \ldots, \bs_N\}$ of $N$
sequences, and given some sequence $\br$, the {\em support} of $\br$ in
the database $\bD$ is def\/{i}ned as the total number of sequences in $\bD$
that contain $\br$
\begin{align*}
\tcbhighmath{
  \supp(\br) = \bcard{\Big}{\bigl\{\bs_i \in \bD | \br \subseteq
  \bs_i\bigr\}}
}
\end{align*}

\medskip
The {\em relative support} of $\br$ is the fraction of sequences that
contain $\br$
\begin{align*}
  \rsupp(\br) = \supp(\br)/N
\end{align*}

\medskip
Given a user-specif\/{i}ed $\minsup$ threshold, we say that a sequence $\br$
is {\em frequent} in database $\bD$ if $\supp(\br) \ge \minsup$.  A
frequent sequence is {\em maximal} if it is not a subsequence of any
other frequent sequence, and a frequent sequence is {\em closed} if it
is not a subsequence of any other frequent sequence with the same
support.

\end{frame}


\begin{frame}{Mining Frequent Sequences}

  For sequence mining the order of the symbols matters, and thus we have
to consider all possible {\em permutations} of the symbols as the
possible frequent candidates. Contrast this with itemset mining, where
we had only to consider {\em combinations} of the items.  

The sequence
search space can be organized in a pref\/{i}x search tree.  The root of the
tree, at level 0, contains the empty sequence, with each symbol $x \in
\Sigma$ as one of its children. As such, a node labeled with the
sequence $\bs=s_1s_2\ldots s_k$ at level $k$ has children of the form
$\bs' = s_1s_2\ldots s_ks_{k+1}$ at level $k+1$. In other words, $\bs$
is a pref\/{i}x of each child $\bs'$, which is also called an {\em
extension} of $\bs$.

\end{frame}


\begin{frame}{Example Sequence Database}
\begin{center}
\begin{tabular}[h]{|l|l|}
    \hline
    Id & Sequence\\ \hline
    \hline
    $\bs_1$ & $\mathit{CAGAAGT}$\\
    \hline
    $\bs_2$ & $\mathit{TGACAG}$\\
    \hline
    $\bs_3$ & $\mathit{GAAGT}$\\
    \hline
  \end{tabular}
\end{center}

\begin{block}
  
Using $\minsup=3$, the set of frequent subsequences is given as:
\begin{align*}
  \Fk{1} & = A(3), G(3), T(3)\\
  \Fk{2} & = AA(3), AG(3), GA(3), GG(3)\\
  \Fk{3} & = AAG(3), GAA(3), GAG(3)\\
  \Fk{4} & = GAAG(3)
\end{align*}

\end{block}

\end{frame}


\ifdefined\wox \begin{frame} \titlepage \end{frame} \fi


\begin{frame}{Level-wise Sequence Mining: GSP Algorithm}
\small
  The GSP algorithm searches the sequence pref\/{i}x tree using a
  level-wise or breadth-f\/{i}rst search.  Given the set of frequent
  sequences at level $k$, we generate all possible sequence extensions
  or {\em candidates} at level $k+1$.  We next compute the support of
  each candidate and prune those that are not frequent. The search stops
  when no more frequent extensions are possible.

\medskip
  The pref\/{i}x search tree at level $k$ is
denoted $\Ck{k}$. Initially $\Ck{1}$ comprises all the symbols in
$\Sigma$. Given the current set of candidate $k$-sequences $\Ck{k}$, the
method f\/{i}rst computes their support.

\medskip
For each database sequence $\bs_i
\in \bD$, we check whether a  candidate sequence $\br \in \Ck{k}$
is a subsequence of $\bs_i$. If so, we increment the support of $\br$.
Once the frequent sequences at level $k$ have been found, we generate
the candidates for level $k+1$.

\medskip
For the extension, each leaf $\br_a$ is extended with the last symbol of
any other leaf $\br_b$ that shares the same pref\/{i}x (i.e., has the same
parent), to obtain the new candidate $(k+1)$-sequence
$\br_{ab} = \br_a + \br_b[k]$.
If the new candidate $\br_{ab}$ contains any infrequent $k$-sequence, we
prune it.
\end{frame}


\begin{frame}[fragile]{Algorithm \alggsp}
\begin{tightalgo}[H]{\textwidth-18pt}
\SetKwInput{AlgorithmA}{\alggsp ($\bD$, $\Sigma$, $\minsup$)}
\AlgorithmA{}
$\cF \assign \emptyset$\;
$\Ck{1} \assign \{\emptyset\}$ \tcp{Initial prefix tree with single
symbols}
\lForEach{$s \in \Sigma$}{
Add $s$ as child of $\emptyset$ in $\Ck{1}$ with $sup(s) \assign 0$
}
$k \assign 1$ \tcp{$k$ denotes the level}
\While{$\Ck{k} \not= \emptyset$}{
  \algSeqcomputesupport ($\Ck{k}, \bD$) \nllabel{alg:fpm:seq:gsp:csup}\;
  \ForEach{ leaf $\bs \in \Ck{k}$}{
    \lIf{$\supp(\br) \ge \minsup$}{
      $\cF \assign \cF \cup \bigl\{(\br, \supp(\br))\bigr\}$
    }
    \lElse{
    remove $\bs$ from $\Ck{k}$
    }
  }
 $\Ck{k+1} \assign$ \algSeqextprefix ($\Ck{k}$)
 \nllabel{alg:fpm:seq:gsp:ext} \;
  $k \assign k+1$
}
\Return{$\Fk{k}$}
\end{tightalgo}
\end{frame}


\begin{frame}[fragile]{Algorithm \algSeqcomputesupport}
\begin{tightalgo}[H]{\textwidth-18pt}
\SetKwInput{AlgorithmB}{\algSeqcomputesupport ($\Ck{k}, \bD$)}
\AlgorithmB{}
\ForEach{$\bs_i \in \bD$}{
\ForEach{$\br \in \Ck{k}$}{
    \lIf{$\br \subseteq \bs_i$}{
      $\supp(\br) \assign \supp(\br)+1$
    }
  }
}
\end{tightalgo}

\begin{tightalgo}[H]{\textwidth-18pt}
\SetKwInput{AlgorithmC}{\algSeqextprefix ($\Ck{k}$)}
\AlgorithmC{}
\ForEach{$ \text{leaf } \br_a \in \Ck{k}$}{
\ForEach{$\text{leaf } \br_b \in
\textsc{Children}(\textsc{Parent}(\br_a))$}{
$\br_{ab} \assign \br_a + \br_b[k]$ \tcp{ extend $\br_a$ with last item of
$\br_b$} \nllabel{alg:fpm:seq:gsp:cand}
    \tcp{prune if there are any infrequent subsequences}
    \If{$\br_c \in \Ck{k}, \textbf{ for all }
    \br_c \subset \br_{ab}, \text{ such that }
    \card{\br_c} = \card{\br_{ab}}-1$}{
     Add $\br_{ab}$ as child of $\br_a$ with
     $sup(\br_{ab}) \assign 0$}
   }
   \If{no extensions from $\br_a$}{remove $\br_a$,
   and all ancestors of $\br_a$ with no extensions, from $\Ck{k}$}
}
\Return $\Ck{k}$\;
\end{tightalgo}
\end{frame}



\begin{frame}{Sequence Search Space}
  \framesubtitle{shaded ovals are infrequent sequences}
\begin{center}
  \psset{unit=0.75in}
  \centerline{
  \scalebox{0.5}{%
  \psset{treesep=0.05in,levelsep=*1in,treefit=tight}
  \pstree[]{\Toval{$\emptyset(3)$}}{
    %
    \pstree[]{\Toval{$A$(3)}}{
      \pstree[]{\Toval{$AA$(3)}}{
        \Toval[fillcolor=lightgray,fillstyle=solid]{$AAA(1)$}
        \pstree[]{\Toval{$AAG(3)$}}{
          \Toval[fillcolor=lightgray,fillstyle=solid]{$AAGG$}
        }
      }
      \pstree[]{\Toval{$AG$(3)}}{
        \Toval[fillcolor=lightgray,fillstyle=solid]{$AGA(1)$}
        \Toval[fillcolor=lightgray,fillstyle=solid]{$AGG(1)$}
      }
      \Toval[fillcolor=lightgray,fillstyle=solid]{$AT(2)$}
    }
    %
    \tspace{1.5in}
    \Toval[fillcolor=lightgray,fillstyle=solid]{$C(2)$}
    %
    \pstree[]{\Toval{$G$(3)}}{
      \pstree[]{\Toval{$GA$(3)}}{
        \pstree[]{\Toval{$GAA$(3)}}{
          \Toval[fillcolor=lightgray,fillstyle=solid]{$GAAA$}
          \Toval{$GAAG(3)$}
        }
        \pstree[]{\Toval{$GAG$(3)}}{
          \Toval[fillcolor=lightgray,fillstyle=solid]{$GAGA$}
          \Toval[fillcolor=lightgray,fillstyle=solid]{$GAGG$}
        }
      }
      \pstree[]{\Toval{$GG$(3)}}{
        \Toval[fillcolor=lightgray,fillstyle=solid]{$GGA(0)$}
        \Toval[fillcolor=lightgray,fillstyle=solid]{$GGG(0)$}
      }
      \Toval[fillcolor=lightgray,fillstyle=solid]{$GT(2)$}
    }
    %
    \pstree[]{\Toval{$T$(3)}}{
      \Toval[fillcolor=lightgray,fillstyle=solid]{$TA(1)$}
      \Toval[fillcolor=lightgray,fillstyle=solid]{$TG(1)$}
      \Toval[fillcolor=lightgray,fillstyle=solid]{$TT(0)$}
    }
  }
  }}
\end{center}
\end{frame}

\ifdefined\wox \begin{frame} \titlepage \end{frame} \fi

\begin{frame}{Vertical Sequence Mining: Spade}
The Spade algorithm uses a vertical database representation for sequence
mining. For each symbol $s \in \Sigma$, we keep a
set of tuples of the form $\tup{i, pos(s)}$, where
$pos(s)$ is the set of positions in the database sequence
$\bs_i \in \bD$ where symbol $s$
appears. 

\medskip
Let $\cL(s)$ denote the set of such sequence-position tuples
for symbol $s$,
which we refer to as the {\em poslist}.
The set of poslists for each symbol $s \in \Sigma$ thus constitutes a
vertical representation of the input database.

\medskip
Given $k$-sequence
$\br$, its poslist $\cL(\br)$
maintains the list of positions for the occurrences
of the last symbol $\br[k]$ in each database sequence $\bs_i$, provided
$\br \subseteq \bs_i$. The support of sequence $\br$ is simply the
number of distinct sequences in which $\br$ occurs, that is,
$\supp(\br) =
\card{\cL(\br)}$.

\end{frame}

\begin{frame}{Spade Algorithm}
Support computation in Spade is done via {\em sequential join}
operations.  

\medskip
Given the poslists for any two
$k$-sequences $\br_a$ and $\br_b$ that share the same $(k-1)$ length
pref\/{i}x, a sequential join on the poslists is used to
compute the support for the new $(k+1)$ length candidate sequence
$\br_{ab} = \br_a + \br_b[k]$.  

\medskip 
Given a tuple $\ltup{\big}{i,pos\bigl(\br_b[k]\bigr)} \in
\cL(\br_b)$, we f\/{i}rst check if there exists a tuple
$\ltup{\big}{i,pos\bigl(\br_a[k]\bigr)} \in \cL(\br_a)$, that is, both
sequences must occur in the same database sequence $\bs_i$. 

\smallskip
Next, for
each position $p \in pos\bigl(\br_b[k]\bigr)$, we check whether there
exists a position $q \in pos\bigl(\br_a[k]\bigr)$ such that $q < p$. If
yes, this means that the symbol $\br_b[k]$ occurs after the last
position of $\br_a$ and thus we retain $p$ as a valid occurrence of
$\br_{ab}$. The poslist $\cL(\br_{ab})$ comprises all such valid
occurrences.  

\medskip
We keep track of positions only for the last
symbol in the candidate sequence since we extend sequences
from a common pref\/{i}x, and so there is no need to keep track of all the
occurrences of the symbols in the pref\/{i}x.  

\medskip
We denote the sequential
join as $\cL(\br_{ab}) = \cL(\br_a) \cap \cL(\br_b)$.

\end{frame}


\begin{frame}[fragile]{Spade Algorithm}
\begin{tightalgo}[H]{\textwidth-18pt}
\SetKwInOut{AlgorithmT}{\algspade ($P$, $\minsup$, $\cF$, $k$)}
\tcp{Initial Call: $\cF \assign \emptyset$, $k \assign 0$,
$P \assign \bigl\{\tup{s,\cL(s)} \mid
  s \in \Sigma, \supp(s) \ge \minsup\bigr\}$}
\AlgorithmT{}
\ForEach{$\br_a \in P$}{
    $\cF \assign \cF \cup \bigl\{(\br_{a}, \supp(\br_{a}))\bigr\}$\;
    $P_a \assign \emptyset$\;
    \ForEach{$\br_b \in P$}{
      $\br_{ab} = \br_a + \br_b[k]$\;
      $\cL(\br_{ab}) = \cL(\br_a) \cap \cL(\br_b)$\;
      \If{$\supp(\br_{ab}) \geq \minsup$}{
          $P_a \assign P_a \cup \bigl\{ \tup{\br_{ab}, \cL(\br_{ab})} \bigr\}$\;
        }
    }
    \lIf{$P_a \ne \emptyset$}{
      \algspade($P_a$, $\minsup$, $\cF$, $k+1$)
    }
}
\end{tightalgo}
\end{frame}



\begin{frame}{Sequence Mining via Spade}
  \large
  \centering
 \def\nA{\Tr{
    \begin{tabular}{|c|c|}
        \hline
        \multicolumn{2}{|c|}{$A$}\\
        \hline
        1 & 2,4,5\\
        2 & 3,5\\
        3 & 2,3\\
        \hline
    \end{tabular}}
  }
 \def\nC{\Tr{
    \begin{tabular}{|c|c|}
        \hline
        \multicolumn{2}{|c|}{$C$}\\
        \hline
        1 & 1\\
        2 & 4\\
        \hline
    \end{tabular}}
  }
 \def\nG{\Tr{
    \begin{tabular}{|c|c|}
        \hline
        \multicolumn{2}{|c|}{$G$}\\
        \hline
        1 & 3,6\\
        2 & 2,6\\
        3 & 1,4\\
        \hline
    \end{tabular}}
  }
 \def\nT{\Tr{
    \begin{tabular}{|c|c|}
        \hline
        \multicolumn{2}{|c|}{$T$}\\
        \hline
        1 & 7\\
        2 & 1\\
        3 & 5\\
        \hline
    \end{tabular}}
  }
 \def\nAA{\Tr{
    \begin{tabular}{|c|c|}
        \hline
        \multicolumn{2}{|c|}{$\mathit{AA}$}\\
        \hline
        1 & 4,5\\
        2 & 5\\
        3 & 3\\
        \hline
    \end{tabular}}
  }
 \def\nAG{\Tr{
    \begin{tabular}{|c|c|}
        \hline
        \multicolumn{2}{|c|}{$\mathit{AG}$}\\
        \hline
        1 & 3,6\\
        2 & 6\\
        3 & 4\\
        \hline
    \end{tabular}}
  }
 \def\nAT{\TR*[fillcolor=lightgray,fillstyle=solid]{
    \begin{tabular}{|c|c|}
        \hline
        \multicolumn{2}{|c|}{$\mathit{AT}$}\\
        \hline
        1 & 7\\
        3 & 5\\
        \hline
    \end{tabular}}
  }
 \def\nGA{\Tr{
    \begin{tabular}{|c|c|}
        \hline
        \multicolumn{2}{|c|}{$\mathit{GA}$}\\
        \hline
        1 & 4,5\\
        2 & 3,5\\
        3 & 2,3\\
        \hline
    \end{tabular}}
  }
 \def\nGG{\Tr{
    \begin{tabular}{|c|c|}
        \hline
        \multicolumn{2}{|c|}{$\mathit{GG}$}\\
        \hline
        1 & 6\\
        2 & 6\\
        3 & 4\\
        \hline
    \end{tabular}}
  }
 \def\nGT{\TR*[fillcolor=lightgray,fillstyle=solid]{
    \begin{tabular}{|c|c|}
        \hline
        \multicolumn{2}{|c|}{$\mathit{GT}$}\\
        \hline
        1 & 7\\
        3 & 5\\
        \hline
    \end{tabular}}
  }
 \def\nTA{\TR*[fillcolor=lightgray,fillstyle=solid]{
    \begin{tabular}{|c|c|}
        \hline
        \multicolumn{2}{|c|}{$\mathit{TA}$}\\
        \hline
        2 & 3,5\\
        \hline
    \end{tabular}}
  }
 \def\nTG{\TR*[fillcolor=lightgray,fillstyle=solid]{
    \begin{tabular}{|c|c|}
        \hline
        \multicolumn{2}{|c|}{$\mathit{TG}$}\\
        \hline
        2 & 2,6\\
        \hline
    \end{tabular}}
  }
 \def\nAAA{\TR*[fillcolor=lightgray,fillstyle=solid]{
    \begin{tabular}{|c|c|}
        \hline
        \multicolumn{2}{|c|}{$\mathit{AAA}$}\\
        \hline
        1 & 5\\
        \hline
    \end{tabular}}
  }
 \def\nAAG{\Tr{
    \begin{tabular}{|c|c|}
        \hline
        \multicolumn{2}{|c|}{$\mathit{AAG}$}\\
        \hline
        1 & 6\\
        2 & 6\\
        3 & 4\\
        \hline
    \end{tabular}}
  }
 \def\nAGA{\TR*[fillcolor=lightgray,fillstyle=solid]{
    \begin{tabular}{|c|c|}
        \hline
        \multicolumn{2}{|c|}{$\mathit{AGA}$}\\
        \hline
        1 & 5\\
        \hline
    \end{tabular}}
  }
 \def\nAGG{\TR*[fillcolor=lightgray,fillstyle=solid]{
    \begin{tabular}{|c|c|}
        \hline
        \multicolumn{2}{|c|}{$\mathit{AGG}$}\\
        \hline
        1 & 6\\
        \hline
    \end{tabular}}
  }
 \def\nGAA{\Tr{
    \begin{tabular}{|c|c|}
        \hline
        \multicolumn{2}{|c|}{$\mathit{GAA}$}\\
        \hline
        1 & 5\\
        2 & 5\\
        3 & 3\\
        \hline
    \end{tabular}}
  }
 \def\nGAG{\Tr{
    \begin{tabular}{|c|c|}
        \hline
        \multicolumn{2}{|c|}{$\mathit{GAG}$}\\
        \hline
        1 & 6\\
        2 & 6\\
        3 & 4\\
        \hline
    \end{tabular}}
  }
 \def\nGAAG{\Tr{
    \begin{tabular}{|c|c|}
        \hline
        \multicolumn{2}{|c|}{$\mathit{GAAG}$}\\
        \hline
        1 & 6\\
        2 & 6\\
        3 & 4\\
        \hline
    \end{tabular}}
  }
  \centerline{
  \scalebox{0.5}{%
  \psset{unit=0.75in}
  \psmatrix[colsep=0.5,rowsep=1]
  &&&[name=empty]$\emptyset$\psspan{2}\\\empty
  &&[name=A]\nA & [name=C]\nC & [name=G]\nG & [name=T]\nT\\\empty
  [name=AA]\nAA & [name=AG]\nAG &
  [name=AT]\nAT  &  [name=GA]\nGA
  & [name=GG]\nGG & [name=GT]\nGT& [name=TA]\nTA  &
  [name=TG]\nTG\\\empty
   [name=AAA]\nAAA & [name=AAG]\nAAG & [name=AGA]\nAGA
   & [name=AGG]\nAGG & [name=GAA]\nGAA & [name=GAG]\nGAG \\\empty
   &&&&[name=GAAG]\nGAAG
  \endpsmatrix
  \ncline{empty}{A}
  \ncline{empty}{C}
  \ncline{empty}{G}
  \ncline{empty}{T}
  \ncline{A}{AA}
  \ncline{A}{AG}
  \ncline{A}{AT}
  \ncline{G}{GA}
  \ncline{G}{GG}
  \ncline{G}{GT}
  \ncline{T}{TA}
  \ncline{T}{TG}
  \ncline{AA}{AAA}
  \ncline{AA}{AAG}
  \ncline{AG}{AGA}
  \ncline{AG}{AGG}
  \ncline{GA}{GAA}
  \ncline{GA}{GAG}
  \ncline{GAA}{GAAG}
  }}
\end{frame}

\ifdefined\wox \begin{frame} \titlepage \end{frame} \fi

\begin{frame}{Projection-Based Sequence Mining: Pref\/{i}xSpan}
Let $\bD$ denote a database, and let $s \in \Sigma$ be any
symbol. The {\em projected database} with respect to $s$, denoted
$\bD_s$, is obtained by f\/{i}nding the
f\/{i}rst occurrence of $s$ in $\bs_i$, say at position $p$. Next,
we retain in $\bD_s$ only the suff\/{i}x of $\bs_i$
starting at position $p+1$.  Further, any infrequent
symbols are removed from the suff\/{i}x.
This is done for each sequence
$\bs_i \in \bD$.

\medskip
Pref\/{i}xSpan computes the support for only the
individual symbols in the projected database $\bD_s$; it then
performs recursive projections on the
frequent symbols in a depth-f\/{i}rst manner. 

\medskip
Given a frequent subsequence $\br$, let $\bD_\br$ be the projected
dataset for $\br$. Initially $\br$ is empty and $\bD_\br$ is the entire
input dataset $\bD$.
Given a database of (projected) sequences $\bD_\br$,
Pref\/{i}xSpan f\/{i}rst f\/{i}nds all the frequent symbols in the projected
dataset. For each such symbol $s$, we extend $\br$ by appending $s$
to obtain the new frequent subsequence $\br_s$. Next, we create the
projected dataset $\bD_s$ by projecting $\bD_\br$ on
symbol $s$. A recursive call to Pref\/{i}xSpan is
then made with $\br_s$ and $\bD_s$.
\end{frame}


\begin{frame}[fragile]{PrefixSpan Algorithm}
\begin{tightalgo}[H]{\textwidth-18pt}
\SetKwInOut{AlgorithmS}{\algprefixspan ($\bD_\br$, $\br$, $\minsup$,
$\cF$)}
\tcp{Initial Call: $\bD_\br \assign \bD$,
$\br \assign \emptyset$, $\cF \assign \emptyset$}
\AlgorithmS{}
\ForEach{$s \in \Sigma$ such that $\supp(s, \bD_\br) \ge \minsup$}{
  $\br_s = \br + s$ \tcp{ extend $\br$ by symbol $s$}
  $\cF \assign \cF \cup \bigl\{ (\br_s, \supp(s, \bD_\br))\bigr\}$\;
  $\bD_s \assign \emptyset$ \tcp{ create projected data for symbol $s$}
  \ForEach{$\bs_i \in \bD_\br$}{
    $\bs'_i \assign $ projection of $\bs_i$ w.r.t symbol $s$\;
    Remove any infrequent symbols from $\bs'_i$\;
    Add $\bs'_i$ to $\bD_s$ if $\bs'_i \ne \emptyset$
  }
  \lIf{$\bD_s \ne \emptyset$}{
    \algprefixspan($\bD_s$, $\br_s$, $\minsup$, $\cF$)
  }
}
\end{tightalgo}
\end{frame}



\begin{frame}{Projection-based Sequence Mining: PrefixSpan}
\centering
\scalebox{0.5}{%
\psset{levelsep=1.5in}
\pstree[]{%
  \TR{%
    \psframebox{
      \begin{tabular}{ll}
          \multicolumn{2}{c}{$\bD_{\emptyset}$}\\
        \hline
        $\bs_1$ & $\mathit{CAGAAGT}$\\
        $\bs_2$ & $\mathit{TGACAG}$\\
        $\bs_3$ & $\mathit{GAAGT}$\\
        \hline\hline
        \multicolumn{2}{c}{$A(3)$, \st{$C(2)$}, $G(3)$, $T(3)$}
      \end{tabular}}}%
}{%
  \pstree[]{%
    \TR{%
      \psframebox{
        \begin{tabular}{ll}
          \multicolumn{2}{c}{$\bD_A$}\\
          \hline
          $\bs_1$ & $\mathit{GAAGT}$\\
          $\bs_2$ & $\mathit{AG}$\\
          $\bs_3$ & $\mathit{AGT}$\\
          \hline\hline
          \multicolumn{2}{c}{$A(3)$, $G(3)$, \st{$T(2)$}}
        \end{tabular}}}%
  }{%
    \pstree[]{%
      \TR{%
        \psframebox{
          \begin{tabular}{ll}
            \multicolumn{2}{c}{$\bD_{\mathit{AA}}$}\\
            \hline
            $\bs_1$ & $\mathit{AG}$\\
            $\bs_2$ & $G$\\
            $\bs_3$ & $G$\\
            \hline\hline
            \multicolumn{2}{c}{\st{$A(1)$}, $G(3)$}
          \end{tabular}}}%
    }{%
      \TR{%
        \psframebox{
        \begin{tabular}{c}
          $\bD_{\mathit{AAG}}$\\
          \hline
          $\emptyset$
        \end{tabular}}}%
    }%
    \TR{%
      \psframebox{
        \begin{tabular}{ll}
          \multicolumn{2}{c}{$\bD_{\mathit{AG}}$}\\
          \hline
          $\bs_1$ & $\mathit{AAG}$\\
          \hline\hline
          \multicolumn{2}{c}{\st{$A(1)$}, \st{$G(1)$}}
        \end{tabular}}}%
  }%
  \pstree[]{%
    \TR{%
      \psframebox{
        \begin{tabular}{ll}
          \multicolumn{2}{c}{$\bD_{G}$}\\
          \hline
          $\bs_1$ & $\mathit{AAGT}$\\
          $\bs_2$ & $\mathit{AAG}$\\
          $\bs_3$ & $\mathit{AAGT}$\\
          \hline\hline
          \multicolumn{2}{c}{$A(3)$, $G(3)$, \st{$T(2)$}}
        \end{tabular}}}%
  }{%
    \pstree[]{%
      \TR{%
        \psframebox{
          \begin{tabular}{ll}
            \multicolumn{2}{c}{$\bD_{\mathit{GA}}$}\\
            \hline
            $\bs_1$ & $\mathit{AG}$\\
            $\bs_2$ & $\mathit{AG}$\\
            $\bs_3$ & $\mathit{AG}$\\
            \hline\hline
            \multicolumn{2}{c}{$A(3)$, $G(3)$}
          \end{tabular}}}%
    }{%
      \pstree[thislevelsep=1in]{%
        \TR{%
          \psframebox{
            \begin{tabular}{ll}
              \multicolumn{2}{c}{$\bD_{\mathit{GAA}}$}\\
              \hline
              $\bs_1$ & $G$\\
              $\bs_2$ & $G$\\
              $\bs_3$ & $G$\\
              \hline\hline
              \multicolumn{2}{c}{$G(3)$}
            \end{tabular}}}%
      }{%
        \TR{%
          \psframebox{
            \begin{tabular}{c}
            $\bD_{\mathit{GAAG}}$\\
            \hline
            $\emptyset$
          \end{tabular}}}%
      }%

      \TR{%
        \psframebox{
          \begin{tabular}{c}
            $\bD_{\mathit{GAG}}$\\
            \hline
            $\emptyset$
          \end{tabular}}}%
    }%
    \TR{%
      \psframebox{
        \begin{tabular}{c}
          $\bD_{\mathit{GG}}$\\
          \hline
          $\emptyset$
        \end{tabular}}}%
  }%
  \TR{%
    \psframebox{
      \begin{tabular}{ll}
        \multicolumn{2}{c}{$\bD_{T}$}\\
        \hline
        $\bs_2$ & $\mathit{GAAG}$\\
        \hline\hline
        \multicolumn{2}{c}{\st{$A(1)$}, \st{$G(1)$}}
      \end{tabular}}}%
} }
\end{frame}

\ifdefined\wox \begin{frame} \titlepage \end{frame} \fi


\begin{frame}{Substring Mining via Suff\/{i}x Trees}

  Let $\bs$ be a sequence having length $n$, then
there are at most $O(n^2)$ possible distinct
substrings contained in $\bs$. This is a much smaller search space compared to subsequences, and
consequently we can design more eff\/{i}cient algorithms for solving
the frequent substring mining task. 

\bigskip
Naively, we can mine all the
frequent substrings in worst case $O(Nn^2)$ time for a dataset
$\bD=\{\bs_1, \bs_2, \ldots, \bs_N\}$ with $N$ sequences.

\bigskip
We will show that all sequences can be mined in $O(Nn)$ time via Suffix
Trees.
\end{frame}



\begin{frame}{Suff\/{i}x Tree}
Given a sequence $\bs$, we append a terminal character 
$\$ \not\in \Sigma$ so that
$\bs = s_{1}s_{2} \dots s_{n}s_{n+1}$, where $s_{n+1}=\$$, and
the $j$th suff\/{i}x of $\bs$ is given as
$\bs[j:n+1] = s_{j}s_{j+1} \dots s_{n+1}$.

\medskip
The {\em suff\/{i}x tree} of the sequences in the database $\bD$,
denoted $\cT$,
 stores all the
suff\/{i}xes for each $\bs_i \in \bD$ in a tree structure, where suff\/{i}xes
that share a common pref\/{i}x lie on the same path from the root of the
tree.

\medskip
The substring obtained by
concatenating all the symbols from the root node to a node $v$ is
called the {\em node label} of $v$, and is denoted as $L(v)$.
The substring
that appears on an edge $(v_a,v_b)$ is called an {\em edge label},
and is denoted as $L(v_a,v_b)$.

\medskip
A suff\/{i}x
tree has two kinds of nodes: internal and leaf nodes.  An internal
node in the suff\/{i}x tree (except for the root) has at least two children,
where each edge label to a child begins with a different symbol.  Because
the terminal character is unique, there are as many leaves in the
suff\/{i}x tree as there are unique suff\/{i}xes over all the sequences.
Each leaf node corresponds to a suff\/{i}x shared by one or
more sequences in $\bD$.
\end{frame}


\begin{frame}[fragile]{Suffix Tree Construction for $\bs=CAGAAGT\$$}
  \framesubtitle{Insert each suffix $j$ per step}
\begin{figure}
  \scriptsize
  \def\Tb#1{\Tr{\psframebox{#1}}}
  \psset{unit=1in}
  \psset{tpos=0.5,nrot=:U,treesep=0.15,levelsep=*1,radius=0.05}
  \centerline{
  \scalebox{0.9}{%
    \subfloat[$j=1$]{ \label{fig:fpm:seq:nst1a}
      \fbox{
        \pstree[]{\TC*}{%
          \Tb{(1,1)}\ncput*{$CAGAAGT\$$}
        }
      }
    }
    \subfloat[$j=2$]{      \label{fig:fpm:seq:nst1b}
      \fbox{
        \pstree[]{\TC*}{%
          \Tb{(1,2)}\ncput*{$AGAAGT\$$}
          \Tb{(1,1)}\ncput*{$CAGAAGT\$$}
        }
      }
    }
    \subfloat[$j=3$]{ \label{fig:fpm:seq:nst1c}
      \fbox{
        \pstree[]{\TC*}{%
          \Tb{(1,2)}\ncput*{$AGAAGT\$$}
          \Tb{(1,1)}\ncput*{$CAGAAGT\$$}
          \Tb{(1,3)}\ncput*{$GAAGT\$$}
        }
      }
    }
    \subfloat[$j=4$]{ \label{fig:fpm:seq:nst1d}
      \fbox{
        \pstree[]{\TC*}{%
          \pstree[]{\TC*\ncput*{$A$}}{%
            \Tb{(1,4)}\ncput*{$AGT\$$}
            \Tb{(1,2)}\ncput*{$GAAGT\$$}
          }
          \tspace{0.2}
          \Tb{(1,1)}\ncput*{$CAGAAGT\$$}
          \Tb{(1,3)}\ncput*{$GAAGT\$$}
        }
      }
    }
    }}%centerline
\end{figure}
\end{frame}

\begin{frame}[fragile]{Suffix Tree Construction  for $\bs=CAGAAGT\$$}
  \framesubtitle{Insert each suffix $j$ per step}
\begin{figure}
  \scriptsize
  \def\Tb#1{\Tr{\psframebox{#1}}}
  \psset{unit=1in}
  \psset{tpos=0.5,nrot=:U,treesep=0.15,levelsep=*1,radius=0.05}
  \centerline{
  \scalebox{0.7}{%
    \subfloat[$j=5$]{ \label{fig:fpm:seq:nst1e}
      \fbox{
        \pstree[]{\TC*}{%
          \pstree[]{\TC*\ncput*{$A$}}{%
            \Tb{(1,4)}\ncput*{$AGT\$$}
            \pstree[]{\TC*\ncput*{$G$}}{%
              \Tb{(1,2)}\ncput*{$AAGT\$$}
              \Tb{(1,5)}\ncput*{$T\$$}
            }
          }
          \tspace{0.2}
          \Tb{(1,1)}\ncput*{$CAGAAGT\$$}
          \Tb{(1,3)}\ncput*{$GAAGT\$$}
        }
      }
    }
    \subfloat[$j=6$]{ \label{fig:fpm:seq:nst1f}
      \fbox{
        \pstree[]{\TC*}{%
          \pstree[]{\TC*\ncput*{$A$}}{%
            \Tb{(1,4)}\ncput*{$AGT\$$}
            \pstree[]{\TC*\ncput*{$G$}}{%
              \Tb{(1,2)}\ncput*{$AAGT\$$}
              \Tb{(1,5)}\ncput*{$T$}
            }
          }
          \tspace{0.2}
          \Tb{(1,1)}\ncput*{$CAGAAGT\$$}
          \pstree[]{\TC*\ncput*{$G$}}{%
            \Tb{(1,3)}\ncput*{$AAGT\$$}
            \Tb{(1,6)}\ncput*{$T\$$}
          }
        }
      }
    }
    \subfloat[$j=7$]{  \label{fig:fpm:seq:nst1g}
      \fbox{
        \pstree[]{\TC*}{%
          \pstree[]{\TC*\ncput*{$A$}}{%
            \Tb{(1,4)}\ncput*{$AGT\$$}
            \pstree[]{\TC*\ncput*{$G$}}{
              \Tb{(1,2)}\ncput*{$AAGT\$$}
              \Tb{(1,5)}\ncput*{$T\$$}
            }
          }
          \tspace{0.2}
          \Tb{(1,1)}\ncput*{$CAGAAGT\$$}
          \pstree[]{\TC*\ncput*{$G$}}{%
            \Tb{(1,3)}\ncput*{$AAGT\$$}
            \Tb{(1,6)}\ncput*{$T\$$}
          }
          \Tb{(1,7)}\ncput*{$T\$$}
        }
      }
    }
    }}%centerline
\end{figure}
\end{frame}


\newcommand\TB[2]{\Tr{\psframebox{%
      \begin{tabular}{c}
        #1\\#2
      \end{tabular}}}}
 \def\Tb#1{\Tr{\psframebox{#1}}}
\begin{frame}{Suffix Tree for Entire Database}
\framesubtitle{$\bD = \{\bs_1=\mathit{CAGAAGT}, 
    \bs_2 = \mathit{TGACAG},
	\bs_3=\mathit{GAAGT} \}$}
  \scriptsize
  \psset{unit=1in}
  \psset{tpos=0.5,nrot=:U,treesep=0.2,radius=0.05,levelsep=*0.75}
  \centerline{
  \scalebox{0.9}{%
      \pstree[]{\Tcircle{3}}{%
        \pstree[]{\Tcircle{3}\ncput*{$A$}}{%
          \TB{(1,4)}{(3,2)}\ncput*{$AGT\$$}
          \Tb{(2,3)}\ncput*[]{$CAG\$$}
          \pstree[]{\Tcircle{3}\ncput*{$G$}}{
            \Tb{(1,2)}\ncput*{$AAGT\$$}
            \TB{(1,5)}{(3,3)}\ncput*{$T\$$}
            \Tb{(2,5)}\ncput*{$\$$}
          }
        }
        \pstree[]{\Tcircle{2}\ncput*{$CAG$}}{%
          \Tb{(1,1)}\ncput*{$AAGT\$$}
          \Tb{(2,4)}\ncput*{$\$$}
        }
        \pstree[]{\Tcircle{3}\ncput*{$G$}}{%
          \pstree[]{\Tcircle{3}\ncput*{$A$}}{
            \TB{(1,3)}{(3,1)}\ncput*{$AGT\$$}
            \Tb{(2,2)}\ncput*{$CAG\$$}
          }
          \TB{(1,6)}{(3,4)}\ncput*{$T\$$}
          \Tb{(2,6)}\ncput*{$\$$}
        }
        \pstree[]{\Tcircle{3}\ncput*{$T$}}{
          \Tb{(2,1)}\ncput*{$GACAG\$$}
          \TB{(1,7)}{(3,5)}\ncput*{$\$$}
        }
      }
    }}%centerline
\end{frame}


\begin{frame}{Frequent Substrings}
Once the suff\/{i}x tree is built, we can compute all the frequent
substrings by checking how many different sequences appear in a leaf
node or under an internal node.

\medskip
The node labels for the nodes with support at least $\minsup$ yield the
set of frequent substrings; all the pref\/{i}xes of such node labels are
also frequent.


\medskip
The suff\/{i}x tree can also support ad hoc queries for
f\/{i}nding all the occurrences in the database for any query substring
$\bq$.
For each symbol in $\bq$, we follow the path from the root until all
symbols in $\bq$ have been seen, or until there is a mismatch at any
position. If $\bq$ is found, then the set of leaves under that path is the
list of occurrences of the query $\bq$. On the other hand, if there is
mismatch that means the
query does not occur in the database.

\medskip
Because we have to match each
character in $\bq$, we immediately get $O(|\bq|)$
as the time bound (assuming
that $\card{\Sigma}$ is a constant), which is
{\em independent} of the size of the database. Listing all
the matches takes additional time, for a total time complexity of
$O(|\bq|+k)$, if there are $k$ matches.


\end{frame}

\ifdefined\wox \begin{frame} \titlepage \end{frame} \fi

\begin{frame}{Ukkonen's Linear Time Suffix Tree Algorithm}

{\bf Achieving Linear Space:}
If an algorithm stores all the
symbols on each edge label, then the space complexity is $O(n^2)$, and
we cannot achieve
linear time construction either.

\bigskip
The trick is to not explicitly store
all the edge labels, but rather to use an {\em edge-compression}
technique, where we store only the starting and ending positions of
the edge label in the input string $\bs$. That is, if an edge label
is given as $\bs[i:j]$, then we represent is as the
interval $[i,j]$.
\end{frame}

  \def\Tb#1{\Tr{\psframebox{#1}}}
  \renewcommand\TB[2]{\Tr{\psframebox{%
        \begin{tabular}{c}
          #1\\#2
        \end{tabular}}}}
\begin{frame}[fragile]{Suffix Tree using Edge-compression:
  $\bs=CAGAAGT\$$}
\setcounter{subfigure}{0}
  \begin{figure}
  \psset{unit=1in}
  \psset{tpos=0.5,nrot=:U,treesep=0.2,radius=0.05,levelsep=*1.0,treefit=loose}
    \subfloat[Full Tree]{
	  \scalebox{0.4}{
      \fbox{
        \pstree[]{\TC*}{%
          \pstree[]{\TC*\ncput*{$A$}}{%
            \Tb{(1,4)}\ncput*{$AGT\$$}
            \pstree[]{\TC*\ncput*{$G$}}{
              \Tb{(1,2)}\ncput*{$AAGT\$$}
              \Tb{(1,5)}\ncput*{$T\$$}
            }
          }
          \tspace{0.2}
          \Tb{(1,1)}\ncput*{$CAGAAGT\$$}
          \pstree[]{\TC*\ncput*{$G$}}{%
            \Tb{(1,3)}\ncput*{$AAGT\$$}
            \Tb{(1,6)}\ncput*{$T\$$}
          }
          \Tb{(1,7)}\ncput*{$T\$$}
        }
      }
	  }}
	\subfloat[Compressed Tree]{
     \scalebox{0.7}{
	 \fbox{
      \pstree[]{\TC*~[tnpos=l]{$v_1$}}{%
        \pstree[]{\TC*~[tnpos=l]{$v_2$}\ncput*{$[2,2]$}}{%
          \Tb{(1,4)}\ncput*{$[5,8]$}
          \pstree[]{\TC*~[tnpos=l]{$v_3$}\ncput*{$[3,3]$}}{
            \Tb{(1,2)}\ncput*{$[4,8]$}
            \Tb{(1,5)}\ncput*{$[7,8]$}
          }
        }
        \Tb{(1,1)}\ncput*{$[1,8]$}
        \pstree[]{\TC*~[tnpos=l]{$v_4$}\ncput*{$[3,3]$}}{%
          \Tb{(1,3)}\ncput*{$[4,8]$}
          \Tb{(1,6)}\ncput*{$[7,8]$}
        }
        \Tb{(1,7)}\ncput*{$[7,8]$}
		}}
	  }}
\end{figure}
\end{frame}


\begin{frame}{Ukkonen Algorithm: Achieving Linear Time}
Ukkonen's method is an {\em online} algorithm, that is,
given a string $\bs = s_1s_2\ldots s_{n}\$$ it constructs the full
suff\/{i}x tree in phases. 

\medskip
Phase $i$ builds the tree up to the $i$-th symbol
in $\bs$.
Let $\cT_{i}$ denote the
suff\/{i}x tree up to the $i$th pref\/{i}x $\bs[1:i]$, with $1 \le i \le
n$. Ukkonen's algorithm constructs $\cT_i$ from
$\cT_{i-1}$, by making sure
that all suff\/{i}xes including the {\em current} character $s_{i}$ are in
the new intermediate tree $\cT_{i}$. 

\medskip
In other words, in the $i$th
phase, it inserts all the suff\/{i}xes $\bs[j:i]$ from $j=1$ to
$j=i$ into the tree $\cT_{i}$. Each such insertion is called the $j$th
{\em extension} of the $i$th {\em phase}.

\medskip
Once we process the terminal
character at position $n+1$ we obtain the f\/{i}nal suff\/{i}x tree $\cT$
for $\bs$.


\medskip
However, this naive Ukkonen method has cubic time complexity
because to obtain $\cT_i$ from $\cT_{i-1}$ takes $O(i^2)$ time,
with the last phase requiring $O(n^2)$ time. With
$n$ phases, the total time is $O(n^3)$.
We will show that this time can be reduced to $O(n)$.

\end{frame}


\begin{frame}[fragile]{Algorithm \algUkkonenSimple}
\begin{tightalgo}[H]{\textwidth-18pt}
\SetKwInOut{AlgorithmU}{\algUkkonenSimple ($\bs$)} \AlgorithmU{}
$n \gets \card{\bs}$\; $\bs[n+1] \gets \$$ \tcp{ append terminal character} 
$\cT \assign \emptyset$ \tcp{ add empty string as root} 
\ForEach(\tcp*[h]{phase $i$ - construct
$\cT_i$}){$i=1,\ldots,n+1$}{
  \ForEach(\tcp*[h]{extension $j$ for phase $i$}){$j=1,\ldots,i$}{
    \tcp{Insert $\bs[j:i]$ into the suff\/{i}x tree}
    F{i}nd end of the path with label $\bs[j:i-1]$ in $\cT$\;
    Insert $s_i$ at end of path;
  }
}
\Return{$\cT$}\;
\end{tightalgo}
\end{frame}

\begin{frame}{Ukkonen's Linear Time Algorithm: Implicit Suffixes}
This optimization states that, in phase
$i$, if the $j$th extension $\bs[j:i]$ is found in the tree, then any
subsequent extensions will also be found, and consequently there is
no need to process further extensions in phase $i$.

\bigskip
Thus, the suff\/{i}x tree $\cT_i$ at the end of phase $i$ has
{\em implicit suff\/{i}xes} corresponding to extensions
$j+1$ through $i$.

\bigskip
It is important to note that all suff\/{i}xes will
become explicit the f\/{i}rst
time we encounter a new substring that does not already exist in the
tree. This will surely happen in phase $n+1$ when
we process the terminal character
$\$$, as it cannot occur anywhere else in
$\bs$ (after all, $\$ \not\in \Sigma$).
\end{frame}




\begin{frame}{Ukkonen's Algorithm: Implicit Extensions}
Let the current phase be $i$, and let $l \le i-1$ be
the last explicit suff\/{i}x in the previous tree $\cT_{i-1}$.

\bigskip
All explicit suff\/{i}xes in $\cT_{i-1}$ have
edge labels of the form $[x,i-1]$ leading to the corresponding
leaf nodes, where the starting position
$x$ is node specif\/{i}c, but the ending position
must be $i-1$ because $s_{i-1}$ was added to the end of these
paths in phase $i-1$. 

\bigskip
In the current phase $i$, we would have to
extend these paths by adding $s_i$ at the end.
However, instead of explicitly incrementing all the
ending positions, we can replace the ending position by a pointer $e$ which
keeps track of the current phase being
processed. 

\bigskip
If we replace $[x,i-1]$ with $[x,e]$, then
in phase $i$, if we
set $e=i$, then immediately all the $l$ existing suff\/{i}xes get
{\em implicitly} extended to $[x,i]$. Thus, in one operation of
incrementing $e$ we have, in effect, taken care of
extensions $1$ through $l$ for phase $i$.
\end{frame}


\begin{frame}[fragile]{Implicit Extensions: $\bs=CAGAAGT\$$, 
  Phase ${\it i}=7$}
\setcounter{subfigure}{0}
\begin{figure}
  \def\Tb#1{\Tr{\psframebox{#1}}}
  \def\TSb#1{\TCircle[radius=0.2,doubleline=true]{#1}}
  \psset{unit=1in}
  \psset{tpos=0.5,nrot=:U,treesep=0.5,levelsep=*1.5,%
    tnpos=a,radius=0.075}
    \centerline{
    \scalebox{0.8}{
    \subfloat[$\cT_6$]{ \label{fig:fpm:seq:impexts1}
    \scalebox{0.8}{%
        \pstree[thislevelsep=2]{\TC*[name=r]}{%
          \pstree[]{\TC*[name=A]\ncput*{$[2,2]=A$}}{%
              \TSb{(1,4)}\ncput*{$[5,e]=AG$}
              \Tb{(1,2)}\ncput*{$[3,e]=GAAG$}
          }
          \tspace{0.4}
          \Tb{(1,1)}\ncput*{$[1,e]=CAGAAG$}
          \Tb{(1,3)}\ncput*{$[3,e]=GAAG$}
      }
      }}
    \hspace{0.25in}
    \subfloat[$\cT_7$, extensions $j=1,\ldots,4$]{ \label{fig:fpm:seq:impexts2}
    \scalebox{0.8}{%
        \pstree[thislevelsep=2]{\TC*[name=r]}{%
          \pstree[]{\TC*[name=A]\ncput*{$[2,2]=A$}}{%
            \TSb{(1,4)}\ncput*{$[5,e]=AGT$}
            \Tb{(1,2)}\ncput*{$[3,e]=GAAGT$}
          }
          \tspace{0.6}
          \Tb{(1,1)}\ncput*{$[1,e]=CAGAAGT$}
          \Tb{(1,3)}\ncput*{$[3,e]=GAAGT$}
        }
      }
      }}
      }
\end{figure}
\end{frame}


\begin{frame}{Ukkonen's Algorithm: Skip/count Trick}
For the $j$th extension of phase $i$,
we have to search for the substring $\bs[j:i-1]$
so that we can add
$s_i$ at the end. 

\medskip
Note that
this string must exist in $\cT_{i-1}$ because we have already processed
symbol $s_{i-1}$ in the previous phase.
  Thus, instead of searching for each
  character in $\bs[j:i-1]$ starting from the root, we f\/{i}rst
  {\em count} the number of symbols on
  the edge beginning with character
  $s_{j}$; let this length be $m$. If
  $m$ is longer than the length of the substring (i.e., if $m > i-j$), then
  the substring must end on this edge, so we simply jump to position
  $i-j$ and insert $s_i$.

  \medskip
  On the other hand, if $m \le i-j$, then we can
  {\em skip} directly to the child node,
  say $v_c$, and
  search for the remaining string $\bs[j+m:i-1]$ from
  $v_c$ using the same skip/count technique.

  \medskip
  With this
  optimization, the cost of an extension becomes proportional
  to the number of nodes on the path, as opposed to the number of
  characters in $\bs[j:i-1]$.
\end{frame}
  
  

\begin{frame}{Ukkonen's Algorithm: Suffix Links}
We can avoid searching for the substring $\bs[j:i-1]$ 
from the root via the use of {\em suff\/{i}x links}.

\medskip
For each internal node $v_a$ we maintain a link to the
internal node $v_b$, where $L(v_b)$ is the immediate suff\/{i}x of $L(v_a)$.

\medskip
In extension $j-1$, let $v_p$ denote the internal node under which we
f\/{i}nd $\bs[j-1:i]$, and let $m$ be the length of the node label of $v_p$.
To insert the $j$th extension $\bs[j:i]$,
we follow the suff\/{i}x link from $v_p$ to another node, say $v_s$, and
search for the
remaining substring $\bs[j+m-1:i-1]$ from $v_s$.

\medskip
The use of suff\/{i}x links allows us to jump internally within the tree for
different extensions, as opposed to searching from the root each time.
\end{frame}

\begin{frame}[fragile]{Linear Time Ukkonen Algorithm}
\begin{tightalgo}[H]{\textwidth-18pt}
\SetKw{Break}{break} 
\SetKwInOut{AlgorithmU}{\algUkkonen ($\bs$)}
\AlgorithmU{} 
$n \gets \card{\bs}$ \; 
$\bs[n+1] \gets \$$ \tcp{ append terminal character}  
$\cT \assign \emptyset$ \tcp{ add empty string as root} 
$l \assign 0$ \tcp{ last explicit suff\/{i}x} 
\ForEach(\tcp*[h]{phase $i$ - construct $\cT_i$}){$i=1,\ldots,n+1$}{ 
$e \assign i$ \tcp{ implicit extensions} \nllabel{alg:fpm:seq:UkkonenImpExt}
  \ForEach(\tcp*[h]{extension $j$ for phase $i$}){$j=l+1,\ldots,i$}{
    \tcp{Insert $\bs[j:i]$ into the suff\/{i}x tree}
    F{i}nd end of $\bs[j:i-1]$ in $\cT$ via skip/count
    and suff\/{i}x links \nllabel{alg:fpm:seq:UkkonenSkip}\;
    \eIf(\tcp*[h]{implicit suff\/{i}xes}){$s_i \in \cT$}{
      \Break \nllabel{alg:fpm:seq:UkkonenImpSuff}\;
      }{
      Insert $s_i$ at end of path\;
      Set last explicit suff\/{i}x $l$ if needed\;
    }
  }
}
\Return{$\cT$}\;
\end{tightalgo}
\end{frame}
  

\def\Tb#1{\Tr{\psframebox{#1}}}
  \def\TSb#1{\Tr{\pscirclebox[doubleline=true]{#1}}}
\begin{frame}[fragile]{Ukkonen's Suffix Tree Construction:
  $\bs=CAGAAGT\$$}
\setcounter{subfigure}{0}
\begin{figure}
  %\small
  \psset{unit=1in}
  \psset{tpos=0.5,nrot=:U,treesep=0.2,levelsep=*1.25,tnpos=a,radius=0.05}
  \centerline{
    \subfloat[$\cT_1$]{ \label{fig:fpm:seq:ukka}
  \scalebox{0.6}{%
      \fbox{
        \pstree[]{\TC*~{{\pscirclebox{$C$}$AGAAGT\$$}, $e=1$}}{%
          \TSb{(1,1)}\ncput*{$[1,e]=C$}
        }
      }
      }}
    \subfloat[$\cT_2$]{ \label{fig:fpm:seq:ukkb}
  \scalebox{0.6}{%
      \fbox{
        \pstree[]{\TC*~{{$C$\pscirclebox{$A$}$GAAGT\$$}, $e=2$}}{%
          \TSb{(1,2)}\ncput*{$[2,e]=A$}
          \Tb{(1,1)}\ncput*{$[1,e]=CA$}
        }
      }
      }}
    \subfloat[$\cT_3$]{ \label{fig:fpm:seq:ukkc}
  \scalebox{0.6}{%
      \fbox{
        \pstree[]{\TC*~{{$CA$\pscirclebox{$G$}$AAGT\$$}, $e=3$}}{%
          \Tb{(1,2)}\ncput*{$[2,e]=AG$}
          \tspace{0.3}
          \Tb{(1,1)}\ncput*{$[1,e]=CAG$}
          \TSb{(1,3)}\ncput*{$[3,e]=G$}
        }
      }
      }}
    \subfloat[$\cT_4$]{ \label{fig:fpm:seq:ukkd}
  \scalebox{0.6}{%
      \fbox{
        \pstree[]{\TC*~{{$CAG$\pscirclebox{$A$}$AGT\$$}, $e=4$}}{%
          \Tb{(1,2)}\ncput*{$[2,e]=AGA$}
          \tspace{0.3}
          \Tb{(1,1)}\ncput*{$[1,e]=CAGA$}
          \tspace{0.3}
          \TSb{(1,3)}\ncput*{$[3,e]=GA$}
        }
      }
    }
    }
  }%centerline
\end{figure}
\end{frame}

\begin{frame}[fragile]{Ukkonen's Suffix Tree Construction:
  $\bs=CAGAAGT\$$}
\begin{figure}
  \psset{unit=1in}
  \psset{tpos=0.5,nrot=:U,treesep=0.2,levelsep=*1.25,tnpos=a,radius=0.05}
  \centerline{
    \subfloat[$\cT_5$]{ \label{fig:fpm:seq:ukke}
  \scalebox{0.5}{%
      \fbox{
        \pstree[thislevelsep=*1.45]{\TC*[name=r]~{{$CAGA$\pscirclebox{$A$}$GT\$$},
        $e=5$}}{%
          \pstree[]{\TC*[name=A]\ncput*{$[2,2]=A$}}{%
              \TSb{(1,4)}\ncput*{$[5,e]=A$}
              \Tb{(1,2)}\ncput*{$[3,e]=GAA$}
          }
          \tspace{0.3}
          \Tb{(1,1)}\ncput*{$[1,e]=CAGAA$}
          \tspace{0.2}
          \Tb{(1,3)}\ncput*{$[3,e]=GAA$}
        }
        \ncarc[arcangle=50,linestyle=dashed,arrows=->,arrowscale=2]{A}{r}
      }
      }}
    \subfloat[$\cT_6$]{ \label{fig:fpm:seq:ukkf}
  \scalebox{0.5}{%
      \fbox{
        \pstree[thislevelsep=*1.45]{\TC*[name=r]~{{$CAGAA$\pscirclebox{$G$}$T\$$},
        $e=6$}}{%
          \pstree[]{\TC*[name=A]\ncput*{$[2,2]=A$}}{%
              \TSb{(1,4)}\ncput*{$[5,e]=AG$}
              \Tb{(1,2)}\ncput*{$[3,e]=GAAG$}
          }
          \tspace{0.4}
          \Tb{(1,1)}\ncput*{$[1,e]=CAGAAG$}
          \tspace{0.3}
          \Tb{(1,3)}\ncput*{$[3,e]=GAAG$}
        }
        \ncarc[arcangle=50,linestyle=dashed,arrows=->,arrowscale=2]{A}{r}
      }
      }}
    \subfloat[$\cT_7$]{  \label{fig:fpm:seq:ukkg}
  \scalebox{0.5}{%
      \fbox{
        \pstree[thislevelsep=*1.45]{\TC*[name=r]~{{$CAGAAG$\pscirclebox{$T$}$\$$},
        $e=7$}}{%
          \pstree[]{\TC*[name=A]\ncput*{$[2,2]=A$}}{%
            \Tb{(1,4)}\ncput*{$[5,e]=AGT$}
            \pstree[]{\TC*[name=AG]\ncput*{$[3,3]=G$}}{%
              \Tb{(1,2)}\ncput*{$[4,e]=AAGT$}
              \Tb{(1,5)}\ncput*{$[7,e]=T$}
            }
          }
          \tspace{0.5}
          \Tb{(1,1)}\ncput*[]{$[1,e]=CAGAAGT$}
          \tspace{0.8}
          \pstree[]{\TC*[name=G]\ncput*{$[3,3]=G$}}{%
            \Tb{(1,3)}\ncput*{$[4,e]=AAGT$}
            \Tb{(1,6)}\ncput*{$[7,e]=T$}
          }
          \tspace{0.3}
          \TSb{(1,7)}\ncput*{$[7,e]=T$}
        }
        \ncarc[arcangle=50,linestyle=dashed,arrows=->,arrowscale=2]{A}{r}
        \ncline[linestyle=dashed,arrows=->,arrowscale=2]{AG}{G}
        \ncarc[arcangle=20,linestyle=dashed,arrows=->,arrowscale=2]{G}{r}
      }
    }
    }
  }%centerline
\end{figure}
\end{frame}


\begin{frame}[fragile]{Extensions in Phase $i=7$}
\setcounter{subfigure}{0}
\begin{figure}
\centerline{
  \def\Tb#1{\Tr{\psframebox{#1}}}
  \def\TSb#1{\TCircle[radius=0.2,doubleline=true]{#1}}
  \psset{unit=1in}
  \psset{tpos=0.5,nrot=:U,treesep=0.15,levelsep=*1.5,tnpos=a,radius=0.075}
    \subfloat[]{\label{fig:fpm:seq:exts1}
    \scalebox{0.41}{%
      \psframebox[framesep=10pt]{
        \pstree[thislevelsep=*2]{\TC*[name=r]~{Extensions 1--4}}{%
          \pstree[]{\TC*[name=A]~[tnpos=l]{$v_A$}\ncput*{$[2,2]=A$}}{%
            \TSb{(1,4)}\ncput*{$[5,e]=AGT$}
          \tspace{0.4}
            \Tb{(1,2)}\ncput*{$[3,e]=GAAGT$}
          }
          \tspace{0.4}
          \Tb{(1,1)}\ncput*{$[1,e]=CAGAAGT$}
          \tspace{0.3}
          \Tb{(1,3)}\ncput*{$[3,e]=GAAGT$}
        }
        \ncarc[arcangle=30,linestyle=dashed,arrows=->,arrowscale=2]{A}{r}
      }
      }}
    \subfloat[]{\label{fig:fpm:seq:exts2}
    \scalebox{0.41}{%
      \psframebox[framesep=10pt]{
        \pstree[thislevelsep=*2]{\TC*[name=r]~{{Extension 5: $AGT$}}}{%
          \pstree[]{\TC*[name=A]~[tnpos=l]{$v_A$}\ncput*{$[2,2]=A$}}{%
            \Tb{(1,4)}\ncput*{$[5,e]=AGT$}
            \pstree[]{\TC*[name=AG]~[tnpos=l]{$v_{AG}$}\ncput*{$[3,3]=G$}}{%
              \Tb{(1,2)}\ncput*{$[4,e]=AAGT$}
          \tspace{0.4}
              \TSb{(1,5)}\ncput*{$[7,e]=T$}
            }
          }
          \tspace{0.4}
          \Tb{(1,1)}\ncput*{$[1,e]=CAGAAGT$}
          \tspace{0.3}
          \Tb{(1,3)}\ncput*{$[3,e]=GAAGT$}
        }
        \ncarc[arcangle=30,linestyle=dashed,arrows=->,arrowscale=2]{A}{r}
      }
      }}
    \subfloat[]{\label{fig:fpm:seq:exts3}
    \scalebox{0.41}{%
      \psframebox[framesep=10pt]{
        \pstree[thislevelsep=*2]{\TC*[name=r]~{{Extension 6: $GT$}}}{%
          \pstree[]{\TC*[name=A]~[tnpos=l]{$v_A$}\ncput*{$[2,2]=A$}}{%
            \Tb{(1,4)}\ncput*{$[5,e]=AGT$}
            \pstree[]{\TC*[name=AG]~[tnpos=l]{$v_{AG}$}\ncput*{$[3,3]=G$}}{%
              \Tb{(1,2)}\ncput*{$[4,e]=AAGT$}
          \tspace{0.4}
              \Tb{(1,5)}\ncput*{$[7,e]=T$}
            }
          }
          \tspace{0.3}
          \Tb{(1,1)}\ncput*{$[1,e]=CAGAAGT$}
          \tspace{0.3}
          \pstree[]{\TC*[name=G]~[tnpos=l]{$v_{G}$}\ncput*{$[3,3]=G$}}{%
            \Tb{(1,3)}\ncput*{$[4,e]=AAGT$}
          \tspace{0.3}
            \TSb{(1,6)}\ncput*{$[7,e]=T$}
          }
        }
        \ncarc[arcangle=30,linestyle=dashed,arrows=->,arrowscale=2]{A}{r}
        \ncline[linestyle=dashed,arrows=->,arrowscale=2]{AG}{G}
      }
    }}
   }
\end{figure}
\end{frame}
