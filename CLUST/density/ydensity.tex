\lecture{density}{density}

\date{Chapter 15: Density-based Clustering}

\begin{frame}
\titlepage
\end{frame}


\readdata{\dataT}{CLUST/density/figs/t7-4k.txt}
\begin{frame}{Density-based Clustering}
Density-based methods 
are able to mine nonconvex clusters, where distance-based methods may
have difficulty.
\begin{figure}
\begin{center}
\scalebox{0.65}{
\psset{unit=1in}
\psset{fillcolor=lightgray}
\psset{xAxisLabel=$X_1$, yAxisLabel=$X_2$}
  \hspace*{-60pt}
  \pspicture[](-2,-0.5)(2,3.5)
 \psgraph[Ox=0,Oy=20,Dx=100,Dy=75]{->}(0,20)(700,475){4in}{3in}
      \listplot[plotstyle=dots,dotstyle=Bo,
           plotNo=1,plotNoMax=2]{\dataT}
  \endpsgraph
  \endpspicture
  }
\end{center}
\end{figure}
\end{frame}



\begin{frame}{The DBSCAN Approach}
\framesubtitle{Neighborhood and Core Points}
Def\/{i}ne a ball of radius $\epsilon$ around a point
$\bx \in \setR^d$, called the $\epsilon$-{\em neighborhood} of $\bx$:
\begin{align*}
\tcbhighmath{
  N_{\epsilon}(\bx) = B_d(\bx,\epsilon) = \{\by \mid
  \;\dist(\bx,\by) \leq \epsilon\}
}
\end{align*}
Here $\dist(\bx, \by)$ represents the distance between points
$\bx$ and $\by$.
which is usually assumed to be the Euclidean

\medskip
We say that $\bx$ is a {\em core
point} if there are at least $\minpts$ points in its
$\epsilon$-neighborhood, i.e., if 
$|N_{\epsilon}(\bx)| \geq
\minpts$.

\medskip
A {\em border point}
does not meet the $\minpts$ threshold, i.e.,
$|N_{\epsilon}(\bx)| < \minpts$, but it belongs to the $\epsilon$-neighborhood of some core point $\bz$, that is, $\bx \in N_\epsilon(\bz)$.

\medskip
If a point is neither a core nor a border point, then it
is called a {\em noise point} or an outlier.
\end{frame}


\begin{frame}[fragile]{Core, Border and Noise Points}
\begin{figure}[!ht]%fig15.2
  \centerline{
    \subfloat[Neighborhood of a Point]{
    \psset{unit=0.3in, dotscale=2}
    \psset{fillcolor=lightgray}
    \pspicture[](0,0)(4,4)
    \pscircle[linewidth=1pt](2,2){2}
    \pnode(2,2){x}
    \pnode(4,2){r}
    \ncline[linewidth=1pt,arrowscale=1.5]{->}{x}{r}
    \nbput{$\epsilon$}
    \psdots[dotstyle=Bo](2,2)(1,2.5)(2.25,3)(1,1)(3,2.5)(3,0.5)
    \uput[-90](2,2){$\bx$}
    \endpspicture
  }
  \hspace{1in}
  \subfloat[Core, Border, and Noise Points]{
    \psset{unit=0.3in, dotscale=2}
    \psset{fillcolor=lightgray}
    \pspicture[](0,-1.5)(6.5,4)
    \pscircle[linewidth=1pt](2,2){2}
    \pscircle[linewidth=1pt,linestyle=dashed](3,0.5){2}
    \pscircle[linewidth=1pt,linestyle=dotted](5,1){2}
    \psdots[dotstyle=Bo](2,2)(3,0.5)(5,1)(1,2.5)(2.25,3)(1,1)(3,2.5)
    \uput[-90](2,2){$\bx$}
    \uput[45](3,0.5){$\by$}
    \uput[45](5,1){$\bz$}
    \endpspicture
    }}
\end{figure}
\end{frame}

\begin{frame}{The DBSCAN Approach}
\framesubtitle{Reachability and Density-based Cluster}

  A point $\bx$ is {\em directly density reachable}
from another point $\by$ if $\bx \in N_{\epsilon}(\by)$ and
$\by$ is a core point. 

\medskip
A point $\bx$ is {\em density
    reachable} from $\by$ if there exists a chain of points,
$\bx_0, \bx_1, \ldots, \bx_l$, such that
$\bx = \bx_0$ and $\by = \bx_l$, and
$\bx_i$ is
directly density reachable from $\bx_{i-1}$ for all $i=1,\ldots,l$.
In other words,
there is set of core points leading from $\by$ to $\bx$.

\medskip
Two points $\bx$ and $\by$ are
{\em density connected}
if there exists a core point $\bz$,
such that both $\bx$ and $\by$ are density reachable from $\bz$.

\medskip
A {\em density-based cluster} is def\/{i}ned as a maximal set of
density connected points.
\end{frame}


\begin{frame}{DBSCAN Density-based Clustering Algorithm}

DBSCAN computes the
$\epsilon$-neighborhood $N_\epsilon(\bx_i)$ for each point $\bx_i$ in
the dataset $\bD$, and checks if it is a core point.
It also sets the cluster id $id(\bx_i)=\emptyset$ for all points,
indicating that they are not assigned to any cluster.  

\medskip
Starting
from each unassigned core point, the method recursively f\/{i}nds all its
density connected points, which are assigned to the same cluster.

\medskip
Some border point may be
reachable from core points in more than one cluster; they may either be
arbitrarily assigned to one of the clusters or to all of them (if
overlapping clusters are allowed).  

\medskip
Those points that do not belong to
any cluster are treated as outliers or noise.

\medskip
Each DBSCAN cluster is a maximal connected component over the core point
graph.

\medskip
DBSCAN is sensitive to the choice of
$\epsilon$, in particular if clusters have different densities.
The
overall complexity of DBSCAN is $O(n^2)$.
\end{frame}


\newcommand{\dbscan}{\textsc{dbscan}\xspace}
\newcommand{\getcluster}{\textsc{DensityConnected}\xspace}
\begin{frame}[fragile]{DBSCAN Algorithm}
  \scalebox{0.9}{
\begin{tightalgo}[H]{\textwidth-18pt}
  \SetKwInput{Algorithm}{\dbscan ($\bD$, $\epsilon$, $\minpts$)}
  \Algorithm{}
  $\mathit{Core} \assign \emptyset$\;
  \ForEach(\tcp*[h]{F{i}nd the core points}){$\bx_i \in \bD$ \nllabel{alg:clust:den:dbscan:coresA}}{
      Compute $N_\epsilon(\bx_i)$\;
      $\mathit{id}(\bx_i) \assign \emptyset$ \tcp{cluster id for $\bx_i$}
      \lIf{$N_\epsilon(\bx_i) \ge \minpts$}{
        $\mathit{Core} \assign \mathit{Core} \cup \{ \bx_i \}$ \nllabel{alg:clust:den:dbscan:coresB}
      }
  }
  $k \assign 0$ \tcp{cluster id}
  \ForEach{$\bx_i \in \mathit{Core}$, \textit{such that} $\mathit{id}(\bx_i) = \emptyset$}{
    $k \assign k + 1$\;
    $\mathit{id}(\bx_i) \assign k$ \tcp{assign $\bx_i$ to cluster id $k$}
    \getcluster($\bx_i, k$)\nllabel{alg:clust:den:dbscan:getcluster}\;
  }
  $\cC \assign \{ C_i \}_{i=1}^k$, where $C_i \assign
  \{\bx \in \bD\mid \; id(\bx)=i \}$\;
  $\mathit{Noise} \assign \{ \bx \in \bD\mid\; \mathit{id}(\bx) = \emptyset\}$\;
  $\mathit{Border} \assign \bD \setminus \{\mathit{Core} \cup \mathit{Noise}\}$\;
  \Return $\cC, \mathit{Core}, \mathit{Border}, \mathit{Noise}$\;
  \BlankLine
  \BlankLine
 
  \SetKwInput{AlgorithmB}{\getcluster ($\bx$, $k$)}
  \AlgorithmB{}
  \ForEach{$\by \in N_\epsilon(\bx)$}{
    $\mathit{id}(\by) \assign k$ \tcp{assign $\by$ to cluster id $k$}
    \lIf{$\by \in Core$}{
      \getcluster($\by, k$)
    }
  }
\end{tightalgo}
}
\end{frame}



\readdata{\dataTC}{CLUST/density/figs/t7-4k.cluster.e15-m10.txt}
\begin{frame}{Density-based Clusters}
  \framesubtitle{$\epsilon = 15$ and $\minpts = 10$}
\begin{figure}
\scalebox{0.77}{
\psset{unit=1in}
\psset{fillcolor=lightgray}
\psset{xAxisLabel=$X_1$, yAxisLabel=$X_2$}
\psset{plotstyle=dots,dotscale=1}
\centerline{
\pspicture[](-3,-0.1)(3,3)
    \psgraph[Ox=0,Oy=20,Dx=100,Dy=75]{->}(0,20)(700,475){4in}{3in}
%C0
    \psset{dotstyle=Bsquare,fillcolor=white}
    \listplot[nStart=1,nEnd=233]{\dataTC}
    \listplot[nStart=234,nEnd=253]{\dataTC}
%C1
    \psset{dotstyle=Bo,fillcolor=lightgray}
    \listplot[nStart=254,nEnd=1273]{\dataTC}
    \listplot[nStart=1274,nEnd=1371]{\dataTC}
%C2
    \psset{dotstyle=Btriangle,fillcolor=lightgray}
    \listplot[nStart=1372,nEnd=2201]{\dataTC}
    \listplot[nStart=2202,nEnd=2266]{\dataTC}
%C3
    \psset{dotstyle=Bsquare,fillcolor=black}
    \listplot[nStart=2267,nEnd=2489]{\dataTC}
    \listplot[nStart=2490,nEnd=2508]{\dataTC}
%C4
    \psset{dotstyle=Btriangle,fillcolor=white}
    \listplot[nStart=2509,nEnd=2636]{\dataTC}
    \listplot[nStart=2637,nEnd=2652]{\dataTC}
%C5
    \psset{dotstyle=Btriangle,fillcolor=black}
    \listplot[nStart=2653,nEnd=2777]{\dataTC}
    \listplot[nStart=2778,nEnd=2791]{\dataTC}
%C6
    \psset{dotstyle=Bsquare,fillcolor=lightgray}
    \listplot[nStart=2792,nEnd=3167]{\dataTC}
    \listplot[nStart=3168,nEnd=3202]{\dataTC}
%C7
    \psset{dotstyle=Bo,fillcolor=black}
    \listplot[nStart=3203,nEnd=3309]{\dataTC}
    \listplot[nStart=3310,nEnd=3322]{\dataTC}
%C8
    \psset{dotstyle=Bo,fillcolor=white}
    \listplot[nStart=3323,nEnd=3706]{\dataTC}
    \listplot[nStart=3707,nEnd=3743]{\dataTC}
%noise
    \psset{dotstyle=+, dotscale=0.75}
    \listplot[nStart=3744]{\dataTC}
%
    \endpsgraph
    \endpspicture
    }}
\end{figure}
\end{frame}


\begin{frame}[fragile]{DBSCAN Clustering}
\framesubtitle{Iris Dataset}

There is a trade off between $\epsilon$ and $\minpts$.

\setcounter{subfigure}{0}
\begin{figure}
\psset{dotscale=1.5,fillcolor=lightgray,
        arrowscale=2,PointName=none}
\psset{xAxisLabel=$X_1$, yAxisLabel= $X_2$}
\centerline{
\captionsetup[subfloat]{captionskip=15pt}
\subfloat[$\epsilon = 0.2$, $\minpts = 5$]{
\label{fig:clust:den:iris:dbscanA}
\readdata{\dataICa}{CLUST/density/figs/iris.cluster.e0.2-m5.txt}
\scalebox{0.6}{%
\psgraph[tickstyle=bottom,Ox=4,Oy=2,Dx=1,Dy=0.5]{->}(4,2)(7.5,4.5){3in}{2.5in}%
\listplot[plotstyle=dots,dotstyle=Bo,showpoints=true,nEnd=44]{\dataICa}
\listplot[plotstyle=dots,dotstyle=Bo,fillcolor=white,
  showpoints=true, nStart=45, nEnd=48]{\dataICa}
\listplot[plotstyle=dots,dotstyle=Btriangle,showpoints=true,
        nStart=49,nEnd=73]{\dataICa}
\listplot[plotstyle=dots,dotstyle=Btriangle,fillcolor=white,
  showpoints=true,nStart=74, nEnd=79]{\dataICa}
\listplot[plotstyle=dots,dotstyle=Bsquare,
      showpoints=true, nStart=80,nEnd=99]{\dataICa}
\listplot[plotstyle=dots,dotstyle=Bsquare,
  showpoints=true, nStart=100,nEnd=103]{\dataICa}
\listplot[plotstyle=dots,dotstyle=+,dotscale=1, showpoints=true,
        nStart=104]{\dataICa}
\endpsgraph
}}
\hspace{0.5in}
\subfloat[$\epsilon = 0.36$, $\minpts = 3$]{
\label{fig:clust:den:iris:dbscanB}
\readdata{\dataICb}{CLUST/density/figs/iris.cluster.e0.36-m3.txt}
\scalebox{0.6}{
\psgraph[tickstyle=bottom,Ox=4,Oy=2,Dx=1,Dy=0.5]{->}(4,2)(7.5,4.5){3in}{2.5in}%
\listplot[plotstyle=dots,dotstyle=Bo,showpoints=true,nEnd=94]{\dataICb}
\listplot[plotstyle=dots,dotstyle=Bo,fillcolor=white,
  showpoints=true, nStart=95, nEnd=97]{\dataICb}
\listplot[plotstyle=dots,dotstyle=Btriangle,showpoints=true,
        nStart=98,nEnd=144]{\dataICb}
\listplot[plotstyle=dots,dotstyle=Btriangle,fillcolor=white,
  showpoints=true,nStart=145, nEnd=146]{\dataICb}
\listplot[plotstyle=dots,dotstyle=+,dotscale=1, showpoints=true,
        nStart=147]{\dataICb}
\endpsgraph
}}}
\end{figure}
\end{frame}

\ifdefined\wox \begin{frame} \titlepage \end{frame} \fi

\begin{frame}{Kernel Density Estimation}

There is a close connection between density-based clustering and density
estimation. The
goal of density estimation is to determine the unknown probability
density \hbox{function} by f\/{i}nding the dense regions of points, which can in
turn be used for clustering. 

\medskip
Kernel density estimation is a
nonparametric technique that does not assume any f\/{i}xed probability
model of the clusters. Instead, it tries to
directly infer the underlying probability density at each point in the
dataset.
\end{frame}



\begin{frame}{Univariate Density Estimation}
Assume that $X$ is a continuous random variable, and let $x_1, x_2,
\ldots, x_n$ be a random sample.
We directly estimate the cumulative distribution function
from the data by counting how many points are less than or equal
to $x$:
\begin{align*}
  \hF(x)= {1\over n}  \sum_{i=1}^n I(x_i \leq x)
\end{align*}
where $I$ is an indicator function.

\medskip
We 
estimate the density function by taking the derivative of $\hF(x)$
\begin{align*}
  \hf(x) & =
  \frac{\hF\lB(x+\frac{h}{2}\rB)-\hF\lB(x-\frac{h}{2}\rB)}{h}
  = {{k/n} \over h} = {k \over nh}
\end{align*}
where $k$
is the number of points
that lie in the window of width $h$ centered at $x$.
The density estimate is the ratio of the
fraction of the points in the window ($k/n$) to the volume of the window
($h$). 
\end{frame}


\begin{frame}{Kernel Estimator}
Kernel density estimation relies on a {\em kernel
function} $K$
that is non-negative, symmetric, and integrates to $1$,
that is, $K(x) \ge 0$, $K(-x) = K(x)$
for all values $x$, and
${\int K(x) dx = 1}$.

\medskip
{\bf Discrete Kernel}
Define the {\bf discrete kernel} function $K$, that
computes the number of points
in a window of width $h$
\begin{align*}
\tcbhighmath{
  K(z) = \begin{cases}
    1 & \mbox{If $|z| \leq \frac{1}{2}$}\\
    0 & \mbox{Otherwise}\\
  \end{cases}
}
\end{align*}

The density estimate $\hf(x)$ can
be rewritten in terms
of the kernel function as follows:
\begin{align*}
\tcbhighmath{
    \hf(x) =  \frac{1}{nh}\,\sum_{i=1}^n\,
    K \left(\frac{x-x_i}{h}\right)
}
    %\label{eq:clust:den:denestK}
\end{align*}
\end{frame}



\readdata{\dataSL}{CLUST/density/figs/iris-sl.dat}
%\begin{figure}[!t]%fig15.5
%\vspace*{-12pt}
\readdata{\dataha}{CLUST/density/figs/iris1d-h0.25.dat}
\readdata{\datahb}{CLUST/density/figs/iris1d-h0.50.dat}
\readdata{\datahc}{CLUST/density/figs/iris1d-h1.00.dat}
\readdata{\datahd}{CLUST/density/figs/iris1d-h2.00.dat}
\begin{frame}[fragile]{Kernel Density Estimation}
\framesubtitle{Discrete Kernel (Iris 1D)}
\setcounter{subfigure}{0}
\begin{figure}
\vspace*{-0.7cm}
\psset{unit=0.6in}
\psset{arrowscale=2, dotscale=1.5}
\centerline{
\subfloat[$h=0.25$]{
\label{fig:clust:den:kde1dD:a}
\scalebox{0.60}{
\pstScalePoints(1,3){}{}
\begin{pspicture}(3,-0.5)(9,3.5)
\psaxes[Ox=4,Dy=0.33,dy=1]{->}(4,0)(8.5,2.5)[$x$,0][$f(x)$,90]
\listplot[plotstyle=dots,dotstyle=Bo,
        fillcolor=lightgray, showpoints = true]{\dataSL}
\listplot[linewidth=1.5pt]{\dataha}
\end{pspicture}
}}
\hspace{-0.2in}
\subfloat[$h=0.5$]{
\label{fig:clust:den:kde1dD:b}
\scalebox{0.60}{
\pstScalePoints(1,4.3){}{}
\begin{pspicture}(3,-0.5)(9,3.5)
\psaxes[Ox=4,Dy=0.22,dy=1]{->}(4,0)(8.5,2.5)[$x$,0][$f(x)$,90]
\listplot[plotstyle=dots,dotstyle=Bo,
        fillcolor=lightgray, showpoints = true]{\dataSL}
\listplot[linewidth=1.5pt]{\datahb}
\end{pspicture}
}}
}
\vspace*{-0.7cm}
\centerline{
\subfloat[$h=1.0$]{
\label{fig:clust:den:kde1dD:c}
\scalebox{0.60}{
\pstScalePoints(1,4.7){}{}
\begin{pspicture}(3,-0.5)(9,3.5)
\psaxes[Ox=4,Dy=0.21,dy=1]{->}(4,0)(8.5,2.5)[$x$,0][$f(x)$,90]
\listplot[plotstyle=dots,dotstyle=Bo,
        fillcolor=lightgray, showpoints = true]{\dataSL}
\listplot[linewidth=1.5pt]{\datahc}
\end{pspicture}
}}
\hspace{-0.2in}
\subfloat[$h=2.0$]{
\label{fig:clust:den:kde1dD:d}
\scalebox{0.60}{
\pstScalePoints(1,5.2){}{}
\begin{pspicture}(3,-0.5)(9,3.5)
\psaxes[Ox=4,Dy=0.2,dy=1]{->}(4,0)(8.5,2.5)[$x$,0][$f(x)$,90]
\listplot[plotstyle=dots,dotstyle=Bo,
        fillcolor=lightgray, showpoints = true]{\dataSL}
\listplot[linewidth=1.5pt]{\datahd}
\end{pspicture}
}}
}
%}
\end{figure}
The discrete
  kernel yields a non-smooth (or jagged) density function.

\end{frame}


\begin{frame}{Kernel Density Estimation}
\framesubtitle{Gaussian Kernel}
The width $h$ is a parameter that
denotes the spread or smoothness of the density estimate.
The discrete kernel function has abrupt changes. 

\medskip
Def\/{i}ne a more smooth
transition of influence via a Gaussian kernel:
\begin{align*}
  K\left(z\right) =
  \frac{1}{\sqrt{2\pi}}\,\exp\lB\{-\frac{z^2}{2}\rB\}
\end{align*}
Thus, we have
\begin{align*}
\tcbhighmath{
  K\left(\frac{x-x_i}{h}\right) =
  \frac{1}{\sqrt{2\pi}}\,\exp\lB\{-\frac{(x-x_i)^2}{2h^2}\rB\}
}
\end{align*}
Here $x$, which is at the center of the window, plays the role of
the mean, and $h$ acts as the standard deviation.
\end{frame}


\begin{frame}[fragile]{Kernel Density Estimation}
\framesubtitle{Gaussian Kernel (Iris 1D)}
\setcounter{subfigure}{0}
\vspace*{-0.7cm}
\begin{figure}[!t]%fig15.6
\psset{unit=0.6in}
\psset{arrowscale=2, dotscale=1.5}
\centerline{
\subfloat[$h=0.1$]{
\scalebox{0.55}{
\def\DyStep{0.27}
\def\myh{0.1}
\def\scaleF{3.7}
\input{CLUST/density/density-plot-1d-Gaussian}
}}
\hspace{-0.3in}
\subfloat[$h=0.15$]{
\scalebox{0.55}{
\def\DyStep{0.23}
\def\myh{0.15}
\def\scaleF{4.3}
\input{CLUST/density/density-plot-1d-Gaussian}
}}
}
\vspace*{-0.7cm}
\centerline{
\subfloat[$h=0.25$]{
\scalebox{0.55}{
\def\DyStep{0.2}
\def\myh{0.25}
\def\scaleF{5}
\input{CLUST/density/density-plot-1d-Gaussian}
}}
\hspace{-0.3in}
\subfloat[$h=0.5$]{
\scalebox{0.55}{
\def\DyStep{0.19}
\def\myh{0.5}
\def\scaleF{5}
\input{CLUST/density/density-plot-1d-Gaussian}
}}
}
\end{figure}
\small
  When $h$ is small
the density function has many
  local maxima. A large $h$ results in a unimodal distribution.

\end{frame}


\begin{frame}{Multivariate Density Estimation}
To estimate the probability density at a $d$-dimensional point $\bx =
(x_{1}, x_{2},\ldots, x_{d})^T$, we def\/{i}ne the $d$-dimensional ``window'' as a
hypercube in $d$ dimensions, that is, a \hbox{hypercube}
centered at $\bx$ with edge length $h$.
The volume of such a $d$-dimensional hypercube is given as
\begin{align*}
  \vol(H_d(h)) = h^d
\end{align*}

\medskip
The density is estimated as the fraction of the point weight
lying within the
$d$-dimensional window centered at $\bx$, divided by the volume of the
hypercube:
\begin{align*}
  \hf(\bx) = \frac{1}{n h^d} \sum_{i=1}^n
  K\left(\frac{\bx-\bx_i}{h}\right)
\end{align*}
where the multivariate kernel function $K$ satisf\/{i}es the condition
$\int K(\bz) d\bz = 1$.
\end{frame}


\begin{frame}{Multivariate Density Estimation}
\framesubtitle{Discrete and Gaussian Kernel}

{\bf Discrete Kernel:}
For any $d$-dimensional vector $\bz = (z_1,z_2, \ldots, z_d)^T$,
the discrete kernel function in $d$-dimensions is given as
\begin{align*}
\tcbhighmath{
  K(\bz) = \begin{cases}
    1 & \mbox{If $|z_{j}| \leq \frac{1}{2}$, for all dimensions
    $j=1, \ldots, d$}\\
    0 & \mbox{Otherwise}
  \end{cases}
}
\end{align*}

\medskip
{\bf Gaussian Kernel:}
The $d$-dimensional Gaussian kernel is
given as
\begin{align*}
  K\left(\bz\right) =
  \frac{1}{(2\pi)^{d/2}}\,\exp\lB\{-\frac{\bz^T\bz}{2}\rB\}
\end{align*}
\end{frame}


\begin{frame}[fragile]{Density Estimation}
\framesubtitle{Iris 2D Data (Gaussian Kernel)}
\setcounter{subfigure}{0}
\begin{figure}
\psset{unit=0.5in}
\psset{viewpoint=30 -120 30 rtp2xyz,Decran=50}
\psset{lightsrc=viewpoint,opacity=0.3,incolor=white}
\psset{dotstyle=Bo,fillcolor=gray}
\psset{dotsize=0.15}
\centerline{
\subfloat[$h=0.1$]{
\label{fig:clust:den:kde2dGa}
    \def\scaleF{4}
    \def\myh{0.1}
    \scalebox{0.4}{%
    \input{CLUST/density/density-plot-2d}
    }
}
\vspace{0.1in}
\subfloat[$h=0.2$]{
\label{fig:clust:den:kde2dGb}
    \def\scaleF{5}
    \def\myh{0.2}
    \scalebox{0.4}{%
    \input{CLUST/density/density-plot-2d}
    }
    }
    }
\centerline{
\subfloat[$h=0.35$]{
\label{fig:clust:den:kde2dGc}
    \def\scaleF{6.5}
    \def\myh{0.35}
    \scalebox{0.4}{%
    \input{CLUST/density/density-plot-2d}
    }
    }
    \vspace{0.1in}
\subfloat[$h=0.6$]{
\label{fig:clust:den:kde2dGd}
    \def\scaleF{7.5}
    \def\myh{0.6}
    \scalebox{0.4}{%
    \input{CLUST/density/density-plot-2d}
    }
    }
    }
\end{figure}
\end{frame}



\begin{frame}{Density Estimation}
\framesubtitle{Density-based Dataset}
  \framesubtitle{Gaussian kernel, $h=20$}
\setcounter{subfigure}{0}
\begin{figure}
\begin{center}
\hspace*{-1cm}
\subfloat[Original Points]{
\scalebox{0.45}{
\psset{unit=1in}
\psset{fillcolor=lightgray}
\psset{xAxisLabel=$X_1$, yAxisLabel=$X_2$}
  \hspace*{-60pt}
  \pspicture[](-2,-0.5)(2,3.5)
 \psgraph[Ox=0,Oy=20,Dx=100,Dy=75]{->}(0,20)(700,475){4in}{3in}
      \listplot[plotstyle=dots,dotstyle=Bo,
           plotNo=1,plotNoMax=2]{\dataT}
  \endpsgraph
  \endpspicture
  }
}
\hspace*{-2cm}
\subfloat[Gaussian Density Estimation]{
\psset{unit=0.5in}
\psset{viewpoint=70 90 91 rtp2xyz,Decran=65}
\psset{lightsrc=viewpoint}
\defFunction[algebraic]{myfunc}(u,v){u}{v}{((v^2)-(u^2))/4}
\scalebox{0.55}{
\begin{pspicture}(-1,-0.5)(7,6)
%\psSolid[object=surfaceparametree, action=writeobj,
%  file=CLUST/density/figs/mysurf,
%  ngrid=1.0 1.0,incolor=white, fillcolor=lightgray,
%  linewidth=0.5\pslinewidth,function=myfunc,base=0.0 7.0 0.0 5.0]
\psPoint(0,0,0){O2}
\psPoint(7.75,0,0){X2}
\psPoint(0,5.5,0){Y2}
\psline[arrows=->,arrowscale=2](O2)(X2)
\psline[arrows=->,arrowscale=2](O2)(Y2)
\uput[r](X2){$X_1$}
\uput[u](Y2){$X_2$}
%\psset{dotstyle=Bo,fillcolor=gray,linecolor=lightgray}
%\psset{dotsize=0.05}
%\input{CLUST/density/figs/t7-4k-pspoints3.0}
%\psset{fillcolor=white,linecolor=black}
\psset{fillcolor=gray!10, incolor=white, opacity=0.3}
\psSolid[object=objfile, file=CLUST/density/figs/t7-4k-h20surf,
  transform={1 1 1.75 scaleOpoint3d}, RotY=-4,
  linewidth=0.01pt,base=0 7.0 0 5.0]
\multido{\ix=0+1}{8}{%
    %\pstdivide{200pt}{100pt}\myx
    \pstmymultiply{\ix}{100}\myx
        \psPoint(\ix\space,-0.2,0){X1}
        \psPoint(\ix\space,0.0,0){X2}
        \psline(X1)(X2)
    \uput[d](X1){\scriptsize \myx}}
\multido{\iy=0+1}{6}{%
    \pstmymultiply{\iy}{100}\myy
        \psPoint(-0.2,\iy\space,0){Y1}
        \psPoint(0,\iy\space,0){Y2}
        \psline(Y1)(Y2)\uput[l](Y1){\scriptsize \myy}}
\end{pspicture}
	}
	}
\end{center}
\end{figure}
\end{frame}



\begin{frame}{Nearest Neighbor Density Estimation}

  In kernel density estimation we implicitly f\/{i}xed the
volume by f\/{i}xing the width $h$, and we used the
kernel function to f\/{i}nd out the number or weight of points that lie
inside the f\/{i}xed volume region. 

\medskip
An alternative approach to density
estimation is to f\/{i}x $k$, the number of points required to estimate the
density, and allow the volume of the enclosing region to vary to
accommodate those $k$ points. This
approach is called the $k$ nearest neighbors (KNN) approach to density
estimation. 

\medskip
Given $k$, the number of neighbors, we estimate the
density at $\bx$ as follows:
\begin{align*}
  \hf(\bx) = \frac{k}{n \vol(S_d(h_\bx))}
\end{align*}
where $h_\bx$ is the distance from $\bx$ to its $k$th nearest
neighbor, and
$\vol(S_d(h_\bx))$ is the volume of the
$d$-dimensional hypersphere $S_d(h_\bx)$ centered at $\bx$,
with radius $h_\bx$.
\end{frame}


\ifdefined\wox \begin{frame} \titlepage \end{frame} \fi

\begin{frame}{DENCLUE Density-based Clustering}
\framesubtitle{Attractor and Gradient}
A point $\bx^*$ is called a {\em density attractor} if it is a
local maxima of the probability density function $f$.  

\medskip
The density gradient at a point $\bx$ is the multivariate
derivative of the probability density estimate
\begin{align*}
    \grad \hf(\bx) = {\partial \over \partial \bx} \hf(\bx) =
     {1 \over n h^d}
     \sum_{i=1}^n {\partial \over \partial \bx}
     K\lB(\frac{\bx-\bx_i}{h}\rB)
\end{align*}

\medskip
For the Gaussian kernel the gradient at a point $\bx$ is given as
 \begin{align*}
\tcbhighmath{
     \grad \hf(\bx)  = {1 \over n h^{d+2}} \sum_{i=1}^n
        K\left(\frac{\bx-\bx_i}{h}\right)\cdot (\bx_i - \bx)
}
\end{align*}
This equation can be thought of as having two parts for each point: a
vector $(\bx_i-\bx)$ and a scalar {\em influence} value
$K(\frac{\bx-\bx_i}{h})$.
\end{frame}



\begin{frame}{The Gradient Vector}


We f\/{i}rst compute the
direction away from $\bx$ to $\bx_i$, i.e., the vector $(\bx_i - \bx)$. 

Next,
we scale it using the Gaussian kernel value
as the weight $K\left(\frac{\bx-\bx_i}{h}\right)$.

$\grad \hf(\bx)$ is the net influence at
$\bx$, i.e., the weighted sum of the difference vectors.


\begin{figure}
%\begin{figure}[!ht]%fig15.9
  \centerline{
	  \scalebox{0.8}{
    \psset{yunit=0.5in,xunit=0.5in,arrows=->, arrowscale=2}
    \pspicture[](0,0)(6,4)
    \psaxes{->}(0,0)(0,0)(6,4)
    \qdisk(1,1){3pt}
    \qdisk(5,1){3pt}
    \qdisk(3,3){3pt}
    \qdisk(1,3){3pt}
    \uput[dl](1,1){$\bx$}
    \uput[r](5,1){$\bx_1$}
    \uput[ur](3,3){$\bx_2$}
    \uput[u](1,3){$\bx_3$}
    \psline[](1,1)(5,1)
    \psline[](1,1)(1,3)
    \psline[](1,1)(3,3)
    \psline[linecolor=gray,linewidth=2pt](1,1)(1.5,1)
    \psline[linecolor=gray,linewidth=2pt](1,1)(2,2)
    \psline[linecolor=gray,linewidth=2pt](1,1)(1,2.75)
    \psline[linewidth=2pt](1,1)(2,2.75)
    \uput[u](2,2.75){$\grad \hf(\bx)$}
    \endpspicture
    }
  }
\end{figure}
\end{frame}


\begin{frame}{DENCLUE: Density Attractor}
We say that $\bx^*$ is a {\em density attractor} for $\bx$, or
alternatively that $\bx$ is {\em density attracted} to
$\bx^*$, if a hill climbing process started at $\bx$
converges to $\bx^*$.

\medskip
That is, there exists a sequence of points
$\bx=\bx_0 \rightarrow \bx_1 \rightarrow \ldots \rightarrow
\bx_m$,
starting from $\bx$ and ending at $\bx_m$,
such that $\|\bx_m - \bx^*\| \leq \epsilon$,
that is, $\bx_m$ converges to the attractor $\bx^*$.

\medskip
Setting the gradient to the zero vector leads to the following {\em
mean-shift} update rule:
\begin{empheq}[box=\tcbhighmath]{align*}
\begin{split}
  \bx_{t+1} & = \frac{\sum_{i=1}^n
        K\left(\frac{\bx_t-\bx_i}{h}\right) \bx_i}
        {\sum_{i=1}^n K\left(\frac{\bx_t-\bx_i}{h}\right)}
\end{split}
\end{empheq}
where $t$ denotes the current iteration and $\bx_{t+1}$ is the updated value for the current vector $\bx_t$.  
\end{frame}

\newcommand{\gradientascent}{\textsc{F{i}ndAttractor}\xspace}
\begin{frame}[fragile]{The DENCLUE Algorithm: Find Attractor}
\begin{tightalgo}[H]{\textwidth-18pt}
 \SetKwInput{AlgorithmGA}{\gradientascent ($\bx, \bD, h, \epsilon$)}
  \AlgorithmGA{}
  \lnl{alg:clust:den:denclue:FAs}$t \assign 0$\;
  $\bx_t \assign \bx$\;
  \Repeat{$\norm{\bx_t - \bx_{t-1}} \le \epsilon$}{%
    \lnl{alg:clust:den:denclue:DU}
    $\bx_{t+1} \assign \frac{\sum_{i=1}^n
      K\lB({\bx_t - \bx_i \over h}\rB) \cdot \bx_t}
      {\sum_{i=1}^n K\lB({\bx_t - \bx_i \over h}\rB)}$\;
      $t \assign t+1$\;
  }
  \lnl{alg:clust:den:denclue:FAe}\Return{$\bx_t$}
\end{tightalgo}
\end{frame}


\begin{frame}{DENCLUE: Density-based Cluster}
A cluster $C \subseteq \bD$, is called a {\em center-def\/{i}ned
cluster} if all the points $\bx \in C$ are
density attracted to a unique density attractor $\bx^*$, such that
$\hf(\bx^*) \geq \xi$, where $\xi$ is a user-def\/{i}ned minimum
density threshold.  

\medskip
An arbitrary-shaped cluster $C \subseteq \bD$ is called a {\em
density-based cluster} if there exists a set of density attractors
$\bx_1^*, \bx_2^*, \ldots, \bx_m^*$, such that
\begin{enumerate}
\item Each point $\bx \in C$ is attracted to some attractor $\bx_i^*$.
\item Each density attractor has density above $\xi$.
\item Any two density attractors $\bx_i^*$ and  $\bx_{j}^*$ are {\em
  density reachable}, that is, there exists a path from
  $\bx_i^*$ to $\bx_{j}^*$, such that for all points $\by$ on the path,
  $\hf(\by) \geq \xi$.
\end{enumerate}
\end{frame}



\newcommand{\denclue}{\textsc{denclue}\xspace}
\begin{frame}[fragile]{The DENCLUE Algorithm}
\begin{tightalgo}[H]{\textwidth-18pt}
  \SetKwInput{Algorithm}{\denclue ($\bD, h, \xi, \epsilon$)}
  \Algorithm{}
  $\cA \assign \emptyset$\;
  \ForEach(\tcp*[h]{f\/{i}nd density attractors}){$\bx \in \bD$}{
    \lnl{alg:clust:den:denclue:fa}
    $\bx^* \assign \text{\gradientascent}(\bx, \bD, h,\epsilon)$\;
    \If{$\hf(\bx^*) \ge \xi$}{
    \lnl{alg:clust:den:denclue:A}
      $\cA \assign \cA \cup \{\bx^*\}$\;
      \lnl{alg:clust:den:denclue:R}
      $R(\bx^*) \assign R(\bx^*) \cup \{\bx\}$\;
    }
  }
  \lnl{alg:clust:den:denclue:cC}
  $\cC \assign \{\text {maximal } C\subseteq \cA \mid\; \forall \bx_i^*, \bx_{j}^* \in C,
    \bx_i^* \text{ and } \bx_{j}^* \text { are density
    reachable}\}$\;
  \ForEach(\tcp*[h]{density-based clusters}){$C \in \cC$}{
    \lForEach{$\bx^* \in C$}{$C \assign C \cup R(\bx^*)$}
  }
  \Return{$\cC$}
\end{tightalgo}
\end{frame}


\begin{frame}{DENCLUE: Iris 2D Data}

Iris 2D dataset comprising the {\tt sepal length}
  and {\tt sepal width} attributes. 

The results were obtained with
  $h=0.2$ and $\xi=0.08$, using a Gaussian
  kernel.

\begin{figure}
\psset{viewpoint=40 -135 25 rtp2xyz,Decran=50}
\psset{lightsrc=viewpoint}
\psset{incolor=white}
\psset{opacity=0.2}
\psset{fillcolor=gray!10}
\begin{pspicture}(-1,1)(7,7)
\psSolid[object=objfile, file=CLUST/density/figs/iris-h0.2surf,
  linewidth=0.01pt,base=3.5 8.5 1.0 5.0]
\psset{opacity=0.5}
\psSolid[object=objfile, file=CLUST/density/figs/iris-h0.2xi0.8surf,
  linewidth=0.01pt,base=3.5 8.5 1.0 5.0]
\psPoint(3.5,1.0,0){O2}
\psPoint(8.5,1,0){X2}
\psPoint(3.5,5.0,0){Y2}
\psPoint(3.5,1.0,1.5){Z2}
\psPoint(3.5,1.0,0.8){xi}
\psline[arrows=->,arrowscale=1.5](O2)(X2)
\psline[arrows=->,arrowscale=1.5](O2)(Y2)
\psline[arrows=->,arrowscale=1.5](O2)(Z2)
\uput[r](X2){$X_1$}
\uput[ul](Y2){$X_2$}
\uput[u](Z2){$f(\bx)$}
%\uput[r](xi){$\xi=0.08$}
\multido{\nx=3.5+1.0}{5}{%
        \psPoint(\nx\space,0.8,0){X1}
        \psPoint(\nx\space,1.0,0){X2}
        \psline(X1)(X2)
    \uput[d](X1){\scriptsize \nx}}
\multido{\ny=1+1}{4}{%
        \psPoint(3.3,\ny\space,0){Y1}
        \psPoint(3.5,\ny\space,0){Y2}
        \psline(Y1)(Y2)\uput[l](Y1){\scriptsize \ny}}
\end{pspicture}
\end{figure}
\end{frame}


\begin{frame}{DENCLUE: Density-based Dataset}
Using the parameters $h=10$ and
  $\xi=9.5 \times 10^{-5}$, with a Gaussian kernel, we obtain eight
  clusters.

\vspace*{-2cm}

\begin{figure}
\psset{unit=0.5in}
\psset{viewpoint=70 90 91 rtp2xyz,Decran=65}
\psset{lightsrc=viewpoint}
\psset{incolor=white}
\psset{opacity=0.15}
\begin{pspicture}(-1,-0.5)(7,6)
	\scalebox{0.85}{
\psset{fillcolor=white}
\psSolid[object=objfile, file=CLUST/density/figs/t7-4k-h10-xi1.9surf,
  transform={1 1 1.75 scaleOpoint3d},
  linewidth=0.01pt,base=0 7.0 0 5.0]
\psPoint(0,0,0){O2}
\psPoint(7.5,0,0){X2}
\psPoint(0,5.5,0){Y2}
\psline[arrows=->,arrowscale=2](O2)(X2)
\psline[arrows=->,arrowscale=2](O2)(Y2)
\uput[r](X2){$X_1$}
\uput[u](Y2){$X_2$}
\psset{dotstyle=Bo,fillcolor=gray,linecolor=lightgray}
\psset{dotsize=0.05}
\input{CLUST/density/figs/t7-4k-pspoints3.0}
\psset{fillcolor=white,linecolor=black}
\multido{\ix=0+1}{8}{%
    \pstmymultiply{\ix}{100}\myx
        \psPoint(\ix\space,-0.2,0){X1}
        \psPoint(\ix\space,0.0,0){X2}
        \psline(X1)(X2)
    \uput[d](X1){\scriptsize \myx}}
\multido{\iy=0+1}{6}{%
    \pstmymultiply{\iy}{100}\myy
        \psPoint(-0.2,\iy\space,0){Y1}
        \psPoint(0,\iy\space,0){Y2}
        \psline(Y1)(Y2)\uput[l](Y1){\scriptsize \myy}}
	}
\end{pspicture}
\end{figure}
\end{frame}
