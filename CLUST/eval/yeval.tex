\lecture{eval}{eval}

\date{Chapter 17: Clustering Validation}
\newcommand{\NMI}{\mathit{NMI}}
\newcommand{\VI}{\mathit{VI}}

\begin{frame}
\titlepage
\end{frame}


\begin{frame}{Clustering Validation and Evaluation}
Cluster validation and assessment encompasses three main tasks:
{\em clustering
evaluation} seeks to assess the goodness or quality of the clustering,
{\em clustering stability} seeks to understand the sensitivity of the clustering result to various algorithmic parameters,
for example, the number of clusters,
and {\em clustering tendency} assesses
the suitability of applying clustering in the f\/{i}rst place, that is, whether
the data has any inherent grouping structure.

\bigskip
Validity measures can be divided into three main types:

\begin{description}

\item [External:] External validation measures employ criteria that are
  not inherent to the dataset, e.g., class labels. 

\item [Internal:] Internal validation measures employ criteria that are
  derived from the data itself, e.g., intracluster
  and intercluster distances.

\item [Relative:] Relative validation measures aim to directly compare
  different clusterings, usually those
  obtained via different parameter settings for the same algorithm.
\end{description}
\end{frame}



\begin{frame}{External Measures}
External measures assume that the correct or
ground-truth clustering is known {\it a priori}, which 
is used to evaluate a
given clustering.  

\medskip
Let $\bD = \{\bx_i\}_{i=1}^n$ be a dataset consisting of $n$ points in a
$d$-dimensional space, partitioned into $k$ clusters.  Let $y_i \in
\left\{ 1, 2,\ldots , k \right\}$ denote the ground-truth cluster
membership or label information for each point.  

\medskip
The ground-truth
clustering is given as 
$\cT = \left\{ T_1, T_2, \ldots, T_k \right\}$,
where the cluster $T_{j}$ consists of all the points with label $j$, 
i.e.,
$T_{j} = \left\{ \bx_i \in \bD | y_i = j \right\}$.  
We refer to $\cT$ as the ground-truth {\em
partitioning}, and to each $T_i$ as a {\em partition}.



\medskip
Let $\cC=\{C_1,
\ldots, C_r\}$ denote a clustering of the same dataset into $r$
clusters, obtained via some clustering algorithm, and let $\hy_i \in
\left\{ 1, 2, \ldots, r \right\}$ denote the cluster label for $\bx_i$.
\end{frame}

\begin{frame}{External Measures}
External evaluation measures try
capture the extent to which points from the same partition appear in the
same cluster, and the extent to which points from different partitions
are grouped in different clusters. 

\medskip
All of the external measures rely on the $r\times k$
{\em contingency table} $\bN$ that is
induced by a clustering $\cC$ and the ground-truth partitioning $\cT$, def\/{i}ned as follows
\begin{align*}
  \bN(i,j) = n_{ij}  = \left| C_i \cap T_{j} \right|
\end{align*}
The count $n_{ij}$ denotes the number of points that are
common to cluster $C_i$ and ground-truth partition $T_{j}$.

\medskip
Let $n_{i} = |C_i|$ denote the number of points in cluster
$C_i$, and let $m_{j} = |T_{j}|$ denote the number of points in partition
$T_{j}$.  

\medskip
The contingency table can be computed from $\cT$ and $\cC$ in
$O(n)$ time by examining the partition and cluster labels, $y_i$ and
$\hy_i$, for each point $\bx_i \in \bD$ and incrementing the
corresponding count $n_{y_i\hy_i}$.
\end{frame}


\begin{frame}{Matching Based Measures: Purity}
Purity quantif\/{i}es the extent to which a
cluster $C_i$ contains entities from only one partition: 
\begin{align*}
  \mathit{purity}_i = \frac{1}{n_{i}}\max_{j=1}^k\; \{n_{ij}\}
\end{align*}

\bigskip
The purity of clustering $\cC$ is def\/{i}ned as the weighted sum of the
clusterwise purity values:
\begin{align*}
\tcbhighmath{
  \mathit{purity} = \sum_{i=1}^r \frac{n_{i}}{n} \mathit{purity}_i =
  \frac{1}{n}\sum_{i=1}^r \max_{j=1}^k \{n_{ij}\}
}
\end{align*}
where the ratio $\tfrac{n_{i}}{n}$ denotes the fraction of points in
cluster $C_i$.  
\end{frame}


\begin{frame}{Matching Based Measures: Maximum Matching}
The maximum matching
measure selects the mapping between clusters and partitions, such
that the sum of the number of common points ($n_{ij}$) is
maximized, provided that only one cluster can match with a given
partition.

\medskip
Let $G$ be a bipartite graph over the vertex set $V = \cC \cup \cT$, and
let the edge set be $E = \{ (C_i, T_{j}) \}$ with 
edge weights $w(C_i,T_{j}) = n_{ij}$.
A {\em matching} $M$ in $G$ is a subset of $E$, such that the edges in
$M$ are pairwise nonadjacent, that is, they do not have a common vertex.

\medskip
The {\em maximum weight matching} in $G$ is given as:
\begin{align*}
  \mathit{match} = \arg \max_M \lB\{ \frac{w(M)}{n} \rB\}
\end{align*}
where $w(M)$ is the sum of the sum of
all the edge weights in matching $M$, given as $w(M) = \sum_{e \in M} w(e)$
\end{frame}




\begin{frame}{Matching Based Measures: F-measure}
  \small
Given cluster $C_i$, let $j_i$ denote the
partition that contains the maximum number of points from $C_i$,
that is, $j_i = \max_{j=1}^k \{ n_{ij} \}$.  

\medskip
The {\em precision} of a
cluster $C_i$ is the same as its purity:
\begin{align*}
\tcbhighmath{
  \mathit{prec}_i = \frac{1}{n_{i}}\max_{j=1}^k \left\{ n_{ij} \right\} =
  \frac{n_{ij_i}}{n_i}
}
\end{align*}

\medskip
The {\em recall} of cluster $C_i$ is def\/{i}ned as
\begin{align*}
\tcbhighmath{
  \mathit{recall}_i = \frac{n_{ij_i}}{\card{T_{j_i}}} =\frac{n_{ij_i}}{m_{j_i}}
}
 \end{align*}
 where $m_{j_i} = \card{T_{j_i}}$.

\medskip
\end{frame}

\begin{frame}{Matching Based Measures: F-measure}
The F-measure is the harmonic mean of the precision and recall values for
each $C_i$
\begin{align*}
\tcbhighmath{
  F_i = \frac{2}{\frac{1}{\mathit{prec}_i} + \frac{1}{\mathit{recall}_i}} =
  \frac{2 \cdot \mathit{prec}_i \cdot \mathit{recall}_i}{\mathit{prec}_i + \mathit{recall}_i} =
  \frac{2 \; n_{ij_i}}{n_{i} + m_{j_i}}
}
\end{align*}

\medskip
The F-measure for the clustering $\cC$ is the mean of clusterwise
F-meaure values:
\begin{align*}
  F = \frac{1}{r} \sum_{i=1}^r F_i
\end{align*}
\end{frame}




\begin{frame}[fragile]{K-means: Iris Principal Components Data}
\framesubtitle{Good Case}
\setcounter{subfigure}{0}
\begin{figure}
    \centering
    \captionsetup[subfloat]{captionskip=20pt}
    \def\pshlabel#1{ {\footnotesize $#1$}}
    \def\psvlabel#1{ {\footnotesize $#1$}}
    \psset{xAxisLabel=$\bu_1$,yAxisLabel= $\bu_2$}
    \psset{xunit=0.5in,yunit=0.65in,dotscale=1.5,arrowscale=2,PointName=none}
    \centerline{
	\scalebox{0.6}{
    %\subfloat[K-means: good]{\label{fig:clust:eval:kexgood}
        %\pspicture[](-5,-2)(3.5,1.5)
        %\psaxes[tickstyle=bottom,Dx=1,Ox=-4,Dy=0.5,Oy=-1.5]{->}(-4,-1.5)(3.5,1.5)
        \psgraph[tickstyle=bottom,Dx=1,Ox=-4,Dy=0.5,Oy=-1.5]{->}%
        (-4,-1.5)(3.5,1.5){3.5in}{2in}
        \psset{dotstyle=Bsquare,fillcolor=lightgray}
        \input{CLUST/eval/irisPCgood-C1}
        \psset{fillcolor=white}
        \input{CLUST/eval/irisPCgood-W1}
        \psset{dotstyle=Bo,fillcolor=lightgray}
        \input{CLUST/eval/irisPCgood-C2}
        \psset{fillcolor=white}
        \input{CLUST/eval/irisPCgood-W2}
        \psset{dotstyle=Btriangle,fillcolor=lightgray}
        \input{CLUST/eval/irisPCgood-C3}
        \psset{fillcolor=white}
        \input{CLUST/eval/irisPCgood-W3}
        \psset{fillcolor=black}
        \pstGeonode[PointSymbol=Bo, dotscale=2](2.64,0.19){A}
        \pstGeonode[PointSymbol=Btriangle,
        dotscale=2](-2.35,0.27){B}
        \pstGeonode[PointSymbol=Bsquare,
        dotscale=2](-0.66,-0.33){C}
        \psclip{\psframe[](-4,-1.5)(3.5,1.5)}%
        {
        \psset{linestyle=none, PointSymbol=none}
        \pstMediatorAB{A}{B}{K}{KP}
        \pstMediatorAB{C}{A}{J}{JP}
        \pstMediatorAB{B}{C}{I}{IP}
        \pstInterLL[PointSymbol=none]{I}{IP}{J}{JP}{O}
        \psset{linewidth=1pt,linestyle=dashed}
        \pstGeonode[PointSymbol=none](-4,-1.5){a}(-4,1.5){b}(3.5,1.5){c}(3.5,-1.5){d}
        \pstInterLL[PointSymbol=none]{O}{I}{a}{b}{oi}
        \pstLineAB{O}{oi}
        \pstInterLL[PointSymbol=none]{O}{J}{a}{d}{oj}
        \pstLineAB{O}{oj}
        \pstInterLL[PointSymbol=none]{O}{K}{b}{c}{ok}
        \pstLineAB{O}{ok}
        }
        \endpsclip
        %\endpspicture
        \endpsgraph
    }}
  \end{figure}
    
Contingency table:
\small
\begin{align*}
  \begin{array}{c|ccc|c}
  & 
  \texttt{iris-setosa} & \texttt{iris-versicolor}
  & \texttt{iris-virginica} &\\
  &  T_1 & T_2 & T_3 & n_i\\
  \hline
  C_1 \text{(squares)} & 0 & 47 & 14 & 61\\
  C_2 \text{(circles)} & 50 &  0 &  0 & 50\\
  C_3 \text{(triangles)} & 0 &  3 & 36 &  39\\
\hline
m_{j} & 50 & 50 & 50 & n=100
\end{array}
\end{align*}
\normalsize
$\mathit{purity} = 0.887$, $\mathit{match} = 0.887$, $F = 0.885$.
\end{frame}

\begin{frame}[fragile]{K-means: Iris Principal Components Data}
\framesubtitle{Bad Case}
\begin{figure}
    \captionsetup[subfloat]{captionskip=20pt}
    \def\pshlabel#1{ {\footnotesize $#1$}}
    \def\psvlabel#1{ {\footnotesize $#1$}}
    \psset{xAxisLabel=$\bu_1$,yAxisLabel= $\bu_2$}
    \psset{xunit=0.5in,yunit=0.65in,dotscale=1.5,arrowscale=2,PointName=none}
    \centerline{
	\scalebox{0.6}{
        %\pspicture[](-5,-2)(3.5,1.5)
        %\psaxes[tickstyle=bottom,Dx=1,Ox=-4,Dy=0.5,Oy=-1.5]{->}(-4,-1.5)(3.5,1.5)
        \psgraph[tickstyle=bottom,Dx=1,Ox=-4,Dy=0.5,Oy=-1.5]{->}%
        (-4,-1.5)(3.5,1.5){3.5in}{2in}
        \psset{dotstyle=Bsquare,fillcolor=lightgray}
        \input{CLUST/eval/irisPCbad-C1}
        \psset{fillcolor=white}
        \input{CLUST/eval/irisPCbad-W1}
        \psset{dotstyle=Bo,fillcolor=lightgray}
        \input{CLUST/eval/irisPCbad-C2}
        \psset{fillcolor=white}
        \input{CLUST/eval/irisPCbad-W2}
        \psset{dotstyle=Btriangle,fillcolor=lightgray}
        \input{CLUST/eval/irisPCbad-C3}
        \psset{fillcolor=white}
        \input{CLUST/eval/irisPCbad-W3}
        \psset{fillcolor=black}
        \pstGeonode[PointSymbol=Bsquare, dotscale=2](2.562,0.486){A}
        \pstGeonode[PointSymbol=Bo,
        dotscale=2](2.419,-0.379){B}
        \pstGeonode[PointSymbol=Btriangle,
        dotscale=2](-1.405,-0.057){C}
        \psclip{\psframe[](-4,-1.5)(3.5,1.5)}%
        {
        \psset{linestyle=none, PointSymbol=none}
        \pstMediatorAB{A}{B}{K}{KP}
        \pstMediatorAB{C}{A}{J}{JP}
        \pstMediatorAB{B}{C}{I}{IP}
        \pstInterLL[PointSymbol=none]{I}{IP}{J}{JP}{O}
        \psset{linewidth=1pt,linestyle=dashed}
        \pstGeonode[PointSymbol=none](-4,-1.5){a}(-4,1.5){b}(3.5,1.5){c}(3.5,-1.5){d}
        \pstInterLL[PointSymbol=none]{O}{I}{a}{b}{oi}
        \pstLineAB{O}{oi}
        \pstInterLL[PointSymbol=none]{O}{J}{b}{c}{oj}
        \pstLineAB{O}{oj}
        \pstInterLL[PointSymbol=none]{O}{K}{a}{d}{ok}
        \pstLineAB{O}{ok}
        }
        \endpsclip
        %\endpspicture
        \endpsgraph
    }}
\end{figure}

Contingency table:
  \small
\begin{align*}
  \begin{array}{c|ccc|c}
  & \texttt{iris-setosa} & \texttt{iris-versicolor}
  & \texttt{iris-virginica} & \\
  & T_1 & T_2 & T_3  & n_i\\
  \hline
C_1 (squares) & 30 & 0 & 0 & 30\\
C_2 (circles) & 20 &  4 &  0 & 24\\
C_3 (triangles) & 0 &  46 & 50 & 96\\
\hline
m_{j} &50 & 50 & 50 &n=150\\
  \end{array}
\end{align*}
\normalsize
$\mathit{purity} = 0.667$, $\mathit{match} = 0.560$, $F=0.658$
\end{frame}



\begin{frame}{Entropy-based Measures: Conditional Entropy}
The entropy of a clustering $\cC$  and partitioning $\cT$ is given as
\begin{align*}
  H(\cC) & = - \sum_{i=1}^r p_{C_i} \log p_{C_i} &
  H(\cT) & = - \sum_{j=1}^k p_{T_{j}} \log p_{T_{j}}
\end{align*}
where $p_{C_i} = \tfrac{n_i}{n}$ and $p_{T_{j}} = \tfrac{m_{j}}{n}$
are the probabilities of cluster $C_i$ and partition $T_{j}$.

\bigskip
The cluster-specif\/{i}c entropy of $\cT$, that is, the conditional entropy of
$\cT$ with respect to cluster $C_i$ is def\/{i}ned as
\begin{align*}
  H(\cT|C_i) = - \sum_{j=1}^k
  \lB( \frac{n_{ij}}{n_i} \rB)
  \log \lB( \frac{n_{ij}}{n_i} \rB)
\end{align*}
\end{frame}

\begin{frame}{Entropy-based Measures: Conditional Entropy}
The conditional entropy of $\cT$ given clustering $\cC$ is def\/{i}ned
as the weighted sum:
\begin{align*}
  H(\cT|\cC) & = \sum_{i=1}^r \frac{n_{i}}{n} H(\cT|C_i)
  = 
\tcbhighmath{
-\sum_{i=1}^r \sum_{j=1}^k p_{ij} \log \lB(
  \frac{p_{ij}}{p_{C_i}}\rB)
}
\\
  & = H(\cC,\cT) - H(\cC)
\end{align*}
where $p_{ij}= \tfrac{n_{ij}}{n}$ is the probability that a point in
cluster $i$ also belongs to partition and 
where $H(\cC,\cT) = - \sum_{i=1}^r \sum_{j=1}^k p_{ij} \log p_{ij}$ is
the joint entropy of $\cC$ and $\cT$.  

\bigskip
$H(\cT|\cC) = 0$ if and only if
$\cT$ is completely determined by $\cC$, corresponding to the ideal
clustering.
If $\cC$ and $\cT$ are independent of
each other, then $H(\cT|\cC) = H(\cT)$.
\end{frame}


\begin{frame}{Entropy-based Measures: Normalized Mutual Information} 
The {\em mutual
information} tries to quantify the amount of shared information
between the clustering $\cC$ and partitioning $\cT$, and it is
def\/{i}ned as
\begin{align*}
\tcbhighmath{
  I(\cC,\cT) = \sum_{i=1}^r \sum_{j=1}^k p_{ij} \log
  \lB(\frac{p_{ij}}{p_{C_i} \cdot p_{T_{j}}} \rB)
}
\end{align*}
When $\cC$ and $\cT$
are independent then $p_{ij} = p_{C_i} \cdot p_{T_{j}}$, and thus
$I(\cC,\cT) = 0$. However, there is no upper bound on the mutual
information.


\bigskip
The {\em normalized mutual information} (NMI) is def\/{i}ned as
the geometric mean:
\begin{align*}
\tcbhighmath{
  \NMI(\cC,\cT) = \sqrt{\frac{I(\cC,\cT)}{H(\cC)} \cdot \frac{I(\cC,\cT)}{H(\cT)}} =
  \frac{I(\cC,\cT)}{\sqrt{H(\cC)\cdot H(\cT)}}
}
\end{align*}
The NMI value lies in the range $[0,1]$. Values close to $1$ indicate a
good clustering.
\end{frame}




\begin{frame}{Entropy-based Measures: Variation of Information}
This criterion is based on the mutual information
between the clustering $\cC$ and the ground-truth partitioning
$\cT$, and their entropy; it is def\/{i}ned as
\begin{align*}
  \VI(\cC,\cT) &= (H(\cT)-I(\cC,\cT)) + (H(\cC)-I(\cC,\cT))\nonumber\\
    & = H(\cT)+H(\cC)-2I(\cC,\cT)
\end{align*}
Variation of information (VI) is zero only when $\cC$ and $\cT$ are identical. Thus, the
lower the VI value the better the clustering $\cC$.

\bigskip
VI can also be expressed as:
\begin{align*}
  \VI(\cC,\cT) & = H(\cT|\cC) + H(\cC|\cT)\\
\end{align*}
\vspace*{-1.0cm}
\begin{empheq}[box=\tcbhighmath]{align*}
\begin{split}
  \VI(\cC,\cT) & = 2H(\cT,\cC)-H(\cT)-H(\cC)
\end{split}
\end{empheq}
\end{frame}


\begin{frame}[fragile]{K-means: Iris Principal Components Data}
\framesubtitle{Good Case}

\setcounter{subfigure}{0}
\begin{figure}
    \centering
    \captionsetup[subfloat]{captionskip=20pt}
    \def\pshlabel#1{ {\footnotesize $#1$}}
    \def\psvlabel#1{ {\footnotesize $#1$}}
    \psset{xAxisLabel=$\bu_1$,yAxisLabel= $\bu_2$}
    \psset{xunit=0.5in,yunit=0.65in,dotscale=1.5,arrowscale=2,PointName=none}
    \centerline{
    \subfloat[K-means: good]{
	\scalebox{0.6}{
        %\pspicture[](-5,-2)(3.5,1.5)
        %\psaxes[tickstyle=bottom,Dx=1,Ox=-4,Dy=0.5,Oy=-1.5]{->}(-4,-1.5)(3.5,1.5)
        \psgraph[tickstyle=bottom,Dx=1,Ox=-4,Dy=0.5,Oy=-1.5]{->}%
        (-4,-1.5)(3.5,1.5){3.5in}{2in}
        \psset{dotstyle=Bsquare,fillcolor=lightgray}
        \input{CLUST/eval/irisPCgood-C1}
        \psset{fillcolor=white}
        \input{CLUST/eval/irisPCgood-W1}
        \psset{dotstyle=Bo,fillcolor=lightgray}
        \input{CLUST/eval/irisPCgood-C2}
        \psset{fillcolor=white}
        \input{CLUST/eval/irisPCgood-W2}
        \psset{dotstyle=Btriangle,fillcolor=lightgray}
        \input{CLUST/eval/irisPCgood-C3}
        \psset{fillcolor=white}
        \input{CLUST/eval/irisPCgood-W3}
        \psset{fillcolor=black}
        \pstGeonode[PointSymbol=Bo, dotscale=2](2.64,0.19){A}
        \pstGeonode[PointSymbol=Btriangle,
        dotscale=2](-2.35,0.27){B}
        \pstGeonode[PointSymbol=Bsquare,
        dotscale=2](-0.66,-0.33){C}
        \psclip{\psframe[](-4,-1.5)(3.5,1.5)}%
        {
        \psset{linestyle=none, PointSymbol=none}
        \pstMediatorAB{A}{B}{K}{KP}
        \pstMediatorAB{C}{A}{J}{JP}
        \pstMediatorAB{B}{C}{I}{IP}
        \pstInterLL[PointSymbol=none]{I}{IP}{J}{JP}{O}
        \psset{linewidth=1pt,linestyle=dashed}
        \pstGeonode[PointSymbol=none](-4,-1.5){a}(-4,1.5){b}(3.5,1.5){c}(3.5,-1.5){d}
        \pstInterLL[PointSymbol=none]{O}{I}{a}{b}{oi}
        \pstLineAB{O}{oi}
        \pstInterLL[PointSymbol=none]{O}{J}{a}{d}{oj}
        \pstLineAB{O}{oj}
        \pstInterLL[PointSymbol=none]{O}{K}{b}{c}{ok}
        \pstLineAB{O}{ok}
        }
        \endpsclip
        %\endpspicture
        \endpsgraph
		}} \hspace{0.25in}
		\subfloat[K-means: bad]{
	  \scalebox{0.6}{
        %\pspicture[](-5,-2)(3.5,1.5)
        %\psaxes[tickstyle=bottom,Dx=1,Ox=-4,Dy=0.5,Oy=-1.5]{->}(-4,-1.5)(3.5,1.5)
        \psgraph[tickstyle=bottom,Dx=1,Ox=-4,Dy=0.5,Oy=-1.5]{->}%
        (-4,-1.5)(3.5,1.5){3.5in}{2in}
        \psset{dotstyle=Bsquare,fillcolor=lightgray}
        \input{CLUST/eval/irisPCbad-C1}
        \psset{fillcolor=white}
        \input{CLUST/eval/irisPCbad-W1}
        \psset{dotstyle=Bo,fillcolor=lightgray}
        \input{CLUST/eval/irisPCbad-C2}
        \psset{fillcolor=white}
        \input{CLUST/eval/irisPCbad-W2}
        \psset{dotstyle=Btriangle,fillcolor=lightgray}
        \input{CLUST/eval/irisPCbad-C3}
        \psset{fillcolor=white}
        \input{CLUST/eval/irisPCbad-W3}
        \psset{fillcolor=black}
        \pstGeonode[PointSymbol=Bsquare, dotscale=2](2.562,0.486){A}
        \pstGeonode[PointSymbol=Bo,
        dotscale=2](2.419,-0.379){B}
        \pstGeonode[PointSymbol=Btriangle,
        dotscale=2](-1.405,-0.057){C}
        \psclip{\psframe[](-4,-1.5)(3.5,1.5)}%
        {
        \psset{linestyle=none, PointSymbol=none}
        \pstMediatorAB{A}{B}{K}{KP}
        \pstMediatorAB{C}{A}{J}{JP}
        \pstMediatorAB{B}{C}{I}{IP}
        \pstInterLL[PointSymbol=none]{I}{IP}{J}{JP}{O}
        \psset{linewidth=1pt,linestyle=dashed}
        \pstGeonode[PointSymbol=none](-4,-1.5){a}(-4,1.5){b}(3.5,1.5){c}(3.5,-1.5){d}
        \pstInterLL[PointSymbol=none]{O}{I}{a}{b}{oi}
        \pstLineAB{O}{oi}
        \pstInterLL[PointSymbol=none]{O}{J}{b}{c}{oj}
        \pstLineAB{O}{oj}
        \pstInterLL[PointSymbol=none]{O}{K}{a}{d}{ok}
        \pstLineAB{O}{ok}
        }
        \endpsclip
        %\endpspicture
        \endpsgraph
    }}
	}
\end{figure}
\begin{center}
\begin{tabular}{|l|ccc|ccc|}
  \hline
  & $\mathit{purity}$ & $\mathit{match}$ & $F$ &
  $H(\cT|\cC)$ & $\NMI$  & $\VI$\\
  \hline
  (a) \text{ Good } & 0.887 & 0.887 & 0.885 & 0.418 & 0.742   & 0.812\\
  (b) \text{ Bad } & 0.667 & 0.560 & 0.658 & 0.743 &  0.587  & 1.200\\
  \hline
\end{tabular}
\end{center}
\end{frame}

\ifdefined\wox \begin{frame} \titlepage \end{frame} \fi

\begin{frame}{Pairwise Measures}
Given clustering $\cC$ and ground-truth partitioning $\cT$, let
$\bx_i, \bx_{j} \in \bD$ be any two points, with $i\ne j$. 
Let $y_i$ denote
the true partition label and let $\hy_i$ denote the cluster label for
point $\bx_i$.

\medskip
If both $\bx_i$ and $\bx_{j}$ belong to the same cluster, that is, $\hy_i =
\hy_{j}$, we call it
a {\em positive} event, and if they do not belong to the same cluster,
that is, $\hy_i \ne \hy_{j}$, we call that a {\em negative} event.
Depending on whether there is agreement between the cluster labels and
partition labels, there are four possibilities to consider:

\begin{description}

\item[\textit{True Positives:}] 
  $\bx_i$ and $\bx_{j}$ belong to the same
  partition in $\cT$, and they are also in the same cluster in $\cC$. The number of true
  positive pairs is given as
  \begin{align*}
\tcbhighmath{
  \mathit{TP} = \bigl|\{(\bx_i, \bx_{j}):\; y_i = y_{j}
  \text{ and } \hy_i = \hy_{j} \}\bigr|
}
  \end{align*}

\item[\textit{False Negatives:}] 
  $\bx_i$ and $\bx_{j}$ belong to the same
  partition in $\cT$, but they do not belong to the  same cluster in $\cC$.
The
  number of all false negative pairs is given as
  \begin{align*}
\tcbhighmath{
  \mathit{FN} = \bigl|\{(\bx_i, \bx_{j}):\; y_i = y_{j}
  \text{ and } \hy_i \ne \hy_{j} \}\bigr|
}
  \end{align*}
\end{description}
\end{frame}

\begin{frame}{Pairwise Measures}
  \begin{description}
\item[\textit{False Positives:}]
  $\bx_i$ and $\bx_{j}$ do not belong to the
  same partition in $\cT$, but they do belong to the same cluster in $\cC$.
  The
  number of false positive pairs is given as
  \begin{align*}
\tcbhighmath{
  \mathit{FP} = \bigl|\{(\bx_i, \bx_{j}):\; y_i \ne y_{j}
  \text{ and } \hy_i = \hy_{j} \}\bigr|
}
  \end{align*}

\item[\textit{True Negatives:}] $\bx_i$ and $\bx_{j}$ neither belong to
  the same partition in $\cT$, nor do they belong to the same cluster in
  $\cC$.  The number of such true negative pairs is given as
    \begin{align*}
\tcbhighmath{
  \mathit{TN} = \bigl|\{(\bx_i, \bx_{j}):\; y_i \ne y_{j}
  \text{ and } \hy_i \ne \hy_{j} \}\bigr|
}
  \end{align*}
\end{description}

Because there are $N = {n \choose 2} = \frac{n(n-1)}{2}$ pairs of points, we
have the following identity:
\begin{align*}
  N = \mathit{TP} + \mathit{FN} + \mathit{FP} + \mathit{TN}
\end{align*}
\end{frame}

\begin{frame}{Pairwise Measures: TP, TN, FP, FN}
They can be computed eff\/{i}ciently using the
contingency table $\bN = \left\{ n_{ij} \right\}$.
The number of true positives is given as
\begin{align*}
  \mathit{TP} 
  & = \frac{1}{2} \biggl(\Bigl(\sum_{i=1}^r \sum_{j=1}^k n_{ij}^2\Bigr)
  - n \biggr)
\end{align*}
The false negatives can be computed as
\begin{align*}
  \mathit{FN} 
  & = \frac{1}{2}\biggl(
  \sum_{j=1}^k m_{j}^2 - \sum_{i=1}^r \sum_{j=1}^k n_{ij}^2 \biggr)
\end{align*}
The number of false positives are:
\begin{align*}
\mathit{FP} & =
  \frac{1}{2}\biggl(
  \sum_{i=1}^r n_i^2 - \sum_{i=1}^r \sum_{j=1}^k n_{ij}^2 \biggr)
\end{align*}

F{i}nally, the number of true negatives can be obtained via
\begin{align*}
  \mathit{TN} = N - (\mathit{TP} + \mathit{FN} + \mathit{FP}) = \frac{1}{2} \biggl(
  n^2 - \sum_{i=1}^r n_i^2 - \sum_{j=1}^k m_{j}^2 +
  \sum_{i=1}^r \sum_{j=1}^k n_{ij}^2
  \biggr)
\end{align*}
\end{frame}



\begin{frame}{Pairwise Measures: Jaccard Coeff\/{i}cient, Rand Statistic}
\small
{\bf Jaccard Coeff\/{i}cient:} measures the fraction of true positive point pairs,
but after ignoring the true negative:
\begin{align*}
\tcbhighmath{
\mathit{Jaccard} = \frac{\mathit{TP}}{\mathit{TP} + \mathit{FN} + \mathit{FP}}
}
\end{align*}

\medskip
{\bf Rand Statistic:} 
measures the fraction of true positives and true negatives over
all point pairs:
\begin{align*}
\tcbhighmath{
\mathit{Rand} = \frac{\mathit{TP} + \mathit{TN}}{N}
}
\end{align*}
\end{frame}

\begin{frame}{Pairwise Measures: FM Measure}

\medskip {\bf Fowlkes-Mallows Measure:} 
Def\/{i}ne the overall
{\em pairwise
precision} and
{\em pairwise recall} values for a clustering $\cC$, as follows:
\begin{align*}
  \mathit{prec} & = \mathit{TP}/\mathit{TP}+\mathit{FP} &
  \mathit{recall} & = \mathit{TP}/\mathit{TP}+\mathit{FN}
\end{align*}
The Fowlkes--Mallows (FM) measure is def\/{i}ned as the
geometric mean of the pairwise precision and recall
\begin{align*}
\tcbhighmath{
\mathit{FM} = \sqrt{\mathit{prec}\cdot \mathit{recall}} = \frac{\mathit{TP}}{\sqrt{ (\mathit{TP}+\mathit{FN})(\mathit{TP}+\mathit{FP}) }}
}
\end{align*}
\end{frame}


\begin{frame}[fragile]{K-means: Iris Principal Components Data}
\framesubtitle{Good Case}

\setcounter{subfigure}{0}
\begin{columns}
  \column{0.5\textwidth}
\begin{figure}
    \centering
    \captionsetup[subfloat]{captionskip=20pt}
    \def\pshlabel#1{ {\footnotesize $#1$}}
    \def\psvlabel#1{ {\footnotesize $#1$}}
    \psset{xAxisLabel=$\bu_1$,yAxisLabel= $\bu_2$}
    \psset{xunit=0.5in,yunit=0.65in,dotscale=1.5,arrowscale=2,PointName=none}
    \centerline{
	\scalebox{0.5}{
    %\subfloat[K-means: good]{\label{fig:clust:eval:kexgood}
        %\pspicture[](-5,-2)(3.5,1.5)
        %\psaxes[tickstyle=bottom,Dx=1,Ox=-4,Dy=0.5,Oy=-1.5]{->}(-4,-1.5)(3.5,1.5)
        \psgraph[tickstyle=bottom,Dx=1,Ox=-4,Dy=0.5,Oy=-1.5]{->}%
        (-4,-1.5)(3.5,1.5){3.5in}{2in}
        \psset{dotstyle=Bsquare,fillcolor=lightgray}
        \input{CLUST/eval/irisPCgood-C1}
        \psset{fillcolor=white}
        \input{CLUST/eval/irisPCgood-W1}
        \psset{dotstyle=Bo,fillcolor=lightgray}
        \input{CLUST/eval/irisPCgood-C2}
        \psset{fillcolor=white}
        \input{CLUST/eval/irisPCgood-W2}
        \psset{dotstyle=Btriangle,fillcolor=lightgray}
        \input{CLUST/eval/irisPCgood-C3}
        \psset{fillcolor=white}
        \input{CLUST/eval/irisPCgood-W3}
        \psset{fillcolor=black}
        \pstGeonode[PointSymbol=Bo, dotscale=2](2.64,0.19){A}
        \pstGeonode[PointSymbol=Btriangle,
        dotscale=2](-2.35,0.27){B}
        \pstGeonode[PointSymbol=Bsquare,
        dotscale=2](-0.66,-0.33){C}
        \psclip{\psframe[](-4,-1.5)(3.5,1.5)}%
        {
        \psset{linestyle=none, PointSymbol=none}
        \pstMediatorAB{A}{B}{K}{KP}
        \pstMediatorAB{C}{A}{J}{JP}
        \pstMediatorAB{B}{C}{I}{IP}
        \pstInterLL[PointSymbol=none]{I}{IP}{J}{JP}{O}
        \psset{linewidth=1pt,linestyle=dashed}
        \pstGeonode[PointSymbol=none](-4,-1.5){a}(-4,1.5){b}(3.5,1.5){c}(3.5,-1.5){d}
        \pstInterLL[PointSymbol=none]{O}{I}{a}{b}{oi}
        \pstLineAB{O}{oi}
        \pstInterLL[PointSymbol=none]{O}{J}{a}{d}{oj}
        \pstLineAB{O}{oj}
        \pstInterLL[PointSymbol=none]{O}{K}{b}{c}{ok}
        \pstLineAB{O}{ok}
        }
        \endpsclip
        %\endpspicture
        \endpsgraph
    }}
  \end{figure}
    
  \column{0.5\textwidth}
Contingency table:
\small
\begin{align*}
  \matr{ & \vline& \textbf{setosa} & \textbf{versicolor} &
  \textbf{virginica}\\[-3pt]
  & \vline & T_1 & T_2 & T_3\\
  \hline
C_1 &\vline& 0 & 47 & 14\\
C_2 &\vline& 50 &  0 &  0\\
C_3 &\vline& 0 &  3 & 36\\
}
\end{align*}
\end{columns}
The number of true
positives is:
\begin{align*}
  \mathit{TP} & = {47 \choose 2} + {14 \choose 2} + {50 \choose 2} + {3 \choose 2}
  + {36 \choose 2} = 3030
\end{align*}
Likewise, we have
$\mathit{FN} = 645$, 
$\mathit{FP} = 766$,
$\mathit{TN} =6734$, and $N = {150 \choose 2} = 11175$.

We therefore have:
$\mathit{Jaccard} = 0.682$, $\mathit{Rand}=0.887$, 
$\mathit{FM} = 0.811$.

For the ``bad'' clustering, we have:
$\mathit{Jaccard} = 0.477$, $\mathit{Rand}=0.717$, 
$\mathit{FM} = 0.657$.

\end{frame}


\begin{frame}{Correlation Measures: Hubert statistic}
\small
  Let $\bX$ and $\bY$ be two symmetric $n \times n$ matrices,
and let $N = {n \choose 2}$.
Let $\bx, \by \in \setR^{N}$ denote the
vectors obtained by
linearizing the upper triangular elements (excluding the main diagonal)
of $\bX$ and $\bY$.

\medskip
Let $\mu_X$ denote the element-wise mean of $\bx$,
given as
\begin{align*}
  \mu_X & = \frac{1}{N}\sum_{i=1}^{n-1}\sum_{j=i+1}^n
  \bX(i,j) = \frac{1}{N} \bx^T\bx
\end{align*}
and let $\bz_x$ denote the centered $\bx$ vector, def\/{i}ned as
$ \bz_x = \bx - \bone \cdot \mu_X$

\medskip
The Hubert statistic is def\/{i}ned as 
\begin{align*}
\tcbhighmath{
  \Gamma =
  \frac{1}{N} \sum_{i=1}^{n-1} \sum_{j=i+1}^n
\bX(i,j)\cdot \bY(i,j) = \frac{1}{N} \bx^T\by
}
\end{align*}

\medskip
The normalized Hubert statistic is def\/{i}ned as the
element-wise correlation
\begin{align*}
\tcbhighmath{
  \Gamma_n =  \frac{\bz_x^T \bz_y}{\norm{\bz_x} \cdot \norm{\bz_y}} =
  \cos \theta
}
\end{align*}
\end{frame}



\begin{frame}{Correlation-based Measure: Discretized Hubert Statistic}
Let $\bT$ and $\bC$ be the $n \times n$ matrices def\/{i}ned as
\begin{align*}
  \bT(i,j) & =
  \begin{cases}
    1 & \text{if } y_i = y_{j}, i \ne j\\
    0 & \text{otherwise}
  \end{cases} &
  \bC(i,j) & =
  \begin{cases}
    1 & \text{if } \hy_i = \hy_{j}, i \ne j\\
    0 & \text{otherwise}
 \end{cases}
\end{align*}
Let $\bt, \bc \in \setR^N$ denote the $N$-dimensional vectors
comprising the upper triangular elements (excluding the diagonal)
of $\bT$ and $\bC$.
Let $\bz_t$ and $\bz_c$ denote the centered $\bt$ and $\bc$
vectors.

\medskip
The discretized Hubert statistic is computed
by setting $\bx = \bt$ and $\by = \bc$:
\begin{align*}
  \Gamma = \frac{1}{N} \bt^T\bc = \frac{\mathit{TP}}{N}
\end{align*}

\medskip
The
normalized version of the discretized Hubert statistic is simply
the correlation between $\bt$ and $\bc$
\begin{align*}
  \Gamma_n &
  = \frac{\bz_t^T\bz_c}
  {\norm{\bz_t}\cdot \norm{\bz_c}} 
=\frac{\tfrac{\mathit{TP}}{N} - \mu_T \mu_C}
  {\sqrt{\mu_T\mu_C (1-\mu_T)(1-\mu_C)}}
\end{align*}
where $\mu_T = \tfrac{\mathit{TP}+\mathit{FN}}{N}$ and $\mu_C =
\tfrac{\mathit{TP}+\mathit{FP}}{N}$.
\end{frame}

\ifdefined\wox \begin{frame} \titlepage \end{frame} \fi

\begin{frame}{Internal Measures}
Internal evaluation measures do not have recourse to the ground-truth
partitioning. To evaluate the
quality of the clustering, internal measures
therefore have to utilize notions of intracluster similarity
or compactness, contrasted with notions of intercluster separation,
with usually a trade-off in maximizing these two aims.

\medskip
The internal measures are based on the $n\times n$ {\em
distance matrix}, also called the {\em proximity matrix}, of all
pairwise distances among the $n$ points:

\begin{align}
    \tcbhighmath{
    \bW = \Bigl\{ \norm{\bx_i- \bx_{\!j}} \Bigr\}_{i,j=1}^n}
  \label{eq:clust:eval:proxW}
\end{align}
where $\norm{\bx_i - \bx_{\!j}}$
is the Euclidean
distance between $\bx_i, \bx_{j} \in \bD$.


\medskip
The proximity matrix $\bW$ is 
the adjacency matrix of the weighted complete graph $G$ over
the $n$ points, that is, with nodes $V = \{\bx_i \mid \bx_i \in \bD\}$, edges
$E=\{(\bx_i, \bx_{j})\mid \bx_i, \bx_{j} \in \bD\}$, and edge weights
$w_{ij} = \bW(i, j)$ for all $\bx_i, \bx_{j}\in \bD$.
\end{frame}

\begin{frame}{Internal Measures}
The clustering $\cC$ can be considered as a
$k$-way cut in $G$.
Given any subsets $S,R \subset V$, def\/{i}ne $W(S,R)$ as the sum of the
weights on all edges with one vertex in $S$ and the other in $R$, given
as
\begin{align*}
  W(S, R) = \sum_{\bx_i \in S} \sum_{\bx_{j} \in R} w_{ij}
\end{align*}
We denote by $\ol{S} = V - S$ the complementary set
of vertices.

\medskip
The sum of all the intracluster and intercluster 
weights are given as
\begin{align*}
  W_{in} & = \frac{1}{2} \sum_{i=1}^k W(C_i,C_i) &
  W_{out} & = \frac{1}{2} \sum_{i=1}^k W(C_i, \ol{C_i})
  = \sum_{i=1}^{k-1} \sum_{j > i} W(C_i, C_{j})
\end{align*}

The number of distinct intracluster and intercluster edges is given as
\begin{align*}
  N_{in} &= \sum_{i=1}^k {n_i \choose 2} & 
  N_{out} &= \sum_{i=1}^{k-1} \sum_{j=i+1}^k n_i \cdot n_{j} 
\end{align*}
\end{frame}



\begin{frame}[fragile]{Clusterings as Graphs: Iris}
\framesubtitle{Only intracluster edges shown.}
\setcounter{subfigure}{0}

Good clustering.

\begin{figure}
    \centering
    \captionsetup[subfloat]{captionskip=20pt}
    \def\pshlabel#1{ {\footnotesize $#1$}}
    \def\psvlabel#1{ {\footnotesize $#1$}}
    \psset{xAxisLabel=$\bu_1$,yAxisLabel= $\bu_2$}
    \psset{xunit=0.5in,yunit=0.65in,dotscale=1.5,arrowscale=2,PointName=none}
    \centerline{
	\scalebox{0.55}{
    %\subfloat[K-means: good]{\label{fig:clust:eval:kexgood}
        %\pspicture[](-5,-2)(3.5,1.5)
        %\psaxes[tickstyle=bottom,Dx=1,Ox=-4,Dy=0.5,Oy=-1.5]{->}(-4,-1.5)(3.5,1.5)
        \psgraph[tickstyle=bottom,Dx=1,Ox=-4,Dy=0.5,Oy=-1.5]{->}%
        (-4,-1.5)(3.5,1.5){3.5in}{2in}
        \psset{dotstyle=Bsquare,fillcolor=lightgray}
        \input{CLUST/eval/irisPCgood-C1}
        \psset{fillcolor=white}
        \input{CLUST/eval/irisPCgood-W1}
        \psset{dotstyle=Bo,fillcolor=lightgray}
        \input{CLUST/eval/irisPCgood-C2}
        \psset{fillcolor=white}
        \input{CLUST/eval/irisPCgood-W2}
        \psset{dotstyle=Btriangle,fillcolor=lightgray}
        \input{CLUST/eval/irisPCgood-C3}
        \psset{fillcolor=white}
        \input{CLUST/eval/irisPCgood-W3}
        \psset{fillcolor=black}
        \pstGeonode[PointSymbol=Bo, dotscale=2](2.64,0.19){A}
        \pstGeonode[PointSymbol=Btriangle,
        dotscale=2](-2.35,0.27){B}
        \pstGeonode[PointSymbol=Bsquare,
        dotscale=2](-0.66,-0.33){C}
        \psclip{\psframe[](-4,-1.5)(3.5,1.5)}%
        {
        \psset{linestyle=none, PointSymbol=none}
        \pstMediatorAB{A}{B}{K}{KP}
        \pstMediatorAB{C}{A}{J}{JP}
        \pstMediatorAB{B}{C}{I}{IP}
        \pstInterLL[PointSymbol=none]{I}{IP}{J}{JP}{O}
        \psset{linewidth=1pt,linestyle=dashed}
        \pstGeonode[PointSymbol=none](-4,-1.5){a}(-4,1.5){b}(3.5,1.5){c}(3.5,-1.5){d}
        \pstInterLL[PointSymbol=none]{O}{I}{a}{b}{oi}
        \pstLineAB{O}{oi}
        \pstInterLL[PointSymbol=none]{O}{J}{a}{d}{oj}
        \pstLineAB{O}{oj}
        \pstInterLL[PointSymbol=none]{O}{K}{b}{c}{ok}
        \pstLineAB{O}{ok}
        }
        \endpsclip
        %\endpspicture
        \endpsgraph
    }
	\hspace{0.2in}
	\scalebox{0.55}{
        %\pspicture[](-5,-2)(3.5,1.5)
        %\psaxes[tickstyle=bottom,Dx=1,Ox=-4,Dy=0.5,Oy=-1.5]{->}
          %(-4,-1.5)(3.5,1.5)
        \psgraph[tickstyle=bottom,Dx=1,Ox=-4,Dy=0.5,Oy=-1.5]{->}%
        (-4,-1.5)(3.5,1.5){3.5in}{2in}
        \input{CLUST/eval/irisPCgood-graph}
        \endpsgraph
    }}
\end{figure}
%Only intracluster edges shown.
%\end{frame}

\vspace*{-0.4cm}

Bad clustering.

%\begin{frame}[fragile]{Clusterings as Graphs: Iris (Bad Case)}
%\setcounter{subfigure}{0}
\begin{figure}
    \captionsetup[subfloat]{captionskip=20pt}
    \def\pshlabel#1{ {\footnotesize $#1$}}
    \def\psvlabel#1{ {\footnotesize $#1$}}
    \psset{xAxisLabel=$\bu_1$,yAxisLabel= $\bu_2$}
    \psset{xunit=0.5in,yunit=0.65in,dotscale=1.5,arrowscale=2,PointName=none}
    \centerline{
	\scalebox{0.55}{
        %\pspicture[](-5,-2)(3.5,1.5)
        %\psaxes[tickstyle=bottom,Dx=1,Ox=-4,Dy=0.5,Oy=-1.5]{->}(-4,-1.5)(3.5,1.5)
        \psgraph[tickstyle=bottom,Dx=1,Ox=-4,Dy=0.5,Oy=-1.5]{->}%
        (-4,-1.5)(3.5,1.5){3.5in}{2in}
        \psset{dotstyle=Bsquare,fillcolor=lightgray}
        \input{CLUST/eval/irisPCbad-C1}
        \psset{fillcolor=white}
        \input{CLUST/eval/irisPCbad-W1}
        \psset{dotstyle=Bo,fillcolor=lightgray}
        \input{CLUST/eval/irisPCbad-C2}
        \psset{fillcolor=white}
        \input{CLUST/eval/irisPCbad-W2}
        \psset{dotstyle=Btriangle,fillcolor=lightgray}
        \input{CLUST/eval/irisPCbad-C3}
        \psset{fillcolor=white}
        \input{CLUST/eval/irisPCbad-W3}
        \psset{fillcolor=black}
        \pstGeonode[PointSymbol=Bsquare, dotscale=2](2.562,0.486){A}
        \pstGeonode[PointSymbol=Bo,
        dotscale=2](2.419,-0.379){B}
        \pstGeonode[PointSymbol=Btriangle,
        dotscale=2](-1.405,-0.057){C}
        \psclip{\psframe[](-4,-1.5)(3.5,1.5)}%
        {
        \psset{linestyle=none, PointSymbol=none}
        \pstMediatorAB{A}{B}{K}{KP}
        \pstMediatorAB{C}{A}{J}{JP}
        \pstMediatorAB{B}{C}{I}{IP}
        \pstInterLL[PointSymbol=none]{I}{IP}{J}{JP}{O}
        \psset{linewidth=1pt,linestyle=dashed}
        \pstGeonode[PointSymbol=none](-4,-1.5){a}(-4,1.5){b}(3.5,1.5){c}(3.5,-1.5){d}
        \pstInterLL[PointSymbol=none]{O}{I}{a}{b}{oi}
        \pstLineAB{O}{oi}
        \pstInterLL[PointSymbol=none]{O}{J}{b}{c}{oj}
        \pstLineAB{O}{oj}
        \pstInterLL[PointSymbol=none]{O}{K}{a}{d}{ok}
        \pstLineAB{O}{ok}
        }
        \endpsclip
        %\endpspicture
        \endpsgraph
    }
	\hspace{0.2in}
	\scalebox{0.55}{
        %\pspicture[](-5,-2)(3.5,1.5)
        %\psaxes[tickstyle=bottom,Dx=1,Ox=-4,Dy=0.5,Oy=-1.5]{->}(-4,-1.5)(3.5,1.5)
        \psgraph[tickstyle=bottom,Dx=1,Ox=-4,Dy=0.5,Oy=-1.5]{->}%
        (-4,-1.5)(3.5,1.5){3.5in}{2in}
        \input{CLUST/eval/irisPCbad-graph}
        \endpsgraph
    }}
\end{figure}
\end{frame}


\begin{frame}{Internal Measures: BetaCV and C-index}

{\bf BetaCV Measure:} The BetaCV measure is the
ratio of the mean intracluster distance to the mean intercluster
distance:
\begin{align*}
\tcbhighmath{
  BetaCV = \frac{W_{in}/N_{in}}{W_{out}/N_{out}} =
  \frac{N_{out}}{N_{in}} \cdot \frac{W_{in}}{W_{out}}
  = \frac{N_{out}}{N_{in}}
  \frac{\sum_{i=1}^k W(C_i, C_i)}{\sum_{i=1}^k
  W(C_i, \ol{C_i})}
}
\end{align*}
The smaller the BetaCV ratio, the better the clustering.

\bigskip
{\bf C-index:} Let $W_{\min}(N_{in})$ be the sum of the
smallest $N_{in}$ distances in the proximity matrix $\bW$, where
$N_{in}$ is the total number of intracluster edges, or point
pairs. Let $W_{\max}(N_{in})$ be the sum of the largest $N_{in}$
distances in $\bW$.

\smallskip
The C-index measures to what extent the clustering puts together the
$N_{in}$
points that are the closest across the $k$ clusters.
It is def\/{i}ned as
\begin{align*}
\tcbhighmath{
  Cindex = \frac{W_{in} -
  W_{\min}(N_{in})}{W_{\max}(N_{in})-W_{\min}(N_{in})}
}
\end{align*}
The C-index lies in the range $[0,1]$.
The smaller the C-index, the better the clustering.
\end{frame}


\begin{frame}{Internal Measures: Normalized Cut and Modularity}

  {\bf Normalized Cut Measure:} The normalized cut
objective for graph clustering can
also be used as an internal clustering evaluation measure:
\begin{align*}
\tcbhighmath{
NC = \sum_{i=1}^k {W(C_i, \ol{C_i}) \over vol(C_i)} =
\sum_{i=1}^k {W(C_i, \ol{C_i}) \over W(C_i, V)}
}
\end{align*}
where $vol(C_i) = W(C_i,V)$ is the volume of cluster $C_i$.
The higher the normalized cut value the better.

\bigskip
{\bf Modularity:}
The modularity objective is given as
\begin{align*}
\tcbhighmath{
Q = \sum_{i=1}^k \Biggl(
    {W(C_i,C_i) \over W(V,V)} -
    \lB({W(C_i,V) \over W(V,V)}\rB)^2
    \Biggr)
}
\end{align*}
The smaller the modularity measure the better the clustering.
\end{frame}


\begin{frame}{Internal Measures: Dunn Index}
The Dunn index is def\/{i}ned as the ratio between the
minimum distance between point pairs from different clusters and
the maximum distance between point pairs from the same cluster
\begin{align*}
\tcbhighmath{
  Dunn = \frac{\displaystyle W_{out}^{\min}}{\displaystyle W_{in}^{\max}}
}
\end{align*}
where $W_{out}^{\min}$ is the minimum intercluster distance:
\begin{align*}
  \displaystyle
  W_{out}^{\min} = \min_{i, j>i} \; \bigl\{
  w_{ab} | \bx_a \in C_i, \bx_b \in C_{j}\bigr\}
\end{align*}
and $W_{in}^{\max}$ is the maximum intracluster distance:
\begin{align*}
  \displaystyle
  W_{in}^{\max} = \max_{i} \; \bigl\{ w_{ab} | \bx_a,\bx_b \in C_i   \bigr\}
\end{align*}
The larger the Dunn index the better the clustering because it means even
the closest distance between points in different clusters is much larger
than the farthest distance between points in the same cluster.
\end{frame}



\begin{frame}{Internal Measures: Davies-Bouldin Index}
Let $\mu_i$ denote the cluster mean
\begin{align*}
\mu_i = \frac{1}{n_i} \sum_{\bx_{j} \in C_i} \bx_{j}
\end{align*}
Let
$\sigma_{\mu_i}$ denote the dispersion or spread of the points around the cluster
mean
\begin{align*}
\sigma_{\mu_i} = \sqrt{\frac{\sum_{\bx_{j} \in C_i} \dist(\bx_{j},
\mu_i)^2}{n_i}} = \sqrt{var(C_i)}
\end{align*}

The Davies--Bouldin measure for a pair of clusters $C_i$ and $C_{j}$ is
def\/{i}ned as the ratio
\begin{align*}
\tcbhighmath{
\mathit{DB}_{ij} = \frac{\sigma_{\mu_i}+\sigma_{\mu_{j}}}{\dist(\mu_i,\mu_{j})}
}
\end{align*}
$\mathit{DB}_{ij}$ measures how compact the clusters are compared to the distance
between the cluster means.
The Davies--Bouldin index is then def\/{i}ned as
\begin{align*}
  \mathit{DB} = \frac{1}{k} \sum_{i=1}^k \max_{j\ne i} \{\mathit{DB}_{ij} \}
\end{align*}
The smaller the DB value the better the clustering.
\end{frame}


\begin{frame}{Silhouette Coefficient}
  \small
Define the
silhoutte coeff\/{i}cient of a point $\bx_i$ as
\begin{align*}
\tcbhighmath{
  s_i = \frac{\displaystyle \mu_{out}^{\min}(\bx_i)-\mu_{in}(\bx_i)}
  {\displaystyle \max
  \Bigl\{\mu_{out}^{\min}(\bx_i), \mu_{in}(\bx_i)\Bigr\} }
}
\end{align*}
where
$\mu_{in}(\bx_i)$ is the mean distance from $\bx_i$ to points in its own cluster $\hy_i$:
\begin{align*}
  \mu_{in}(\bx_i) = \frac{\sum_{\bx_{j} \in C_{\hy_i}, j \ne i}
  \dist(\bx_i,\bx_{j})}{n_{\hy_i}-1}
\end{align*}
and $\mu_{out}^{\min}(\bx_i)$ is the mean of the distances from
$\bx_i$ to points in the closest cluster:
\begin{align*}
  \mu_{out}^{\min}(\bx_i) = \min_{j\ne \hy_i} \lB\{
  \frac{\sum_{\by \in C_{j}} \dist(\bx_i,\by)}{n_{j}}
  \rB\}
\end{align*}

\medskip
The $s_i$ value lies in the interval $[-1, +1]$. A value
close to
$+1$ indicates that $\bx_i$ is much closer to points in its own
cluster, a value close to zero indicates $\bx_i$ is
close to the boundary, and a value close to $-1$
indicates that $\bx_i$ is much closer to another cluster, 
and therefore
may be mis-clustered.

\medskip
The silhouette coeff\/{i}cient is the mean $s_i$ value:
  $SC = \frac{1}{n} \sum_{i=1}^n s_i$.
A value close to $+1$ indicates a good clustering.
\end{frame}


\begin{frame}[fragile]{Iris Data: Good vs.\ Bad Clustering}
\setcounter{subfigure}{0}
\begin{figure}
    \centering
    \captionsetup[subfloat]{captionskip=20pt}
    \def\pshlabel#1{ {\footnotesize $#1$}}
    \def\psvlabel#1{ {\footnotesize $#1$}}
    \psset{xAxisLabel=$\bu_1$,yAxisLabel= $\bu_2$}
    \psset{xunit=0.5in,yunit=0.65in,dotscale=1.5,arrowscale=2,PointName=none}
    \centerline{
	\subfloat[Good]{
	\scalebox{0.55}{
        \psgraph[tickstyle=bottom,Dx=1,Ox=-4,Dy=0.5,Oy=-1.5]{->}%
        (-4,-1.5)(3.5,1.5){3.5in}{2in}
        \input{CLUST/eval/irisPCgood-graph}
        \endpsgraph
		}}
	\hspace{0.4in}
	\subfloat[Bad]{
	\scalebox{0.55}{
        \psgraph[tickstyle=bottom,Dx=1,Ox=-4,Dy=0.5,Oy=-1.5]{->}%
        (-4,-1.5)(3.5,1.5){3.5in}{2in}
        \input{CLUST/eval/irisPCbad-graph}
        \endpsgraph
		}}}
\end{figure}
  \begin{center}
  \scalebox{0.8}{%
\begin{tabular}{|l|cccc|ccccc|}
  \hline
  & \multicolumn{4}{c|}{Lower better} &
  \multicolumn{5}{c|}{Higher better}\\\cline{2-10}
 & $BetaCV$ &
 $Cindex$ & $Q$ & $\mathit{DB}$ & $\mathit{NC}$  & $Dunn$  & $\mathit{SC}$ & $\Gamma$ & $\Gamma_n$\\
\hline
(a) Good & 0.24 & 0.034 & $-$0.23 & 0.65 & 2.67  & 0.08
     & 0.60 & 8.19&  0.92\\
(b) Bad & 0.33 & 0.08 & $-$0.20 & 1.11 & 2.56  & 0.03  & 0.55 & 7.32
& 0.83\\
\hline
\end{tabular}
}
\end{center}
\end{frame}


\ifdefined\wox \begin{frame} \titlepage \end{frame} \fi

\begin{frame}{Relative Measures: Silhouette Coeff\/{i}cient}
The silhouette coeff\/{i}cient for each
point $s_{j}$, and the average SC value
can be used to estimate the number of clusters
in the data. 

\medskip
The approach consists of plotting the $s_{j}$ values in
descending order for each cluster, and to note the overall $SC$ value
for a particular value of $k$, as well as clusterwise SC values:
\begin{align*}
  SC_i = \frac{1}{n_i}\sum_{\bx_{j} \in C_i} s_{j}
\end{align*}
We then pick the value $k$ that yields the best clustering, with
many points having high $s_{j}$ values within each cluster, as well as
high values for $SC$ and $SC_i$ ($1 \le i \le k$).
\end{frame}



\readdata{\dataKa}{CLUST/eval/figs/irisKmeans-K2-Sil.txt}
\readdata{\dataKb}{CLUST/eval/figs/irisKmeans-K3-Sil.txt}
\readdata{\dataKc}{CLUST/eval/figs/irisKmeans-K4-Sil.txt}
\begin{frame}[fragile]{Iris K-means: Silhouette Coeff\/{i}cient Plot
  ($k=2$)}
\setcounter{subfigure}{0}
\begin{figure}
\captionsetup[subfloat]{captionskip=0.5in}
\centering %\small
\psset{xAxisLabel=$~$,yAxisLabel=silhouette coeff\/{i}cient,%
xAxisLabelPos={c,-0.05},yAxisLabelPos={-20,c}} \psset{xLabels={}}
\centerline{ \subfloat[$k=2$, $SC=0.706$]{
\scalebox{0.9}{
\label{fig:clust:eval:silplotK2}
\begin{psgraph}[Dy=0.1,labels=y,ticks=y]{->}(0,0)(150,1.1){4in}{1.75in}
    \listplot[%plotstyle=bar, barwidth=0.01cm,
    %fillcolor=lightgray, fillstyle=solid,
    plotstyle=dots, dotscale=0.75,
    plotNoMax=2, plotNo=1]{\dataKa}
    \psline{<->}(0,-0.05)(96,-0.05)
    \uput[d](50,-0.05){\scriptsize
    $\begin{array}{c}
        SC_1=0.662\\
        n_1=97
      \end{array}$
      }
    \psline[](96.5,-0.05)(96.5,0.95)
    \psline{<->}(98,-0.05)(150,-0.05)
    \uput[d](120,-0.05){\scriptsize
      $\begin{array}{c}
        SC_2=0.785\\
        n_2=53
      \end{array}$
      }
\end{psgraph}
}} }
\end{figure}
 $k=2$ yields the highest silhouette coeff\/{i}cient, with the two
 clusters essentially well separated.
  $C_1$ starts out with high $s_i$ values, which gradually drop as we
  get to border points. $C_2$
  is even better separated, since it has a higher silhouette coeff\/{i}cient
  and the pointwise scores are all high, except for the last three
  points.

\end{frame}

\begin{frame}[fragile]{Iris K-means: Silhouette Coeff\/{i}cient Plot
  ($k=3$)}
\begin{figure}
\captionsetup[subfloat]{captionskip=0.5in}

\centerline{ \subfloat[$k=3$, $SC=0.598$]{
\scalebox{0.9}{
\label{fig:clust:eval:silplotK3}
\begin{psgraph}[Dy=0.1,labels=y,ticks=y]{->}(0,0)(150,1){4in}{1.75in}
    \listplot[%plotstyle=bar, barwidth=0.01cm,
    %fillcolor=lightgray, fillstyle=solid,
    plotstyle=dots, dotscale=0.75,
    plotNoMax=2, plotNo=1]{\dataKb}
    \psline{<->}(0,-0.05)(60,-0.05)
    \uput[d](30,-0.05){\scriptsize
      $\begin{array}{c}
        SC_1=0.466\\
        n_1=61
      \end{array}$
      }
    \psline[](60.5,-0.05)(60.5,0.95)
    \psline{<->}(62,-0.05)(110,-0.05)
    \uput[d](85,-0.05){\scriptsize
      $\begin{array}{c}
        SC_2=0.818\\
        n_2=50
      \end{array}$
      }
    \psline[](110.5,-0.05)(110.5,0.95)
    \psline{<->}(111,-0.05)(150,-0.05)
    \uput[d](130,-0.05){\scriptsize
      $\begin{array}{c}
        SC_3=0.52\\
        n_3=39
      \end{array}$
      }
\end{psgraph}
} } }
\end{figure}
 $C_1$ from $k=2$ has been split into two
  clusters for $k=3$, namely $C_1$ and $C_3$. Both of these have many
  bordering points, whereas $C_2$ is well separated with high silhouette
  coeff\/{i}cients across all points.

\end{frame}

\begin{frame}[fragile]{Iris K-means: Silhouette Coeff\/{i}cient Plot
  ($k=4$)}
\begin{figure}
\captionsetup[subfloat]{captionskip=0.5in}
\centerline{ \subfloat[$k=4$, $SC=0.559$]{
\scalebox{0.9}{
\label{fig:clust:eval:silplotK4}
\begin{psgraph}[Dy=0.1,labels=y,ticks=y]{->}(0,0)(150,1){4in}{1.75in}
    \listplot[%plotstyle=bar, barwidth=0.01cm,
    %fillcolor=lightgray, fillstyle=solid,
    plotstyle=dots,dotscale=0.75,
    plotNoMax=2, plotNo=1]{\dataKc}
    \psline{<->}(0,-0.05)(48,-0.05)
    \uput[d](20,-0.05){\scriptsize
      $\begin{array}{c}
        SC_1=0.376\\
        n_1=49
      \end{array}$
      }
    \psline[](48.5,-0.05)(48.5,0.95)
    \psline{<->}(49,-0.05)(76,-0.05)
    \uput[d](65,-0.05){\scriptsize
      $\begin{array}{c}
        SC_2=0.534\\
        n_2=28
      \end{array}$
      }
    \psline[](76.5,-0.05)(76.5,0.95)
    \psline{<->}(77,-0.05)(126,-0.05)
    \uput[d](100,-0.05){\scriptsize
      $\begin{array}{c}
        SC_3=0.787\\
        n_3=50
      \end{array}$
      }
    \psline[](126.5,-0.05)(126.5,0.95)
    \psline{<->}(127,-0.05)(150,-0.05)
    \uput[d](140,-0.05){\scriptsize
      $\begin{array}{c}
        SC_4=0.484\\
        n_4=23
      \end{array}$
      }
\end{psgraph}
}}} 
\end{figure}
 $C_3$ is the well
separated cluster, corresponding to $C_2$ (in $k=2$ and $k=3$), and the
remaining clusters are essentially subclusters of $C_1$ for $k=2$. 
Cluster $C_1$ also has two
  points with negative $s_i$ values, indicating that they are
  probably misclustered.

\end{frame}



\begin{frame}{Relative Measures: Calinski--Harabasz Index}
  Given the dataset $\bD =\{\bx_i\}_{i=1}^n$,
the scatter matrix for $\bD$ is given as
\begin{align*}
  \bS = n\cov = \sum_{j=1}^n \lB(\bx_{j} - \bmu\rB)
  \lB(\bx_{j} - \bmu\rB)^T
\end{align*}
where $\bmu = \tfrac{1}{n} \sum_{j=1}^n \bx_{j} $ is the mean and $\cov$
is the covariance matrix.  The scatter matrix can
be decomposed into two matrices $\bS = \bS_W + \bS_B$,
where $\bS_W$ is the within-cluster scatter
matrix and $\bS_B$ is the between-cluster scatter matrix, given
as
\begin{align*}
  \bS_W & = \sum_{i=1}^k \sum_{\bx_{j} \in C_i}
  \lB(\bx_{j} - \bmu_i\rB) \lB(\bx_{j} - \bmu_i\rB)^T\\
  \bS_B & = \sum_{i=1}^k n_i
  \lB(\bmu_i - \bmu\rB) \lB(\bmu_i - \bmu\rB)^T
\end{align*}
where $\bmu_i = \tfrac{1}{n_i} \sum_{\bx_{j} \in C_i} \bx_{j}$ is
the mean for cluster $C_i$.
\end{frame}


\begin{frame}{Relative Measures: Calinski--Harabasz Index}
The Calinski--Harabasz (CH) variance ratio criterion for a given value of $k$
is def\/{i}ned as follows:
\begin{align*}
\tcbhighmath{
  \mathit{CH}(k) = \frac{tr(\bS_B)/(k-1)}{tr(\bS_W)/(n-k)} =
  \frac{n-k}{k-1} \cdot \frac{tr(\bS_B)}{tr(\bS_W)}
}
\end{align*}
where $tr$ is the trace of the matrix.

\medskip
We plot the $\mathit{CH}$ values and look for a large increase in the
value followed by little or no gain. We choose
the value $k>3$ that minimizes the term
\begin{align*}
  \Delta(k) = \Bigl(\mathit{CH}(k+1)-\mathit{CH}(k)\Bigr) - \Bigl(\mathit{CH}(k)-\mathit{CH}(k-1)\Bigr)
\end{align*}
The intuition is that we want to f\/{i}nd the value of $k$ for which $\mathit{CH}(k)$ is much higher than
$\mathit{CH}(k-1)$ and there is only a
little improvement or a decrease in the
$\mathit{CH}(k + 1)$ value.

\end{frame}



\readdata{\dataCH}{CLUST/eval/figs/CHvals.txt}
\begin{frame}[fragile]{Calinski--Harabasz Variance Ratio}
  \small
  CH ratio for various values
  of $k$ on the Iris principal components data, using the K-means
  algorithm, with the best results chosen from 200 runs.
\begin{figure}
    \centering
    \def\pshlabel#1{ {\footnotesize #1}}
    \def\psvlabel#1{ {\footnotesize #1}}
\psset{dotscale=2,fillcolor=lightgray}
\psset{xAxisLabel=$k$,yAxisLabel=$\mathit{CH}$,%
xAxisLabelPos={c,-40},yAxisLabelPos={-1.25,c}}
\scalebox{0.8}{
\begin{psgraph}[Dy=50,Oy=550,Ox=1,showorigin=false]{->}(0,0)(9,250){3.5in}{2in}
\pstScalePoints(1,1){-1 add}{-550 add}
  \listplot[showpoints=true, dotstyle=Bsquare]{\dataCH}
\end{psgraph}
}
\end{figure}
\vspace{0.2in}
  The successive
  $\mathit{CH}(k)$ and $\Delta(k)$ values are as follows:
  \begin{align*}
    \begin{array}{c|cccccccc}
    k & 2 & 3 & 4 & 5 & 6 & 7 & 8 & 9\\
    \hline
    \mathit{CH}(k) & 570.25 & 692.40 & 717.79 & 683.14 & 708.26 & 700.17 & 738.05
    & 728.63\\
    \Delta(k) & \text{--} & -96.78 & -60.03 & 59.78 & -33.22 & 45.97 &
    -47.30 & \text{--}\\
    \end{array}
  \end{align*}
$\Delta(k)$ suggests $k=3$ as the best (lowest) value.
\end{frame}


\begin{frame}{Relative Measures: Gap Statistic}
The gap statistic compares the sum of
intracluster
weights $W_{in}$ 
for different values of $k$ with their expected values assuming
no apparent clustering structure, which forms the null hypothesis.

\bigskip
Let $\cC_k$ be the clustering obtained for a specif\/{i}ed value of $k$.
Let $W_{in}^k(\bD)$ denote
the sum of intracluster weights (over all clusters) for $\cC_k$ on the
input dataset $\bD$.

\bigskip
We would like to compute the probability of the observed
$W_{in}^k$ value under the null hypothesis.
To obtain an empirical distribution for $W_{in}$, we resort to Monte
Carlo simulations of the sampling process.  
\end{frame}

\begin{frame}{Relative Measures: Gap Statistic}
We generate $t$
random samples comprising $n$ points.
Let $\bR_i \in \setR^{n \times d}$, $1 \le i \le t$ denote the $i$th sample.
Let $W_{in}^k(\bR_i)$ denote the sum of intracluster weights for a
given clustering of $\bR_i$ into $k$ clusters.

\medskip
From each sample
dataset $\bR_i$, we generate clusterings for different values of $k$,
and record the intracluster values
$W_{in}^k(\bR_i)$.

\medskip
Let $\mu_W(k)$ and $\sigma_W(k)$ denote the
mean and standard deviation of these intracluster weights for each
value of $k$.
The {\em gap statistic} for a given $k$ is then def\/{i}ned as
\begin{align*}
\tcbhighmath{
  gap(k) = \mu_W(k) - \log  W_{in}^k(\bD)
}
\end{align*}

\medskip
Choose $k$ as follows:
\begin{align*}
  k^* = \arg\min_k \Bigl\{ gap(k) \ge gap(k+1) - \sigma_W(k+1)\Bigr\}
\end{align*}
\end{frame}


\readdata{\dataG}{CLUST/eval/figs/gapstatistic.txt}
\readdata{\dataGE}{CLUST/eval/figs/gapstatistic-errs.txt}
\begin{frame}[fragile]{Gap Statistic: Randomly Generated Data}
\setcounter{subfigure}{0}

A random sample of $n=150$ points, which does not have any apparent
cluster structure.

\begin{figure}
%\begin{figure}[!t]%fig17.5
    \centering
    \def\pshlabel#1{ {\footnotesize $#1$}}
    \def\psvlabel#1{ {\footnotesize $#1$}}
\captionsetup[subfloat]{captionskip=0.4in}
\centering \psset{dotscale=2,fillcolor=lightgray} \centerline{
\subfloat[Randomly generated data ($k=3$)]{
\label{fig:clust:eval:gapdata}
  \pspicture[](-5,-2.25)(3.5,2.75)
\psset{xunit=0.5in,yunit=0.65in,dotscale=1.5,arrowscale=2,PointName=none}
  \psaxes[tickstyle=bottom,Dx=1,Ox=-4,Dy=0.5,Oy=-1.5]{->}(-4,-1.5)(3.5,1.5)
  \psset{dotstyle=Bsquare,fillcolor=lightgray}
  \input{CLUST/eval/gapdata-C1}
  \psset{dotstyle=Bo,fillcolor=lightgray}
  \input{CLUST/eval/gapdata-C2}
  \psset{dotstyle=Btriangle,fillcolor=lightgray}
  \input{CLUST/eval/gapdata-C3}
  \psset{fillcolor=black}
  \pstGeonode[PointSymbol=Bsquare, dotscale=2]%
      (1.95,-0.022){A}
  \pstGeonode[PointSymbol=Bo,dotscale=2]%
      (-0.078,0.15){B}
  \pstGeonode[PointSymbol=Btriangle,dotscale=2]%
      (-2.51,-0.028){C}
    \psclip{\psframe[](-4,-1.5)(3.5,1.5)}%
    {
    \psset{linestyle=none, PointSymbol=none}
    \pstMediatorAB{A}{B}{K}{KP}
    \pstMediatorAB{C}{A}{J}{JP}
    \pstMediatorAB{B}{C}{I}{IP}
    \pstInterLL[PointSymbol=none]{I}{IP}{K}{KP}{O}
    \psset{linewidth=1pt,linestyle=dashed}
    \pstGeonode[PointSymbol=none](-4,-1.5){a}(-4,1.5){b}(3.5,1.5){c}(3.5,-1.5){d}
    \pstInterLL[PointSymbol=none]{O}{I}{b}{c}{oi}
    \pstLineAB{O}{oi}
    %\pstInterLL[PointSymbol=none]{O}{J}{a}{d}{oj}
    %\pstLineAB{O}{oj}
    \pstInterLL[PointSymbol=none]{O}{K}{b}{c}{ok}
    \pstLineAB{O}{ok}
    }
    \endpsclip
    \endpspicture
  }}
\end{figure}
\end{frame}


\begin{frame}[fragile]{Gap Statistic: Intracluster Weights and Gap Values}

We generate $t=200$ random datasets, and compute  both the expected and the
observed (Iris) intracluster weight $\mu_W(k)$, for each value of $k$.  
The observed $W_{in}^k(\bD)$ values are smaller than the expected values $\mu_W(k)$.

\begin{figure}
%\begin{figure}[!t]%fig17.5
    \centering
    \def\pshlabel#1{ {\footnotesize $#1$}}
    \def\psvlabel#1{ {\footnotesize $#1$}}
\captionsetup[subfloat]{captionskip=0.4in}
\centering \psset{dotscale=2,fillcolor=lightgray} 
\centerline{\hskip30pt\scalebox{0.85}{ \subfloat[Intracluster
weights]{ \label{fig:clust:eval:WK}
\psset{xAxisLabel=$k$,yAxisLabel=$\log_2 W_{in}^k$,%
xAxisLabelPos={c,-1},yAxisLabelPos={-2,c}}
\pslegend[rt]{\rule[1ex]{2em}{1pt}%
\psdot[dotstyle=Btriangle,dotscale=1.5](-0.6,0.2) & expected:
$\mu_W(k)$\\
\rule[1ex]{2em}{1pt}%
\psdot[dotstyle=Bo,dotscale=1.5](-0.6,0.2) & observed: $W_{in}^k$}
\begin{psgraph}[Dy=1,Oy=10]{->}(0,0)(10,6){2.2in}{2in}
    \pstScalePoints(1,1){}{-10 add}
    \listplot[showpoints=true, dotstyle=Bo,
    plotNoMax=6, plotNo=4]{\dataG}
    \listplot[showpoints=true, dotstyle=Btriangle,
    plotNoMax=6, plotNo=5]{\dataG}
\end{psgraph}
} \hspace{0.4in} \subfloat[Gap statistic]{
\label{fig:clust:eval:gap}
\psset{xAxisLabel=$k$,yAxisLabel=$gap(k)$,%
xAxisLabelPos={c,-0.15},yAxisLabelPos={-2,c}}
\begin{psgraph}[Dy=0.1]{->}(0,0)(10,1){2.2in}{2in}
    \listplot[showpoints=true, dotstyle=Bsquare, dotscale=1.5,
    plotNoMax=6, plotNo=1]{\dataG}
    \def\DoCoordinate#1#2{}\GetCoordinates{\dataGE}
\end{psgraph}
} }} 
\end{figure}
\end{frame}


\begin{frame}{Gap Statistic as a Function of $\textit{k}$}
% {\tabcolsep10pt
\renewcommand{\arraystretch}{1.1} 
\begin{center}
\begin{tabular}{|c|ccc|}
    \hline
$k$ &  $gap(k)$ & $\sigma_W(k)$ & $gap(k)-\sigma_W(k)$\\
\hline
1 &  0.093 &0.0456 &0.047\\
2 &  0.346 &0.0486 &0.297\\
3 &  0.679 &0.0529 &0.626\\
4 &  0.753 &0.0701 &0.682\\
5 &  0.586 &0.0711 &0.515\\
6 &  0.715 &0.0654 &0.650\\
7 &  0.808 &0.0611 &0.746\\
8 &  0.680 &0.0597 &0.620\\
9 &  0.632 &0.0606 &0.571\\
\hline
  \end{tabular}%}
\end{center}
The optimal value for the number of clusters is
$k=4$ because
$$gap(4) = 0.753 > gap(5)-\sigma_W(5) = 0.515$$

\medskip
However, if we
relax the gap test to be within two standard deviations, then the
optimal value is $k=3$ because
$$gap(3) = 0.679 >
gap(4)-2\sigma_W(4) = 0.753-2\cdot0.0701 = 0.613$$
\end{frame}

\ifdefined\wox \begin{frame} \titlepage \end{frame} \fi

\begin{frame}{Cluster Stability}
The main idea behind cluster stability is that the clusterings obtained
from several datasets sampled from the same underlying distribution as
$\bD$ should be similar or ``stable.''

\medskip
Stability can be
used to f\/{i}nd  a good value for
$k$, the correct number of clusters.

\medskip
We generate $t$ samples of size $n$ by sampling from $\bD$
with replacement.
Let $\cC_k(\bD_i)$ denote the
clustering obtained from sample $\bD_i$, for a given value of $k$. 

\medskip
Next,
we compare the distance between all pairs of clusterings
$\cC_k(\bD_i)$ and $\cC_k(\bD_{j})$ using several 
of the external cluster evaluation measures.
From these values we compute the
expected pairwise distance for each value of $k$.  F{i}nally, the value
$k^*$ that exhibits the least deviation between the clusterings obtained
from the resampled datasets is the best choice for $k$ because it
exhibits the most stability.
\end{frame}



\newcommand{\StabClus}{{\textsc{ClusteringStability}}}
\begin{frame}[fragile]{Clustering Stability Algorithm}
\begin{tightalgo}[H]{\textwidth-18pt}
\SetKwInOut{Algorithm}{\StabClus\ ($A, t, k^{\max}, \bD$)}
\Algorithm{}
$n \gets \card{\bD}$\;
%\tcp{Generate $t$ samples}
\For{$i = 1,2,\ldots,t$}{
  $\bD_i \gets$ sample $n$ points from $\bD$ with replacement\;
}
%\tcp{Generate clusterings for different values of $k$}
\For{$i = 1,2,\ldots,t$}{
  \For{$k =2,3,\ldots,k^{\max}$}{
    $\cC_k(\bD_i) \assign$ cluster $\bD_i$ into $k$ clusters using
    algorithm $A$\;
  }
}
%\tcp{Compute mean difference between clusterings for each $k$}
\ForEach{pair $\bD_i, \bD_{j}$ with $j > i$ }{
  $\bD_{ij} \gets \bD_i \cap \bD_{j}$ \tcp{create common dataset}\;%using Eq.\nosp\eqref{eq:clus:eval:Dij}}\; 
\For{$k = 2,3,\ldots,k^{\max}$}{
  $d_{ij}(k) \gets d\bigl(\cC_k(\bD_i),\cC_k(\bD_{j}),\bD_{ij}\bigr)$ \tcp{distance between clusterings}\;
  }
}
\For{$k = 2,3,\ldots,k^{\max}$}{
  $\mu_d(k) \assign
  \frac{2}{t(t-1)} \sum_{i=1}^t \sum_{j>i} d_{ij}(k)$
  %\nllabel{alg:clust:eval:clusterstability:mud} \tcp{expected pairwise
  %distance}\;
}
%\tcp{Choose best $k$}
$k^* \gets \arg\min_k \bigl\{ \mu_d(k) \bigr\}$
\end{tightalgo}
\end{frame}


\readdata{\dataCS}{CLUST/eval/figs/stabilityvals.txt}
\begin{frame}[fragile]{Clustering Stability: Iris Data}
  \framesubtitle{$t=500$ bootstrap samples; best K-means from 100 runs}

Both the Variation of Information and the Fowlkes-Mallows  
  measures indicate that $k=2$ is the best value. VI
  indicates the least expected distance between pairs of
  clusterings, and FM indicates the most
  expected similarity between clusterings.

\begin{figure}[H]
    \centering
    \def\pshlabel#1{ {\footnotesize #1}}
    \def\psvlabel#1{ {\footnotesize #1}}
\psset{dotscale=2,fillcolor=lightgray,arrowscale=2}
\scalebox{0.9}{
\centerline{
\begin{pspicture}(-3,-1)(3,5)
\psset{xAxisLabel=$k$,yAxisLabel=Expected Value,%
xAxisLabelPos={c,-0.15},yAxisLabelPos={-1.5,c}}
\pslegend[rb]{\rule[1ex]{2em}{1pt}%
\psdot[dotstyle=Bo,dotscale=1.5](-0.5,0.03) & $\mu_s(k): \mathit{FM}$\\
\rule[1ex]{2em}{1pt}%
\psdot[dotstyle=Btriangle,dotscale=1.5](-0.5,0.03) & $\mu_d(k): VI$}
\begin{psgraph}[Dy=0.1]{->}(0,0)(10,1){3in}{2in}
    \listplot[showpoints=true, dotstyle=Bo,
    plotNoMax=4, plotNo=3]{\dataCS}
    \listplot[showpoints=true, dotstyle=Btriangle,
    plotNoMax=4, plotNo=1]{\dataCS}
\end{psgraph}
\end{pspicture}
}
}
\end{figure}
\end{frame}



\begin{frame}{Clustering Tendency: Spatial Histogram}

Clustering tendency or clusterability
aims to determine whether the dataset $\bD$ has any
meaningful groups to begin with. 

\medskip
Let $X_1, X_2, \ldots, X_d$ denote the $d$ dimensions. Given
$b$, the number of bins for each dimension, we divide each dimension
$X_{j}$ into $b$ equi-width bins, and simply count how many points lie in
each of the $b^d$ $d$-dimensional cells.  

\medskip
From this spatial histogram, we
can obtain the empirical joint probability mass function (EPMF) for the
dataset $\bD$
\begin{align*}
  f(\bi) = P(\bx_{j} \in \text{cell }\bi) =
  \frac{\bigl|\{\bx_{j} \in \text{cell }\bi\}\bigr|}{n}
\end{align*}
where $\bi = (i_1,i_2,\ldots,i_d)$ denotes a cell index, with $i_{j}$
denoting
the bin index along dimension $X_{j}$.
\end{frame}


\begin{frame}{Clustering Tendency: Spatial Histogram}
We generate $t$ random samples, each comprising $n$ points within
the same $d$-dimensional space as the input dataset $\bD$.  
Let $\bR_{j}$
denote the $j$th such random sample. 
We then compute the
corresponding EPMF $g_{j}(\bi)$ for each $\bR_{j}$, $1\le j\le t$.

\medskip
We next compute how much the distribution $f$ differs from
$g_{j}$ (for $j=1,\ldots,t$),
using the Kullback--Leibler (KL) divergence from $f$ to
$g_{j}$, def\/{i}ned as
\index{Kullback-Leibler divergence}
\index{KL divergence|see{Kullback-Leibler divergence}}
\begin{align*}
  \mathit{KL}(f|g_{j}) = \sum_{\bi} f(\bi) \log \lB(\frac{f(\bi)}{g_{j}(\bi)} \rB)
\end{align*}
The KL divergence is zero only when $f$ and $g_{j}$ are the same
distributions. Using these divergence values, we can compute how much
the dataset $\bD$ differs from a random dataset.

\medskip

Its main limitation is that the number of cells
($b^d$) increases exponentially with the dimensionality, and, with a f\/{i}xed sample size $n$, most
of the cells will have none or one point, making it hard
to estimate the divergence. The method is also sensitive to the choice
of parameter $b$.
\end{frame}



\readdata{\dataHtwoD}{CLUST/eval/figs/tendency2dhist.txt}
\readdata{\dataKLtwoD}{CLUST/eval/figs/kl2d.txt}
\begin{frame}[fragile]{Spatial Histogram: Iris PCA Data versus Uniform}
\framesubtitle{Uniform has $n=150$ points}
\setcounter{subfigure}{0}
\begin{figure}
%\begin{figure}[!p]\vspace*{12pt}%fig17.7
  \centering
  \captionsetup[subfloat]{captionskip=35pt}
    \psset{xAxisLabel=$\bu_1$,yAxisLabel= $\bu_2$}
    \def\pshlabel#1{ {\footnotesize $#1$}}
    \def\psvlabel#1{ {\footnotesize $#1$}}
    \centerline{
        \psset{xunit=0.3in,yunit=0.6in,
          dotscale=1.5,dotstyle=Bo,fillcolor=lightgray,
          arrowscale=2}
    \subfloat[Iris: spatial cells]{
    \label{fig:clust:eval:IrisSpatialCells}
    \scalebox{0.8}{%
        %\pspicture[](-3,-1.25)(3,1.5)
        %\psaxes[tickstyle=bottom,Dx=1,Ox=-4,Dy=0.5,Oy=-1.5]{->}%
        %  (-4,-1.5)(3.5,1.5)
        \psgraph[tickstyle=bottom,Dx=1,Ox=-4,Dy=0.5,Oy=-1.5]{->}%
        (-4,-1.5)(3.5,1.5){2.5in}{2in}
        \input{CLUST/eval/irisPCgood-C1}
        \input{CLUST/eval/irisPCgood-W1}
        \input{CLUST/eval/irisPCgood-C2}
        \input{CLUST/eval/irisPCgood-W2}
        \input{CLUST/eval/irisPCgood-C3}
        \input{CLUST/eval/irisPCgood-W3}
        \psset{linestyle=dashed}
        \multido{\ni=-3.795+1.403}{6}{%
        \psline(\ni,-1.262)(\ni,1.371)}
        \multido{\ni=-1.262+0.527}{6}{%
        \psline(-3.795,\ni)(3.225,\ni)}
        \endpsgraph
        }}
    \hspace{0.5in}
  \subfloat[Uniform: spatial cells]{
  \label{fig:clust:eval:RandomSpatialCells}
  \scalebox{0.8}{%
  %\pspicture[](-3,-1.25)(3,1.5)
  %\psaxes[tickstyle=bottom,Dx=1,Ox=-4,Dy=0.5,Oy=-1.5]{->}%
%   (-4,-1.5)(3.5,1.5)
  \psgraph[tickstyle=bottom,Dx=1,Ox=-4,Dy=0.5,Oy=-1.5]{->}%
  (-4,-1.5)(3.5,1.5){2.5in}{2in}
  \input{CLUST/eval/gapdata-C1}
  \input{CLUST/eval/gapdata-C1}
  \input{CLUST/eval/gapdata-C2}
  \input{CLUST/eval/gapdata-C3}
        \psset{linestyle=dashed}
        \multido{\ni=-3.795+1.403}{6}{%
        \psline(\ni,-1.262)(\ni,1.371)}
        \multido{\ni=-1.262+0.527}{6}{%
        \psline(-3.795,\ni)(3.225,\ni)}
    \endpsgraph
    }}}
\end{figure}
\end{frame}

\begin{frame}[fragile]{Spatial Histogram: Empirical PMF}
\framesubtitle{5 bins results in 25 spatial cells}
\begin{figure}
  \centering
  \captionsetup[subfloat]{captionskip=35pt}
    \psset{xAxisLabel=$\bu_1$,yAxisLabel= $\bu_2$}
    \def\pshlabel#1{ {\footnotesize $#1$}}
    \def\psvlabel#1{ {\footnotesize $#1$}}
  \centerline{
  \psset{arrowscale=2}
  \subfloat[Empirical probability mass function]{%
\def\pshlabel#1{\scriptsize {#1}}
\def\psvlabel#1{\scriptsize {#1}}
  \label{fig:clust:eval:irisSpatialdistr}
  \psset{xAxisLabel=Spatial Cells,yAxisLabel=Probability,%
  xAxisLabelPos={c,-0.04},yAxisLabelPos={-3.5,c}}
  \pslegend[rt]{%
  \psline[linewidth=3pt,linecolor=darkgray](0,0)(1.15,0) & $\qquad$Iris
  ($f$)\\
  \psline[linewidth=3pt,linecolor=lightgray](0,0)(1.15,0) &
  $\qquad$Uniform ($g_{j}$)}
  \begin{psgraph}[Dy=0.02,showorigin=true]{->}(0,0)(25,0.2){3.5in}{1.75in}
    \listplot[plotstyle=bar,barwidth=0.1cm,
          fillcolor=darkgray,fillstyle=solid,
          plotNoMax=2,plotNo=1]{\dataHtwoD}
          \pstScalePoints(1,1){0.25 add}{}
          \listplot[plotstyle=bar,barwidth=0.1cm,
          fillcolor=lightgray,fillstyle=solid,
          plotNoMax=2,plotNo=2]{\dataHtwoD}
          \listplot[showpoints=false,linestyle=dashed,
          plotNoMax=2,plotNo=2]{\dataHtwoD}
          \pstScalePoints(1,1){}{}
          \listplot[showpoints=false,linewidth=1.5pt,
          plotNoMax=2,plotNo=1]{\dataHtwoD}
  \end{psgraph}
  }}
\end{figure}
\end{frame}

\begin{frame}[fragile]{Spatial Histogram: KL Divergence Distribution}
\begin{figure}
  \centering
  \captionsetup[subfloat]{captionskip=35pt}
    \psset{xAxisLabel=$\bu_1$,yAxisLabel= $\bu_2$}
    \def\pshlabel#1{ {\footnotesize $#1$}}
    \def\psvlabel#1{ {\footnotesize $#1$}}
  \centerline{
  \psset{arrowscale=2}
    \subfloat[KL-divergence distribution]{
    \label{fig:clust:eval:irisSpatialKL}
    \psset{xAxisLabel=KL Divergence,yAxisLabel=Probability,%
  xAxisLabelPos={c,-0.07},yAxisLabelPos={-0.2,c}}
  \begin{psgraph}[Dy=0.05,Dx=0.15,Ox=0.65,showorigin=true]{->}(0,0)(1.2,0.3){3in}{1.5in}
    \pstScalePoints(1,1){-0.65 add}{}
    \listplot[plotstyle=bar,barwidth=0.2cm,
          fillcolor=lightgray,fillstyle=solid,
          plotNoMax=2,plotNoX=2,plotNo=2]{\dataKLtwoD}
  \end{psgraph}
}}
\end{figure}
  We generated $t=500$ random samples from the null
  distribution, and computed the KL divergence from $f$ to $g_{j}$ for
  each $1 \le j \le t$.

  \smallskip
 The mean KL value is
  $\mu_{\mathit{KL}} = 1.17$, with a standard deviation of $\sigma_{\mathit{KL}}=0.18$, that is, Iris PCA is clusterable.
\end{frame}


\begin{frame}{Clustering Tendency: Distance Distribution}
We can compare the pairwise point
distances from $\bD$, with those from the randomly generated
samples $\bR_i$ from the null distribution. 

\medskip

We create the EPMF from the proximity matrix $\bW$ for $\bD$ by binning the distances into $b$ bins:

\begin{align*}
  f(i) = P(w_{pq} \in \text{ bin } i \;|\; \bx_p, \bx_q \in \bD, p<q) =
  \frac{\bigl|\{w_{pq} \in \text{ bin } i\}\bigr|}{n(n-1)/2}
\end{align*}
Likewise, for each of the samples $\bR_{j}$, we determine the
EPMF for the pairwise distances, denoted $g_{j}$. 

\medskip
F{i}nally, we 
compute the KL divergences between $f$ and $g_{j}$.
The expected divergence
indicates the extent to which $\bD$ differs from the null (random)
distribution.
\end{frame}


\readdata{\dataHoneD}{CLUST/eval/figs/tendency1dhist.txt}
\readdata{\dataKLoneD}{CLUST/eval/figs/kl1d.txt}
\begin{frame}[fragile]{Iris PCA Data $\times$ Uniform: Distance Distribution}
\setcounter{subfigure}{0}
The distance distribution is obtained by binning the edge weights
between all pairs of points using $b=25$ bins.

\begin{figure}
  \centering
  \captionsetup[subfloat]{captionskip=30pt}
    \def\pshlabel#1{ {\footnotesize #1}}
    \def\psvlabel#1{ {\footnotesize #1}}
  \centerline{
  \subfloat[]{
    \label{fig:clust:eval:irisDistdist}
  \scalebox{0.9}{%
  \psset{xAxisLabel=Pairwise distance,yAxisLabel=Probability,%
  xAxisLabelPos={c,-0.015},yAxisLabelPos={-0.9,c}}
  \pslegend[rt]{%
  \psline[linewidth=3pt,linecolor=darkgray](0,0)(0.3,0) & $\qquad$Iris
  ($f$)\\
  \psline[linewidth=3pt,linecolor=lightgray](0,0)(0.3,0) &
  $\qquad$Uniform ($g_{j}$)}
  \begin{psgraph}[Dy=0.01]{->}(0,0)(7,0.11){3.75in}{2in}
    \listplot[plotstyle=bar,barwidth=0.1cm,
          fillcolor=darkgray,fillstyle=solid,
          plotNoMax=3,plotNoX=2,plotNo=2]{\dataHoneD}
          \pstScalePoints(1,1){0.09 add}{}
          \listplot[plotstyle=bar,barwidth=0.1cm,
          fillcolor=lightgray,fillstyle=solid,
          plotNoMax=3,plotNoX=2,plotNo=3]{\dataHoneD}
          \listplot[showpoints=false,linestyle=dashed,
          plotNoMax=3,plotNoX=2,plotNo=3]{\dataHoneD}
          \pstScalePoints(1,1){}{}
          \listplot[showpoints=false,linewidth=1.5pt,
          plotNoMax=3,plotNoX=2,plotNo=2]{\dataHoneD}
  \end{psgraph}
  }}}
\end{figure}
\end{frame}

\begin{frame}[fragile]{Iris PCA Data $\times$ Uniform: Distance Distribution}

We compute the KL divergence from $\bD$ to each $\bR_{\!j}$, over
 $t=500$ samples. 
 The mean divergence is
 $\mu_{\mathit{KL}} = 0.18$, with standard deviation $\sigma_{\mathit{KL}}=0.017$.
 Even though the Iris dataset
 has a good clustering tendency, the KL divergence is not very large.

\begin{figure}
  \centering
  \captionsetup[subfloat]{captionskip=30pt}
    \def\pshlabel#1{ {\footnotesize #1}}
    \def\psvlabel#1{ {\footnotesize #1}}
    \centerline{
    \subfloat[]{
    \label{fig:clust:eval:randomKL}
  \scalebox{0.9}{%
  \psset{xAxisLabel=KL divergence,yAxisLabel=Probability,%
  xAxisLabelPos={c,-0.0475},yAxisLabelPos={-0.02,c}}
  \begin{psgraph}[Dy=0.05,Dx=0.02,Ox=0.12]{->}(0,0)(0.12,0.25){3.5in}{1.75in}
    \pstScalePoints(1,1){-0.12 add}{}
    \listplot[plotstyle=bar,barwidth=0.2cm,
          fillcolor=lightgray,fillstyle=solid,
          plotNoMax=2,plotNoX=2,plotNo=2]{\dataKLoneD}
  \end{psgraph}
  }}}
\end{figure}

We conclude that, at least for the Iris dataset, the distance distribution is
not as discriminative as the spatial histogram approach for
clusterability analysis.
\end{frame}



\begin{frame}{Clustering Tendency: Hopkins Statistic}
Given a dataset $\bD$ comprising $n$ points,
we generate $t$ uniform subsamples $\bR_i$ of $m$ points each, 
sampled from the same dataspace as $\bD$.

\medskip
We also generate $t$ subsamples of $m$ points
directly from $\bD$, using sampling without replacement. Let $\bD_i$
denote the $i$th direct subsample. 

\medskip
Next, we compute the minimum distance
between each point $\bx_{j} \in \bD_i$ and points in $\bD$
\begin{align*}
  \dist_{\min}(\bx_{\!j}) = \min_{\bx_i \in \bD, \bx_i \ne \bx_{\!j}}
  \Bigl\{\norm{\bx_{\!j}- \bx_i}\Bigr\}
\end{align*}

We
also compute the minimum distance $\dist_{\min}(\by_{j})$
between a point $\by_{j} \in \bR_i$ and
points in $\bD$.

\medskip
The Hopkins statistic (in $d$ dimensions) for the $i$th pair of samples $\bR_i$ and $\bD_i$ is then def\/{i}ned as
\begin{align*}
  \mathit{HS}_i = \frac{\sum_{\by_{j} \in \bR_i} \lB(\dist_{\min}(\by_{j})\rB)^d}
  {\sum_{\by_{j} \in \bR_i} \lB(\dist_{\min}(\by_{j})\rB)^d +
  \sum_{\bx_{j} \in \bD_i} \lB(\dist_{\min}(\bx_{j})\rB)^d}
\end{align*}
If the data is well clustered we expect $\dist_{\min}(\bx_{j})$ values to
be smaller compared to the $\dist_{\min}(\by_{j})$ values, and in this
case $\mathit{HS}_i$ tends to 1. 
\end{frame}


\readdata{\dataHS}{CLUST/eval/figs/hopkinsdat.txt}
\psset{arrowscale=2}
\def\pshlabel#1{ {\footnotesize #1}}
\def\psvlabel#1{ {\footnotesize #1}}
\begin{frame}{Iris PCA Data $\times$ Uniform: Hopkins Statistic Distribution}
\framesubtitle{Number of sample pairs $t=500$, subsample size $m=30$.}
\begin{figure}
\begin{center}
  \hspace*{-40pt}
	\scalebox{0.9}{
    \begin{pspicture}(-3,-1)(3,5)
  \psset{xAxisLabel=Hopkins Statistic,yAxisLabel=Probability,%
  xAxisLabelPos={c,-0.025},yAxisLabelPos={-0.025,c}}
  \begin{psgraph}[Dy=0.05,Dx=0.02,Ox=0.82,showorigin=false]{->}(0,0)(0.18,0.15){4in}{2in}
    \pstScalePoints(1,1){-0.82 add}{}
    \listplot[plotstyle=bar,barwidth=0.1cm,
          fillcolor=lightgray,fillstyle=solid,
          plotNoMax=2,plotNoX=2,plotNo=2]{\dataHS}
  \end{psgraph}
  \end{pspicture}
  }
\end{center}
\end{figure}
  The  Hopkins statistic has $\mu_{\mathit{HS}} = 0.935$ and $\sigma_{\mathit{HS}}=0.025$. 

\medskip

Given the high value of the
  statistic, we conclude that the Iris dataset has a good clustering
  tendency.

\end{frame}
