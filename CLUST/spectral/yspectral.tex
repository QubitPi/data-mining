\lecture{spectral}{spectral}

\date{Chapter 16: Spectral \& Graph Clustering}

\begin{frame}
\titlepage
\end{frame}

\begin{frame}{Graphs and Matrices: Adjacency Matrix}
Given a dataset $\bD = \{\bx_i \}_{i=1}^n$ consisting of $n$ points in $\setR^d$, let $\bA$
denote the $n \times n$ symmetric {\em similarity matrix} between
the points, given as
\begin{align*}
    \bA = \matr{
        a_{11} & a_{12} & \cdots & a_{1n}\\
        a_{21} & a_{22} & \cdots & a_{2n}\\
        \vdots & \vdots & \cdots & \vdots\\
        a_{n1} & a_{n2} & \cdots & a_{nn}\\
    }
\end{align*}
where $\bA(i,j) = a_{ij}$ denotes the similarity or aff\/{i}nity
between points $\bx_i$ and $\bx_{j}$. 

\medskip
We require the similarity to
be symmetric and non-negative, that is, $a_{ij} = a_{ji}$ and $a_{ij}
\ge 0$, respectively. 

\medskip
The matrix $\bA$ is 
the {\em weighted adjacency matrix} for the data graph.
If all aff\/{i}nities are 0 or 1, then $\bA$ represents
the regular adjacency relationship between the vertices.
\end{frame}


\begin{frame}{Iris Similarity Graph: Mutual Nearest Neighbors}
  \framesubtitle{$|V|=n=150$, $|E|=m=1730$}
\begin{figure}
    \centerline{
        \scalebox{0.75}{
        \psset{unit=0.75in,dotscale=2}
        \begin{pspicture}(5,5)
            \input{CLUST/spectral/iris-graph}
        \end{pspicture}
        }
    }
\end{figure}
Edge weight given as
\begin{align*}
    a_{ij} =
    \exp \lB\{-{\norm{\bx_i - \bx_{j}}^2 \over 2\sigma^2} \rB\}
\end{align*}
using $\sigma=1$. An edge exists if both nodes are mutual nearest
neighbors among top 15 neighbors. Self-loops not shown.
\end{frame}



\begin{frame}{Graphs and Matrices: Degree Matrix}
For a vertex $\bx_i$, let $d_i$ denote the {\em degree} of the
vertex, def\/{i}ned as
\begin{align*}
    d_i = \sum_{j=1}^n a_{ij}
\end{align*}

We def\/{i}ne the {\em
degree matrix} $\bDelta$ of graph $G$ as the $n \times n$ diagonal
matrix:
\begin{align*}
    \bDelta = \matr{
      d_1 & 0 & \cdots & 0 \\
      0 & d_2 & \cdots & 0\\
      \vdots & \vdots & \ddots & \vdots\\
      0 & 0 &\cdots  & d_n \\
    } =
    \matr{
      \sum_{j=1}^n a_{1j} & 0 & \cdots & 0 \\
      0 & \sum_{j=1}^n a_{2j} & \cdots & 0\\
      \vdots & \vdots & \ddots & \vdots\\
      0 & 0 &\cdots  & \sum_{j=1}^n a_{nj} \\
    }
\end{align*}
$\bDelta$ can be compactly written as $\bDelta(i,i) = d_i$ for all
$1 \le i \le n$.
\end{frame}



\begin{frame}{Graphs and Matrices: Normalized Adjacency Matrix}
The normalized adjacency
matrix is obtained by dividing each row of the adjacency matrix by
the degree of the corresponding node. Given the weighted 
adjacency
matrix $\bA$ for a graph $G$, its normalized adjacency matrix is
def\/{i}ned as
\begin{empheq}[box=\tcbhighmath]{align*}
\begin{split}
    \bM =  \bDelta^{-1} \bA & =
    \matr{
        {a_{11}\over d_1} &  {a_{12}\over d_1} &
        \cdots &  {a_{1n}\over d_1}\\[1ex]
        {a_{21}\over d_2} &  {a_{22}\over d_2} &
        \cdots &  {a_{2n}\over d_2}\\[1ex]
        \vdots & \vdots & \ddots & \vdots\\
        {a_{n1}\over d_n} &  {a_{n2}\over d_n} & \cdots &
        {a_{nn}\over d_n}\\
    }
\end{split}
\end{empheq}
Because $\bA$ is assumed to have non-negative elements, this implies
that each element of $\bM$, namely $m_{ij}$ is also non-negative,
as $m_{ij} = \tfrac{a_{ij}}{d_i} \ge 0$. 


\medskip
Each row in $\bM$ sums to $1$, which implies that $1$ is an
eigenvalue of $\bM$. In fact, $\lambda_1 = 1$ is the largest
eigenvalue of $\bM$, and the other eigenvalues satisfy the
property that $|\lambda_i| \le 1$. 
Because $\bM$
is not symmetric, its eigenvectors are not necessarily orthogonal.
\end{frame}


\begin{frame}{Example Graph}
\framesubtitle{Adjacency and Degree Matrices}
\begin{figure}
    \centerline{
	\scalebox{0.7}{
        \psset{unit=0.75in,dotscale=2,fillcolor=lightgray,dotstyle=Bo}
        \begin{pspicture}(0,-0.25)(0,2.25)
            \psmatrix[mnode=circle]
                & [name=a] 1 & & & [name=f]6\\
                [name=b]2 & & [name=d]4 & [name=e]5 & [mnode=none]\\
                & [name=c]3  & & & [name=g]7
            \endpsmatrix
   %         \circlenode*(2,3){a}{1}
            %\circlenode*(1,2){b}{2}
            %\circlenode*(2,1){c}{3}
            %\circlenode*(3,2){d}{4}
            %\circlenode*(4,2){e}{5}
            %\circlenode*(5,3){f}{6}
            %\circlenode*(5,1){g}{7}
            \ncline{a}{b}
            \ncline{a}{d}
            \ncline{a}{f}
            \ncline{b}{c}
            \ncline{b}{d}
            \ncline{c}{d}
            \ncline{c}{g}
            \ncline{d}{e}
            \ncline{e}{f}
            \ncline{e}{g}
            \ncline{f}{g}
        \end{pspicture}
		}}
 \end{figure}
 \small
    Its adjacency and degree matrices are given as
    \begin{align*}
        \bA &= \matr{0 & 1 & 0 & 1 & 0 & 1 & 0\\
                    1 & 0 & 1 & 1 & 0 & 0 & 0\\
                    0 & 1 & 0 & 1 & 0 & 0 & 1\\
                    1 & 1 & 1 & 0 & 1 & 0 & 0\\
                    0 & 0 & 0 & 1 & 0 & 1 & 1\\
                    1 & 0 & 0 & 0 & 1 & 0 & 1\\
                    0 & 0 & 1 & 0 & 1 & 1 & 0} &
        \bDelta & = \matr{
                3 & 0 & 0 & 0 & 0 & 0 & 0\\
                0 & 3 & 0 & 0 & 0 & 0 & 0\\
                0 & 0 & 3 & 0 & 0 & 0 & 0\\
                0 & 0 & 0 & 4 & 0 & 0 & 0\\
                0 & 0 & 0 & 0 & 3 & 0 & 0\\
                0 & 0 & 0 & 0 & 0 & 3 & 0\\
                0 & 0 & 0 & 0 & 0 & 0 & 3}
    \end{align*}
\end{frame}


\begin{frame}{Example Graph}
\framesubtitle{Normalized Adjacency Matrix}
\begin{figure}
    \centerline{
	\scalebox{0.7}{
        \psset{unit=0.75in,dotscale=2,fillcolor=lightgray,dotstyle=Bo}
        \begin{pspicture}(0,-0.25)(0,2.25)
            \psmatrix[mnode=circle]
                & [name=a] 1 & & & [name=f]6\\
                [name=b]2 & & [name=d]4 & [name=e]5 & [mnode=none]\\
                & [name=c]3  & & & [name=g]7
            \endpsmatrix
   %         \circlenode*(2,3){a}{1}
            %\circlenode*(1,2){b}{2}
            %\circlenode*(2,1){c}{3}
            %\circlenode*(3,2){d}{4}
            %\circlenode*(4,2){e}{5}
            %\circlenode*(5,3){f}{6}
            %\circlenode*(5,1){g}{7}
            \ncline{a}{b}
            \ncline{a}{d}
            \ncline{a}{f}
            \ncline{b}{c}
            \ncline{b}{d}
            \ncline{c}{d}
            \ncline{c}{g}
            \ncline{d}{e}
            \ncline{e}{f}
            \ncline{e}{g}
            \ncline{f}{g}
        \end{pspicture}
		}}
		\vspace{-0.2in}
 \end{figure}
 \small
    The normalized adjacency matrix is as follows:
    \begin{align*}
        \bM & = \bDelta^{-1}\bA =
        \amatr{r}{
        0& 0.33& 0& 0.33& 0& 0.33& 0\\
        0.33& 0& 0.33& 0.33& 0& 0& 0\\
        0& 0.33& 0& 0.33& 0& 0& 0.33\\
        0.25& 0.25& 0.25& 0& 0.25& 0& 0\\
        0& 0& 0& 0.33& 0& 0.33& 0.33\\
        0.33& 0& 0& 0& 0.33& 0& 0.33\\
        0& 0& 0.33& 0& 0.33& 0.33& 0\\
        }
    \end{align*}
    The eigenvalues of $\bM$ are:
    $\lambda_1   = 1 $,
    $\lambda_2  =0.483 $,
    $\lambda_3  =0.206$,
    $\lambda_4  = -0.045$,
    $\lambda_5  =-0.405 $,
    $\lambda_6  = -0.539 $,
    $\lambda_7 =-0.7$
  \end{frame}



\begin{frame}{Graph Laplacian Matrix}
The {\em
Laplacian matrix} of a graph is def\/{i}ned as
\begin{align*}
    \bL & = \bDelta - \bA \nonumber\\ %\label{eq:clus:spectral:L}\\
    & =     \matr{
      \sum_{j=1}^n a_{1j} & 0 & \cdots & 0 \\
      0 & \sum_{j=1}^n a_{2j} & \cdots & 0\\
      \vdots & \vdots & \ddots & \vdots\\
      0 & 0 &\cdots  & \sum_{j=1}^n a_{nj} \\
    }
    -
    \matr{
        a_{11} & a_{12} & \cdots & a_{1n}\\
        a_{21} & a_{22} & \cdots & a_{2n}\\
        \vdots & \vdots & \cdots & \vdots\\
        a_{n1} & a_{n2} & \cdots & a_{nn}\\
    } \notag\\
     & =    
\tcbhighmath{\matr{
    \sum_{j\ne1} a_{1j} & -a_{12} & \cdots & -a_{1n} \\
    -a_{21} & \sum_{j\ne2} a_{2j} & \cdots & -a_{2n}\\
      \vdots & \vdots & \cdots & \vdots\\
      -a_{n1} & -a_{n2} &\cdots  & \sum_{j\ne n} a_{nj} \\
    } 
}
\end{align*}

$\bL$ is a symmetric, positive
semidef\/{i}nite matrix.
This means that $\bL$ has $n$ real, non-negative eigenvalues,
which can be arranged in decreasing order as follows: $\lambda_1
\ge \lambda_2 \ge \cdots \ge \lambda_n \ge 0$.  Because $\bL$ is
symmetric, its eigenvectors are orthonormal. 
The rank of $\bL$ is at most $n-1$, and the
smallest eigenvalue is $\lambda_n = 0$.
\end{frame}

\begin{frame}{Graph Laplacian Matrix}

The Laplacian matrix is used to compute some properties of its corresponding graph. For example:

\begin{enumerate} 
	\item The number of connected components of graph G is equal to the multiplicity of the Laplacian eigenvalue 0.
	\item The cofactor of any element of L(G) is equal to the number of spanning trees of G.
\end{enumerate}

\medskip

Laplacian matrix is usually used when the graph is irregular. Its main advantage is that it enables us to study certain properties of irregular graphs (unlike adjacent matrix).

\medskip

Given a graph function, the Laplacian measures how much the function on a graph differs at a vertex from the average of the values of the same function over the neighbors of the vertex. 

\end{frame}



\begin{frame}{Example Graph: Laplacian Matrix}
\begin{figure}
    \centerline{
	\scalebox{0.7}{
        \psset{unit=0.75in,dotscale=2,fillcolor=lightgray,dotstyle=Bo}
        \begin{pspicture}(0,-0.25)(0,2.25)
            \psmatrix[mnode=circle]
                & [name=a] 1 & & & [name=f]6\\
                [name=b]2 & & [name=d]4 & [name=e]5 & [mnode=none]\\
                & [name=c]3  & & & [name=g]7
            \endpsmatrix
   %         \circlenode*(2,3){a}{1}
            %\circlenode*(1,2){b}{2}
            %\circlenode*(2,1){c}{3}
            %\circlenode*(3,2){d}{4}
            %\circlenode*(4,2){e}{5}
            %\circlenode*(5,3){f}{6}
            %\circlenode*(5,1){g}{7}
            \ncline{a}{b}
            \ncline{a}{d}
            \ncline{a}{f}
            \ncline{b}{c}
            \ncline{b}{d}
            \ncline{c}{d}
            \ncline{c}{g}
            \ncline{d}{e}
            \ncline{e}{f}
            \ncline{e}{g}
            \ncline{f}{g}
        \end{pspicture}
		}}
		\vspace{-0.2in}
 \end{figure}
\small
 The graph Laplacian is given as
    \begin{align*}
        \bL = \bDelta-\bA =
        \amatr{r}{
                 3 &  -1 &   0 &  -1 &   0 &  -1 &   0\\
                -1 &   3 &  -1 &  -1 &   0 &   0 &   0\\
                 0 &  -1 &   3 &  -1 &   0 &   0 &  -1\\
                -1 &  -1 &  -1 &   4 &  -1 &   0 &   0\\
                 0 &   0 &   0 &  -1 &   3 &  -1 &  -1\\
                -1 &   0 &   0 &   0 &  -1 &   3 &  -1\\
                 0 &   0 &  -1 &   0 &  -1 &  -1 &   3}
    \end{align*}
    The eigenvalues of $\bL$ are as follows:
    $\lambda_1= 5.618$,
    $\lambda_2=4.618$,
    $\lambda_3= 4.414 $,
    $\lambda_4=3.382$,
    $\lambda_5=2.382 $,
    $\lambda_6=1.586 $,
    $\lambda_7= 0$  
\end{frame}


\begin{frame}{Normalized Laplacian Matrices}
  \small
The {\em normalized
symmetric Laplacian matrix} of a graph is def\/{i}ned as
\begin{empheq}[box=\tcbhighmath]{align*}
\begin{split}
    \bL^s & = \bDelta^{-1/2}\bL\bDelta^{-1/2}
          =    \matr{
     {\sum_{j\ne1} a_{1j} \over \sqrt{d_1d_1}} & -{a_{12} \over
     \sqrt{d_1d_2}} & \cdots & -{a_{1n} \over \sqrt{d_1d_n}}
     \\[1ex]
     -{a_{21} \over \sqrt{d_2d_1}} & {\sum_{j\ne2}a_{2j} \over
     \sqrt{d_2d_2}} & \cdots & -{a_{2n} \over \sqrt{d_2d_n}} \\
      \vdots & \vdots & \ddots & \vdots\\
      -{a_{n1} \over \sqrt{d_nd_1}} & -{a_{n2} \over
      \sqrt{d_nd_2}} & \cdots & {\sum_{j\ne n}a_{nj} \over
      \sqrt{d_nd_n}} \\
      } 
\end{split}
\end{empheq}
$\bL^s$ is a symmetric, positive semidef\/{i}nite matrix, with 
rank at most $n-1$. The
smallest eigenvalue $\lambda_n = 0$.
The {\em normalized asymmetric Laplacian} matrix is def\/{i}ned as
\begin{empheq}[box=\tcbhighmath]{align*}
\begin{split}
    \bL^a &= \bDelta^{-1} \bL
     = \matr{
    {\sum_{j\ne1} a_{1j} \over d_1} & -{a_{12} \over d_1} &
    \cdots & -{a_{1n} \over d_1} \\[1ex]
    -{a_{21}\over d_2} & {\sum_{j\ne2} a_{2j} \over d_2}
    & \cdots & -{a_{2n} \over d_2}\\
      \vdots & \vdots & \ddots & \vdots\\
      -{a_{n1} \over d_n} & -{a_{n2} \over d_n} &\cdots
      & {\sum_{j\ne n} a_{nj} \over d_n} \\
    }
\end{split}
\end{empheq}
$\bL^a$ is also a
positive semi-def\/{i}nite matrix with $n$ real eigenvalues $\lambda_1
\ge \lambda_2 \ge \dots \ge \lambda_n = 0$.
\end{frame}



\begin{frame}{Example Graph}
\framesubtitle{Normalized Symmetric Laplacian Matrix}
\begin{figure}
    \centerline{
	\scalebox{0.6}{
        \psset{unit=0.75in,dotscale=2,fillcolor=lightgray,dotstyle=Bo}
        \begin{pspicture}(0,-0.25)(0,2.25)
            \psmatrix[mnode=circle]
                & [name=a] 1 & & & [name=f]6\\
                [name=b]2 & & [name=d]4 & [name=e]5 & [mnode=none]\\
                & [name=c]3  & & & [name=g]7
            \endpsmatrix
            \ncline{a}{b}
            \ncline{a}{d}
            \ncline{a}{f}
            \ncline{b}{c}
            \ncline{b}{d}
            \ncline{c}{d}
            \ncline{c}{g}
            \ncline{d}{e}
            \ncline{e}{f}
            \ncline{e}{g}
            \ncline{f}{g}
        \end{pspicture}
		}}
		\vspace{-0.2in}
 \end{figure}
\small
The    normalized symmetric Laplacian is given as
    \begin{align*}
        \bL^s & = \amatr{r}{
     1 &-0.33 & 0 &-0.29 & 0 &-0.33 & 0\\
    -0.33 & 1 &-0.33 &-0.29 & 0 & 0 & 0\\
     0 &-0.33 & 1 &-0.29 & 0 & 0 &-0.33\\
    -0.29 &-0.29 &-0.29 & 1 &-0.29 & 0 & 0\\
     0 & 0 & 0 &-0.29 & 1 &-0.33 &-0.33\\
    -0.33 & 0 & 0 & 0 &-0.33 & 1 &-0.33\\
     0 & 0 &-0.33 & 0 &-0.33 &-0.33 & 1\\
    }
    \end{align*}
    The eigenvalues of $\bL^s$ are as follows:
    $\lambda_1 = 1.7$,
    $\lambda_2 =1.539$,
    $\lambda_3 = 1.405$,
    $\lambda_4 =1.045$,
    $\lambda_5 =0.794$,
    $\lambda_6 =0.517$,
    $\lambda_7 = 0 $
\end{frame}



\begin{frame}{Example Graph}
\framesubtitle{Normalized Asymmetric Laplacian Matrix}
\begin{figure}
    \centerline{
	\scalebox{0.6}{
        \psset{unit=0.75in,dotscale=2,fillcolor=lightgray,dotstyle=Bo}
        \begin{pspicture}(0,-0.25)(0,2.25)
            \psmatrix[mnode=circle]
                & [name=a] 1 & & & [name=f]6\\
                [name=b]2 & & [name=d]4 & [name=e]5 & [mnode=none]\\
                & [name=c]3  & & & [name=g]7
            \endpsmatrix
            \ncline{a}{b}
            \ncline{a}{d}
            \ncline{a}{f}
            \ncline{b}{c}
            \ncline{b}{d}
            \ncline{c}{d}
            \ncline{c}{g}
            \ncline{d}{e}
            \ncline{e}{f}
            \ncline{e}{g}
            \ncline{f}{g}
        \end{pspicture}
		}}
		\vspace{-0.2in}
 \end{figure}
\small
    The normalized asymmetric Laplacian matrix is given as
    \begin{align*}
        \bL^a = \bDelta^{-1}\bL =
        \amatr{r}{
         1   & -0.33&  0   & -0.33&  0   & -0.33&  0\\
        -0.33&  1   & -0.33& -0.33&  0   &  0   &  0\\
         0   & -0.33&  1   & -0.33&  0   &  0   & -0.33\\
        -0.25& -0.25& -0.25&  1   & -0.25&  0   &  0\\
         0   &  0   &  0   & -0.33&  1   & -0.33& -0.33\\
        -0.33&  0   &  0   &  0   & -0.33&  1   & -0.33\\
         0   &  0   & -0.33&  0   & -0.33& -0.33&  1}
    \end{align*}
    The eigenvalues of $\bL^a$ are identical to those for $\bL^s$,
    namely
    $\lambda_1 = 1.7$,
    $\lambda_2 =1.539$,
    $\lambda_3 = 1.405$,
    $\lambda_4 =1.045$,
    $\lambda_5 =0.794$,
    $\lambda_6 =0.517$,
    $\lambda_7 = 0 $
  \end{frame}

\ifdefined\wox \begin{frame} \titlepage \end{frame} \fi

\begin{frame}{Clustering as Graph Cuts}
A {\em $k$-way cut} in a graph is a partitioning or clustering of
the vertex set, given as $\cC = \{C_1, \ldots, C_k\}$.
We require $\cC$ to optimize
some objective function that captures the intuition that nodes
within a cluster should have high similarity, and nodes from
different clusters should have low similarity.

\medskip
Given a weighted graph $G$ def\/{i}ned by its similarity matrix $\bA$,
let $S, T \subseteq V$ be any two
subsets of the vertices. We denote by $W(S,T)$ the sum of the
weights on all edges with one vertex in $S$ and the other in $T$,
given as
\begin{align*}
\tcbhighmath{
    W(S, T) = \sum_{v_i \in S} \sum_{v_{j} \in T} a_{ij}
}
\end{align*}

\medskip
Given $S \subseteq V$, we denote by $\ol{S}$ the complementary set
of vertices, that is, $\ol{S} = V - S$. A {\em (vertex) cut} in a
graph is def\/{i}ned as a partitioning of $V$ into $S \subset V$ and
$\ol{S}$. The {\em weight of the cut} or {\em cut weight} is
def\/{i}ned as the sum of all the weights on
edges between vertices in $S$
and $\ol{S}$, given as $W(S, \ol{S})$.
\end{frame}


\begin{frame}{Cuts and Matrix Operations}
  \small
Given a clustering $\cC = \{C_1,\dots,C_k\}$ comprising $k$ clusters.
Let $\bc_i \in \{0,1\}^n$ be the {\em cluster indicator vector}
that records the cluster membership for cluster $C_i$, def\/{i}ned as
\begin{align*}
    c_{ij} =
        \begin{cases}
            1 & \text{if $v_{j} \in C_i$}\\
            0 & \text{if $v_{j} \not\in C_i$}\\
        \end{cases}
\end{align*}
The cluster size can be written as
\begin{align*}
\tcbhighmath{
    \card{C_i} = \bc_i^T\bc_i = \norm{\bc_i}^2
}
\end{align*}
The {\em volume} of a cluster
$C_i$ is def\/{i}ned as the sum of all the weights on edges with one
end in cluster $C_i$:
\begin{align*}
\tcbhighmath{
    vol(C_i) = W(C_i, V)  = \sum_{v_r \in C_i} d_r
    = \sum_{v_r \in C_i} c_{ir} d_r c_{ir}
     = \sum_{r=1}^n \sum_{s=1}^n c_{ir} \bDelta_{rs} c_{is}
     = \bc_i^T \bDelta \bc_i
}
\end{align*}
\end{frame}

\begin{frame}{Cuts and Matrix Operations}
The sum of weights of all internal edges is:
\begin{empheq}[box=\tcbhighmath]{align*}
\begin{split}
    W(C_i, C_i) & = \sum_{v_r \in C_i} \sum_{v_s \in C_i}
    a_{rs}
     = \sum_{r=1}^n \sum_{s=1}^n c_{ir} a_{rs} c_{is} = \bc_i^T
    \bA \bc_i
\end{split}
\end{empheq}
We can get the sum of weights for all the external edges as follows:
\begin{empheq}[box=\tcbhighmath]{align*}
\begin{split}
    W(C_i, \ol{C_i}) & = \sum_{v_r \in C_i} \sum_{v_s \in
    V- C_i} a_{rs} = W(C_i,V) - W(C_i,C_i)
     = \bc_i (\bDelta - \bA) \bc_i = \bc_i^T \bL \bc_i
\end{split}
\end{empheq}
\end{frame}



\begin{frame}{Clustering Objective Functions: Ratio Cut}
The clustering objective function can be formulated as an
optimization \hbox{problem} over the $k$-way cut $\cC =
\{C_1,\dots,C_k\}$. 

The {\em ratio cut} objective is def\/{i}ned
over a $k$-way cut as follows:
\begin{empheq}[box=\tcbhighmath]{align*}
\begin{split}
    \min_\cC \; J_{rc}(\cC) & = \sum_{i=1}^k {W(C_i, \ol{C_i}) \over
    \card{C_i}} =  \sum_{i=1}^k {\bc_i^T \bL \bc_i \over
    \bc_i^T\bc_i} = \sum_{i=1}^k {\bc_i^T \bL \bc_i \over
    \norm{\bc_i}^2}
\end{split}
\end{empheq}
Ratio cut tries to minimize the sum of the similarities from a
cluster $C_i$ to other points not in the cluster $\ol{C_i}$,
taking into account the size of each cluster. 


Unfortunately, for binary cluster indicator vectors $\bc_i$, the
ratio cut objective is NP-hard. An obvious relaxation is to allow
$\bc_i$ to take on any real value. In this case, we can rewrite
the objective as
\begin{align*}
    \min_\cC \; J_{rc}(\cC) = \sum_{i=1}^k {\bc_i^T \bL \bc_i
    \over \norm{\bc_i}^2} =
    \sum_{i=1}^k \cramped{\lB({\bc_i \over \norm{\bc_i}}\rB)^T}
        \bL
        \cramped{\lB({\bc_i \over \norm{\bc_i}}\rB)}
        =  \sum_{i=1}^k \bu_i^T \bL \bu_i
\end{align*}
The optimal solution comprises the eigenvectors corresponding to the 
$k$ smallest eigenvalues of $\bL$, i.e., the eigenvectors
$\bu_n, \bu_{n-1}, \dots, \bu_{n-k+1}$ represent the relaxed
cluster indicator vectors.
\end{frame}


\begin{frame}{Clustering Objective Functions: Normalized Cut}

{\em Normalized cut} is similar to
ratio cut, except that it divides the cut weight of each cluster
by the volume of a cluster instead of its size. The objective
function is given as
\begin{align*}
\tcbhighmath{
    \min_\cC \; J_{nc}(\cC) = \sum_{i=1}^k {W(C_i, \ol{C_i}) \over
    vol(C_i)} = \sum_{i=1}^k {\bc_i^T \bL \bc_i  \over
    \bc_i^T \bDelta \bc_i}
}
\end{align*}
We can obtain an optimal solution by allowing $\bc_i$ to 
be an arbitrary real vector. 

\medskip
The optimal solution comprise the eigenvectors corresponding to the 
$k$ smallest eigenvalues of either the normalized symmetric or asymmetric
 Laplacian matrices, $\bL^s$ and $\bL^a$.
\end{frame}




\begin{frame}[fragile]{Spectral Clustering Algorithm}

The spectral clustering algorithm takes a dataset $\bD$ as
input and computes the similarity matrix $\bA$.  
For normalized cut we chose either $\bL^s$
or $\bL^a$, whereas for ratio cut we choose $\bL$. 
Next, we
compute the $k$ smallest eigenvalues and corresponding 
eigenvectors of the chosen matrix.

\medskip
The main problem is that the eigenvectors $\bu_i$
are not binary, and thus it is not immediately clear how we can
assign points to clusters. 

\medskip
One solution to this problem is to
treat the $n \times k$ matrix of eigenvectors as a new data matrix:
\begin{align*}
    \bU = \matr{
    | & | &  & | \\
    \bu_n & \bu_{n-1} & \cdots & \bu_{n-k+1}\\
    | & | &  & | \\
    }
	\rightarrow
	\text{ normalize rows }
	\rightarrow
	\matr{
    \mbox{---} & \by_1^T &\mbox{---}\\
    \mbox{---} & \by_2^T &\mbox{---}\\
    &\vdots&\\
    \mbox{---} & \by_n^T &\mbox{---}\\
    }
	= \bY
\end{align*}
We then cluster the new points in $\bY$ into $k$ clusters via
the K-means algorithm or any other fast clustering method to obtain
binary cluster indicator vectors $\bc_i$.
\end{frame}



\newcommand{\spectral}{\textsc{Spectral Clustering}}
\begin{frame}[fragile]{Spectral Clustering Algorithm}
\begin{tightalgo}[H]{\textwidth-18pt}
\SetKwInOut{Algorithm}{\spectral\ ($\bD, k$)} \Algorithm{} Compute
the similarity matrix $\bA \in \setR^{n\times n}$\; 
\lIf{ratio cut}{
    \nllabel{alg:clus:spectral:spectral:lineBs} $\bB \assign \bL$}
\lElseIf{normalized cut}{$\bB \assign \bL^s \text{ or } \bL^a$}
Solve $\bB \bu_i = \lambda_i \bu_i$ for $i=n,\dots,n-k+1$, where
$\lambda_n \le \lambda_{n-1} \le \dots \le \lambda_{n-k+1}$
\nllabel{alg:clus:spectral:spectral:eig}\;
$\bU \assign \matr{\bu_n & \bu_{n-1}& \cdots & \bu_{n-k+1}}$\;
$\bY \assign$ normalize rows of $\bU$ \; %using Eq.\nosp\eqref{eq:clus:spectral:yi}\;
 $\cC \assign \{C_1,\dots,C_k\}$ via K-means on $\bY$\;
\end{tightalgo}
\end{frame}



\begin{frame}{Spectral Clustering on Example Graph}
  \framesubtitle{$k=2$, normalized cut (normalized asymmetric Laplacian)}
\begin{figure}
    \centerline{
	\scalebox{0.6}{
        \psset{unit=0.75in,dotscale=2,fillcolor=lightgray,dotstyle=Bo}
        \begin{pspicture}(0,-0.25)(0,2.25)
            \psmatrix[mnode=circle]
                & [name=a] 1 & & & [name=f]6\\
                [name=b]2 & & [name=d]4 & [name=e]5 & [mnode=none]\\
                & [name=c]3  & & & [name=g]7
            \endpsmatrix
            \ncline{a}{b}
            \ncline{a}{d}
            \ncline{a}{f}
            \ncline{b}{c}
            \ncline{b}{d}
            \ncline{c}{d}
            \ncline{c}{g}
            \ncline{d}{e}
            \ncline{e}{f}
            \ncline{e}{g}
            \ncline{f}{g}
        \end{pspicture}
		}}
		\vspace{0.2in}
    \centerline{
	\scalebox{0.6}{
    \psset{xunit=5in,yunit=1in,
        dotscale=1.5,arrowscale=2,PointName=none}
    \psset{xAxisLabel=$\bu_1$,yAxisLabel= $\bu_2$}
    \psgraph[tickstyle=bottom,Dx=0.1,Ox=-1,Dy=0.5,Oy=-1]{->}%
    (-1,-1)(-1,-1)(-0.5,1){3in}{2in}
    \psset{dotstyle=Bo,fillcolor=lightgray}
    \psdots[](-0.8585673,-0.5127008)(-0.6036752,-0.7972304)%
    (-0.8585673,-0.5127008)(-0.8117383,-0.5840214)%
    (-0.6644345,0.7473465)(-0.6482062,0.7614649)%
    (-0.6482062,0.7614649)
      \uput[90](-0.86,-0.5){\small $1,3$}
        \uput[90](-0.6,-0.8){\small $2$}
        \uput[90](-0.81,-0.584){\small $4$}
        \uput[90](-0.66,0.747){\small $5$}
        \uput[30](-0.65,0.761){\small $6,7$}
    \psset{fillcolor=black}
    \pstGeonode[PointSymbol=none, dotscale=2](-0.654,0.756){A}
    \pstGeonode[PointSymbol=none, dotscale=2](-0.783,-0.602){B}
    \psclip{\psframe[](-1,-1)(-0.5,1)}%
    {
    \psset{linestyle=none, PointSymbol=none}
    \pstMediatorAB{A}{B}{K}{KP}
    \psset{linewidth=1pt,linestyle=dashed}
    \pstGeonode[PointSymbol=none](-1,-1){a}(-1,1){b}(-0.5,1){c}(-0.5,-1){d}
    \pstInterLL[PointSymbol=none]{K}{KP}{b}{c}{ku}
    \pstLineAB{K}{ku}
    \pstInterLL[PointSymbol=none]{K}{KP}{a}{d}{kd}
    \pstLineAB{K}{kd}
    }
    \endpsclip
    \endpsgraph
	}}
\end{figure}
\end{frame}


\begin{frame}{Normalized Cut on Iris Graph}
  \framesubtitle{$k=3$, normalized asymmetric Laplacian}
\begin{figure}
    \centerline{
        \scalebox{0.55}{
            \psset{unit=0.75in,dotscale=2}
            \begin{pspicture}(5,5)
                \input{CLUST/spectral/iris-graph-normalized}
            \end{pspicture}
        }
    }
\end{figure}
\centerline{
\begin{tabular}{|c|c|c|c|}
        \hline
         & {\tt setosa} & {\tt virginica} & {\tt versicolor}\\
         \hline
        $C_1$ (triangle) & 50 & 0 & 4\\
        $C_2$ (square) & 0 & 36 & 0\\
        $C_3$ (circle) & 0 & 14 & 46 \\
        \hline
    \end{tabular}
	}
\end{frame}


\begin{frame}{Maximization Objectives: Average Cut}
 The {\em average weight} objective is
def\/{i}ned as
\begin{empheq}[box=\tcbhighmath]{align*}
\begin{split}
    \max_{\cC}\; J_{aw}(\cC) & =
    \sum_{i=1}^k {W(C_i,C_i) \over |C_i|} =
    \sum_{i=1}^k {\bc^T_i \bA \bc_i
    \over \bc_i^T\bc_i}
	= \sum_{i=1}^k \bu_i^T \bA \bu_i
\end{split}
\end{empheq}
where $\bu_i$ is an arbitrary real vector, which is a relaxation of the
binary cluster indicator vectors $\bc_i$.

\medskip
We can maximize
the objective by selecting the $k$ largest eigenvalues of $\bA$,
and the corresponding eigenvectors.

\begin{align*}
    \max_{\cC} \; J_{aw}(\cC) &=
    \bu^T_1 \bA \bu_1 + \dots + \bu^T_k \bA \bu_k\\
    & = \lambda_1 + \dots + \lambda_k
\end{align*}
where $\lambda_1 \ge \lambda_2 \ge \cdots \ge \lambda_n$.
In general, while $\bA$ is symmetric, it may not be 
positive semidef\/{i}nite. This means
that $\bA$ can have negative eigenvalues, and to maximize the objective 
we must consider only the positive eigenvalues and the
corresponding eigenvectors.
\end{frame}

\begin{frame}{Maximization Objectives: Modularity}
Given $\bA$, the weighted adjacency
matrix, the modularity of a clustering is the
difference between the observed and expected fraction of weights
on edges within the clusters. 
The clustering objective is given as
\begin{align*}
    \max_{\cC} \; J_{Q}(\cC) &= \sum_{i=1}^k
    \lB( {\bc_i^T \bA \bc_i \over tr(\bDelta)} -
        {(\bd_i^T \bc_i)^2 \over tr(\bDelta)^2}
    \rB)
    = \sum_{i=1}^k \bc_i^T \bQ \bc_i
\end{align*}
where $\bQ$ is the {\em modularity matrix}:
\begin{align*}
    \bQ = {1 \over tr(\bDelta)} \lB(\bA - {\bd \cdot
    \bd^T \over tr(\bDelta)} \rB)
\end{align*}
The optimal solution comprises the eigenvectors corresponding to the $k$
largest eigenvalues of $\bQ$.
Since $\bQ$ is
symmetric, but not positive semidef\/{i}nite, we use only 
the positive eigenvalues.
\end{frame}

\ifdefined\wox \begin{frame} \titlepage \end{frame} \fi

\begin{frame}{Markov Chain Clustering}
A Markov chain is a discrete-time stochastic process
over a set of states, in our case the set of vertices $V$. 

\medskip
The
Markov chain makes a transition from one node to another at
discrete timesteps $t=1,2,\dots$, with the probability of making
a transition from node $i$ to node $j$ given as $m_{ij}$. 

\medskip
Let the
random variable $X_t$ denote the state at time $t$. The Markov
property means that the probability distribution of $X_t$ over the
states at time $t$ depends only on the probability distribution of
$X_{t-1}$, that is,
\begin{align*}
P(X_{t}=i | X_0, X_1, \dots, X_{t-1}) = P(X_{t}=i | X_{t-1})
\end{align*}
Further, we assume that the Markov chain is {\em homogeneous},
that is, the transition probability
\begin{align*}
P(X_t = j | X_{t-1} = i) = m_{ij}
\end{align*}
is independent of the time step $t$.
\end{frame}



\begin{frame}{Markov Chain Clustering: Markov Matrix}
The normalized adjacency matrix
$\bM = \bDelta^{-1} \bA$ can be interpreted as
the $n\times n$ {\em transition matrix} where the entry $m_{ij} =
{a_{ij} \over d_i}$ is the probability of
transitioning or jumping from node $i$ to node $j$ in the graph
$G$. 

\medskip
The matrix $\bM$ is thus the transition matrix for a {\em Markov
chain} 
or a Markov random walk on
graph $G$. That is,
given node $i$ the transition matrix $\bM$ specif\/{i}es the
probabilities of reaching any other node $j$ in one time step.


\medskip
In general, the transition probability matrix
for
 $t$ time steps is given as
\begin{align*}
     \bM^{t-1} \cdot \bM = \bM^t
\end{align*}
\end{frame}

\begin{frame}{Markov Chain Clustering: Random Walk}
A random walk on $G$ thus corresponds to taking successive powers
of the transition matrix $\bM$. 

\medskip
Let $\bpi_0$ specify the initial
state probability vector at time $t=0$.
The state probability vector after 
$t$ steps is
\begin{align*}
    \bpi_t^T & = \bpi_{t-1}^T \bM =
    \bpi_{t-2}^T\bM^2 = \cdots
     = \bpi_0^T \bM^{t}
\end{align*}
Equivalently, taking transpose on both sides, we
get
  $$\bpi_t  = (M^t)^T \bpi_0 = (\bM^T)^t \bpi_0$$
The state probability vector thus converges to the dominant
eigenvector of $\bM^T$.
\end{frame}


\begin{frame}{Markov Clustering Algorithm}
Consider a variation of the random walk, where the
probability of transitioning from node $i$ to $j$ is inflated by
taking each element $m_{ij}$ to the power $r \ge 1$. Given a
transition matrix $\bM$, def\/{i}ne the inflation operator $\Upsilon$
as follows:
\begin{align*}
\tcbhighmath{
    \Upsilon(\bM,r) = \lB\{
        \frac{(m_{ij})^r}{\sum_{a=1}^n (m_{ia})^r}
    \rB\}_{i,j=1}^n
}
\end{align*}
The net
effect of the inflation operator is to increase the higher
probability transitions and decrease the lower probability
transitions.

\medskip
The Markov clustering
algorithm (MCL) is an iterative method that interleaves matrix expansion
and inflation steps. Matrix expansion corresponds to taking
successive powers of the transition matrix, leading to random
walks of longer lengths. On the other hand, matrix inflation makes
the higher probability transitions even more likely and reduces
the lower probability transitions.

\medskip
MCL
takes as input the inflation parameter $r\ge1$. Higher values lead
to more, smaller clusters, whereas smaller values lead to fewer,
but larger clusters.  

\end{frame}


\begin{frame}{Markov Clustering Algorithm: MCL}
The f\/{i}nal clusters are found by enumerating the weakly connected
components in the directed graph induced by the converged
transition matrix $\bM_t$, where the edges are defined as:
\begin{align*}
    E = \bigl\{(i,j) \mid \bM_t(i,j) > 0\bigr\}
\end{align*}
A directed edge $(i,j)$ exists only if node $i$
can transition to node $j$ within $t$ steps of the expansion and
inflation process. 

\medskip
A node $j$ is called an {\em attractor} if
$\bM_t(j,j) > 0$, and we say that node $i$ is attracted to
attractor $j$ if $\bM_t(i,j) > 0$. The MCL process yields a set of
attractor nodes, $V_a \subseteq V$, such that other nodes are
attracted to at least one attractor in $V_a$. 

\medskip
To extract the
clusters from $G_t$, MCL f\/{i}rst f\/{i}nds the strongly
connected components $S_1, S_2, \dots, S_q$ over the set of
attractors $V_a$. Next, for each strongly connected set of
attractors $S_{j}$, MCL f\/{i}nds the weakly connected
 components consisting of all nodes $i \in V_t-V_a$
 attracted to an attractor in $S_{j}$. If a node $i$ is attracted to
 multiple strongly connected components, it is added to each such
 cluster, resulting in possibly overlapping clusters.
\end{frame}




\newcommand{\MCL}{{\textsc{Markov Clustering}}}
\begin{frame}{Algorithm \MCL}
\begin{tightalgo}[H]{\textwidth-18pt}
\SetKwInOut{Algorithm}{\MCL\ ($\bA, r, \epsilon$)} \Algorithm{}
$t\assign 0$\; Add self-edges to $\bA$ if they do not exist\;
$\bM_t \assign  \bDelta^{-1} \bA$\; \Repeat{$\norm{\bM_t -
\bM_{t-1}}_F \le \epsilon$}{
    $t \assign  t+1$\;
    $\bM_t \assign  \bM_{t-1} \cdot \bM_{t-1}$\;
    $\bM_t \assign  \Upsilon(\bM_t, r)$\;
} $G_t \assign  \text{directed graph induced by } \bM_t$\; $\cC
\assign  \{ \text{weakly connected components in } G_t\}$\;
\end{tightalgo}
\end{frame}


\begin{frame}{MCL Attractors and Clusters}
  \framesubtitle{$r=2.5$}
\begin{figure}
    \centerline{
	\scalebox{0.6}{
        \psset{unit=0.75in,dotscale=2,fillcolor=lightgray,dotstyle=Bo}
        \begin{pspicture}(0,-0.25)(0,2.25)
            \psmatrix[mnode=circle]
                & [name=a] 1 & & & [name=f]6\\
                [name=b]2 & & [name=d]4 & [name=e]5 & [mnode=none]\\
                & [name=c]3  & & & [name=g]7
            \endpsmatrix
            \ncline{a}{b}
            \ncline{a}{d}
            \ncline{a}{f}
            \ncline{b}{c}
            \ncline{b}{d}
            \ncline{c}{d}
            \ncline{c}{g}
            \ncline{d}{e}
            \ncline{e}{f}
            \ncline{e}{g}
            \ncline{f}{g}
        \end{pspicture}
		}}
		\vspace{0.2in}
\centerline{
\scalebox{0.8}{
\psset{unit=0.75in,fillcolor=lightgray,arrowscale=1.5}
\begin{pspicture}(0,-0.25)(0,2.25)
    \psmatrix[mnode=circle]
        & [name=a] 1 & & & [name=f,fillstyle=solid]6\\
        [name=b]2 & & [name=d,fillstyle=solid]4 & [name=e]5 & [mnode=none]\\
        & [name=c]3  & & & [name=g,fillstyle=solid]7
    \endpsmatrix
    \ncline{->}{a}{d}\naput{1}
    \ncline{->}{b}{d}\naput{1}
    \ncline{->}{c}{d}\naput{1}
    \ncline{->}{e}{f}\naput{0.5}
    \ncline{->}{e}{g}\naput{0.5}
    \ncline{-}{f}{g}\naput{0.5}
    \nccircle[angleA=-90]{->}{d}{0.15in}
    \nccircle[angleA=-90]{->}{f}{0.15in}
    \nccircle[angleA=-90]{->}{g}{0.15in}
\end{pspicture}
} }
\end{figure}
\end{frame}


\begin{frame}[fragile]{MCL on Iris Graph}
	\framesubtitle{Comparison between two values of $r$}
\begin{figure}
    \begin{center}
        \subfloat[$r=1.3$]{
        \label{fig:clust:spectral:irisMCLa}
        \scalebox{0.55}{
            \psset{unit=0.75in,dotscale=2}
            \begin{pspicture}(5,5)
                \input{CLUST/spectral/iris-graph-MCL1.3}
            \end{pspicture}
        }}
        \hspace{0.1in}
        \subfloat[$r=2$]{
        \label{fig:clust:spectral:irisMCLb}
        \scalebox{0.55}{
            \psset{unit=0.75in,dotscale=2}
            \begin{pspicture}(5,5)
                \input{CLUST/spectral/iris-graph-MCL2}
            \end{pspicture}
        }}
    \end{center}
\end{figure}
\end{frame}


\begin{frame}{Contingency Table: MCL Clusters versus Iris Types}
  \framesubtitle{$r=1.3$}

	There is a quite significant number of \texttt{iris-virginica} misplaced.

	\medskip

\renewcommand{\arraystretch}{1.1}
\begin{center}
\begin{tabular}{|c|c|c|c|}
        \hline
         & {\tt iris-setosa} & {\tt iris-virginica} & {\tt
         iris-versicolor}\\
         \hline
        $C_1$ (triangle) & 50 & 0 & 1\\
        $C_2$ (square) & 0 & 36 & 0\\
        $C_3$ (circle) & 0 & 14 & 49 \\
        \hline
    \end{tabular}
\end{center}
\end{frame}
