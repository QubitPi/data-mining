\lecture{lda}{lda}

\date{Chap.\ 20: Linear Discriminant Analysis}

\begin{frame}
\titlepage
\end{frame}

\begin{frame}{Linear Discriminant Analysis}
Given
labeled data consisting of $d$-dimensional points $\bx_i$ along
with their classes $y_i$, the goal of linear discriminant analysis
(LDA) is to f\/{i}nd a vector $\bw$ that maximizes the separation
between the classes after projection onto $\bw$. 

\medskip
The key difference between principal component analysis and LDA is
that the former deals with unlabeled data and tries to maximize
variance, whereas the latter deals with labeled data and tries to
maximize the discrimination between the classes.
\end{frame}


\begin{frame}{Projection onto a Line}
 Let $\bD_i$ denote the subset of points
labeled with class $c_i$, i.e., $\bD_i = \{\bx_{j} | y_{j} = c_i\}$,
and let $\card{\bD_i} = n_i$ denote the number of points with
class $c_i$. We assume that there are only $k=2$ classes. 


\medskip
The projection  of any $d$-dimensional
point $\bx_i$ onto a unit vector $\bw$ is given as
\begin{align*}
  \bx'_i = \lB( \frac{\bw^T \bx_i}{\bw^T\bw} \rB) \bw
    = \lB( \bw^T \bx_i \rB) \bw = a_i \bw
\end{align*}
where $a_i$ specif\/{i}es the offset or coordinate of $\bx'_i$ along
the line $\bw$:
\begin{align*}
  a_i  = \bw^T\bx_i
\end{align*}

\medskip
The set of $n$ scalars $\{a_1, a_2, \ldots, a_n\}$ represents the mapping from $\setR^d$ to $\setR$, that is, from the original
$d$-dimensional space to a 1-dimensional space (along $\bw$).
\end{frame}


\begin{frame}{Projection onto $\bw$: Iris 2D Data}
  \framesubtitle{{\tt iris-setosa} as class $c_1$
(circles), and the other two Iris types as class $c_2$
(triangles)}
\begin{figure}[!t]
    \centering
    \scalebox{0.75}{%
    \psset{dotscale=1.5,arrowscale=2,PointName=none,
    dotsep=0.05, unit=1.0in}
    \begin{pspicture}(4,1.5)(8.5,5)
    \psaxes[Dx=0.5,Dy=0.5,Ox=4.0,Oy=1.5]{->}(4.0,1.5)(8.5,5)
    \pstGeonode[PointSymbol=none](8.04,4.15){na}
    \pstGeonode[PointSymbol=none](3.94,2.1){nb}
    \pstLineAB[linewidth=2pt,arrows=<-]{na}{nb}
    \nput{90}{na}{$\bw$}
    \input{CLASS/lda/figs/proj}
        \psset{dotscale=2,fillcolor=black}
    \pstGeonode[PointSymbol=none,
        fillcolor=lightgray](5.01,3.42){mu1}
    \pstProjection[PointSymbol=Bo]{na}{nb}{mu1}[m1]
    \pstGeonode[PointSymbol=none,
        fillcolor=lightgray](6.26,2.87){mu2}
    \pstProjection[PointSymbol=Btriangle]{na}{nb}{mu2}[m2]
    \end{pspicture}
    }
\end{figure}
\end{frame}



\begin{frame}{Iris 2D Data: Optimal Linear Discriminant Direction}
\begin{figure}[!t]\vspace*{-14pt}
    \centering
    %\vspace{0.2in}
    \scalebox{0.8}{%
    \psset{dotscale=1.5,dotsep=0.03,arrowscale=2,PointName=none,unit=1.0in}
    \begin{pspicture}(4,1.5)(8.5,5)
    \psaxes[Dx=0.5,Dy=0.5,Ox=4.0,Oy=1.5]{->}(4.0,1.5)(8.5,5)
    %\begin{pspicture}(4,1.5)(8,4.5)
    %\psaxes[Dx=0.5,Dy=0.5,Ox=4.0,Oy=1.5]{->}(4.0,1.5)(8.0,4.5)
    \pstGeonode[PointSymbol=none](4.88,4.51){na}
    \pstGeonode[PointSymbol=none](6.89,1.47){nb}
    \pstLineAB[linewidth=2pt,arrows=->]{na}{nb}
    \uput[90](6.9,1.6){$\bw$}
    \input{CLASS/lda/figs/proj}
    \psset{dotscale=2,fillcolor=black}
    \pstGeonode[PointSymbol=none,
        fillcolor=lightgray](5.01,3.42){mu1}
    \pstProjection[PointSymbol=Bo]{na}{nb}{mu1}[m1]
    \pstGeonode[PointSymbol=none,
        fillcolor=lightgray](6.26,2.87){mu2}
    \pstProjection[PointSymbol=Btriangle]{na}{nb}{mu2}[m2]
    \end{pspicture}
    }
\end{figure}
\end{frame}



\begin{frame}{Optimal Linear Discriminant}
The mean of the projected points is given as:
\begin{align*}
    m_1  = & \bw^T\bmu_1 & 
	m_2 = & \bw^T\bmu_2
\end{align*}

To maximize the separation between the classes, we 
maximize the difference between the projected means,
$|m_1 - m_2|$. However, for good separation,
the variance of the projected points for each class should also
not be too large. 
LDA
maximizes the separation by ensuring that the {\em
  scatter} $s_i^2$ for the projected points within each class is
small, where scatter is def\/{i}ned as
\begin{align*}
  s_i^2 = \dsum_{\bx_{j} \in \bD_i} (a_{j} - m_i)^2 = n_i \sigma_i^2
\end{align*}
where $\sigma_i^2$ is the
variance for class $c_i$.
\end{frame}



\begin{frame}{Linear Discriminant Analysis: F{i}sher Objective}
  \small
We incorporate the two LDA criteria, namely, maximizing the
distance between projected means and minimizing the sum of
projected scatter, into a single maximization criterion called the
{\em F{i}sher LDA objective}: 
\begin{align*}
\tcbhighmath{
    \max_{\bw} \;\; J(\bw) = \frac{(m_1 - m_2)^2}{s_1^2 + s_2^2}
}
\end{align*}

\medskip
In matrix terms, we can rewrite $(m_1 - m_2)^2$ as follows:
\begin{align*}
  (m_1 - m_2)^2 = & \lB(\bw^T(\bmu_1 - \bmu_2)\rB)^2 = \bw^T\bB\bw
\end{align*}
where $\bB = (\bmu_1 - \bmu_2) (\bmu_1 - \bmu_2)^T$ is a $d\times
d$ rank-one matrix called the {\em between-class scatter matrix}.

\medskip
The projected scatter for class $c_i$ is given as
\begin{align*}
  s_i^2 = &
  \dsum_{\bx_j \in \bD_1} (\bw^T\bx_j - \bw^T\bmu_i)^2
  =  \bw^T \lB(\dsum_{\bx_j \in \bD_i} (\bx_j - \bmu_i)(\bx_j -
  \bmu_i)^T \rB) \bw
  =  \bw^T\bS_i\bw
\end{align*}
where $\bS_i$ is the {\em scatter matrix} for $\bD_i$.
\end{frame}


\begin{frame}{Linear Discriminant Analysis: F{i}sher Objective}
The combined scatter for both classes is given as
\begin{align*}
  s_1^2 + s_2^2 = \bw^T \bS_1 \bw + \bw^T \bS_2 \bw
  = \bw^T (\bS_1 + \bS_2) \bw= \bw^T \bS \bw
\end{align*}
where the symmetric positive semidef\/{i}nite matrix 
$\bS = \bS_1 + \bS_2$ denotes the {\em within-class scatter
matrix} for the pooled data. 

The LDA objective function in matrix form is
\begin{align*}
\tcbhighmath{
    \max_{\bw} \;\; J(\bw) = \frac{\bw^T\bB\bw}{\bw^T\bS\bw}
}
\end{align*}

\medskip
To solve for the best direction $\bw$, we differentiate the objective
function with respect to $\bw$; after simplification it yields the 
{\em generalized eigenvalue problem}
\begin{align*}
  \bB \bw  = &\; \lambda \bS \bw
\end{align*}
where $\lambda = J(\bw)$ is
a generalized
eigenvalue of $\bB$ and $\bS$.
To maximize the objective
$\lambda$ should be chosen to be the largest generalized
eigenvalue, and $\bw$ to be the corresponding eigenvector.
\end{frame}



\newcommand{\LDA}{\textsc{LinearDiscriminant}\xspace}
\begin{frame}[fragile]{Linear Discriminant Algorithm}
\begin{tightalgo}[H]{\textwidth-18pt}
\SetKwInOut{Algorithm}{\LDA($\bD = \{(\bx_i, y_i)\}_{i=1}^n$)}
\Algorithm{} 
$\bD_i \assign \bigl\{\bx_{j} \mid y_{j}=c_i, j=1,\ldots,n\bigr\}, i=1,2$ \tcp{class-specif\/{i}c subsets} 
$\bmu_i \assign \text{mean}(\bD_i), i=1,2$ \tcp{class means} 
$\bB \assign (\bmu_1-\bmu_2)(\bmu_1-\bmu_2)^T$ \tcp{between-class scatter matrix} 
$\bZ_i \assign \bD_i - \bone_{n_i} \bmu_i^T, i=1,2$ \tcp{center class matrices}
$\bS_i  \assign \bZ_i^T\bZ_i, i=1,2$ \tcp{class scatter matrices} 
$\bS \assign \bS_1 + \bS_2$ \tcp{within-class scatter matrix} 
$\lambda_1, \bw \assign \text{eigen}(\bS^{-1}\bB)$ \tcp{compute dominant eigenvector}
\end{tightalgo}
\end{frame}


\begin{frame}{Linear Discriminant Direction: Iris 2D Data}
  \begin{columns}
	\column{0.5\textwidth}
\begin{figure}[!t]\vspace*{-14pt}
    \centering
    %\vspace{0.2in}
    \scalebox{0.45}{%
    \psset{dotscale=1.5,dotsep=0.03,arrowscale=2,PointName=none,unit=1.0in}
    \begin{pspicture}(4,1.5)(8.5,5)
    \psaxes[Dx=0.5,Dy=0.5,Ox=4.0,Oy=1.5]{->}(4.0,1.5)(8.5,5)
    %\begin{pspicture}(4,1.5)(8,4.5)
    %\psaxes[Dx=0.5,Dy=0.5,Ox=4.0,Oy=1.5]{->}(4.0,1.5)(8.0,4.5)
    \pstGeonode[PointSymbol=none](4.88,4.51){na}
    \pstGeonode[PointSymbol=none](6.89,1.47){nb}
    \pstLineAB[linewidth=2pt,arrows=->]{na}{nb}
    \uput[90](6.9,1.6){$\bw$}
    \input{CLASS/lda/figs/proj}
    \psset{dotscale=2,fillcolor=black}
    \pstGeonode[PointSymbol=none,
        fillcolor=lightgray](5.01,3.42){mu1}
    \pstProjection[PointSymbol=Bo]{na}{nb}{mu1}[m1]
    \pstGeonode[PointSymbol=none,
        fillcolor=lightgray](6.26,2.87){mu2}
    \pstProjection[PointSymbol=Btriangle]{na}{nb}{mu2}[m2]
    \end{pspicture}
    }
\end{figure}

\column{0.5\textwidth} 
\scriptsize
The between-class scatter matrix is
\begin{align*}
    \bB & = \amatr{r}{1.587 & -0.693\\-0.693 & 0.303}
\end{align*}
and the within-class scatter matrix is
\begin{align*}
    \bS_1 & = \matr{6.09 & 4.91\\4.91 & 7.11}\\
    \bS_2 & = \matr{43.5 & 12.09\\12.09 & 10.96}\\
    \bS & = \matr{49.58 & 17.01\\17.01 & 18.08}
\end{align*}
The direction of most separation between $c_1$ and $c_2$ is the
dominant eigenvector corresponding to the largest eigenvalue of
the matrix $\bS^{-1} \bB$. The solution is
\begin{align*}
    J(\bw) & = \lambda_1 = 0.11\\
    \bw & = \amatr{r}{0.551\\-0.834}
\end{align*}
\end{columns}
\end{frame}


\begin{frame}{Linear Discriminant Analysis: Two Classes}
For the two class scenario, if $\bS$ is nonsingular, we can
directly solve for $\bw$ without computing the eigenvalues and
eigenvectors. 

\medskip
The between-class scatter matrix $\bB$
points in the same direction as $(\bmu_1-\bmu_2)$ because
\begin{align*}
  \bB\bw = & \Bigl( (\bmu_1 - \bmu_2) (\bmu_1 - \bmu_2)^T\Bigr) \bw\notag\\
  = & (\bmu_1 - \bmu_2) \Bigl((\bmu_1 - \bmu_2)^T \bw\Bigr)\notag\\
  = & b (\bmu_1 - \bmu_2)
\end{align*}

\medskip
The generalized eigenvectors equation can then be 
rewritten as 
\begin{align*}
  \bw = & \frac{b}{\lambda} \bS^{-1} (\bmu_1 - \bmu_2)
\end{align*}
Because ${b \over \lambda}$ is just a scalar, we can solve for the
best linear discriminant as
\begin{empheq}[box=\tcbhighmath]{align*}
\begin{split}
  \bw = & \bS^{-1} (\bmu_1 - \bmu_2)
\end{split}
\end{empheq}
We can finally normalize $\bw$ to be a unit vector.
\end{frame}


\begin{frame}{Linear Discriminant Direction: Iris 2D Data}

We can directly compute $\bw$ as follows:
    \begin{align*}
        \bw & = \bS^{-1}(\bmu_1-\bmu_2)\\
        & =
        \amatr{r}{0.066 & -0.029\\-0.100 & 0.044}
        \amatr{r}{-1.246\\ 0.546} = \amatr{r}{-0.0527 \\ 0.0798}
    \end{align*}
    After normalizing, we have
    \begin{align*}
        \bw = {\bw \over \norm{\bw}} =
        {1 \over 0.0956} \amatr{r}{-0.0527 \\ 0.0798} =
        \amatr{r}{-0.551\\0.834}
    \end{align*}
    Note that even though the sign is reversed for $\bw$,
    they represent the same direction; only the scalar multiplier is
    different.

\end{frame}

\ifdefined\wox \begin{frame} \titlepage \end{frame} \fi
 
\begin{frame}{Kernel Discriminant Analysis} 
The goal of kernel LDA is to f\/{i}nd the direction vector $\bw$ in
feature space that maximizes
\begin{align*}
        \max_{\bw} \;\; J(\bw) = \frac{(m_1 - m_2)^2}{s_1^2 + s_2^2}
\end{align*}
It is well known that 
$\bw$ can be expressed as a linear combination of the
points in feature space
\begin{align*}
  \bw 
  & = \sum_{j=1}^n a_{j} \phi(\bx_{j})
\end{align*}
The mean for class $c_i$ in feature space is given as
\begin{align*}
\bmu^\phi_i = {1 \over n_i} \sum_{\bx_{j} \in \bD_i} \phi(\bx_{j})
\end{align*}
and the covariance matrix for class $c_i$ in feature space is
\begin{align*}
  \cov_i^\phi = {1 \over n_i} \sum_{\bx_{j} \in \bD_i}
  \Bigl(\phi(\bx_{j}) - \bmu_i^\phi\Bigr)
  \Bigl(\phi(\bx_{j})- \bmu_i^\phi\Bigr)^T
\end{align*}
\end{frame}



\begin{frame}{Kernel Discriminant Analysis} 
The between-class scatter matrix in feature space is
\begin{align*}
  \bB_\phi & = \Bigl(\bmu_1^\phi - \bmu_2^\phi\Bigr)
  \Bigl(\bmu_1^\phi - \bmu_2^\phi\Bigr)^T
\end{align*}
and the within-class scatter matrix in feature space is
\begin{align*}
  \bS_\phi &= n_1\cov_1^\phi + n_2\cov_2^\phi
\end{align*}
$\bS_\phi$ is a $d \times d$ symmetric, positive semidef\/{i}nite
matrix, where $d$ is the dimensionality of the feature space. 

\medskip
The best linear
discriminant vector $\bw$ in feature space is the dominant
eigenvector, which satisf\/{i}es the expression
\begin{align*}
\tcbhighmath{
  \lB(\bS_\phi^{-1}\bB_\phi\rB) \bw = \lambda \bw
}
 \label{eq:class:lda:eigeqF}
\end{align*}
where we assume that $\bS_\phi$ is non-singular. 
\end{frame}


\begin{frame}{LDA Objective via Kernel Matrix: Between-class Scatter}
\small
The projected mean for class $c_i$ is given as
\begin{align*}
   m_i & = \bw^T \bmu^\phi_i
   = {1 \over n_i} \dsum_{j=1}^n \dsum_{\bx_k \in \bD_i} a_{j}
  K(\bx_{j},\bx_k) = \ba^T \bmm_i 
\end{align*}
where $\ba = (a_1, a_2, \ldots, a_n)^T$ is the weight vector, and
\begin{align*}
  \bmm_i & = {1 \over n_i}
  \matr{
      \sum_{\bx_k \in \bD_i} K(\bx_1,\bx_k)\\
      \sum_{\bx_k \in \bD_i} K(\bx_2,\bx_k)\\
      \vdots\\
      \sum_{\bx_k \in \bD_i} K(\bx_n,\bx_k)
      }  = {1\over n_i}\bK^{c_i} \bone_{n_i}
\end{align*}
where $\bK^{c_i}$ is the $n \times n_i$ subset of the kernel
matrix, restricted to columns belonging to points only in $\bD_i$,
and $\bone_{n_i}$ is the $n_i$-dimensional vector all of whose
entries are one. 

\medskip
The separation between the projected means is thus
\begin{align*}
  (m_1 - m_2)^2  & =  \lB(\ba^T\bmm_1 - \ba^T\bmm_2\rB)^2 
     =  \ba^T\bM\ba
\end{align*}
where $\bM = (\bmm_1 - \bmm_2)(\bmm_1-\bmm_2)^T$ is the
between-class scatter matrix.
\end{frame}


\begin{frame}{LDA Objective via Kernel Matrix: Within-class Scatter}

We can compute the projected scatter for each class, $s_1^2$
and $s_2^2$, purely in terms of the kernel function, as follows
\begin{align*}
    s_1^2 & = \sum_{\bx_i \in \bD_1}
    \norm{\bw^T\phi(\bx_i) - \bw^T\bmu^\phi_1}^2
    = \ba^T \biggl(
    \Bigl(\sum_{\bx_i \in \bD_1} \bK_i\bK_i^T\Bigr)
    - n_1 \bmm_1 \bmm_1^T \biggr) \ba
     = \ba^T \bN_1 \ba
\end{align*}
where $\bK_i$ is the $i$th column of the kernel matrix, and $\bN_1$
is the class scatter matrix for $c_1$. 

\medskip
The sum of projected scatter values is then given as
\begin{align*}
  s_1^2 + s_2^2 = \ba^T (\bN_1 + \bN_2) \ba = \ba^T \bN \ba
\end{align*}
where $\bN$ is the $n\times n$ within-class scatter matrix.
\end{frame}

\begin{frame}{Kernel LDA}
The kernel LDA maximization condition is
\begin{align*}
\tcbhighmath{
    \max_{\bw} J(\bw) = \max_{\ba} J(\ba) =
    {\ba^T\bM\ba \over \ba^T\bN\ba}
}
\end{align*}


\medskip
The weight vector $\ba$ is the eigenvector
corresponding to the largest eigenvalue of the generalized
eigenvalue problem:
\begin{align*}
    \bM\ba = \lambda_1\bN\ba
\end{align*}

\medskip
When there are only two classes $\ba$ can be
obtained directly:
\begin{align*}
    \ba = \bN^{-1} (\bmm_1 - \bmm_2)
\end{align*}

\medskip
To normalize $\bw$ to be a unit vector  we
scale $\ba$ by $1 \over \sqrt{\ba^T\bK\ba}$.


\medskip
We can project any point $\bx$ onto the discriminant
direction as follows:
\begin{align*}
\tcbhighmath{
  \bw^T\phi(\bx) = \dsum_{j=1}^n a_{j}\phi(\bx_{j})^T\phi(\bx) =
  \dsum_{j=1}^n a_{j} K(\bx_{j},\bx)
}
\end{align*}
\end{frame}



\newcommand{\KLDA}{\textsc{KernelDiscriminant}}
\begin{frame}[fragile]{Kernel Discriminant Analysis Algorithm}
\begin{tightalgo}[H]{\textwidth-18pt}
\SetKwInOut{Algorithm}{\KLDA\ ($\bD = \{(\bx_i, y_i)\}_{i=1}^n, K$)} 
\Algorithm{} 
$\bK \assign \bigl\{K(\bx_i, \bx_{j}) \bigr\}_{i,j=1,\ldots,n}$ \tcp{compute $n \times n$ kernel matrix} 
$\bK^{c_i} \assign \bigl\{\bK(j,k) \mid y_k=c_i, 1\le j,k \le n\bigr\}, i=1,2$ \tcp{class kernel matrix} 
$\bmm_i \assign {1\over n_i}\bK^{c_i}  \bone_{n_i}, i=1,2$ \tcp{class means} 
$\bM \assign (\bmm_1-\bmm_2)(\bmm_1-\bmm_2)^T$ \tcp{between-class scatter matrix} 
$\bN_i  \assign \bK^{c_i} (\bI_{n_i} - {1\over n_i}\bone_{n_i \times n_i}) (\bK^{c_i})^T$, $i = 1, 2$ \tcp{class scatter matrices} 
$\bN \assign \bN_1 + \bN_2$ \tcp{within-class scatter matrix} 
$\lambda_1, \ba \assign \text{eigen}(\bN^{-1}\bM)$ \tcp{compute weight vector} 
$\ba \assign {\ba \over \sqrt{\ba^T \bK \ba}}$ \tcp{normalize $\bw$ to be unit vector} 
\end{tightalgo}
\end{frame}



\readdata{\dataPC}{CLASS/lda/figs/iris-PC.txt}
\readdata{\dataW}{CLASS/lda/figs/projw.dat} 

\begin{frame}[fragile]{Kernel Discriminant Analysis}
\framesubtitle{Quadratic Homogeneous Kernel}

Iris 2D Data: $c_1$ (circles; {\tt iris-virginica}) and
  $c_2$ (triangles; other two types).
 
\smallskip

  Kernel Function: $K(\bx_i, \bx_j) = (\bx_i^T\bx_j)^2$.

\smallskip

Contours of constant projection onto optimal kernel
discriminant, i.e., points along both the curves have the
same value when projected onto $\bw$.


\begin{figure}
\scalebox{0.6}{
\psset{stepFactor=0.3} \psset{dotscale=1.5,fillcolor=lightgray,
            arrowscale=2,PointName=none}
\psset{xAxisLabel=$\bu_1$, yAxisLabel= $\bu_2$}
\psgraph[tickstyle=bottom,Ox=-4,Oy=-1.5,Dx=1,Dy=0.5]{->}(-4.0,-1.5)(4.0,1.5){4in}{3in}%
\listplot[plotstyle=dots,dotstyle=Bo,showpoints=true,
    nEnd=50,plotNo=1,plotNoMax=2]{\dataPC}
\listplot[plotstyle=dots,dotstyle=Btriangle,showpoints=true,
    nStart=51,plotNo=1,plotNoMax=2]{\dataPC}
\psset{algebraic=true}
\psplotImp[algebraic](-4,-1.5)(4,1.5){%
        0.5+0.51*x*y+0.76*x^2-0.4*y^2}
\psplotImp[algebraic](-4,-1.5)(4,1.5){%
        0.1+0.51*x*y+0.76*x^2-0.4*y^2}
\psplotImp[algebraic](-4,-1.5)(4,1.5){%
        -0.1+0.51*x*y+0.76*x^2-0.4*y^2}
\psplotImp[algebraic](-4,-1.5)(4,1.5){%
        -0.5+0.51*x*y+0.76*x^2-0.4*y^2}
\psplotImp[algebraic](-4,-1.5)(4,1.5){%
        -1+0.51*x*y+0.76*x^2-0.4*y^2}
\psplotImp[algebraic](-4,-1.5)(4,1.5){%
        -2+0.51*x*y+0.76*x^2-0.4*y^2}
\psplotImp[algebraic](-4,-1.5)(4,1.5){%
        -3+0.51*x*y+0.76*x^2-0.4*y^2}
\psplotImp[algebraic](-4,-1.5)(4,1.5){%
        -4+0.51*x*y+0.76*x^2-0.4*y^2}
\psplotImp[algebraic](-4,-1.5)(4,1.5){%
        -5+0.51*x*y+0.76*x^2-0.4*y^2}
\psplotImp[algebraic](-4,-1.5)(4,1.5){%
        -6+0.51*x*y+0.76*x^2-0.4*y^2}
\psplotImp[algebraic](-4,-1.5)(4,1.5){%
        -8+0.51*x*y+0.76*x^2-0.4*y^2}
\endpsgraph
}
\end{figure}
\end{frame}



\begin{frame}[fragile]{Kernel Discriminant Analysis}
\framesubtitle{Quadratic Homogeneous Kernel}

Projecting $\bx_i \in \bD$  onto $\bw$, which separates the two classes.

\smallskip

The projected scatters and means for both classes are as follows:
    \begin{align*}
        m_1 & = 0.338 & m_2 & = 4.476\\
        s_1^2 & = 13.862 & s_2^2 & = 320.934
    \end{align*}
    The value of $J(\bw)$ is given as
    \begin{align*}
        J(\bw) = {(m_1-m_2)^2 \over s_1^2+s_2^2} =
        {(0.338-4.476)^2 \over 13.862+320.934} = {17.123 \over
        334.796} = 0.0511
    \end{align*}
    which,  as expected, matches $\lambda_1 = 0.0511$ from above.

\begin{figure}
\scalebox{0.8}{
\psset{dotscale=1.8, arrowscale=2,PointName=none} 
\psset{xAxisLabel=$\bw$, yAxisLabel=$~$}
%\psgraph[tickstyle=bottom,Ox=-1,Oy=-0.1,Dx=2,Dy=-0.1]{->}
%(-1,-0.1)(11,0.1){5in}{1in}%
\pspicture[](-1,-1)(11,1)
%\psline[linewidth=2pt]{->}(-1,0)(11,0)
%\def\tick{\psline(0,0)(0,-0.3)}%
%\multips(-1,0)(2,0){6}{\tick}
\psaxes[Ox=-1]{->}(-1,-0.1)(11,0.1) \uput[0](11.05,0){$\bw$}
\listplot[plotstyle=dots,dotstyle=Btriangle,
    showpoints=true,nStart=51, fillcolor=lightgray]{\dataW}
%\listplot[plotstyle=dots,dotstyle=Bo,
%    showpoints=true,nEnd=50, fillcolor=lightgray]{\dataW}
\psset{dotscale=2.2} \psdot[dotstyle=Bo,fillcolor=white](0.378,0)
\psdot[dotstyle=Btriangle,fillcolor=white](4.476,0)
%\endpsgraph
\endpspicture
}
\end{figure}
\end{frame}



\readdata{\dataQHca}{CLASS/lda/figs/iris-QHc1.dat}
\readdata{\dataQHcb}{CLASS/lda/figs/iris-QHc2.dat}
\readdata{\dataMca}{CLASS/lda/figs/kdamappedPoints-c1.dat}
\readdata{\dataMcb}{CLASS/lda/figs/kdamappedPoints-c2.dat}
\begin{frame}{Kernel Feature Space and Optimal Discriminant}

It is not desirable or possible to obtain an explicit discriminant vector
$\bw$. 

\smallskip

$\bx=(x_1,x_2)^T \in \setR^2$ is mapped
to $\phi(\bx) = (\sqrt{2}x_1x_2, x_1^2, x_2^2)^T \in \setR^3$.

\smallskip

The projection of $\phi(\bx_i)$ onto $\bw$ is also shown, where
\begin{align*}
\bw = 0.511x_1x_2 + 0.761x_1^2 - 0.4 x_2^2
\end{align*}

\begin{figure}[!t]
\scalebox{0.6}{
\centering 
\psset{unit=0.4in} \psset{arrowscale=2}
\psset{Alpha=120,Beta=-150}
%\psset{Alpha=-190,Beta=-45}
\psset{nameX=$~$, nameY=$~$, nameZ=$~$} \scalebox{0.75}{
\hskip15pt\begin{pspicture}(0,-7.5)(11,4.5)
\pstThreeDCoor[xMin=-6.5, xMax= 5, yMin=0,
        yMax=15, zMin=0, zMax=2.3, Dx=2, Dy=3, Dz=.5,
        linewidth=1pt,linecolor=black]
\pstThreeDBox[linecolor=gray](-6.5,0,0)(11,0,0)(0,13,0)(0,0,2)
\pstThreeDPut(5.3,0,0){$X_1X_2$} \pstThreeDPut(0,15.2,0){$X_1^2$}
\pstThreeDPut(0,0,2.5){$X_2^2$}
\input{CLASS/lda/figs/kdaproject.tex}
\psset{dotstyle=Btriangle,dotscale=1.75,fillcolor=lightgray}
\listplotThreeD[plotstyle=dots,showpoints=true]{\dataQHcb}
\psset{dotstyle=Bo,dotscale=1.75,fillcolor=lightgray}
\listplotThreeD[plotstyle=dots,showpoints=true]{\dataQHca}
%w = (0.511, 0.761, -0.4)
\pstThreeDLine[linewidth=2pt,arrows=->](-0.509,-0.758,0.398)
(5.754, 8.566, -4.5) \pstThreeDPut(5.9,8.6,-4.5){$\bw$}
\psset{dotstyle=Btriangle,dotscale=1.75,fillcolor=white}
\listplotThreeD[plotstyle=dots,showpoints=true]{\dataMcb}
\psset{dotstyle=Bo,dotscale=1.75,fillcolor=white}
\listplotThreeD[plotstyle=dots,showpoints=true]{\dataMca}
\psset{dotscale=2,fillcolor=black}
\pstThreeDDot[dotstyle=Bo](0.173,0.257,-0.135)
\pstThreeDDot[dotstyle=Btriangle](2.287,3.405,-1.79)
\end{pspicture}
}}
\end{figure}
\end{frame}
