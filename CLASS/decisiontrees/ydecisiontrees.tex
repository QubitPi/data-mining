\lecture{decisiontrees}{decisiontrees}

\date{Chapter 19: Decision Tree Classifier}

\begin{frame}
\titlepage
\end{frame}



\begin{frame}{Decision Tree Classif\/{i}er}
  \small
Let the training dataset $\bD = \{\bx_i, y_i\}_{i=1}^n$ consist of
$n$ points in a $d$-dimensional space, with $y_i$ being the class
label for point $\bx_i$.  

\medskip
A
decision tree classif\/{i}er is a recursive, partition-based tree
model that predicts the class $\hat{y}_i$ for each point $\bx_i$.

\medskip
Let $\cR$ denote the data space that encompasses the set of input
points $\bD$. A decision tree uses an axis-parallel hyperplane
 to
split the data space $\cR$ into two resulting half-spaces or
regions, say $\cR_1$ and $\cR_2$, which also induces a partition
of the input points into $\bD_1$ and $\bD_2$, respectively. 

\medskip
Each
of these regions is recursively split via axis-parallel
hyperplanes until most of
the points belong to the same class.  


\medskip
To classify a new {\em test} point we have to
recursively evaluate which half-space it belongs to until we reach
a leaf node in the decision tree, at which point we predict its
class as the label of the leaf.
\end{frame}



\readdata{\dataSLW}{CLASS/decisiontrees/figs/iris-slwc.txt}
\begin{frame}{Decision Tree: Recursive Splits}
\begin{figure}
%\begin{figure}[!t]\vspace*{14pt}
%    \vspace{0.2in}
%    \psset{dotscale=1.5,fillcolor=lightgray,
%        arrowscale=2,PointName=none}
%    \centerline{
%    \subfloat[Recursive Splits]{
%  \label{fig:class:dt:irisSlwc:data}

    \hspace*{-1.5cm}
    \scalebox{0.83}{%
    \begin{pspicture}(-3.5,-1)(4,7)
    \psset{xAxisLabel=$X_1$, yAxisLabel= $X_2$}
    %\psgraph[tickstyle=bottom,Dx=0.5,Dy=0.5,
    %Ox=4,Oy=2]{->}(4.0,2.0)(8.5,4.5){4in}{3in}%
    \psgraph[tickstyle=bottom,Dx=0.5,Dy=0.5,
    Ox=4.3,Oy=2]{->}(4.3,2.0)(7.9,4.4){4in}{3in}%
    \listplot[plotstyle=dots,dotstyle=Bo,showpoints=true,
        nEnd=50,plotNo=1,plotNoMax=2]{\dataSLW}
    \listplot[plotstyle=dots,dotstyle=Btriangle,showpoints=true,
        nStart=51,plotNo=1,plotNoMax=2]{\dataSLW}
    %\psline[](4,2)(4,4.5)(8.5,4.5)(8.5,2)(4,2)
    \psline[](4.3,2)(4.3,4.4)(7.9,4.4)(7.9,2)(4.3,2)
    %\psline[linewidth=2pt](5.45,2)(5.45,4.5)
    %\psline[linewidth=2pt,linecolor=lightgray](4,2.8)(5.45,2.8)
    %\psline[linewidth=2pt,linestyle=dashed](4.7,2)(4.7,2.8)
    %\psline[linewidth=2pt,linecolor=lightgray](5.45,3.45)(8.5,3.45)
    %\psline[linewidth=2pt,linestyle=dashed](6.5,3.45)(6.5,4.5)
    \psline[linewidth=2pt](5.45,2)(5.45,4.4)
    \psline[linewidth=2pt,linecolor=lightgray](4.3,2.8)(5.45,2.8)
    \psline[linewidth=2pt,linestyle=dashed](4.7,2)(4.7,2.8)
    \psline[linewidth=2pt,linecolor=lightgray](5.45,3.45)(7.9,3.45)
    \psline[linewidth=2pt,linestyle=dashed](6.5,3.45)(6.5,4.4)
    \uput[0](5.35,4.5){$h_{0}$}
    \uput[0](4.05,2.8){$h_{2}$}
    \uput[0](7.9,3.45){$h_{3}$}
    \uput[90](4.7,2.8){$h_{4}$}
    \uput[0](6.4,4.5){$h_{5}$}
    \uput[90](4.8,4.2){$\cR_{1}$}
    \uput[90](6.8,2.0){$\cR_{2}$}
    \uput[90](4.5,2.0){$\cR_{3}$}
    \uput[90](5.1,2.0){$\cR_{4}$}
    \uput[90](6.0,4.2){$\cR_{5}$}
    \uput[90](7.2,4.2){$\cR_{6}$}
    \psdot[dotstyle=Bsquare,dotscale=2,fillcolor=white](6.75,4.25)
        \uput[45](6.75,4.25){$\bz$}
    %\uput[0](5.15,4.6){$h_{5.45}$}
    %\uput[0](6.25,4.6){$h_{6.5}$}
    %\uput[0](8.5,3.45){$h_{3.45}$}
    %\uput[0](3.55,2.8){$h_{2.8}$}
    %\uput[0](4.55,1.925){\small $h_{4.7}$}
    \vspace{0.3in}
    \endpsgraph
    \end{pspicture}
}
\end{figure}
\end{frame}



\begin{frame}{Decision Tree}

A decision tree consists of internal nodes that represent the
decisions corresponding to the hyperplanes or split points (i.e.,
which half-space a given point lies in), and leaf nodes that
represent regions or partitions of the data space, which are
labeled with the majority class. A region is characterized by the
subset of data points that lie in that region.



\begin{figure}
\hspace*{-2.5cm}
  \scalebox{0.75}{%
  \begin{pspicture}(-4,-7)(4,1)
  \psset{unit=0.75in}
  \def\bedge{\ncline[linewidth=2pt]}
  \def\gedge{\ncline[linewidth=2pt,linecolor=lightgray]}
  \def\dedge{\ncline[linewidth=2pt,linestyle=dashed]}
  \psset{tpos=0.5,treesep=0.5,levelsep=*0.5,treefit=tight}
  \pstree[edge=\bedge]{\Toval[name=D]{$X_1 \le 5.45$}}{%
    \pstree[edge=\gedge]{\Toval[name=L]{$X_2 \le 2.8$}\nbput{Yes}}{%
      \pstree[edge=\dedge]{\Toval[name=LL]{$X_1 \le 4.7$}\naput{Yes}}{%
          \TR[]{%
            \begin{tabular}{|c|c|}
                \hline
            \psdot[dotstyle=Bo](0,0) $~~c_1$ & 1\\
            \hline
            \psdot[dotstyle=Btriangle](0,0) $~~c_2$ & 0\\
            \hline
                \multicolumn{2}{c}{$\cR_3$}
            \end{tabular}
          }\nbput{Yes}
          \TR[]
          }\naput{No}
      }
      \TR[edge=\gedge]{%
            \begin{tabular}{|c|c|}
            \hline
            \psdot[dotstyle=Bo](0,0)$~~c_1$  & 44\\
            \hline
            \psdot[dotstyle=Btriangle](0,0)$~~c_2$  & 1\\
            \hline
                \multicolumn{2}{c}{$\cR_1$}
            \end{tabular}
          }\naput{No}
    }
    \pstree[edge=\gedge]{\Toval[name=R]{$X_2 \le 3.45$}\naput{No}}{%
        \TR[]{
            \begin{tabular}{|c|c|}
            \hline
            \psdot[dotstyle=Bo](0,0)$~~c_1$  & 0\\
            \hline
            \psdot[dotstyle=Btriangle](0,0)$~~c_2$  & 90\\
            \hline
                \multicolumn{2}{c}{$\cR_2$}
            \end{tabular}
        }\nbput{Yes}
        \pstree[edge=\dedge]{\Toval[name=RR]{$X_1 \le 6.5$}\naput{No}}{%
             \TR[]{
            \begin{tabular}{|c|c|}
            \hline
            \psdot[dotstyle=Bo](0,0)$~~c_1$  & 5\\
            \hline
            \psdot[dotstyle=Btriangle](0,0)$~~c_2$  & 0\\
            \hline
                \multicolumn{2}{c}{$\cR_5$}
            \end{tabular}
            }\nbput{Yes}
            \TR[]{
            \begin{tabular}{|c|c|}
            \hline
            \psdot[dotstyle=Bo](0,0)$~~c_1$  & 0\\
            \hline
            \psdot[dotstyle=Btriangle](0,0)$~~c_2$  & 3\\
            \hline
                \multicolumn{2}{c}{$\cR_6$}
            \end{tabular}
            }\naput{No}
        }
    }
  }
  \end{pspicture}
}
\end{figure}
\end{frame}


\begin{frame}{Decision Trees: Axis-Parallel Hyperplanes}

A hyperplane $h(\bx)$ is def\/{i}ned as the set of all points $\bx$ that
satisfy the following equation
\begin{empheq}[box=\tcbhighmath]{align*}
\begin{split}
    h(\bx)\!: & \;\bw^T \bx + b = 0
\end{split}
\end{empheq}
where $\bw \in \setR^d$ is a {\em weight vector} that is normal to
the hyperplane, and $b$ is the offset of the hyperplane from the
origin. 

\medskip
A decision tree considers only {\em axis-parallel
hyperplanes}, that is, the weight vector must be parallel to one
of the original dimensions or axes $X_{j}$:
\begin{align*}
   h(\bx)\!:& \; x_{j} + b = 0
\end{align*}
where the choice of the offset $b$ yields different hyperplanes along
dimension $X_{j}$.
\end{frame}



\begin{frame}{Decision Trees: Split Points} 
A hyperplane specif\/{i}es a decision or {\em
split point} because it splits the data space $\cR$ into two
half-spaces. All points $\bx$ such that $h(\bx) \le 0$ are on the
hyperplane or to one side of the hyperplane, whereas all points
such that $h(\bx) > 0$ are on the other side. 

\medskip
The split point is written as $h(\bx) \le 0$, i.e.
\begin{align*}
  X_{j} \le v
\end{align*}
where $v = -b$ is some value in the domain of attribute $X_{j}$. 

\medskip
The
decision or split point $X_{j} \le v$ thus splits the input data
space $\cR$ into two regions $\cR_Y$ and $\cR_N$, which denote the
set of {\em all possible points} that satisfy the decision and
those that do not.

\bigskip
{\bf Categorical Attributes:}
 For a categorical attribute $X_{j}$, the split points or
decisions are of the $X_{j} \in V$, where $V \subset dom(X_{j})$, and
$dom(X_{j})$ denotes the domain for $X_{j}$. 

\end{frame}



\begin{frame}{Decision Trees: Data Partition and Purity}
Each split of $\cR$ into
$\cR_Y$ and $\cR_N$ also induces a binary partition of the
corresponding input data points $\bD$. A split point of
the form $X_{j} \le v$ induces the data partition
\begin{align*}
  \bD_Y & = \{\bx \mid \; \bx \in \bD, x_{j} \le v\}\\
  \bD_N & = \{\bx \mid \; \bx \in \bD, x_{j} > v\}
\end{align*}

\bigskip
The purity of a region $\cR_{j}$ is the
fraction of points with the majority label in $\bD_{j}$, that is,
\begin{align*}
\tcbhighmath{
  \mathit{purity}(\bD_{j}) = \max_i \lB\{ {n_{ji} \over n_{j}} \rB \}
}
\end{align*}
where $n_{j}=\card{\bD_{j}}$ is the total number of data points in the
region $\cR_{j}$, and $n_{ji}$ is the number of points in $\bD_{j}$
with class label $c_i$.
\end{frame}




\begin{frame}[fragile]{Decision Trees to Rules}
\begin{figure}
  \vspace{-0.3in}
  \scalebox{0.6}{%
  \begin{pspicture}(-4,-7)(4,1)
  \psset{unit=0.75in}
  \def\bedge{\ncline[linewidth=2pt]}
  \def\gedge{\ncline[linewidth=2pt,linecolor=lightgray]}
  \def\dedge{\ncline[linewidth=2pt,linestyle=dashed]}
  \psset{tpos=0.5,treesep=0.5,levelsep=*0.5,treefit=tight}
  \pstree[edge=\bedge]{\Toval[name=D]{$X_1 \le 5.45$}}{%
    \pstree[edge=\gedge]{\Toval[name=L]{$X_2 \le 2.8$}\nbput{Yes}}{%
      \pstree[edge=\dedge]{\Toval[name=LL]{$X_1 \le 4.7$}\naput{Yes}}{%
          \TR[]{%
            \begin{tabular}{|c|c|}
                \hline
            \psdot[dotstyle=Bo](0,0) $~~c_1$ & 1\\
            \hline
            \psdot[dotstyle=Btriangle](0,0) $~~c_2$ & 0\\
            \hline
                \multicolumn{2}{c}{$\cR_3$}
            \end{tabular}
          }\nbput{Yes}
          \TR[]
          }\naput{No}
      }
      \TR[edge=\gedge]{%
            \begin{tabular}{|c|c|}
            \hline
            \psdot[dotstyle=Bo](0,0)$~~c_1$  & 44\\
            \hline
            \psdot[dotstyle=Btriangle](0,0)$~~c_2$  & 1\\
            \hline
                \multicolumn{2}{c}{$\cR_1$}
            \end{tabular}
          }\naput{No}
    }
    \pstree[edge=\gedge]{\Toval[name=R]{$X_2 \le 3.45$}\naput{No}}{%
        \TR[]{
            \begin{tabular}{|c|c|}
            \hline
            \psdot[dotstyle=Bo](0,0)$~~c_1$  & 0\\
            \hline
            \psdot[dotstyle=Btriangle](0,0)$~~c_2$  & 90\\
            \hline
                \multicolumn{2}{c}{$\cR_2$}
            \end{tabular}
        }\nbput{Yes}
        \pstree[edge=\dedge]{\Toval[name=RR]{$X_1 \le 6.5$}\naput{No}}{%
             \TR[]{
            \begin{tabular}{|c|c|}
            \hline
            \psdot[dotstyle=Bo](0,0)$~~c_1$  & 5\\
            \hline
            \psdot[dotstyle=Btriangle](0,0)$~~c_2$  & 0\\
            \hline
                \multicolumn{2}{c}{$\cR_5$}
            \end{tabular}
            }\nbput{Yes}
            \TR[]{
            \begin{tabular}{|c|c|}
            \hline
            \psdot[dotstyle=Bo](0,0)$~~c_1$  & 0\\
            \hline
            \psdot[dotstyle=Btriangle](0,0)$~~c_2$  & 3\\
            \hline
                \multicolumn{2}{c}{$\cR_6$}
            \end{tabular}
            }\naput{No}
        }
    }
  }
  \end{pspicture}
}
  \vspace{-0.4in}
\end{figure}

\small
%\scalebox{0.7}{
A tree is a set of
decision rules; each comprising the
decisions on the path to a leaf:
  \begin{align*}
    \cR_3\!: & \text{ If $X_1 \le 5.45$ and $X_2 \le 2.8$ and
    $X_1 \le 4.7$, then class is $c_1$, or}\\
    \cR_4\!: & \text{ If $X_1 \le 5.45$ and $X_2 \le 2.8$ and
    $X_1 > 4.7$, then class is $c_2$, or}\\
    \cR_1\!: & \text{ If $X_1 \le 5.45$ and $X_2 > 2.8$, then class is
    $c_1$, or}\\
    \cR_2\!: & \text{ If $X_1 > 5.45$ and $X_2 \le 3.45$, then class is
    $c_2$, or}\\
    \cR_5\!: & \text{ If $X_1 > 5.45$ and $X_2 > 3.45$
    and $X_1 \le 6.5$, then class is $c_1$, or}\\
    \cR_6\!: & \text{ If $X_1 > 5.45$ and $X_2 > 3.45$
    and $X_1 > 6.5$, then class is $c_2$}\\[-20pt]
  \end{align*}
 % }
\end{frame}


\begin{frame}{Decision Tree Algorithm}

The method takes as input a training
dataset $\bD$, and two parameters $\eta$ and $\pi$, where $\eta$
is the leaf size and $\pi$ the leaf purity threshold. 

\medskip
Different
split points are evaluated for each attribute in $\bD$. Numeric
decisions are of the form $X_{j} \le v$ for some value $v$ in the value range for attribute $X_{j}$, and categorical decisions are of the
form $X_{j} \in V$ for some subset of values in the domain of $X_{j}$.

\medskip
The best split point is chosen to partition the data into two
subsets, $\bD_Y$ and $\bD_N$, where $\bD_Y$ corresponds to all
points $\bx \in \bD$ that satisfy the split decision, and $\bD_N$
corresponds to all points that do not satisfy the split decision.
The decision tree method is then called recursively on $\bD_Y$ and
$\bD_N$. 

\medskip
We stop the process if the leaf size drops below $\eta$ or if the purity
is at least $\pi$.
\end{frame}




\begin{frame}[fragile]{Decision Tree Algorithm}
\footnotesize{
\begin{tightalgo}[H]{\textwidth-18pt}
    \SetKwInOut{Algorithm}{{\textsc{DecisionTree}} ($\bD, \eta, \pi$)}
    \Algorithm{}
    $n \assign \card{\bD}$ \tcp{partition size}
    $n_i \assign \card{\{ \bx_{j} | \bx_{j} \in \bD, y_{j} = c_i\}}$
    \tcp{size of class $c_i$}
    $\mathit{purity}(\bD) \assign \max_i \lB\{\frac{n_i}{n}\rB\}$\;
    \If(\tcp*[h]{stopping condition}){$n \le \eta$ or
    $\mathit{purity}(\bD) \ge \pi$}{
      $c^* \assign  \arg\max_{c_i} \lB\{\frac{n_i}{n}\rB\}$ \tcp{majority
        class}
        create leaf node, and label it with class $c^*$\;
        \Return\;
    }
    $(\mathit{split\text{ }point}^*, \mathit{score}^*) \assign (\emptyset, 0)$
    \tcp{initialize best split point}
    \ForEach{(attribute $X_{j}$)}{
        \If{($X_{j}$ is numeric)}{
        $(v,\mathit{score}) \assign$  \textsc{Evaluate-Numeric-Attribute}($\bD, X_{j}$)\;
        \lIf{$\mathit{score} > \mathit{score}^*$}{
        $(\mathit{split\text{ }point}^*, \mathit{score}^*) \assign
        (X_{j} \le v, \mathit{score})$
            }
        }
        \ElseIf{($X_{j}$ is categorical)}{
        $(V, \mathit{score}) \assign$
            \textsc{Evaluate-Categorical-Attribute}($\bD, X_{j}$)\;
            \lIf{$\mathit{score} > \mathit{score}^*$}{
            $(\mathit{split\text{ }point}^*, \mathit{score}^*) \assign
            (X_{j} \in V, \mathit{score})$
            }
        }
    }
%    \tcp{partition $\bD$ into $\bD_Y$ and $\bD_N$ using
%    $\mathit{split\text{ }point}^*$, and call recursively}\;
    $\bD_Y \assign \{ \bx \in \bD \mid \bx \text{ satisfies }
    \mathit{split\text{ }point}^* \}$\;
    $\bD_N \assign \{ \bx \in \bD \mid \bx \text{ does not satisfy }
    \mathit{split\text{ }point}^* \}$\;
    create internal node $\mathit{split\text{ }point}^*$, with
    two child nodes, $\bD_Y$ and $\bD_N$\;
    \textsc{DecisionTree}($\bD_Y, \eta, \pi$); \textsc{DecisionTree}($\bD_N,  \eta, \pi$)\;
\end{tightalgo}
}
\end{frame}

\ifdefined\wox \begin{frame} \titlepage \end{frame} 

\begin{frame}[fragile]{Decision Tree Algorithm}
\footnotesize{
\begin{tightalgo}[H]{\textwidth-18pt}
    \SetKwInOut{Algorithm}{{\textsc{DecisionTree}} ($\bD, \eta, \pi$)}
    \Algorithm{}
    $n \assign \card{\bD}$ \tcp{partition size}
    $n_i \assign \card{\{ \bx_{j} | \bx_{j} \in \bD, y_{j} = c_i\}}$
    \tcp{size of class $c_i$}
    $\mathit{purity}(\bD) \assign \max_i \lB\{\frac{n_i}{n}\rB\}$\;
    \If(\tcp*[h]{stopping condition}){$n \le \eta$ or
    $\mathit{purity}(\bD) \ge \pi$}{
      $c^* \assign  \arg\max_{c_i} \lB\{\frac{n_i}{n}\rB\}$ \tcp{majority
        class}
        create leaf node, and label it with class $c^*$\;
        \Return\;
    }
    $(\mathit{split\text{ }point}^*, \mathit{score}^*) \assign (\emptyset, 0)$
    \tcp{initialize best split point}
    \ForEach{(attribute $X_{j}$)}{
        \If{($X_{j}$ is numeric)}{
        $(v,\mathit{score}) \assign$  \textsc{Evaluate-Numeric-Attribute}($\bD, X_{j}$)\;
        \lIf{$\mathit{score} > \mathit{score}^*$}{
        $(\mathit{split\text{ }point}^*, \mathit{score}^*) \assign
        (X_{j} \le v, \mathit{score})$
            }
        }
        \ElseIf{($X_{j}$ is categorical)}{
        $(V, \mathit{score}) \assign$
            \textsc{Evaluate-Categorical-Attribute}($\bD, X_{j}$)\;
            \lIf{$\mathit{score} > \mathit{score}^*$}{
            $(\mathit{split\text{ }point}^*, \mathit{score}^*) \assign
            (X_{j} \in V, \mathit{score})$
            }
        }
    }
%    \tcp{partition $\bD$ into $\bD_Y$ and $\bD_N$ using
%    $\mathit{split\text{ }point}^*$, and call recursively}\;
    $\bD_Y \assign \{ \bx \in \bD \mid \bx \text{ satisfies }
    \mathit{split\text{ }point}^* \}$\;
    $\bD_N \assign \{ \bx \in \bD \mid \bx \text{ does not satisfy }
    \mathit{split\text{ }point}^* \}$\;
    create internal node $\mathit{split\text{ }point}^*$, with
    two child nodes, $\bD_Y$ and $\bD_N$\;
    \textsc{DecisionTree}($\bD_Y, \eta, \pi$); \textsc{DecisionTree}($\bD_N,  \eta, \pi$)\;
\end{tightalgo}
}
\end{frame}


\fi


\begin{frame}{Split Point Evaluation Measures: Entropy}
Intuitively, we want to
select a split point that gives the best separation or
discrimination between the different class labels.

\medskip
Entropy measures the amount of disorder or
uncertainty in a system.
A partition has lower entropy (or low
 disorder) if it is relatively pure, that is, if most of the points have the same label. On the other hand, a partition
has higher entropy (or more
disorder) if the class labels are mixed, and there is no majority class
as such.

\medskip
The entropy of a set of labeled points $\bD$ is def\/{i}ned as
follows:
\begin{align*}
\tcbhighmath{
  H(\bD) = -\sum_{i=1}^k \;P(c_i|\bD) \;\log_2 P(c_i|\bD)
}
\end{align*}
where $P(c_i|\bD)$ is the probability of class $c_i$ in $\bD$, and
$k$ is the number of classes. 

\medskip
If a region is pure, that is, has
points from the same class, then the entropy is zero. On the other
hand, if the classes are all mixed up, and each appears with equal
probability $P(c_i|\bD)=\tfrac{1}{k}$, then the entropy has the
highest value, $H(\bD) = \log_2 k$.
\end{frame}




\begin{frame}{Split Point Evaluation Measures: Entropy}
Def\/{i}ne the {\em split entropy} as the
weighted entropy of each of the resulting partitions
\begin{align*}
\tcbhighmath{
  H(\bD_Y,\bD_N) =
  \frac{n_Y}{n}H(\bD_Y)+\frac{n_N}{n}H(\bD_N)
}
\end{align*}
where $n = |\bD|$ is the number of points in $\bD$, and $n_Y = |\bD_Y|$
and $n_N = |\bD_N|$ are the number of points in $\bD_Y$ and
$\bD_N$.

\medskip
Def\/{i}ne the {\em information gain}
for a split point as
\begin{align*}
\tcbhighmath{
  \mathit{Gain}(\bD,\bD_Y,\bD_N) = H(\bD) - H(\bD_Y,\bD_N)
}
\end{align*}
The higher the information gain, the more the reduction in
entropy, and the better the split point.  

\medskip
We score each split point
and choose the one that gives the highest information gain.
\end{frame}


\begin{frame}{Split Point Evaluation Measures}
\framesubtitle{Gini Index and CART Measure}

{\bf Gini Index:}
The Gini index is def\/{i}ned as follows:
\begin{align*}
\tcbhighmath{
  G(\bD) = 1 - \sum_{i=1}^{k}{P(c_i|\bD)^2}
}
\end{align*}
If the partition is pure, then Gini index is $0$. 
The weighted Gini index is:
\begin{align*}
  G(\bD_Y,\bD_N) = \frac{n_Y}{n}G(\bD_Y)+\frac{n_N}{n}G(\bD_N)
\end{align*}
The lower the Gini index
value, the better the split point.

\bigskip
{\bf CART: }
The CART measure is
\begin{align*}
\tcbhighmath{
  \mathit{CART}(\bD_Y,\bD_N) =
  2 \frac{n_Y}{n} \frac{n_N}{n}
  \sum_{i=1}^{k}{\Bigl|P(c_i|\bD_Y)-P(c_i|\bD_N)\Bigr|}
}
\end{align*}
CART maximizes the
difference between the class PMF for the two partitions; the higher the CART measure, the better the
split point.
\end{frame}



\begin{frame}{Evaluating Split Points: Numeric Attributes}
All of the
split point evaluation measures depend on the class probability mass function (PMF) for $\bD$,
namely, $P(c_i | \bD)$, and the class PMFs for the resulting
partitions $\bD_Y$ and $\bD_N$, namely $P(c_i | \bD_Y)$ and
$P(c_i| \bD_N)$. 

\medskip
We have to evaluate split points of
the form $X \le v$. We consider only the
midpoints between two successive distinct values for $X$ in the
sample $\bD$. 
Let $\{v_1, \ldots, v_m\}$ denote the set of all such midpoints,
such that $v_1 < v_2 < \cdots < v_m$. 

\medskip
For each split point $X \le
v$, we have to estimate the class PMFs:
\begin{align*}
  \hP(c_i|\bD_Y) & = \hP(c_i | X \le v) \\
\hP(c_i|\bD_N) &= \hP(c_i | X > v) 
\end{align*}

\medskip
Using 
Bayes theorem, we have
\begin{align*}
  \hP(c_i | X \le v) = \frac{\hP(X \le v|c_i) \hP(c_i)}{\hP(X \le v)}
  = \frac{\hP(X \le v|c_i) \hP(c_i)}{\sum_{j=1}^k\hP(X \le v|c_{j}) \hP(c_{j})}
\end{align*}
Thus we have to estimate the prior probability and likelihood for each
class in each partition.
\end{frame}


\begin{frame}{Evaluating Split Points: Numeric Attributes}
\small
The prior probability for each class in $\bD$ can be estimated as
\begin{align*}
    \hP(c_i) = {1 \over n} \sum_{j=1}^n I(y_{j}=c_i) =
    {n_i \over n}
\end{align*}
where $y_{j}$ is the class for point $\bx_{j}$,
$n = \card{\bD}$ is the total number of points, and $n_i$ is
the number of points in $\bD$ with class $c_i$.

\medskip
Def\/{i}ne $N_{vi}$ as
the number of points $x_{j} \le v$ with
class $c_i$,
where $x_{j}$ is the value of data point $\bx_{j}$
for the attribute $X$, given as
\begin{align*}
N_{vi} = \sum_{j=1}^n I(x_{j} \le v \text{ and } y_{j}=c_i)
\end{align*}
\end{frame}

\begin{frame}{Evaluating Split Points: Numeric Attributes}
We can estimate  $\hP(X \le v|c_i)$ and $\hP(X > v | c_i)$ as follows:
\begin{align*}
    \hP(X \le v| c_i) & = {N_{vi} \over n_i}\\
  \hP(X > v|c_i) & = 1 - \hP(X \le v | c_i) =
	  {n_i - N_{vi} \over n_i}
\end{align*}

\medskip
Finally, we have
\begin{empheq}[box=\tcbhighmath]{align*}
\begin{split}
  \hP(c_i | \bD_Y) & = \hP(c_i | X \le v) = {N_{vi} \over \sum_{j=1}^k
  N_{vj}}\\
    \hP(c_i | \bD_N)  & =
    \hP(c_i | X > v) = 
    {n_i - N_{vi} \over \sum_{j=1}^k (n_{j} - N_{vj})}
\end{split}
\end{empheq}

The total cost of evaluating a
numeric attribute is $O(n \log n + nk)$, where $k$ is the number of
classes, and $n$ is the number of points.
\end{frame}


\begin{frame}[fragile]{Algorithm \textsc{Evaluate-Numeric-Attribute}}
\footnotesize{
\begin{tightalgo}[H]{\textwidth-18pt}
    \SetKwInOut{Algorithm}{{\textsc{Evaluate-Numeric-Attribute}} ($\bD, X$)}
    \Algorithm{}
    sort $\bD$ on attribute $X$, so that $x_{j} \le x_{j+1},
    \forall j=1,\dots,n-1$ \nllabel{alg:class:dt:ENA:sort}\;
    $\cM \assign \emptyset$ \tcp{set of midpoints}
    \lFor{$i=1,\dots,k$}{$n_i \assign 0$} 
    \For{$j=1,\dots,n-1$}{ \nllabel{alg:class:dt:ENA:For1}
        \lIf{$y_{j} = c_i$}{$n_i \assign n_i + 1$} %\tcp{running count for class $c_i$}
        \If{$x_{j+1} \ne x_{j}$}{
            $v \assign (x_{j+1}\;+\;x_{j})/2$; $\cM \assign \cM \cup \{ v\}$
            \tcp{midpoints}
            \For{$i=1,\dots,k$}{
                $N_{vi} \assign n_i$ \tcp{Number of points such that $x_{j}
                \le v$ and $y_{j}=c_i$}
            }
        }
    }
    \lIf{$y_n = c_i$}{$n_i \assign n_i + 1$}
    %\tcp{evaluate split points of the form $X \le v$}\;
    $v^* \assign \emptyset$; $\mathit{score}^* \assign 0$ \tcp{initialize best split point}
    \ForAll{$v \in \cM$}{\nllabel{alg:class:dt:ENA:For2}
        \For{$i=1,\dots,k$}{
            $\hP(c_i|\bD_Y) \assign {N_{vi} \over \sum_{j=1}^k N_{vj}}$\;
            $\hP(c_i|\bD_N) \assign {n_i-N_{vi} \over \sum_{j=1}^k
            n_{j}-N_{vj}}$\;
        }
        $\mathit{score}(X \le v) \assign Gain(\bD, \bD_Y, \bD_N)$ \;
        %\tcp{use Eq.\nosp\eqref{eq:class:dt:gain}}\;
        \If{$\mathit{score}(X \le v) > \mathit{score}^*$}{
            $v^* \assign v; \mathit{score}^* \assign \mathit{score}(X \le v)$\;
        }
    }
    \Return{$(v^*, \mathit{score}^*)$}
\end{tightalgo}
}
\end{frame}



\readdata{\dataSL}{CLASS/decisiontrees/figs/iris-sl-gain-cdf.dat}
\readdata{\dataSW}{CLASS/decisiontrees/figs/iris-sw-gain-cdf.dat}
\begin{frame}{Iris Data: Class-specific Frequencies ${\it N}_{{\it
  vi}}$}
  \framesubtitle{Classes ${\it c}_1$ and ${\it c}_2$ for 
  attribute $X_1$ - {\tt sepal length}}

We f\/{i}rst compute the frequencies $N_{vi}$.
Consider the split point $X_1 \le 5.45$: 

\begin{figure}[!t]
    \centering
    \hspace{0.4in}
    \psset{arrowscale=2}
    \psset{dotscale=1.5,fillcolor=lightgray}
    \psset{xAxisLabel=Midpoints: $v$,
        yAxisLabel= Frequency: $N_{vi}$,%
        xAxisLabelPos={c,-1.75},yAxisLabelPos={-0.4,c}}
    \pstScalePoints(1,0.1){}{}
    \scalebox{0.80}{%
    \begin{psgraph}[Dy=10,dy=1,Dx=0.5,Ox=4]{->}%
            (4,0)(8,11){4.5in}{2.5in}
    \listplot[showpoints, dotstyle=Btriangle,plotNo=2,plotNoMax=3]{\dataSL}
    \listplot[showpoints, dotstyle=Bo,plotNo=3,%
        plotNoMax=3]{\dataSL}
        \psline[linestyle=dotted](5.45,0)(5.45,6.0)
        \uput[90](5.4,6.1){$v=5.45$}
        \psdot[dotstyle=Btriangle,fillcolor=black](5.45,0.7)
        \psdot[dotstyle=Bo,fillcolor=black](5.45,4.5)
        \uput[120](5.45,4.5){$45$}
        \uput[120](5.45,0.7){$7$}
        \uput[45](6.75,5.25){{\tt iris-setosa} ($c_1$)}
        \uput[45](7.25,10){{\tt other} ($c_2$)}
    \end{psgraph}
    }
\end{figure}
\end{frame}

\begin{frame}{Iris Data: Finding the best split point}
We do have two classes $c_1$ and $c_2$:

    \begin{align*}
    \hP(c_1) & = {50}/{150}={1}/{3}\\[-5pt]
    \hP(c_2) & = {100}/{150}={2}/{3}
    \end{align*}

    The entropy of the dataset $\bD$ is therefore

    \begin{align*}
        H(\bD) = -\lB({1 \over 3} \log_2 {1\over 3} +
            {2 \over 3} \log_2 {2\over 3}\rB) = 0.918
    \end{align*}

	After computing the frequencies $N_{vi}$, we check each split point, e.g., $X_1=5.45$:

    \begin{align*}
      N_{v1} & = 45 & N_{v2} & = 7
    \end{align*}

    \begin{align*}
      \hP(c_1|\bD_Y) & = {N_{v1} \over N_{v1} + N_{v2}} =
        {45 \over 45+7} = 0.865\\
        \hP(c_2|\bD_Y) & = {N_{v2} \over N_{v1} + N_{v2}}
        = {7 \over 45+7} = 0.135
    \end{align*}

\end{frame}

\begin{frame}{Iris Data: Finding the best split point}

    \begin{align*}
        \hP(c_1|\bD_N) & =
        {n_1 - N_{v1} \over (n_1 - N_{v1})+(n_2 - N_{v2})}
        = {50-45 \over (50-45)+(100-7)} = 0.051\\
        \hP(c_2|\bD_N) & =
        {n_2 - N_{v2} \over (n_1 - N_{v1})+(n_2 - N_{v2})}
        = {(100-7) \over (50-45)+(100-7)} = 0.949
    \end{align*}

    Computing the entropy of the partitions $\bD_Y$ and $\bD_N$:

    \begin{align*}
        H(\bD_Y) & = -(0.865 \log_2 0.865 + 0.135 \log_2 0.135) =
        0.571\\
        H(\bD_N) & = -(0.051 \log_2 0.051 + 0.949 \log_2 0.949) =
        0.291
    \end{align*}

    The entropy of $X \le 5.45$ is:

    \begin{align*}
        H(\bD_Y,\bD_N) = {52 \over 150} H(\bD_Y) + {98 \over 150}
        H(\bD_N) = 0.388
    \end{align*}

    The information gain for $X=5.45$ is therefore

    \begin{align*}
        Gain = H(\bD) - H(\bD_Y,\bD_N) = 0.918 - 0.388 = 0.53
    \end{align*}


\end{frame}

\begin{frame}{Iris Data: Information Gain for Different Splits}
\begin{figure}[!t]
  \psset{arrowscale=2}
    \psset{xAxisLabel=Split points: $X_i \le v$,
        yAxisLabel=Information Gain,%
        xAxisLabelPos={c,-1.75},yAxisLabelPos={-0.75,c}}
    \pstScalePoints(1,20){}{}
    \centerline{
    \hspace{0.5in}
    \scalebox{0.90}{%
    \begin{psgraph}[Dy=0.05,dy=1,Dx=0.5,Ox=2]{->}%
            (2,0)(8,12){4.5in}{2.5in}
        \listplot[plotstyle=bar,barwidth=0.2cm,
        fillcolor=lightgray,fillstyle=solid,plotNo=1,plotNoMax=3]{%
        \dataSL
        }
        \listplot[plotstyle=bar,barwidth=0.2cm,
        fillstyle=none,plotNo=1,plotNoMax=3]{%
        \dataSW
        }
        \psline[linewidth=5pt](5.45,0)(5.45,10.6)
        %\pnode(6.5,9){nA}
        %\pnode(5.5,10.6){nB}
        %\ncarc[angle=-60]{->}{nA}{nB}
        \uput[0](6.0,10.6){\small $X_1 \le 5.45$}
        \psline{->}(6.0,10.6)(5.5,10.6)
        \uput[90](3.25,5.5){{\tt sepal-width}($X_2$)}
        \uput[90](5.5,11){{\tt sepal-length}($X_1$)}
    \end{psgraph}
    }}
\end{figure}
\end{frame}



\begin{frame}{Categorical Attributes}
  \small
For categorical $X$ the split points are of 
the form $X \in V$, where $V \subset dom(X)$ and $V
\ne \emptyset$. All distinct partitions of the set of
values of $X$ are considered.

\medskip
If $m = \card{dom(X)}$, then there are $O(2^{m-1})$ distinct
partitions, which can be too many.
One simplif\/{i}cation is to restrict $V$ to be of size one, so
that there are only $m$ split points of the form $X_{j} \in \{v\}$,
where $v \in dom(X_{j})$.

\medskip
Def\/{i}ne $n_{vi}$ as the number of points $\bx_{j} \in \bD$, with
value $x_{j}=v$ for attribute $X$ and having class $y_{j}=c_i$:
\begin{align*}
  n_{vi} = \sum_{j=1}^n I(x_{j} = v \mbox{ and } y_{j}=c_i)
\end{align*}
The class conditional empirical PMF for $X$ is then given as
\begin{align*}
    \hP(X=v|c_i) & =
    {\hP\bigl(X = v \text{ and } c_i\bigr) \over \hP(c_i)}
   = {n_{vi} \over n_i}
\end{align*}

\medskip
We then have
\begin{empheq}[box=\tcbhighmath]{align*}
\begin{split}
\hP(c_i | \bD_Y) = \frac{\sum_{v \in V} n_{vi}} {\sum_{j=1}^k \sum_{v \in V} n_{vj}} \hspace*{1cm}  \hP(c_i | \bD_N)  = \frac{\sum_{v \not\in V} n_{vi}} {\sum_{j=1}^k \sum_{v \not\in V} n_{vj}} \\
\end{split}
\end{empheq}
\end{frame}


\begin{frame}[fragile]{Algorithm \textsc{Evaluate-Categorical-Attribute}}
\begin{tightalgo}[H]{\textwidth-18pt}
  \small
%\begin{tightalgo}[!t]{\textwidth-18pt}
    \SetKwInOut{Algorithm}{{\textsc{Evaluate-Categorical-Attribute}}
    ($\bD, X, l$)}
    \Algorithm{}
    \For{$i=1,\dots,k$}{
        $n_i\assign0$\;
        \lForAll{$v \in dom(X)$}{$n_{vi}\assign0$}
    }
    \For{$j=1,\dots,n$}{\label{alg:class:dt:ECA:For1}
        \lIf{$x_{j} = v \text{ and } y_{j}=c_i$}{
        $n_{vi} \assign n_{vi} + 1$ \tcp{frequency statistics}
        }
    }
    \tcp{evaluate split points of the form $X \in V$}
    $V^* \assign \emptyset$; $\mathit{score}^* \assign 0$ \tcp{initialize best split point}
    \ForAll{$V\subset dom(X)$, such that $1 \le |V| \le l$}{
        \nllabel{alg:class:dt:ECA:For2}
        \For{$i=1,\dots,k$}{
            $\hP(c_i|\bD_Y) \assign \frac{\sum_{v \in V} n_{vi}}
            {\sum_{j=1}^k \sum_{v \in V} n_{vj}}$\;
            $\hP(c_i|\bD_N) \assign \frac{\sum_{v \not\in V} n_{vi}}
                {\sum_{j=1}^k \sum_{v \not\in V} n_{vj}}$\;
        }
        $\mathit{score}(X \in V) \assign Gain(\bD, \bD_Y, \bD_N)$\;
        %\tcp{use Eq.\nosp\eqref{eq:class:dt:gain}}\;
        \If{$\mathit{score}(X \in V) > \mathit{score}^*$}{
            $V^* \assign V; \mathit{score}^* \assign \mathit{score}(X \in V)$\;
        }
    }
    \Return{$(V^*, \mathit{score}^*)$}
\end{tightalgo}
\end{frame}



\begin{frame}{Discretized {\tt sepal length}: Class Frequencies}

Consider the 2-dimensional Iris dataset comprising the {\tt sepal
    length} and {\tt sepal width} attributes.

\medskip

{\tt sepal length} has been discretized as shown in {Table}, where we 
can see the class frequencies $n_{vi}$:

\begin{center}
\renewcommand{\arraystretch}{1.1}
\begin{tabular}{|c|l|c|c|}
                \hline
                \multirow{2}{*}{Bins} & \multirow{2}{*}{$v$: values }
                    & \multicolumn{2}{c|}{Class frequencies
                        ($n_{vi}$)}\\
                    \cline{3-4}
                & & $c_1$:{\tt iris-setosa} & $c_2$:{\tt other}\\
                \hline
                $[4.3, 5.2]$ & Very Short ($a_1$) & 39 & 6\\
                $(5.2, 6.1]$ & Short ($a_2$) & 11 & 39\\
                $(6.1, 7.0]$ & Long ($a_3$) & 0 & 43\\
                $(7.0, 7.9]$ & Very Long ($a_4$) & 0 & 12\\
                \hline
\end{tabular}%}{}
\end{center}
\end{frame}

\begin{frame}{Discretized {\tt sepal length}: Best Split Point}

    Consider the split point $X_1 \in \{a_1,a_3\}$: 
    \begin{align*}
        \hP(c_1|\bD_Y) &=
        \frac{n_{a_1 1}+n_{a_3 1}}{
        (n_{a_1 1}+n_{a_3 1}) + (n_{a_1 2}+n_{a_3 2})}
        = {39+0 \over (39+0)+(6+43)} = 0.443\\
        \hP(c_2|\bD_Y) &= 1 - \hP(c_1 | \bD_Y) = 0.557
    \end{align*}
    with the entropy given as
    \begin{align*}
        H(\bD_Y) = -(0.443 \log_2 0.443 + 0.557 \log_2 0.557) =
        0.991
    \end{align*}
    Compute the class PMF for $\bD_N$:
    \begin{align*}
        \hP(c_1|\bD_N) &
        = \frac{n_{a_2 1}+n_{a_4 1}}{
        (n_{a_2 1}+n_{a_4 1}) + (n_{a_2 2}+n_{a_4 2})}
        = {11+0 \over (11+0)+(39+12)} = 0.177\\
        \hP(c_2|\bD_N) &= 1 - \hP(c_1|\bD_N) = 0.823
    \end{align*}
    with the entropy given as
    \begin{align*}
        H(\bD_N) = -(0.177 \log_2 0.177 + 0.823 \log_2 0.823) =
        0.673
    \end{align*}
\end{frame}

\begin{frame}{Discretized {\tt sepal length}: Best Split Point}

    $V \in \{a_1,a_3\}$ splits the input data $\bD$ into partitions of size

    \begin{align*}
    \card{\bD_Y} = 39+6+43=88
    \end{align*}
    \begin{align*}
    \card{\bD_N} = 150-88=62.
    \end{align*}

    The entropy of the split is therefore given as

    \begin{align*}
        H(\bD_Y, \bD_N) = {88 \over 150} H(\bD_Y) + {62 \over
        150} H(\bD_N) = 0.86
    \end{align*}

    Since the entropy of $\bD$ is $H(\bD) = 0.918$.
    The gain is given as:
    \begin{align*}
        Gain = H(\bD) - H(\bD_Y, \bD_N) = 0.918 - 0.86 = 0.058
    \end{align*}

\end{frame}

\begin{frame}{Categorical Split Points for Discretized {\tt sepal length}}
\begin{center}
\renewcommand{\arraystretch}{1.1}
\begin{tabular}{|c|c|c|}
    \hline
    $V$ & Split entropy & Info.\ gain\\
    \hline
	$\{ a_1 \}$ & \textbf{0.509} & \textbf{0.410}\\
    $\{ a_2 \}$ & 0.897 & 0.217\\
    $\{ a_3 \}$ & 0.711 & 0.207\\
    $\{ a_4 \}$ & 0.869 & 0.049\\
    $\{ a_1, a_2\}$ & 0.632 & 0.286\\
    $\{ a_1, a_3\}$ & 0.860 & 0.058\\
    $\{ a_1, a_4\}$ & 0.667 & 0.251\\
    $\{ a_2, a_3\}$ & 0.667 & 0.251\\
    $\{ a_2, a_4\}$ & 0.860 & 0.058\\
    $\{ a_3, a_4\}$ & 0.632 & 0.286\\
    \hline
    \end{tabular}%}{}
\end{center}
Best split: $X \in \{a_1\}$.
\end{frame}
