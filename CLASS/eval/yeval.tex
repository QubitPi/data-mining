\lecture{eval}{eval}

\date{Chapter 22: Classification Assessment}
\newcommand{\specificity}{\textit{specif\/{i}city}}

\begin{frame}
\titlepage
\end{frame}



\begin{frame}{Classif\/{i}cation Assessment}

A classif\/{i}er is a model or
function $M$ that predicts the class label $\hy$ for a given input
example $\bx$:
\begin{align*}
  \hy = M(\bx)
\end{align*}
where $\bx = (x_1, x_2, \ldots, x_d)^T \in \setR^d$ is a point in
$d$-dimensional space and $\hy \in \{c_1, c_2, \ldots, c_k\}$ is
its predicted class.

\medskip
To build the classif\/{i}cation model $M$ we need a {\em training set} of
points along with their known classes. 

\medskip
Once the model $M$ has been trained,
we assess its performance over a
separate {\em testing set} of points for which we know the
true classes. 

\medskip
F{i}nally, the model
can be deployed to predict the class for future points whose class we
typically do not know.
\end{frame}



\begin{frame}{Classif\/{i}cation Performance Measures}

Let $\bD$ be the testing set comprising $n$ points in a $d$
dimensional space, let $\{c_1, c_2, \ldots, c_k\}$ denote the set
of $k$ class labels, and let $M$ be a classif\/{i}er. For $\bx_i \in
\bD$, let $y_i$ denote its true class, and let $\hy_i = M(\bx_i)$
denote its predicted class.

\medskip{\bf Error Rate:} 
The error rate is the fraction
of incorrect predictions for the classif\/{i}er over the testing set,
def\/{i}ned as
\begin{align*}
\tcbhighmath{
  \textit{Error\ Rate} = {1 \over n} \sum_{i=1}^n I(y_i \ne \hy_i)
}
\end{align*}
where $I$ is an indicator function.
Error rate is an estimate of the probability of misclassif\/{i}cation.
The lower the error rate the better the classif\/{i}er.

\medskip{\bf Accuracy:} 
The accuracy of a classif\/{i}er is
the fraction of correct predictions:
\begin{align*}
\tcbhighmath{
  \textit{Accuracy} ={1\over n} \sum_{i=1}^n I(y_i = \hy_i)  = 1 - \textit{Error\ Rate}
}
\end{align*}
Accuracy gives an estimate of the probability of a correct
prediction; thus, the higher the accuracy, the better the
classif\/{i}er.
\end{frame}



\readdata{\dataI}{CLASS/eval/figs/iris-sorted.txt}
\readdata{\dataT}{CLASS/eval/figs/iris-3cls-te.txt}
\begin{frame}{Iris Data: Full Bayes Classifier}
  \framesubtitle{Training data in grey. Testing data in black.}
  Three Classes: {\tt Iris-setosa} ($c_1$; circles), {\tt
  Iris-versicolor} ($c_2$;  squares) and {\tt Iris-virginica} ($c_3$;
  triangles)\\

  \begin{columns}
	\column{0.7\textwidth}
\begin{figure}[!b]
\scalebox{0.7}{
\centering
\psset{dotscale=1.5,fillcolor=lightgray,
      arrowscale=2,PointName=none}
\psset{xAxisLabel=$X_1$, yAxisLabel= $X_2$}
\psgraph[tickstyle=bottom,Dx=0.5,Dy=0.5,
Ox=4,Oy=2]{->}(4.0,2.0)(8.5,4.5){4in}{3in}%
%original dataset
\listplot[plotstyle=dots,dotstyle=Bo,showpoints=true,
      nStart=1,nEnd=50,plotNo=1,plotNoMax=2]{\dataI}
\listplot[plotstyle=dots,dotstyle=Bsquare,showpoints=true,
          nStart=51,nEnd=100,plotNo=1,plotNoMax=2]{\dataI}
          %\psset{dotscale=1}
\listplot[plotstyle=dots,dotstyle=Btriangle,showpoints=true,
          nStart=101,nEnd=150,plotNo=1,plotNoMax=2]{\dataI}
%testing dataset
 \psset{fillcolor=black}
\listplot[plotstyle=dots,dotstyle=Bo,showpoints=true,
      nStart=1,nEnd=10]{\dataT}
\listplot[plotstyle=dots,dotstyle=Bsquare,showpoints=true,
          nStart=11,nEnd=20]{\dataT}
          %\psset{dotscale=1}
\listplot[plotstyle=dots,dotstyle=Btriangle,showpoints=true,
          nStart=21]{\dataT}
\psset{dotscale=2.5,fillcolor=white}
\psdot[dotstyle=Bo](5.0,3.43)
\psdot[dotstyle=Bsquare](5.9,2.75)
\psdot[dotstyle=Btriangle](6.66,3.0)
\begin{psclip}{%
\psline[](4,2)(4,4.5)(8.5,4.5)(8.5,2)(4,2)}
\psplotImp[algebraic](4,2)(8.5,4.5){%
  18.29*(x-5.0)^2-2*12.54*(x-5.0)*(y-3.43)+15.87*(y-3.43)^2 -1}
\psplotImp[algebraic](4,0)(8.5,10.5){%
  5.26*(x-5.9)^2-2*3.58*(x-5.9)*(y-2.75)+11.88*(y-2.75)^2 -1}
\psplotImp[algebraic](4,0)(8.5,10.5){%
  3.05*(x-6.66)^2-2*2.53*(x-6.66)*(y-3.0)+13.1*(y-3.0)^2 -1}
  \psset{linecolor=lightgray}
\psplotImp[algebraic](4,2)(8.5,4.5){%
  18.29*(x-5.0)^2-2*12.54*(x-5.0)*(y-3.43)+15.87*(y-3.43)^2 -4}
\psplotImp[algebraic](4,0)(8.5,10.5){%
  5.26*(x-5.9)^2-2*3.58*(x-5.9)*(y-2.75)+11.88*(y-2.75)^2 -4}
\psplotImp[algebraic](4,0)(8.5,10.5){%
  3.05*(x-6.66)^2-2*2.53*(x-6.66)*(y-3.0)+13.1*(y-3.0)^2 -4}
\end{psclip}
\endpsgraph
}
\end{figure}
\column{0.3\textwidth}
  Mean (in white) and density contours (1 and 2 standard deviations)
  shown for each class.\\

  The classif\/{i}er
  misclassif\/{i}es 8 out of the 30 test cases. Thus, we have
  \begin{align*}
    \textit{Error\ Rate} &= {8/30} = 0.27\\
    \textit{Accuracy}  & ={22/30} = 0.73
  \end{align*}
\end{columns}
\end{frame}




\begin{frame}{Contingency Table--based Measures}

Let $\cD = \{\bD_1, \bD_2, \ldots, \bD_k\}$ denote a partitioning of the
testing points based on their true class labels, where $\bD_{j} =
\{\bx_i  \in \bD \;| y_i = c_{j}\}$.  Let $n_i = |\bD_i|$ denote the
size of true class $c_i$.

\medskip Let $\cR = \{\bR_1, \bR_2, \ldots, \bR_k\}$ denote a
partitioning of the testing points based on the predicted labels, that
is, $\bR_{j} = \{\bx_i \in \bD \;| \hy_i = c_{j}\}$.  Let $m_{j} =
|\bR_{j}|$ denote the size of the predicted class $c_{j}$.

\medskip
$\cR$ and $\cD$ induce a $k \times k$ contingency table $\bN$,
also called a {\em confusion matrix},
\index{classification!confusion matrix} \index{confusion matrix}
def\/{i}ned as follows:
\begin{align*}
  \bN(i,j) = n_{ij}  = \lB| \bR_i \cap \bD_{j} \rB| =
\Bigl| \bigl\{ \bx_a \in \bD \;| \hy_a = c_i \text{ and } y_a = c_{j}
  \bigr\} \Bigr|
\end{align*}
where $1\le i, j \le k$.
The count $n_{ij}$ denotes the number of points with
predicted class $c_i$ whose true label is $c_{j}$. Thus,
$n_{ii}$ (for $1 \le i \le k$) denotes the number of cases where the
classif\/{i}er agrees on the true label $c_i$. The remaining counts
$n_{ij}$, with $i \ne j$, are cases where the classif\/{i}er and true
labels disagree.
\end{frame}



\begin{frame}{Accuracy/Precision and Coverage/Recall}
\small
The class-specif\/{i}c {\em accuracy} or {\em precision} of the
classif\/{i}er $M$ for class $c_i$ is given as the fraction of correct
predictions over all points predicted to be in class $c_i$
\begin{align*}
\tcbhighmath{
  \textit{acc}_i = \textit{prec}_i = \frac{n_{ii}}{m_{i}}
}
\end{align*}
where $m_i$ is the number of examples predicted as $c_i$ by classif\/{i}er $M$.
The higher the accuracy on class $c_i$ the better the classif\/{i}er.
The overall precision or accuracy of the classif\/{i}er is the
weighted average of class-specif\/{i}c accuracies:
\begin{align*}
\tcbhighmath{
  \textit{Accuracy} = \textit{Precision} = \sum_{i=1}^k \lB(\frac{m_i}{n}\rB) \textit{acc}_i = \frac{1}{n} \sum_{i=1}^k n_{ii}
}
\end{align*}

\medskip
The
class-specif\/{i}c {\em coverage} or {\em recall}  of $M$ for class
$c_i$ is the fraction of correct predictions over all points in
class $c_i$:
\begin{align*}
\tcbhighmath{
  \textit{coverage}_i = \textit{recall}_i = \frac{n_{ii}}{n_{i}}
}
\end{align*}
The higher the coverage the better the classif\/{i}er.
\end{frame}


\begin{frame}{F-measure} 
The {\em class-specif\/{i}c
F-measure} tries to balance the precision and recall values, by
computing their harmonic mean for class $c_i$:
\begin{align*}
\tcbhighmath{
  F_i = \frac{2}{\frac{1}{\textit{prec}_i} + \frac{1}{\textit{recall}_i}} =
  \frac{2 \cdot \textit{prec}_i \cdot \textit{recall}_i}{\textit{prec}_i + \textit{recall}_i} =
  \frac{2 \; n_{ii}}{n_{i} + m_{i}}
}
\end{align*}
The higher the $F_i$ value the better the classif\/{i}er.

\medskip
The overall {\em F-measure} for the classif\/{i}er $M$ is the mean of
the class-specif\/{i}c values:
\begin{align*}
\tcbhighmath{
  F = \frac{1}{k} \sum_{i=1}^r F_i
}
  %\label{eq:clust:eval:F}
\end{align*}
For a perfect classif\/{i}er, the maximum value of the F-measure is 1.
\end{frame}



\begin{frame}{Contingency Table for Iris: Full Bayes Classifier}
%{\tabcolsep6pt
\renewcommand{\arraystretch}{1.1}
\begin{center}
\begin{footnotesize}
\begin{tabular}{|l||c|c|c||l|}
    \hline
    & \multicolumn{3}{c||}{True} & \\
    \hline
     Predicted & Iris-setosa ($c_1$) & Iris-versicolor ($c_2$)&
     Iris-virginica($c_3$) & \\
    \hline\hline
  Iris-setosa ($c_1$)&       10 & 0 & 0 & $m_1=10$\\
  Iris-versicolor ($c_2$)&   0  & 7 & 5 & $m_2 = 12$\\
  Iris-virginica  ($c_3$)&   0  & 3 & 5 & $m_3 = 8$\\
  \hline
  & $n_1 = 10$ & $n_2 = 10$ & $n_3 = 10$ & $n=30$\\
  \hline
  \end{tabular}%}{}
\end{footnotesize}
\end{center}
\small
The class-specif\/{i}c
  precison and recall values are:
  \begin{align*}
    \textit{prec}_1 &= {n_{11} \over m_1}  = {10/10} = 1.0 &
    \textit{recall}_1 &= {n_{11} \over n_1}  = {10/10} = 1.0\\[4.5pt]
    \textit{prec}_2 &= {n_{22} \over m_2}  = {7/12} = 0.583 & 
    \textit{recall}_2 &= {n_{22} \over n_2}  = {7/10} = 0.7\\[4.5pt]
    \textit{prec}_3 &= {n_{33} \over m_3}  = {5/8} = 0.625 &
    \textit{recall}_3 &= {n_{33} \over n_3}  = {5/10} = 0.5
  \end{align*}
  The overall accuracy and F-measure is
  \begin{align*}
    \textit{Accuracy} & = {(n_{11} + n_{22} + n_{33}) \over n} = {(10+7+5) \over
    30} = {22/30} = 0.733\\
    F & = {1\over 3}(1.0 + 0.636 + 0.556)  = {2.192 \over 3} = 0.731
  \end{align*}

\end{frame}


\begin{frame}{Binary Classif\/{i}cation: Positive and Negative Class}
When there are only $k=2$
classes, we call class $c_1$ the positive class and $c_2$ the
negative class. The entries of the resulting $2 \times 2$
confusion matrix are
\begin{center}
\begin{tabular}{|c|c|c|}
\hline
& \multicolumn{2}{c|}{\textbf{True Class}} \\ \hline
\textbf{Predicted Class} & Positive ($c_1$) & Negative ($c_2$) \\ \hline
Positive ($c_1$)         & True Positive ({\it TP})   &  False Positive ({\it FP})  \\ \hline
Negative ($c_2$)        &  False Negative ({\it FN})  & True Negative ({\it TN})   \\ \hline
\end{tabular}
\end{center}
\end{frame}

\begin{frame}{Binary Classif\/{i}cation: Positive and Negative Class}

\begin{itemize}
  \item \textit{True Positives (TP):}
    The number of points that
    the classif\/{i}er correctly predicts as positive:
  \begin{align*}
    \mathit{TP} = n_{11} = \bigl|\{\bx_i \;| \hy_i = y_i = c_1\}\bigr|
  \end{align*}

\item \textit{False Positives (FP):}
The number of points the classif\/{i}er
  predicts to be positive, which in fact belong to the negative
  class:
  \begin{align*}
    \mathit{FP} = n_{12} = \bigl|\{\bx_i\;| \hy_i = c_1 \text{ and } y_i = c_2
    \}\bigr|
  \end{align*}

  \item \textit{False Negatives (FN):}
The number of points the classif\/{i}er
    predicts to be in the negative class, which in fact belong to the
    positive class:
  \begin{align*}
    \mathit{FN} = n_{21} = \bigl|\{\bx_i\;| \hy_i = c_2 \text{ and } y_i = c_1
    \}\bigr|
  \end{align*}

     \item \textit{True Negatives (TN):}
The number of points that
     the classif\/{i}er correctly predicts as negative:
  \begin{align*}
    \mathit{TN} = n_{22} = \bigl|\{\bx_i \;| \hy_i = y_i = c_2\}\bigr|
  \end{align*}
\end{itemize}
\end{frame}


\begin{frame}{Binary Classif\/{i}cation: Assessment Measures}

{\bf Error Rate:} 
The error rate for the
binary classif\/{i}cation case is given as the fraction of mistakes
(or false predictions):
\begin{align*}
\tcbhighmath{
\textit{Error\ Rate} = \frac{\mathit{FP}+\mathit{FN}}{n}
}
\end{align*}

\medskip{\bf Accuracy:} 
The accuracy is the fraction
of correct predictions:
\begin{align*}
\tcbhighmath{
\textit{Accuracy} = \frac{\mathit{TP}+\mathit{TN}}{n}
}
\end{align*}

\subsubsection{Class-specif\/{i}c Precision}
The precision for the positive
and negative class is given as
\begin{empheq}[box=\tcbhighmath]{align*}
\begin{split}
  \textit{prec}_P &= \frac{\mathit{TP}}{\mathit{TP}+\mathit{FP}} = \frac{\mathit{TP}}{m_1}\\
  \textit{prec}_N & = \frac{\mathit{TN}}{\mathit{TN}+\mathit{FN}} = \frac{\mathit{TN}}{m_2}
\end{split}
\end{empheq}
where $m_i = \card{\bR_i}$ is the number of points predicted by $M$ as
having class $c_i$.
\end{frame}

\begin{frame}{Binary Classif\/{i}cation: Assessment Measures}

{\bf Sensitivity or True Positive Rate:}
The fraction of correct predictions with respect
to all points in the positive class, i.e., the
recall for the positive class
\begin{align*}
\tcbhighmath{
  \mathit{TPR} = \textit{sensitivity} = \textit{recall}_P = \frac{\mathit{TP}}{\mathit{TP}+\mathit{FN}} = \frac{\mathit{TP}}{n_1}
}
\end{align*}
where $n_1$ is the size of the positive class.

\medskip{\bf Specif\/{i}city or True Negative Rate:}
The recall for the negative class:
\begin{align*}
\tcbhighmath{
  \mathit{TNR} = \specificity = \textit{recall}_N = \frac{\mathit{TN}}{\mathit{FP}+\mathit{TN}} = \frac{\mathit{TN}}{n_2}
}
\end{align*}
where $n_2$ is the size of the negative class.

\end{frame}

\begin{frame}{Binary Classif\/{i}cation: Assessment Measures}

\medskip{\bf False Negative Rate:} Def\/{i}ned as
\begin{align*}
\tcbhighmath{
  \mathit{FNR} = \frac{\mathit{FN}}{\mathit{TP}+\mathit{FN}} = \frac{\mathit{FN}}{n_1}  = 1 - \textit{sensitivity}
}
\end{align*}

\medskip{\bf False Positive Rate:} Def\/{i}ned as
\begin{align*}
\tcbhighmath{
  \mathit{FPR} = \frac{\mathit{FP}}{\mathit{FP}+\mathit{TN}} = \frac{\mathit{FP}}{n_2}= 1 - \specificity
}
\end{align*}
\end{frame}

\readdata{\dataPC}{CLASS/eval/figs/iris-PC.txt}
\readdata{\dataPCte}{CLASS/eval/figs/iris-PC-te.txt}
\begin{frame}{Iris Principal Components Data: Naive Bayes Classifier}
  \framesubtitle{{\tt Iris-versicolor} (class $c_1$; in circles) and
  other two Irises (class $c_2$; in triangles). Training data in grey
  and testing data in black. Class means in white and density contours.}
\begin{figure}[!t]
%    \vspace{0.2in}
  \scalebox{0.7}{
    \centering
    \psset{dotscale=1.5,fillcolor=lightgray,
        arrowscale=2,PointName=none}
    \psset{xAxisLabel=$\bu_1$, yAxisLabel= $\bu_2$}
    \psgraph[tickstyle=bottom,Ox=-4,Oy=-1.5,Dx=1,Dy=0.5]{->}(-4.0,-1.5)(4.0,
1.5){4in}{3in}%
    \listplot[plotstyle=dots,dotstyle=Bo,showpoints=true,
            nEnd=50,plotNo=1,plotNoMax=2]{\dataPC}
    \listplot[plotstyle=dots,dotstyle=Btriangle,showpoints=true,
            nStart=51,plotNo=1,plotNoMax=2]{\dataPC}
    \psset{fillcolor=black}
    \listplot[plotstyle=dots,dotstyle=Bo,showpoints=true,
            nEnd=10,plotNo=1,plotNoMax=2]{\dataPCte}
    \listplot[plotstyle=dots,dotstyle=Btriangle,showpoints=true,
            nStart=11,plotNo=1,plotNoMax=2]{\dataPCte}
    \psset{dotscale=2.5,fillcolor=white}
    \psdot[dotstyle=Bo](-0.64,-0.20)
    \psdot[dotstyle=Btriangle](0.27,0.14)
%randomseed = 4100000
%        V1         V2
%-0.6405687 -0.2035819
%       V1        V2
%0.2695923 0.1402959
%          [,1]      [,2]
%[1,] 0.2881451 0.0000000
%[2,] 0.0000000 0.1802742
%       [,1]      [,2]
%[1,] 6.1402 0.0000000
%[2,] 0.0000 0.2058456
%[1] "inverses"
%         [,1]     [,2]
%[1,] 3.470474 0.000000
%[2,] 0.000000 5.547105
%          [,1]    [,2]
%[1,] 0.1628611 0.00000
%[2,] 0.0000000 4.85801
        \begin{psclip}{%
          \psline[](-4,-1.5)(-4,1.5)(4,1.5)(4,-1.5)(-4,-1.5)}
                \psplotImp[linewidth=1pt,algebraic](-4,-2)(4,3.5){%
                3.47*(x+0.64)^2+5.55*(y+0.20)^2 -1}
                \psplotImp[linewidth=1pt,algebraic,linecolor=lightgray](-4,-2)(4,3.5){%
                3.47*(x+0.64)^2+5.55*(y+0.20)^2 -4}
                \psplotImp[linewidth=1pt,algebraic](-4,-2)(4,3.5){%
                0.16*(x-0.27)^2+4.86*(y-0.14)^2 -1}
                \psplotImp[linewidth=1pt,algebraic,linecolor=lightgray](-4,-2)(4,3.5){%
                0.16*(x-0.27)^2+4.86*(y-0.14)^2 -4}
        \end{psclip}
    \endpsgraph
}
\end{figure}
\end{frame}



\begin{frame}{Iris Principal Components Data: Assessment Measures}
\renewcommand{\arraystretch}{1.1}
\begin{center}
\begin{tabular}{|l||c|c||l|}
    \hline
    & \multicolumn{2}{c||}{True} & \\
    \hline
     Predicted & Positive ($c_1$) & Negative ($c_2$) & \\
    \hline\hline
  Positive ($c_1$)&   $\mathit{TP}=7$  & $\mathit{FP}=7$ & $m_1 = 14$\\
  Negative  ($c_2$)&   $\mathit{FN}=3$ & $\mathit{TN}=13$ & $m_2 = 16$\\
  \hline
  & $n_1 = 10$ & $n_2 = 20$ & $n=30$\\
  \hline
  \end{tabular}%}{}
\end{center}
\small
The
  naive Bayes classif\/{i}er misclassif\/{i}ed 10 out of the 30 test instances,
  resulting in an error rate and accuracy of
  \begin{align*}
    \textit{Error\ Rate} &= {10/30} = 0.33 & 
    \textit{Accuracy} & = {20/30} = 0.67
  \end{align*}
  
  Other performance measures:
  \begin{align*}
    \textit{prec}_P & = {\mathit{TP} \over \mathit{TP}+\mathit{FP}} = {7
	\over 14} = 0.5 & 
    \textit{prec}_N & = {\mathit{TN} \over \mathit{TN}+\mathit{FN}} = {13 \over 16} = 0.8125\\
    \textit{sensitivity}
    & = {\mathit{TP} \over \mathit{TP}+\mathit{FN}} = {7 \over 10} = 0.7 &
    \specificity
    & = {\mathit{TN} \over \mathit{TN}+\mathit{FP}} = {13 \over 20} = 0.65\\
    \mathit{FNR} & = 1 - \textit{sensitivity} = 0.3 & 
    \mathit{FPR} & = 1 - \specificity = 0.35
  \end{align*}

\end{frame}




\begin{frame}{ROC Analysis}

Receiver
Operating Characteristic (ROC) analysis is a popular strategy for
assessing the performance of classif\/{i}ers when there are two
classes. 

\medskip
ROC analysis requires that a classif\/{i}er output a score
value for the positive class for each point in the testing set.
These scores can then be used to order points in decreasing order.

\medskip
Typically, a binary classif\/{i}er
chooses some positive score threshold $\rho$, and
classif\/{i}es all points with score above $\rho$ as positive,
with the remaining points classif\/{i}ed as negative.

\medskip
ROC analysis plots the
performance of the classif\/{i}er over all possible values of the threshold
parameter $\rho$. 


\medskip
In particular, for each value of $\rho$, it
plots the false positive rate (1-specif\/{i}city) on the $x$-axis
versus the true positive
rate (sensitivity) on the $y$-axis.
The resulting plot is called the {\em ROC curve} or {\em ROC plot} for
the classif\/{i}er.
\end{frame}





\begin{frame}[fragile]{ROC Analysis}
Let $S(\bx_i)$ denote the real-valued score for the positive class
output by a classif\/{i}er $M$ for the point $\bx_i$. Let the maximum
and minimum score thresholds observed on testing dataset $\bD$ be
as follows:
\begin{align*}
\rho^{\min} & = \min_i \{ S(\bx_i) \} &
\rho^{\max} & = \max_i \{ S(\bx_i) \}
\end{align*}


Initially, we classify all points as negative. Both {\it TP} and {\it FP} are
thus initially zero, as given in the confusion matrix:
\begin{center}
\begin{tabular}{|c|c|c|}
  \hline
    & \multicolumn{2}{c|}{True}\\
  \hline
  Predicted & Pos & Neg\\
  \hline
  Pos& 0 & 0\\
  Neg& {\it FN} & {\it TN}\\
  \hline
  \end{tabular}
\end{center}
This results in {\it TPR} and {\it FPR}
rates of zero, which correspond to the point $(0,0)$ at the lower
left corner in the ROC plot. 
\end{frame}

\begin{frame}[fragile]{ROC Analysis}
Next, for each distinct value of
$\rho$ in the range $[\rho^{\min}, \rho^{\max}]$, we tabulate the
set of positive points:
\begin{align*}
  \bR_1(\rho) = \{\bx_i \in \bD : S(\bx_i) > \rho \}
\end{align*}
and we compute the corresponding true and false positive rates, to
obtain a new point in the ROC plot. 

F{i}nally, in the last step, we
classify all points as positive. Both $\mathit{FN}$ and $\mathit{TN}$
are thus zero, as per the confusion matrix
\begin{center}
\begin{tabular}{|c|c|c|}
  \hline
    & \multicolumn{2}{c|}{True}\\
  \hline
  Predicted& Pos & Neg\\
  \hline
  Pos& {\it TP} & {\it FP}\\
  Neg& 0 & 0\\
  \hline
  \end{tabular}
\end{center}
resulting in
$\mathit{TPR}$ and $\mathit{FPR}$ values of 1. This results in the point $(1,1)$ at
the top right-hand corner in the ROC plot. 
\end{frame}



\begin{frame}[fragile]{ROC Analysis}
An ideal classif\/{i}er
corresponds to the top left point $(0,1)$, which corresponds to
the case $\mathit{FPR}=0$ and $\mathit{TPR}=1$, that is, the classif\/{i}er has no false
positives, and identif\/{i}es all true positives (as a consequence, it
also correctly predicts all the points in the negative class).
This case is shown in the confusion matrix:
\begin{center}
\begin{tabular}{|c|c|c|}
  \hline
    & \multicolumn{2}{c|}{True}\\
    \hline
  Predicted & Pos & Neg\\
  \hline
  Pos& {\it TP} & 0\\
  Neg& 0 & {\it TN}\\
  \hline
  \end{tabular}
\end{center}

A classif\/{i}er with a curve closer to the
ideal case, that is, closer to the upper left corner, is a better
classif\/{i}er.


\medskip{\bf Area Under ROC Curve:}
The area under the ROC curve,
abbreviated AUC, can be used as a measure of classif\/{i}er
performance. 
The AUC value is
essentially the probability that the classif\/{i}er will rank a random
positive test case higher than a random negative test instance.
\end{frame}



\begin{frame}[fragile]{ROC: Different Cases for $2 \times 2$ Confusion
  Matrix}
%  \processtable{Different cases for $2 \times 2$ confusion matrix
%  \label{tab:class:eval:CMcases}}

\begin{footnotesize}
{\begin{tabular}{c}
  %\subfloat[Initial: All Negative]
{
  \label{tab:class:eval:CMcases:neg}
  \tabcolsep4pt\renewcommand{\arraystretch}{1.1}\begin{tabular}{|c|c|c|}
  \hline
    & \multicolumn{2}{c|}{True}\\
  \hline
  Predicted & Pos & Neg\\
  \hline
  Pos& 0 & 0\\
  Neg& {\it FN} & {\it TN}\\
  \hline
\multicolumn{3}{c}{\fontsize{8}{8}\selectfont(a) Initial: all negative}\\
  \end{tabular}
  }
  \hspace{0.15in}
%  \subfloat[Final: All Positive]
{
  \label{tab:class:eval:CMcases:pos}
  \tabcolsep4pt\renewcommand{\arraystretch}{1.1}\begin{tabular}{|c|c|c|}
  \hline
    & \multicolumn{2}{c|}{True}\\
  \hline
  Predicted& Pos & Neg\\
  \hline
  Pos& {\it TP} & {\it FP}\\
  Neg& 0 & 0\\
  \hline
\multicolumn{3}{c}{\fontsize{8}{8}\selectfont(b) F{i}nal: all positive}\\
  \end{tabular}}
  \hspace{0.15in}
%  \subfloat[Ideal Classifier]
{
  \label{tab:class:eval:CMcases:ideal}
  \tabcolsep4pt\renewcommand{\arraystretch}{1.1}\begin{tabular}{|c|c|c|}
  \hline
    & \multicolumn{2}{c|}{True}\\
    \hline
  Predicted & Pos & Neg\\
  \hline
  Pos& {\it TP} & 0\\
  Neg& 0 & {\it TN}\\
  \hline
\multicolumn{3}{c}{\fontsize{8}{8}\selectfont(c) Ideal classif\/{i}er}\\
  \end{tabular}
  }
\end{tabular}}%{}
\end{footnotesize}
\end{frame}


\begin{frame}{Random Classif\/{i}er} 
A random
classif\/{i}er corresponds to a diagonal line in the ROC plot. 


\medskip
Consider a classif\/{i}er that randomly guesses the class of a
point as positive half the time, and negative the other half.  We
then expect that half of the true positives and true negatives
will be identif\/{i}ed correctly, resulting in the point $(\mathit{TPR}, \mathit{FPR}) =
(0.5, 0.5)$ for the ROC plot. 

\medskip
In
general, any f\/{i}xed probability of prediction, say $r$, for the
positive class, yields the point $(r,r)$ in ROC space.  

\medskip
The
diagonal line thus represents the performance of a random
classif\/{i}er, over all possible positive class prediction thresholds
$r$.  
\end{frame}




\begin{frame}{ROC/AUC Algorithm}
The ROC/AUC takes as
input the testing set $\bD$, and the classif\/{i}er $M$. 

\medskip
The f\/{i}rst
step is to predict the score $S(\bx_i)$ for the positive class
($c_1$) for each test point $\bx_i \in \bD$. Next, we sort the
$(S(\bx_i), y_i)$ pairs, that is, the score and the true class
pairs, in decreasing order of the scores

\medskip
Initially, we set the
positive score threshold $\rho = \infty$.  We then
examine each pair $(S(\bx_i),
y_i)$ in sorted order, and for each distinct value of the score,
we set $\rho = S(\bx_i)$ and plot the point
$$(\mathit{FPR}, \mathit{TPR}) = \lB(\frac{\mathit{FP}}{n_2}, \frac{\mathit{TP}}{n_1}\rB)$$%\pagebreak

\medskip
As each test point is
examined, the true and false positive values are adjusted based on
the true class $y_i$ for the test point $\bx_i$. If $y_1 = c_1$,
we increment the true positives, otherwise, we increment the false
positives

\end{frame}



\begin{frame}{ROC/AUC Algorithm}
The AUC value is computed as each new point is added to the ROC plot.
The algorithm maintains the previous values of the false and true
positives, $\mathit{FP}_{prev}$ and $\mathit{TP}_{prev}$, for the
previous score threshold $\rho$. 

\medskip
Given the current $\mathit{FP}$ and $\mathit{TP}$
values, we compute the area under the curve def\/{i}ned by the four
points
\begin{align*}
(x_1, y_1) & = \lB(\frac{\mathit{FP}_{prev}}{n_2}, \frac{\mathit{TP}_{prev}}{n_1}\rB) &
(x_2, y_2) & = \lB(\frac{\mathit{FP}}{n_2}, \frac{\mathit{TP}}{n_1}\rB)\\
(x_1,0) & = \lB(\frac{\mathit{FP}_{prev}}{n_2}, 0\rB) &
(x_2, 0) &= \lB(\frac{\mathit{FP}}{n_2}, 0\rB)
\end{align*}
These four points def\/{i}ne a trapezoid, whenever $x_2 > x_1$ and $y_2 >
y_1$, otherwise, they def\/{i}ne a rectangle (which may be degenerate, with
zero area).

\medskip
The area
under the trapezoid is given as $b\cdot h$, where
$b=|x_2-x_1|$ is the length of the base of the trapezoid and $h =
\frac{1}{2}(y_2+y_1)$ is the average height of the trapezoid.
\end{frame}


\newcommand{\algroccurve}{\textsc{ROC-Curve}}
\newcommand{\algtraparea}{\textsc{Trapezoid-Area}}
\begin{frame}[fragile]{Algorithm \algroccurve}
\begin{footnotesize}
\begin{tightalgo}[H]{\textwidth-18pt}
%\begin{tightalgo}[!t]{\textwidth-18pt}
  \SetKwInOut{Algorithm}{\algroccurve ($\bD$, $M$)}
\Algorithm{}
$n_1 \gets \bigl|\{\bx_i \in \bD| y_i = c_1\}\bigr|$ \tcp{size of positive class}
$n_2 \gets \bigl|\{\bx_i \in \bD| y_i = c_2\}\bigr|$ \tcp{size of negative class}
\tcp{classify, score, and sort all test points}
$L \gets \text{sort the set } \{(S(\bx_i), y_i)\!: \bx_i \in \bD\} \text{ by
decreasing scores}$\label{alg:class:eval:roc:sortscores}\;
$\mathit{FP} \gets \mathit{TP} \gets 0$ \;
$\mathit{FP}_{prev} \gets \mathit{TP}_{prev} \gets 0$
\label{alg:class:eval:roc:previnit}\;
$AUC \gets 0$\;
$\rho \gets \infty$ \label{alg:class:eval:roc:thetamax}\;
\ForEach{$(S(\bx_i), y_i) \in L$ \label{alg:class:eval:roc:for}} {
  \If {$\rho > S(\bx_i)$}{
  plot point $\lB(\frac{\mathit{FP}}{n_2},\frac{\mathit{TP}}{n_1}\rB)$ \label{alg:class:eval:roc:plot}\;
  $AUC \gets AUC + \textsc{Trapezoid-Area}\lB( \lB(\frac{\mathit{FP}_{prev}}{n_2}, \frac{\mathit{TP}_{prev}}{n_1}\rB), \lB(\frac{\mathit{FP}}{n_2}, \frac{\mathit{TP}}{n_1}\rB)
  \rB)$ \label{alg:class:eval:roc:area}\;
    $\rho \assign S(\bx_i)$ \;
    $\mathit{FP}_{prev} \gets \mathit{FP}$\;
    $\mathit{TP}_{prev} \gets \mathit{TP}$\;
  }
  \lIf {$y_i = c_1$}{
  $\mathit{TP} \gets \mathit{TP}+1$ \label{alg:class:eval:roc:tp}
  }
  \lElse{
  $\mathit{FP} \gets \mathit{FP}+1$ \label{alg:class:eval:roc:fp}
  }
}
plot point $\lB(\frac{\mathit{FP}}{n_2},\frac{\mathit{TP}}{n_1}\rB)$ \label{alg:class:eval:roc:plotfinal}\;
$AUC \gets AUC + \textsc{Trapezoid-Area}
\lB( \lB(\frac{\mathit{FP}_{prev}}{n_2}, \frac{\mathit{TP}_{prev}}{n_1}\rB), \lB(\frac{\mathit{FP}}{n_2}, \frac{\mathit{TP}}{n_1}\rB) \rB)$ \label{alg:class:eval:roc:areafinal}\;
\end{tightalgo}
\end{footnotesize}
\end{frame}


\begin{frame}{Algorithm  \algtraparea}
\begin{small}
\begin{tightalgo}[H]{\textwidth-18pt}
%\BlankLine
\SetKwInOut{AlgorithmTA}{\algtraparea ($(x_1, y_1), (x_2, y_2)$)}
\AlgorithmTA{}
$b \gets |x_2 - x_1|$ \tcp{base of trapezoid}
$h \gets \frac{1}{2}(y_2 + y_1)$ \tcp{average height of trapezoid}
\Return{$(b\cdot h)$}
\end{tightalgo}
\end{small}
\end{frame}




\begin{frame}{Iris PC Data: ROC Analysis}
We use the naive Bayes
  classif\/{i}er to compute the probability that each
  test point belongs to the positive class ($c_1$; {\tt
  iris-versicolor}). 
  
  \medskip
  The score of the classif\/{i}er for test point $\bx_i$
  is therefore $S(\bx_i) = P(c_1 | \bx_i)$.
  The sorted scores (in decreasing order)
  along with the true
  class labels are as follows:

  \begin{center}
  \footnotesize
\begin{tabular}{c}
    \tabcolsep6pt\renewcommand{\arraystretch}{1.1}\begin{tabular}{|c|cccccccccc|}
    \hline
$S(\bx_i)$ & 0.93 & 0.82 & 0.80 & 0.77 & 0.74 & 0.71 & 0.69 & 0.67 & 0.66 & 0.61\\
$y_i$ & $c_2$    &$c_1$    &  $c_2$   & $c_1$   & $c_1$   &$c_1$    &
$c_2$   &
$c_1$   & $c_2$    & $c_2$\\
\hline
\multicolumn{11}{c}{~}\\
\hline
$S(\bx_i)$ & 0.59 &
0.55 &
0.55 &
0.53 &
0.47 &
0.30 &
0.26 &
0.11 &
0.04 &
2.97e-03\\
$y_i$ & $c_2$ & $c_2$& $c_1$& $c_1$& $c_1$& $c_1$& $c_1$&  $c_2$&
$c_2$&  $c_2$\\
\hline
\multicolumn{11}{c}{~}\\
\end{tabular}\\
\tabcolsep6pt\renewcommand{\arraystretch}{1.1}\begin{tabular}{|c|ccccc|}
\hline
$S(\bx_i)$& 1.28e-03 &
2.55e-07 &
6.99e-08 &
3.11e-08 &
3.109e-08\\
$y_i$ & $c_2$ & $c_2$ & $c_2$ & $c_2$ & $c_2$\\
\hline
\multicolumn{6}{c}{~}\\
\hline
$S(\bx_i)$& 1.53e-08 &
9.76e-09 &
2.08e-09 &
1.95e-09 &
7.83e-10 \\
$y_i$ & $c_2$ & $c_2$ & $c_2$ & $c_2$ & $c_2$\\
\hline
  \end{tabular}\\
\end{tabular}%}{}
\end{center}
\end{frame}



\begin{frame}{ROC Plot for Iris PC Data}
AUC for naive Bayes is 0.775, whereas the AUC for the random classifier
(ROC plot in grey) is 0.5.\\
\begin{figure}[!t]\vspace*{-0.4pc}
\centering
\scalebox{0.8}{
\begin{pspicture}(-2,-1)(2,7.5)
\psset{xAxisLabel=False Positive Rate,yAxisLabel=True Positive Rate,%
      xAxisLabelPos={c,-0.125},yAxisLabelPos={-0.15,c}}
  \begin{psgraph}[Dx=0.1,Dy=0.1,axesstyle=frame]{->}(0,0)(1.0,1.0){3.5in}{3in}
\psline[linecolor=lightgray,linewidth=2pt](0,0)(1,1)
\listplot[linewidth=2pt,showpoints=true]{%
0 0
0.05 0
0.05 0.1
0.1 0.1
0.1 0.2
0.1 0.3
0.1 0.4
0.15 0.4
0.15 0.5
0.2 0.5
0.25 0.5
0.3 0.5
0.35 0.5
0.35 0.6
0.35 0.7
0.35 0.8
0.35 0.9
0.35 1
0.4 1
0.45 1
0.5 1
0.55 1
0.6 1
0.65 1
0.7 1
0.75 1
0.8 1
0.85 1
0.9 1
0.95 1
1 1
}
\end{psgraph}
\end{pspicture}}
\end{figure}
\end{frame}


\begin{frame}{ROC Plot and AUC: Trapezoid Region}

  Consider the following sorted scores, along with the true class, for
  some testing
  dataset with $n = 5$, $n_1 = 3$ and $n_2 = 2$.
  \begin{align*}
    (0.9, c_1), (0.8, c_2), (0.8, c_1), (0.8, c_1), (0.1, c_2)
  \end{align*}

Algorithm yields the following points that
  are added to the ROC plot, along with the running AUC:

\begin{tabular}{|c|c|c|c|c|}
    \hline
  $\rho$ & $\mathit{FP}$ & $\mathit{TP}$ & $(\mathit{FPR}, \mathit{TPR})$ & AUC\\
  \hline
  $\infty$ & 0 & 0 & $(0,0)$ & 0\\[-2pt]
  0.9 & 0 & 1 & $(0, 0.333)$ & 0\\[-2pt]
  0.8 & 1 & 3 & $(0.5, 1)$ & 0.333\\[-2pt]
  0.1 & 2 & 3 & $(1, 1)$ & 0.833\\
  \hline
  \end{tabular}

\end{frame}


\begin{frame}{ROC Plot and AUC: Trapezoid Region}
\begin{figure}[!t]\vspace*{-15pt}
\centering
\begin{pspicture}(-2,-1)(2,5.5)
  \psset{xAxisLabel=False Positive Rate,yAxisLabel=True Positive Rate,%
      xAxisLabelPos={c,-0.2},yAxisLabelPos={-0.25,c}}
  \begin{psgraph}[Dx=0.2,Dy=0.2,axesstyle=frame]{->}%
      (0,0)(1.0,1.0){2in}{2in}
\pspolygon[fillstyle=solid,fillcolor=lightgray]%
(0,0)(0,0.333)(0.5,1)(1,1)(1,0)
  \listplot[linewidth=2pt,showpoints=true]{%
  0 0
  0.0 0.333
  0.5 1
  1 1
  }
  \psline[linestyle=dashed](0.5,0)(0.5,1)
  \uput[u](0.25,0.2){$0.333$}
  \uput[u](0.75,0.2){$0.5$}
\end{psgraph}
\end{pspicture}
\end{figure}
\end{frame}



\begin{frame}{Classif\/{i}er Evaluation}

  Consider a 
classif\/{i}er $M$, and some performance measure $\theta$. 
Typically,
the input dataset $\bD$ is randomly split into a disjoint training
set and testing set. The training set is used to learn the model
$M$, and the testing set is used to evaluate the measure $\theta$.


\medskip
How conf\/{i}dent can we be about the classif\/{i}cation
performance? The results may be due to an artifact of the random
split. 

\medskip
Also $\bD$ is
itself a $d$-dimensional multivariate random sample drawn from the
true (unknown) joint probability density function $f(\bx)$ that
represents the population of interest.  Ideally, we would like to
know the expected value $E[\theta]$ of the performance measure
over all possible testing sets drawn from $f$. However, because
$f$ is unknown, we have to estimate $E[\theta]$ from $\bD$.

\medskip
Cross-validation and resampling are two common approaches
to compute the expected value and variance of a given performance
measure.
\end{frame}



\begin{frame}{$K$-fold Cross-Validation}
Cross-validation divides the dataset $\bD$
into $K$ equal-sized parts, called {\em folds}, namely $\bD_1$, $\bD_2$,
$\ldots$, $\bD_K$.

\medskip
Each fold $\bD_i$ is, in turn, treated as the testing set, with
the remaining folds comprising the training set
$\bD\setminus\bD_i = \bigcup_{j \ne i} \bD_{j}$.

\medskip
After training the model $M_i$ on $\bD\setminus\bD_i$, we assess its performance
on the testing set $\bD_i$ to obtain the $i$-th estimate $\theta_i$.

\medskip
The expected value of the performance measure can then be estimated as
\begin{align*}
  \hmu_\theta = E[\theta] = {1\over K} \sum_{i=1}^K \theta_i
\end{align*}
and its variance as
\begin{align*}
  \hsigma^2_\theta  = {1\over K} \sum_{i=1}^K (\theta_i - \hmu_\theta)^2
\end{align*}

\medskip
Usually $K$ is chosen to be 5 or 10. The special case, when $K=n$, is
called {\em leave-one-out}
cross-validation.
\end{frame}


\newcommand{\algcrossvalidation}{\textsc{Cross-Validation}}
\begin{frame}[fragile]{$K$-fold Cross-Validation Algorithm}
%\begin{algorithm}[H]
\begin{tightalgo}[H]{\textwidth-18pt}
\SetKwInOut{Algorithm}{\algcrossvalidation ($K$, $\bD$)}
\Algorithm{}
$\bD \gets $ randomly shuffle $\bD$\;
$\{\bD_1, \bD_2, \ldots, \bD_K\} \gets$ partition $\bD$ in $K$ equal parts\;
\ForEach{$i \in [1, K]$}{
  $M_i \assign $ train classif\/{i}er on $\bD\setminus\bD_i$\;
  $\theta_i \gets $ assess $M_i$ on $\bD_i$\;
}
$\hmu_\theta = {1\over K} \sum_{i=1}^K \theta_i$\;
$\hsigma^2_\theta  = {1\over K} \sum_{i=1}^K (\theta_i - \hmu_\theta)^2$\;
\Return{$\hmu_\theta, \hsigma^2_\theta$}
\end{tightalgo}
\end{frame}

\begin{frame}[fragile]{$K$-fold Cross-Validation}

Consider the 2-dimensional Iris dataset 
with $k=3$ classes. We assess
the error rate of the full Bayes classif\/{i}er via 5-fold
cross-validation, obtaining the following error rates when testing
on each fold:
\begin{align*}
  \theta_1 & = 0.267 & \theta_2 & = 0.133 &
  \theta_3 & = 0.233 & \theta_4 & = 0.367 & \theta_5 = 0.167
\end{align*}

The mean and variance for the error
rate are as follows:
\begin{align*}
  \hmu_\theta &= {1.167\over 5} = 0.233 & \hsigma^2_\theta & = 0.00833
\end{align*}

Performing ten
5-fold cross-validation runs for the Iris dataset results in the mean of the
expected error rate as $0.232$, and the mean of the variance as
$0.00521$, with the variance in both these estimates being
less than~$10^{-3}$.

\end{frame}



\begin{frame}{Bootstrap Resampling}

The bootstrap method draws $K$ random samples
of size $n$ {\em with replacement} from $\bD$.Each sample $\bD_i$
is thus the same size as $\bD$, and has several repeated points.

\medskip
The probability that a particular point $\bx_{j}$ is
not selected even after $n$ tries is given as
\begin{align*}
  P(\bx_{j} \not\in \bD_i) = q^n  = \lB( 1- {1\over n}\rB)^n \approx
  e^{-1} = 0.368
\end{align*}
which implies 
that each bootstrap sample contains approximately $63.2\%$ of the
points from $\bD$.

\medskip
The bootstrap samples can be used to evaluate the classif\/{i}er by
training it on each of samples $\bD_i$ and then using the full
input dataset $\bD$ as the testing set.

\medskip
However, the estimated mean and variance of $\theta$ will be
somewhat optimistic owing to the fairly large overlap between the
training and testing datasets (63.2\%). 
\end{frame}




\newcommand{\algbootstrap}{\textsc{Bootstrap-Resampling}}
\begin{frame}{Bootstrap Resampling Algorithm}
\begin{tightalgo}[H]{\textwidth-18pt}
%\begin{algorithm}[H]
\SetKwInOut{Algorithm}{\algbootstrap ($K$, $\bD$)}
\Algorithm{}
\For{$i \in [1, K]$}{
  $\bD_i \assign$ sample of size $n$ with replacement from $\bD$\;
  $M_i \assign $ train classif\/{i}er on $\bD_i$\;
  $\theta_i \gets $ assess $M_i$ on $\bD$\;
}
$\hmu_\theta = {1\over K} \sum_{i=1}^K \theta_i$\;
$\hsigma^2_\theta  = {1\over K} \sum_{i=1}^K (\theta_i - \hmu_\theta)^2$\;
\Return{$\hmu_\theta, \hsigma^2_\theta$}
\end{tightalgo}
\end{frame}



\def\pshlabel#1{\scriptsize {#1}}
\def\psvlabel#1{\scriptsize {#1}}
\begin{frame}{Iris 2D Data: Bootstrap Resampling using Error Rate}
We apply bootstrap sampling to estimate the error rate
 for the full Bayes classif\/{i}er, using $K=50$ samples. The sampling
  distribution of error rates is:
  \begin{center}
\begin{figure}%[!b]
  \scalebox{0.8}{
\centerline{
  \psset{xAxisLabel=Error Rate,yAxisLabel=Frequency,%
  xAxisLabelPos={c,-1.5},yAxisLabelPos={-0.01,c}}
  \begin{psgraph}[Dx=0.01,Ox=0.17,showorigin=false]{->}(0.17,0)(0.28,9){3.5in}{2in}
    \listplot[plotstyle=bar,barwidth=0.2cm,
    fillcolor=lightgray,fillstyle=solid]{%
0.18 4 %
0.1866667 6 %
0.1933333 4 %
0.2 3 %
0.2066667 7 %
0.2133333 3 %
0.22 8 %
0.2266667 5 %
0.2333333 2 %
0.24 3 %
0.2466667 2 %
0.2533333 2 %
0.2666667 1}
  \end{psgraph}
  }}
\end{figure}
\end{center}
\centerline{~}
\bigskip
The expected value and
  variance of the error rate are
  \begin{align*}
  \hmu_\theta &= 0.213 &
  \hsigma^2_\theta & = 4.815 \times 10^{-4}
  \end{align*}
\end{frame}





\begin{frame}{Conf\/{i}dence Intervals}
We would like to derive conf\/{i}dence bounds on how
much the estimated mean and variance may deviate from the true value.

\medskip
To answer this question we make use of the
central limit theorem, which states that the sum of a large number of
independent and identically distributed (IID) random variables has
approximately a normal distribution, regardless of the distribution of
the individual random variables. 

\medskip
Let $\theta_1, \theta_2, \ldots,
\theta_K$ be a sequence of IID random variables,
representing, for example, the error rate or some other performance
measure over the $K$-folds in cross-validation or $K$ bootstrap samples.

\medskip
Assume that each $\theta_i$ has a f\/{i}nite mean
$E[\theta_i] = \mu$ and f\/{i}nite variance $var(\theta_i) = \sigma^2$.

\medskip
Let $\hmu$ denote the sample mean:
\begin{align*}
  \hmu = {1\over K} (\theta_1 + \theta_2 + \cdots + \theta_K)
\end{align*}
\end{frame}


\begin{frame}{Conf\/{i}dence Intervals}
By linearity of expectation, we have
\begin{align*}
  E[\hmu] = E\lB[{1\over K} (\theta_1 + \theta_2 + \cdots + \theta_K)\rB] =
  {1\over K} \sum_{i=1}^K E[\theta_i] = {1\over K} \lB(K\mu\rB) = \mu
\end{align*}
The
variance of $\hmu$ is given as
\begin{align*}
  var(\hmu) = var\lB({1\over K} (\theta_1 + \theta_2 + \cdots +
  \theta_K)\rB)
  = {1 \over K^2} \sum_{i=1}^K var(\theta_i) =
  {1 \over K^2} \lB(K\sigma^2\rB)
  ={\sigma^2 \over K}
\end{align*}
Thus, the standard deviation of $\hmu$ is given as
\begin{align*}
  std(\hmu)  = \sqrt{var(\hmu)} = {\sigma \over \sqrt{K}}
\end{align*}

\medskip
We are interested in the distribution of the $z$-score of $\hmu$,
which is itself a random variable
\begin{align*}
  Z_K = {\hmu - E[\hmu] \over std(\hmu)} = {\hmu - \mu \over
  \frac{\sigma}{\sqrt{K}}} = \sqrt{K} \lB({\hmu - \mu \over \sigma}\rB)
  %\label{eq:class:eval:ZK}
\end{align*}
$Z_K$ specif\/{i}es the deviation of the estimated mean from the true
mean in terms of its standard deviation. 
\end{frame}


\begin{frame}{Conf\/{i}dence Intervals}
The central limit theorem
states that as the sample size increases, the
random variable $Z_K$ {\em converges in distribution}
to the standard normal
distribution (which has mean 0 and variance 1). That is,
as $K \to \infty$,  for any $x \in \setR$, we have
\begin{align*}
\lim_{K\to\infty} P(Z_K \le x) = \Phi(x)
\end{align*}
where $\Phi(x)$ is the cumulative distribution function for the
standard normal density function $f(x| 0,1)$. 

\medskip
Let $z_{\alpha/2}$
denote the $z$-score value that encompasses $\alpha/2$ of the
probability mass for a standard normal distribution, that is,
\begin{align*}
  P(0 \le Z_K \le z_{\alpha/2}) = \Phi(z_{\alpha/2}) - \Phi(0) =
  \alpha/2
\end{align*}
then, because the normal distribution is symmetric about the mean,
we have
\begin{align*}
  \lim_{K\to\infty} P(-z_{\alpha/2} \le Z_K \le z_{\alpha/2}) =
  2 \cdot P(0 \le Z_K \le z_{\alpha/2}) = \alpha
\end{align*}

\end{frame}

\begin{frame}{Conf\/{i}dence Intervals}
Note that
\begin{align*}
  -z_{\alpha/2} \le Z_K \le z_{\alpha/2} 
  \text{ implies that }
  \lB(\hmu - z_{\alpha/2}{\sigma \over \sqrt{K}}\rB) \le \mu
  \le \lB(\hmu+z_{\alpha/2}{\sigma \over \sqrt{K}}\rB)
\end{align*}

\medskip
We obtain bounds on the value of the true mean $\mu$ in terms of the
estimated value $\hmu$:
\begin{align*}
  \lim_{K\to\infty} P\lB(\hmu - z_{\alpha/2}{\sigma \over \sqrt{K}} \le \mu \le
  \hmu + z_{\alpha/2}{\sigma \over \sqrt{K}} \rB) = 1-\alpha
\end{align*}

\medskip
Thus, for any given {\em level of conf\/{i}dence} $\alpha$, we can compute
the probability that the true mean $\mu$ lies in the $\alpha\%$
conf\/{i}dence interval $\lB(\hmu - z_{\alpha/2}{\sigma \over
\sqrt{K}}, \hmu + z_{\alpha/2}{\sigma \over \sqrt{K}}\rB)$. 

\end{frame}



\begin{frame}{Confidence Intervals: Unknown Variance}
In general we do not know the 
  true variance
$\sigma^2$. However, we can replace $\sigma^2$ by
the sample variance
\begin{align*}
  \hsigma^2 = {1 \over K} \sum_{i=1}^K (\theta_i - \hmu)^2
\end{align*}
because $\hsigma^2$ is a {\em consistent} estimator for
$\sigma^2$, that is, as $K\to\infty$, $\hsigma^2$ converges with
probability 1, also called {\em converges almost surely}, to
$\sigma^2$. 

\medskip
The central limit theorem then states that the random
variable $Z_K^*$ def\/{i}ned below converges in distribution to the
standard normal distribution:
\begin{align*}
  Z_K^* = \sqrt{K} \lB({\hmu - \mu \over \hsigma}\rB)
\end{align*}
and thus, we have
\begin{align*}
\tcbhighmath{
  \lim_{K\to\infty} P\lB(\hmu - z_{\alpha/2}{\hsigma \over \sqrt{K}} \le \mu \le
  \hmu + z_{\alpha/2}{\hsigma \over \sqrt{K}} \rB) = 1-\alpha
}
\end{align*}

\medskip
%The interval
$\lB(\hmu - z_{\alpha/2}{\hsigma \over \sqrt{K}},\;
  \hmu + z_{\alpha/2}{\hsigma \over \sqrt{K}} \rB)$ is the 100(1-$\alpha)\%$
  conf\/{i}dence interval for~$\mu$.
\end{frame}



%\begin{example}
%Consider Example~\ref{ex:class:eval:cv}, where we applied 5-fold
%cross-validation ($K=5$) to assess the error rate of the full
%Bayes classif\/{i}er. The estimated expected value and variance for
%the error rate were as follows:
%\begin{align*}
%  \hmu_\theta &= 0.233 & \hsigma^2_\theta & = 0.00833 &
%  \hsigma_\theta & =
%  \sqrt{0.00833} = 0.0913
%\end{align*}
%Let $\alpha = 0.95$ be the conf\/{i}dence value. It is known that the
%standard normal distribution has 95\% of the probability density within
%$z_{\alpha/2}=1.96$ standard deviations from the mean.
%Thus, in the limit of large sample size, we have
%\begin{align*}
%  P\Biggl( \mu \in \lB(\hmu_\theta - z_{\alpha/2}{\hsigma_\theta \over
%  \sqrt{K}}, \hmu_\theta +
%  z_{\alpha/2}{\hsigma_\theta \over \sqrt{K}}\rB) \Biggr) = 0.95
%\end{align*}
%Because $z_{\alpha/2}{\hsigma_\theta \over \sqrt{K}} = {1.96
%\times 0.0913 \over \sqrt{5}} = 0.08$, we have
%\begin{align*}
%  P\Bigl( \mu \in (0.233 - 0.08, 0.233 + 0.08) \Bigr) =
%  P\Bigl( \mu \in (0.153, 0.313) \Bigr) = 0.95
%\end{align*}
%Put differently, with 95\% conf\/{i}dence, the true expected error rate lies
%in the interval $(0.153,0.313)$.
%
%If we want greater conf\/{i}dence, for example, for $\alpha=0.99$,
%then the corresponding $z$-score value is $z_{\alpha/2} = 2.58$,
%and thus $z_{\alpha/2}{\hsigma_\theta \over \sqrt{K}} = {2.58
%\times 0.0913 \over \sqrt{5}} = 0.105$. The $99\%$ conf\/{i}dence
%interval for $\mu$ is therefore wider $(0.128,0.338)$.
%
%Nevertheless, $K=5$ is not a large sample size, and thus
%the above conf\/{i}dence intervals are not that reliable.
%\label{ex:class:eval:conflarge}
%\end{example}
%


\begin{frame}{Confidence Intervals: Small Sample Size}
The conf\/{i}dence interval
applies only when the sample
size $K\to \infty$. However, in practice for $K$-fold cross-validation
or bootstrap resampling $K$ is small.

\medskip
In the small sample case, instead of the
normal density to derive the conf\/{i}dence interval, we use the
Student's $t$
distribution. 


\medskip
In particular, we choose the value
$t_{\alpha/2,K-1}$ such that the cumulative $t$ distribution
function with $K-1$ degrees of freedom encompasses $\alpha/2$ of
the probability mass, that is,
\begin{align*}
  P(0 \le Z_K^* \le t_{\alpha/2,K-1}) = T_{K-1}(t_{\alpha/2}) - T_{K-1}(0) =
  \alpha/2
\end{align*}
where $T_{K-1}$ is the cumulative distribution function for the
Student's $t$ distribution with $K-1$ degrees of freedom. 

\medskip
The 100(1-$\alpha\%)$ conf\/{i}dence interval for the true mean $\mu$ is thus
$$
\tcbhighmath{
\lB(\hmu - t_{\alpha/2,K-1}{\hsigma \over \sqrt{K}} \le \mu \le
 \hmu + t_{\alpha/2,K-1}{\hsigma \over \sqrt{K}} \rB)
}
$$
Note the dependence of the interval on both $\alpha$ and the sample size
$K$.
\end{frame}



\begin{frame}{Student's ${\it t}$ Distribution: ${\it K}$ Degrees of Freedom}
\begin{figure}[!t]
  \centerline{
\begin{pspicture}(-6,-0.5)(6,5.5)
\psset{linewidth=1pt,plotpoints=100}
\psset{xAxisLabel=$x$,yAxisLabel=$y$}
\pslegend[rt]{%
\psline[linewidth=2pt](0,0)(0.5,0)\hspace{0.2in}  & $f(x|0,1)$\\
\psline[linestyle=dashed](0,0)(0.5,0)\hspace{0.2in} & $t(10)$\\
\psline[linestyle=dotted](0,0)(0.5,0)\hspace{0.2in} & $t(4)$\\
\psline[](0,0)(0.5,0)\hspace{0.2in} & $t(1)$%
}
\begin{psgraph}[Dy=0.1]{->}(0,0)(-5.5,0)(5.5,0.5){4.5in}{2in}
%\psaxes[Dy=0.1]{->}(0,0)(-5.5,0)(5.5,0.5)
\psGauss[mue=0,sigma=1,linewidth=2pt]{-4.5}{4.5}
\psTDist[nue=1]{-4.5}{4.5}
\psTDist[nue=4,linestyle=dotted]{-4.5}{4.5}
\psTDist[nue=10,linestyle=dashed]{-4.5}{4.5}
\end{psgraph}
\end{pspicture}
}
\end{figure}
\end{frame}


\begin{frame}{Iris 2D Data: Confidence Intervals}
We apply 5-fold
cross-validation ($K=5$) to assess the error rate of the full
Bayes classif\/{i}er on the Iris 2D data. 
The estimated expected value and variance for
the error rate are as follows:
\begin{align*}
  \hmu_\theta &= 0.233 & \hsigma^2_\theta & = 0.00833 &
  \hsigma_\theta & =
  \sqrt{0.00833} = 0.0913
\end{align*}
Let $\alpha = 0.95$ be the conf\/{i}dence value. 
It is known that the
standard normal distribution has 95\% of the probability density within
$z_{\alpha/2}=1.96$ standard deviations from the mean.
Thus, we have $z_{\alpha/2}{\hsigma_\theta \over \sqrt{K}} = {1.96
\times 0.0913 \over \sqrt{5}} = 0.08$, and the confidence interval is
\begin{align*}
  P\Bigl( \mu \in (0.233 - 0.08, 0.233 + 0.08) \Bigr) =
  P\Bigl( \mu \in (0.153, 0.313) \Bigr) = 0.95
\end{align*}
\end{frame}


\begin{frame}{Iris 2D Data: Small Sample Confidence Intervals}
Due to the small sample
size ($K=5$), we can get a better conf\/{i}dence interval by using the $t$
distribution. For $K-1=4$ degrees of freedom, for $\alpha=0.95$, we get
$t_{\alpha/2,K-1} = 2.776$.
Thus,
$$t_{\alpha/2,K-1}{\hsigma_\theta \over \sqrt{K}} = 2.776 \times {0.0913
\over \sqrt{5}}  = 0.113$$
The 95\% conf\/{i}dence interval is therefore $$(0.233-0.113,0.233+0.113) =
(0.12, 0.346)$$
which is much wider than the overly optimistic
conf\/{i}dence interval $(0.153,0.313)$
obtained for the large sample case.
\end{frame}



\begin{frame}{Comparing Classif\/{i}ers: Paired $t$-Test}
How can we test for a signif\/{i}cant
difference in the classif\/{i}cation performance of two alternative
classif\/{i}ers, $M^A$ and $M^B$ on a given dataset $\bD$.

\medskip
We can apply $K$-fold
cross-validation (or bootstrap resampling) and tabulate their
performance over each of the $K$ folds, with identical folds for
both classif\/{i}ers. 
That is, we perform a {\em paired test}, with
both classif\/{i}ers trained and tested on the same data. 

\medskip
Let
$\theta_1^A, \theta_2^A, \ldots, \theta_K^A$ and $\theta_1^B,
\theta_2^B, \ldots, \theta_K^B$ denote the performance values for
$M_A$ and $M_B$, respectively. To determine if the two classif\/{i}ers
have different or similar performance, def\/{i}ne the random variable $\delta_i$ as the difference in their performance on the $i$th
dataset:
\begin{align*}
  \delta_i = \theta_i^A - \theta_i^B
\end{align*}

\medskip
The expected difference and the
variance estimates are given as:
\begin{align*}
  \hmu_{\delta} &= {1\over K} \sum_{i=1}^K \delta_i &
  \hsigma^2_{\delta} &= {1\over K} \sum_{i=1}^K (\delta_i -
  \hmu_{\delta})^2
\end{align*}
\end{frame}


\begin{frame}{Comparing Classif\/{i}ers: Paired $t$-Test}
\small
  
The null hypothesis $H_0$ is that the
performance of $M^A$ and $M^B$ is the same.
The alternative hypothesis $H_a$ is that they are
not the same, that is:
\begin{align*}
  H_0\!: & \quad \mu_\delta = 0 & H_a\!: & \quad\mu_\delta \ne 0
\end{align*}

\medskip
Def\/{i}ne the $z$-score random variable for the estimated
expected difference as
\begin{align*}
  Z^*_\delta = \sqrt{K} \lB( {\hmu_\delta  - \mu_\delta \over \hsigma_\delta} \rB)
\end{align*}
$Z^*_\delta$ follows a $t$
distribution with $K-1$ degrees of freedom. However, under the
null hypothesis we have $\mu_\delta = 0$, and thus
\begin{align*}
\tcbhighmath{
  Z^*_\delta = {\sqrt{K} \hmu_\delta \over \hsigma_\delta} \sim t_{K-1}
}
\end{align*}
i.e., $Z^*_\delta$
follows the $t$ distribution with $K-1$ degrees of freedom.

\medskip
Given a desired conf\/{i}dence level $\alpha$, we conclude that
\begin{align*}
  P\lB(-t_{\alpha/2,K-1} \le Z^*_\delta \le t_{\alpha/2,K-1}\rB) =
  \alpha
\end{align*}
Put another way, if $Z^*_\delta \not\in \lB(-t_{\alpha/2,K-1},
t_{\alpha/2,K-1} \rB)$, then we may reject the null hypothesis
with $\alpha\%$ conf\/{i}dence. 
\end{frame}



\newcommand{\algpairedTtest}{\textsc{Paired $t$-Test}}
\begin{frame}[fragile]{Paired ${\it t}$-Test via Cross-Validation}
\begin{tightalgo}[H]{\textwidth-18pt}
%\begin{algorithm}[H]
\SetKwInOut{Algorithm}{\algpairedTtest ($\alpha$, $K$, $\bD$)}
\Algorithm{}
$\bD \gets $ randomly shuffle $\bD$\;
$\{\bD_1, \bD_2, \ldots, \bD_K\} \gets$ partition $\bD$ in $K$ equal parts\;
\ForEach{$i \in [1, K]$}{
  $M^A_i, M^B_i \assign $ train the two different classif\/{i}ers on $\bD \setminus \bD_i$\;
  $\theta^A_i, \theta^B_i \gets $ assess $M^A_i$ and $M^B_i$ on
  $\bD_i$\;
  $\delta_i = \theta^A_i - \theta^B_i$\;
}
$\hmu_\delta = {1\over K} \sum_{i=1}^K \delta_i$\;
$\hsigma^2_\delta  = {1\over K} \sum_{i=1}^K (\delta_i - \hmu_\delta)^2$\;
$Z^*_\delta = {\sqrt{K} \hmu_\delta \over \hsigma_\delta}$\;
\eIf{$Z^*_\delta \in \lB(-t_{\alpha/2,K-1},t_{\alpha/2,K-1} \rB)$}{
  Accept $H_0$; both classif\/{i}ers have similar performance\;
}{
  Reject $H_0$; classif\/{i}ers have signif\/{i}cantly different performance\;
} 
\end{tightalgo}
\end{frame}


\begin{frame}[fragile]{Paired $t$-Test}
Consider the 2-dimensional Iris dataset 
  with $k=3$ classes. We compare
  the naive Bayes ($M^A$) with the full Bayes ($M^B$) classif\/{i}er via
  cross-validation using $K=5$ folds. Using error rate as the performance
  measure:
  \begin{align*}
    \matr{
     i  & \vline & 1 & 2 & 3 & 4 & 5\\
     \hline
    \theta_i^A & \vline & 0.233 & 0.267 & 0.1 & 0.4 & 0.3\\
    \theta_i^B & \vline & 0.2 & 0.2 & 0.167 & 0.333 & 0.233\\
    \delta_i & \vline & 0.033 & 0.067 & -0.067 & 0.067 & 0.067\\
    }
  \end{align*}
The estimated expected difference and variance of the differences are
  \begin{align*}
    \hmu_\delta & = {0.167 \over 5} = 0.033 &
    \hsigma^2_\delta & = 0.00333 &
    \hsigma_\delta & = \sqrt{0.00333} = 0.0577
  \end{align*}
  The $\zscore$ value is given as
  $ Z^*_\delta = {\sqrt{K} \hmu_\delta \over \hsigma_\delta} =
    {\sqrt{5} \times 0.033 \over 0.0577 } = 1.28 $

  For $1-\alpha=0.95$ (or
  $\alpha=0.05$) and
  $K-1=4$ degrees of freedom, we have
  $t_{\alpha/2} = 2.776$.
  Because $Z^*_\delta = 1.28 \in (-2.776, 2.776) =
  \bigl(-t_{\alpha/2}, t_{\alpha/2}\bigr)$ we cannot reject the null
  hypothesis, that  is,
there is no signif\/{i}cant difference
  between the naive and full Bayes classif\/{i}er for this dataset.

\end{frame}


\begin{frame}{Bias-Variance Decomposition}
In many applications there may be costs associated with
making wrong predictions. A {\em loss function}
specif\/{i}es the cost or
penalty of predicting the class to be $\hy=M(\bx)$,
when the true class is $y$.

\medskip
A commonly used loss function for classif\/{i}cation is the {\em zero-one
loss}, def\/{i}ned~as
\begin{align*}
  L(y, M(\bx)) = I(M(\bx) \ne y) =
  \begin{cases}
    0 & \text{if } M(\bx)=y\\
  1 & \text{if } M(\bx) \ne y
 \end{cases}
\end{align*}
Thus, zero-one loss assigns a cost of zero if the prediction is correct, and one otherwise.

\medskip
Another commonly used loss function is the {\em squared loss},
def\/{i}ned
as
\begin{align*}
  L(y, M(\bx)) = \lB( y - M(\bx) \rB)^2
\end{align*}
where we assume that the classes are discrete valued, and not
categorical.
\end{frame}



\begin{frame}{Expected Loss}
An ideal or optimal
classif\/{i}er is the one that minimizes the loss function. Because
the true class is not known for a test case $\bx$, the goal of
learning a classif\/{i}cation model can be cast as minimizing the
expected loss:
\begin{align*}
  E_y\bigl[L(y, M(\bx)) \;| \bx\bigr] =
  \sum_{y} L(y, M(\bx)) \cdot P(y|\bx)
\end{align*}
where $P(y|\bx)$ is the conditional probability of class $y$ given
test point $\bx$, and $E_y$ denotes
that the expectation is taken over the different
class values $y$.

\medskip
Minimizing the expected zero--one loss corresponds to minimizing
the error rate.
Let $M(\bx) =
c_i$, then we have
\begin{align*}
  E_y\bigl[L(y, M(\bx)) \;| \bx\bigr] &
  = \sum_{y} I(y \ne c_i) \cdot P(y|\bx)
   = \sum_{y \ne c_i} P(y|\bx)
   = 1 - P(c_i | \bx)
\end{align*}
Thus, to minimize the expected loss we should choose $c_i$ as the
class that maximizes the posterior probability, that is, $c_i =
\mathop{\arg\max}_{y} P(y|\bx)$. By def\/{i}nition
the error rate is simply an
estimate of the expected zero--one loss; this choice thus
minimizes the error rate.
\end{frame}


\begin{frame}{Bias and Variance}
  \small
The expected loss for the squared loss function offers important
insight into the classif\/{i}cation problem because it can be
decomposed into bias and variance terms. 

\medskip
Intuitively, the {\em
bias} of a classif\/{i}er refers to the
systematic deviation of its predicted decision boundary from the
true decision boundary, whereas the {\em variance}
 of a classif\/{i}er refers to the
deviation among the learned decision boundaries over different
training sets. 

\medskip
Because
 $M$ depends on the training set, given a test point $\bx$,
 we denote its predicted value as
 $M(\bx, \bD)$. Consider the expected square loss:
\begin{align*}
E_y\Bigl[ L\bigl(y,M(\bx, \bD)\bigr) \;\bigl| \bx, \bD \Bigr] 
  & = E_y\Bigl[ \bigl(y - M(\bx, \bD)\bigr)^2 \bigl| \bx,
  \bD\Bigr]\\ 
  & =  \underbrace{E_y\Bigl[ \bigl(y - E_y[y|\bx]\bigr)^2 \;
      \bigl| \bx, \bD\Bigr]}_{var(y|\bx)} +
    \underbrace{\Bigl(M(\bx, \bD) -
    E_y[y|\bx]\Bigr)^2}_{\textit{squared-error}}
\end{align*}
The f\/{i}rst term is simply the variance of $y$ given $\bx$.
The second term
 is the squared error between the
predicted value $M(\bx, \bD)$ and the expected value $E_y[y|\bx]$.
\end{frame}

\begin{frame}{Bias and Variance}
  \small
The squared error depends on the training set. We can eliminate
this dependence by averaging over all possible training tests of
size $n$. The average or expected squared error for a given test
point $\bx$ over all training sets is then given as
\begin{align*}
 E_{\bD}\Bigl[\bigl(M(\bx, \bD) - E_y[y|\bx]\bigr)^2\Big]
&  =  \underbrace{E_{\bD}\Bigl[ \bigl(M(\bx,\bD) -
    E_{\bD}[M(\bx,\bD)]\bigr)^2\Bigr]}_{\rm variance} +
    \underbrace{\Bigl(E_{\bD}[M(\bx, \bD)] - E_y[y|\bx]\Bigr)^2}_{\rm bias}
\end{align*}

The
expected squared loss over all test points $\bx$ and over all
training sets $\bD$ of size $n$ yields the following decomposition:
\begin{empheq}[box=\tcbhighmath]{align*}
\begin{split}
E_{\bx,\bD,y}\Bigl[ \bigl(y - M(\bx, \bD)\bigr)^2\Bigr]
  & = \underbrace{E_{\bx,y}\Bigl[\bigl(y - E_y[y|\bx]\bigr)^2 \Bigr]}_{\rm noise} +
\underbrace{E_{\bx,\bD}\Bigl[ \bigl(M(\bx, \bD) -
E_{\bD}[M(\bx, \bD)]\bigr)^2\Bigr]}_{{\it average\ variance}}\\
    &\qquad \quad \qquad \qquad +
    \underbrace{E_\bx\Bigl[\bigl(E_{\bD}[M(\bx, \bD)] -
    E_y[y|\bx]\bigr)^2\Bigr]}_{{\it average\ bias}}
\end{split}
\end{empheq}

The expected square loss over all test points and training
sets can be decomposed into three terms: noise, average bias, and
average variance. 
\end{frame}

\begin{frame}{Bias and Variance}
\small

The noise term is the average variance
$var(y|\bx)$ over all test points $\bx$. It contributes a f\/{i}xed
cost to the loss independent of the model, and can thus be ignored
when comparing different classif\/{i}ers. 

\medskip
The classif\/{i}er specif\/{i}c loss
can then be attributed to the variance and bias terms. 
Bias indicates whether the model $M$ is correct or incorrect. 

\medskip
If the decision boundary is
nonlinear, and we use a linear classif\/{i}er, then it is likely to
have high bias. A nonlinear (or a
more complex) classif\/{i}er is more likely to capture the correct
decision boundary, and is thus likely to have a low bias.

\medskip
The complex
classif\/{i}er is not necessarily better, since we also have to consider
the variance term, which measures the inconsistency of the
classif\/{i}er decisions. A complex classif\/{i}er induces a more complex
decision boundary and thus may be prone to {\em overf\/{i}tting},
and thus may be
susceptible to small changes in training set, which may result in
high variance.

\medskip
In general, the expected loss can be attributed to high bias or
high variance, with typically a trade-off between the two.
\end{frame}



\readdata{\dataPC}{CLASS/eval/figs/iris-PC.txt}
\readdata{\dataC}{CLASS/eval/figs/BV-C.dat}
\begin{frame}[fragile]{Bias-variance Decomposition: SVM Quadratic Kernels}
  \small
  Iris PC Data: 
{\tt Iris-versicolor} (class $c_1$; in circles) and
  other two Irises (class $c_2$; in triangles). $K=10$ Bootstrap
  samples, trained via SVMs, varying the regularization constant 
  $C$ from $10^{-2}$ to $10^2$. A small value of $C$ emphasizes the
  margin, whereas a large value of $C$ tries to minimize the slack terms.
  The decision boundaries over the 10 samples were as follows:\\
  \normalsize
\begin{figure}[!t]\vspace*{4pt}
  \vspace{0.1in}
  \captionsetup[subfloat]{captionskip=0.3in}
       \def\pshlabel#1{ {\footnotesize $#1$}}
        \def\psvlabel#1{ {\footnotesize $#1$}}
  \centerline{
  \subfloat[$C=0.01$]{
  \label{fig:class:eval:BVsvmC01}
    \scalebox{0.7}{
    \centering
    \psset{dotscale=1.5,fillcolor=lightgray,
            arrowscale=2,PointName=none}
\psset{xAxisLabel=$\bu_1$, yAxisLabel= $\bu_2$}
  \psgraph[tickstyle=bottom,Ox=-4,Oy=-3,Dx=1,Dy=1]{->}(-4.0,-3)(4.0,
3){2.5in}{2in}%
    \listplot[plotstyle=dots,dotstyle=Bo,showpoints=true,
          nEnd=50,plotNo=1,plotNoMax=2]{\dataPC}
\listplot[plotstyle=dots,dotstyle=Btriangle,showpoints=true,
          nStart=51,plotNo=1,plotNoMax=2]{\dataPC}
\begin{psclip}{%
      \psline[](-4,-3)(-4,3)(4,3)(4,-3)(-4,-3)}
%C=0.01
\psplotImp[algebraic](-4.1,-3.1)(4.1,3.1){-0.50*x^2+-0.12*x*y+-0.02*y^2+1.34}
\psplotImp[algebraic](-4.1,-3.1)(4.1,3.1){-0.49*x^2+-0.05*x*y+0.01*y^2+1.21}
\psplotImp[algebraic](-4.1,-3.1)(4.1,3.1){-0.47*x^2+-0.12*x*y+0.02*y^2+1.08}
\psplotImp[algebraic](-4.1,-3.1)(4.1,3.1){-0.51*x^2+-0.19*x*y+0.01*y^2+1.07}
\psplotImp[algebraic](-4.1,-3.1)(4.1,3.1){-0.44*x^2+-0.16*x*y+-0.02*y^2+1.17}
\psplotImp[algebraic](-4.1,-3.1)(4.1,3.1){-0.45*x^2+-0.19*x*y+-0.04*y^2+1.08}
\psplotImp[algebraic](-4.1,-3.1)(4.1,3.1){-0.45*x^2+-0.21*x*y+-0.02*y^2+1.25}
\psplotImp[algebraic](-4.1,-3.1)(4.1,3.1){-0.46*x^2+-0.10*x*y+0.01*y^2+1.28}
\psplotImp[algebraic](-4.1,-3.1)(4.1,3.1){-0.41*x^2+-0.14*x*y+-0.01*y^2+1.20}
\psplotImp[algebraic](-4.1,-3.1)(4.1,3.1){0.45*x^2+0.10*x*y+0.01*y^2+-1.06}
\end{psclip}
    \endpsgraph
    }}
    \hspace{0.4in}
  \subfloat[$C=1$]{
  \label{fig:class:eval:BVsvmC1}
  \scalebox{0.7}{
    \centering
    \psset{dotscale=1.5,fillcolor=lightgray,
            arrowscale=2,PointName=none}
\psset{xAxisLabel=$\bu_1$, yAxisLabel= $\bu_2$}
  \psgraph[tickstyle=bottom,Ox=-4,Oy=-3,Dx=1,Dy=1]{->}(-4.0,-3)(4.0,
3){2.5in}{2in}%
    \listplot[plotstyle=dots,dotstyle=Bo,showpoints=true,
          nEnd=50,plotNo=1,plotNoMax=2]{\dataPC}
\listplot[plotstyle=dots,dotstyle=Btriangle,showpoints=true,
          nStart=51,plotNo=1,plotNoMax=2]{\dataPC}
\begin{psclip}{%
      \psline[](-4,-3)(-4,3)(4,3)(4,-3)(-4,-3)}
%C=1
\psplotImp[algebraic](-4.1,-3.1)(4.1,3.1){-1.24*x^2+-0.93*x*y+-0.82*y^2+2.67}
\psplotImp[algebraic](-4.1,-3.1)(4.1,3.1){-1.58*x^2+-0.35*x*y+-0.35*y^2+3.02}
\psplotImp[algebraic](-4.1,-3.1)(4.1,3.1){-1.52*x^2+-1.51*x*y+0.32*y^2+2.60}
\psplotImp[algebraic](-4.1,-3.1)(4.1,3.1){-1.22*x^2+-1.54*x*y+-1.15*y^2+2.31}
\psplotImp[algebraic](-4.1,-3.1)(4.1,3.1){-1.23*x^2+-1.80*x*y+-1.30*y^2+2.76}
\psplotImp[algebraic](-4.1,-3.1)(4.1,3.1){-1.44*x^2+-1.64*x*y+-1.32*y^2+2.70}
\psplotImp[algebraic](-4.1,-3.1)(4.1,3.1){-1.35*x^2+-1.92*x*y+-0.69*y^2+2.40}
\psplotImp[algebraic](-4.1,-3.1)(4.1,3.1){-1.31*x^2+-1.64*x*y+-0.01*y^2+2.57}
\psplotImp[algebraic](-4.1,-3.1)(4.1,3.1){-1.05*x^2+-0.73*x*y+-0.84*y^2+2.22}
\psplotImp[algebraic](-4.1,-3.1)(4.1,3.1){1.45*x^2+1.48*x*y+0.38*y^2+-2.67}
\end{psclip}
    \endpsgraph
    }}
    }
\end{figure}
\end{frame}



\begin{frame}[fragile]{Bias-variance Decomposition: SVM Quadratic Kernels}
\begin{figure}[!t]\vspace*{4pt}
%MJZ -- see the BV-C.txt for the output
  \vspace{0.1in}
  \captionsetup[subfloat]{captionskip=0.3in}
       \def\pshlabel#1{ {\footnotesize $#1$}}
        \def\psvlabel#1{ {\footnotesize $#1$}}
  \centerline{
  \subfloat[$C=100$]{
  \label{fig:class:eval:BVsvmC100}
    \scalebox{0.7}{
    \centering
    \psset{dotscale=1.5,fillcolor=lightgray,
            arrowscale=2,PointName=none}
\psset{xAxisLabel=$\bu_1$, yAxisLabel= $\bu_2$}
  \psgraph[tickstyle=bottom,Ox=-4,Oy=-3,Dx=1,Dy=1]{->}(-4.0,-3)(4.0,
3){2.5in}{2in}%
    \listplot[plotstyle=dots,dotstyle=Bo,showpoints=true,
          nEnd=50,plotNo=1,plotNoMax=2]{\dataPC}
\listplot[plotstyle=dots,dotstyle=Btriangle,showpoints=true,
          nStart=51,plotNo=1,plotNoMax=2]{\dataPC}
\begin{psclip}{%
      \psline[](-4,-3)(-4,3)(4,3)(4,-3)(-4,-3)}
%C = 100
\psplotImp[algebraic](-4.1,-3.1)(4.1,3.1){-1.71*x^2+-1.16*x*y+-2.56*y^2+4.10}
\psplotImp[algebraic](-4.1,-3.1)(4.1,3.1){-5.09*x^2+1.08*x*y+-4.34*y^2+9.38}
\psplotImp[algebraic](-4.1,-3.1)(4.1,3.1){-2.42*x^2+-2.58*x*y+2.43*y^2+3.55}
\psplotImp[algebraic](-4.1,-3.1)(4.1,3.1){-4.38*x^2+-6.34*x*y+-1.60*y^2+8.00}
\psplotImp[algebraic](-4.1,-3.1)(4.1,3.1){-1.74*x^2+-2.93*x*y+-1.08*y^2+3.56}
\psplotImp[algebraic](-4.1,-3.1)(4.1,3.1){-1.69*x^2+-1.82*x*y+-1.44*y^2+3.09}
\psplotImp[algebraic](-4.1,-3.1)(4.1,3.1){-1.54*x^2+-2.27*x*y+-1.06*y^2+2.91}
\psplotImp[algebraic](-4.1,-3.1)(4.1,3.1){-2.26*x^2+-2.96*x*y+1.86*y^2+4.60}
\psplotImp[algebraic](-4.1,-3.1)(4.1,3.1){-1.71*x^2+-1.36*x*y+-2.53*y^2+4.25}
\psplotImp[algebraic](-4.1,-3.1)(4.1,3.1){2.13*x^2+1.98*x*y+1.87*y^2+-3.97}
\end{psclip}
    \endpsgraph
    }}
    \hspace{0.4in}
  \subfloat[Bias-Variance]{
  \label{fig:class:eval:BVsvmQ}
  \def\pshlabel#1{ {\footnotesize {$#1$}}}
  \def\psvlabel#1{ {\footnotesize {$#1$}}}
  \scalebox{0.7}{
    \centering
\psset{dotscale=1.25,fillcolor=lightgray}
\psset{xAxisLabel=$C$,yAxisLabel=$~$,
    xAxisLabelPos={c,-0.075}}
\pslegend[rt]{%
\rule[1ex]{2em}{1pt}%
\psdot[dotstyle=Bsquare,dotscale=1.5](-0.2,0.01) & loss\\
\rule[1ex]{2em}{1pt}%
\psdot[dotstyle=Btriangle,dotscale=1.5](-0.2,0.01) & bias\\
\rule[1ex]{2em}{1pt}%
\psdot[dotstyle=Bo,dotscale=1.5](-0.2,0.01) & variance}
\pstScalePoints(1,1){log}{}
\begin{psgraph}[Dy=0.1,Ox=-2,xlogBase=10]{->}(-2,0)(2.5,0.41){2.5in}{2in}
  \psset{dotscale=1.5}
\listplot[showpoints=true, dotstyle=Bo,plotNoMax=11,
plotNo=6]{\dataC}
\listplot[showpoints=true, dotstyle=Btriangle,plotNoMax=11,plotNo=7]{\dataC}
\listplot[showpoints=true, dotstyle=Bsquare,plotNoMax=11,plotNo=10]{\dataC}
\end{psgraph}
    }}
    }
\end{figure}

\small
Variance of the SVM model increases as we increase $C$, as seen
from the varying decision boundaries. The figure on the right plots
average variance and average bias for different values of $C$, as well
as the expected loss. The bias-variance tradeoff is clearly visible,
since as the bias reduces, the variance increases.  The lowest expected
loss is obtained when $C=1$.
\end{frame}



\begin{frame}{Ensemble Classif\/{i}ers}
A classif\/{i}er is called {\em unstable} if small perturbations in the
training set result in large changes in the prediction or decision
boundary. 

\medskip
High variance classif\/{i}ers are inherently unstable, since they
tend to overf\/{i}t the data.
On the other hand, high bias methods typically underf\/{i}t the data, and usually have
low variance.

\medskip
In either case, the aim of learning is to reduce classif\/{i}cation
error by reducing the variance or bias, ideally both.

\medskip
Ensemble methods create a {\em combined classif\/{i}er} using the
output of multiple {\em base classif\/{i}ers}, which are
trained on different data subsets. Depending on how the training
sets are selected, and on the stability of the base classif\/{i}ers,
ensemble classif\/{i}ers can help reduce the variance and the bias, leading
to a better overall performance.
\end{frame}



\begin{frame}{Bagging}

{\em Bagging} stands for {\em Bootstrap Aggregation}. It is an
ensemble classif\/{i}cation method that employs multiple bootstrap samples
(with replacement) from the input training data $\bD$ to create
slightly different training sets $\bD_i$, $i=1,2,\ldots,K$.
Different base classif\/{i}ers $M_i$ are
learned, with $M_i$ trained on $\bD_i$.

\medskip Given any test point $\bx$, it is f\/{i}rst classif\/{i}ed
using each of the $K$ base classif\/{i}ers, $M_i$. Let the number of
classif\/{i}ers that predict the class of $\bx$ as $c_{j}$ be given as
\begin{align*}
v_{j}(\bx) =
\Bigl|\bigl\{M_i(\bx) = c_{j} \;\bigl| i = 1,\ldots,K\bigr\} \Bigr|
\end{align*}


The combined classif\/{i}er, denoted $\bM^K$, predicts the class of a
test point $\bx$ by {\em majority voting} among the $k$ classes:
\begin{align*}
  \bM^K(\bx) = \mathop{\arg\max}_{c_{j}} \Bigl\{ v_{j}(\bx) \;\bigl| j=1,\ldots,k \Bigr\}
\end{align*}

Bagging can help reduce the variance, especially if the base
classif\/{i}ers are unstable, due to the averaging effect of majority
voting. It does not, in general, have much effect on the bias.
\end{frame}

\begin{frame}[fragile]{Bagging: Combined SVM Classif\/{i}ers}

  SVM classifiers are trained on $K=10$ bootstrap samples using $C=1$. 

\medskip

    \begin{tabular}{|c|c|c|}
          \hline
      $K$ & \text{Zero--one loss} & \text{Squared loss}\\
    \hline
      3 & 0.047 & 0.187\\
      5 & 0.04 & 0.16\\
     8 & 0.02 & 0.10\\
     10 & 0.027 & 0.113\\
     15 & 0.027 & 0.107\\ \hline
    \end{tabular}
\end{frame}


\begin{frame}[fragile]{Bagging: Combined SVM Classif\/{i}ers}
\setcounter{subfigure}{0}
\small
The combined (average) classif\/{i}er is shown in bold.\\

\normalsize
\begin{figure}[!t]
%MJZ -- see the BV-C.txt for the output
  \vspace{0.1in}
  \captionsetup[subfloat]{captionskip=0.25in}
       \def\pshlabel#1{ {\footnotesize $#1$}}
        \def\psvlabel#1{ {\footnotesize $#1$}}
  \centerline{
  \subfloat[$K=10$]{
  \label{fig:class:eval:BagSvmC1}
  \scalebox{0.8}{
    \centering
    \psset{dotscale=1.5,fillcolor=lightgray,
            arrowscale=2,PointName=none}
\psset{xAxisLabel=$\bu_1$, yAxisLabel= $\bu_2$}
  \psgraph[tickstyle=bottom,Ox=-4,Oy=-3,Dx=1,Dy=1]{->}(-4.0,-3)(4.0,
3){2.5in}{2in}%
    \listplot[plotstyle=dots,dotstyle=Bo,showpoints=true,
          nEnd=50,plotNo=1,plotNoMax=2]{\dataPC}
\listplot[plotstyle=dots,dotstyle=Btriangle,showpoints=true,
          nStart=51,plotNo=1,plotNoMax=2]{\dataPC}
\begin{psclip}{%
      \psline[](-4,-3)(-4,3)(4,3)(4,-3)(-4,-3)}
%C=1
\psplotImp[algebraic](-4.1,-3.1)(4.1,3.1){-1.24*x^2+-0.93*x*y+-0.82*y^2+2.67}
\psplotImp[algebraic](-4.1,-3.1)(4.1,3.1){-1.58*x^2+-0.35*x*y+-0.35*y^2+3.02}
\psplotImp[algebraic](-4.1,-3.1)(4.1,3.1){-1.52*x^2+-1.51*x*y+0.32*y^2+2.60}
\psplotImp[algebraic](-4.1,-3.1)(4.1,3.1){-1.22*x^2+-1.54*x*y+-1.15*y^2+2.31}
\psplotImp[algebraic](-4.1,-3.1)(4.1,3.1){-1.23*x^2+-1.80*x*y+-1.30*y^2+2.76}
\psplotImp[algebraic](-4.1,-3.1)(4.1,3.1){-1.44*x^2+-1.64*x*y+-1.32*y^2+2.70}
\psplotImp[algebraic](-4.1,-3.1)(4.1,3.1){-1.35*x^2+-1.92*x*y+-0.69*y^2+2.40}
\psplotImp[algebraic](-4.1,-3.1)(4.1,3.1){-1.31*x^2+-1.64*x*y+-0.01*y^2+2.57}
\psplotImp[algebraic](-4.1,-3.1)(4.1,3.1){-1.05*x^2+-0.73*x*y+-0.84*y^2+2.22}
\psplotImp[algebraic](-4.1,-3.1)(4.1,3.1){1.45*x^2+1.48*x*y+0.38*y^2+-2.67}
\psplotImp[algebraic, linewidth=3pt](-4.1,-3.1)(4.1,3.1){-1.05*x^2+-1.06*x*y+-0.58*y^2+2.06}
\end{psclip}
    \endpsgraph
    }}
    \hspace{0.4in}
  \subfloat[Effect of $K$]{
  \label{fig:class:eval:BagSvmK}
    \scalebox{0.8}{%
    \psset{dotscale=1.5,fillcolor=lightgray,
            arrowscale=2,PointName=none}
\psset{xAxisLabel=$\bu_1$, yAxisLabel= $\bu_2$}
  \psgraph[tickstyle=bottom,Ox=-4,Oy=-3,Dx=1,Dy=1]{->}(-4.0,-3)(4.0,
3){2.5in}{2in}%
    \listplot[plotstyle=dots,dotstyle=Bo,showpoints=true,
          nEnd=50,plotNo=1,plotNoMax=2]{\dataPC}
\listplot[plotstyle=dots,dotstyle=Btriangle,showpoints=true,
          nStart=51,plotNo=1,plotNoMax=2]{\dataPC}
\begin{psclip}{%
      \psline[](-4,-3)(-4,3)(4,3)(4,-3)(-4,-3)}
\psplotImp[algebraic,linecolor=gray,linewidth=2pt](-4.1,-3.1)(4.1,3.1){-1.45*x^2+-0.93*x*y+-0.29*y^2+2.77} %K=3
\psplotImp[algebraic](-4.1,-3.1)(4.1,3.1){-1.36*x^2+-1.23*x*y+-0.66*y^2+2.67} %K=5
\psplotImp[algebraic,linewidth=2pt](-4.1,-3.1)(4.1,3.1){-1.36*x^2+-1.42*x*y+-0.66*y^2+2.63} %K=8
\psplotImp[algebraic](-4.1,-3.1)(4.1,3.1){-1.05*x^2+-1.06*x*y+-0.58*y^2+2.06} %K=10
\psplotImp[algebraic](-4.1,-3.1)(4.1,3.1){-0.79*x^2+-0.81*x*y+-0.38*y^2+1.55} %K=15
\end{psclip}
    \endpsgraph
    }
  }
}
\end{figure}

\small
The worst training
  performance is obtained for $K=3$ (in thick gray) and the best for
  $K=8$ (in thick black).
\end{frame}



\begin{frame}[fragile]{Random Forest}

A {\em random forest} is an ensemble of $K$ classifiers, $M_1$, \ldots $M_K$, where
each classifier is a decision tree created from a different bootstrap
sample, and 
by sampling a random subset
of the attributes at each internal node in the decision tree. 

Bagging only would generate similar decision trees.
The random sampling of the
attributes results in reducing the correlation between the trees in the
ensemble.

The random forest algorithm uses the $t$th bootstrap sample to learn a
decision tree model $M_t$, but it evaluates just $p$ random attributes 
per split point. 
A typical value $p$ is $\sqrt{d}$.

The $K$ decision trees $M_1, M_2, \cdots, M_K$ 
predict the class of a
test point $\bx$ by majority voting:

\begin{align*}
  \bM^K(\bx) = \mathop{\arg\max}_{c_{\!j}} \Bigl\{ v_{\!j}(\bx) \;\bigl| j=1,\ldots,k \Bigr\}
\end{align*}

where $v_{\!j}$ is the
number of trees that predict the class of $\bx$ as $c_{\!j}$. 

Notice that if $p=d$ then the random forest approach is equivalent to
bagging over decision tree models.
\end{frame}

\begin{frame}[fragile]{Random Forest Algorithm}

\newcommand{\algRandomForest}{\textsc{RandomForest}}
\begin{tightalgo}[H]{\textwidth-24pt}
    \SetKwInOut{Algorithm}{\algRandomForest ($\bD, K, p, \eta, \pi$)}
\SetKw{KwBreak}{break}
\Algorithm{}
\ForEach{$\bx_i \in \bD$}{
    $v_{\!j}(\bx_i) \assign 0, \text{ for all } j=1,2,\ldots, k$}
    \For{$t \in [1,K]$}{
$\bD_t \assign$ sample of size $n$ with replacement from $\bD$ \;
$M_t \assign $ \textsc{DecisionTree} $(\bD_t, \eta, \pi, p)$ \;
\ForEach(\tcp*[h]{out-of-bag votes}){$(\bx_i, y_i) \in \bD \setminus \bD_t$}{
      $\hy_i \assign M_t(\bx_i)$\; 
  \lIf{$\hy_i = c_{\!j}$}{$v_{\!j}(\bx_i) = v_{\!j}(\bx_i) + 1$}
  }
}
$\epsilon_{oob} = \frac{1}{n} \cdot \sum_{i=1}^n I\Bigl(y_i \ne
    \arg\max_{c_{\!j}}
    \bigl\{v_{\!j}(\bx_i) |
(\bx_i, y_i) \in \bD\bigr\}\Bigr)$ \tcp*[l]{OOB error}
\Return{$\{M_1, M_2, \ldots, M_K\}$}
\end{tightalgo}
\end{frame}

\begin{frame}{Random Forest - Out of bag estimation}

Given $\bD_t$, any point in $\bD\setminus \bD_t$ is
called an {\em out-of-bag} point for $M_t$.

The {\em out-of-bag} error rate for each $M_t$ may be calculated 
by considering the prediction over its out-of-bag
points.

The out-of-bag (OOB) error for the random forest is given as:
\begin{align*}
\epsilon_{oob} = \displaystyle \frac{1}{n} \cdot \sum_{i=1}^n I\Bigl(y_i \ne
    max_{c_{\!j}}  \bigl\{v_{\!j}(\bx_i) |
(\bx_i, y_i) \in \bD\bigr\}\Bigr)
\end{align*}

Here $I$ is an indicator function that takes on value $1$ if is arument
is true, and $0$ otherwise. 
We compute the majority out-of-bag class for each point $\bx_i \in
\bD$ and check whether it matches the true class $y_i$. 
The out-of-bag error rate is simply the fraction of points where the
out-of-bag majority class does not match the true class $y_i$.

The out-of-bag error rate approximates the cross-validation error rate
quite well.
\end{frame}

\begin{frame}{Random Forest}
    Consider the Iris principal
    components dataset comprising $n=150$ points in 2-dimensional space.
    The task is to separate {\tt Iris-versicolor} (class $c_1$; in circles) from the other two
  Irises (class $c_2$; in triangles). 

\medskip
  
  Since there are only two attributes in this dataset, we pick $p=1$
  attribute at random for each split-point evaluation in a decision
  tree. 

\medskip
  

Each decision tree is grown using $\eta=3$, that is the
  maximum leaf size is 3 (with default minimum purity $\pi=1.0$).

\medskip
  

  We grow $K=5$ decision trees on different bootstrap samples. The
  decision boundary of the random forest is shown in bold in the
  figure. The error rate on the training data is 2.0\%. However, the
  out-of-bag error rate is 49.33\%, which is overly pessimistic in this
  case, since the dataset has only two attributes, and we use only one
  attribute to evaluate each split point. 
\end{frame}

\begin{frame}[fragile]{Random Forest}

\readdata{\dataPC}{CLASS/eval/figs/iris-PC.txt}
\readdata{\dataB}{CLASS/eval/figs/rf_boundary_irisPC_K5.txt}
\psset{dotscale=1.5,fillcolor=lightgray,arrowscale=2,PointName=none}
\psset{xAxisLabel=$\bu_1$, yAxisLabel= $\bu_2$}
 \centerline{
     \psgraph[tickstyle=bottom,Ox=-4,Oy=-1.4,Dx=1,Dy=1]{->}(-4.0,-1.4)(4.0,1.4){3.5in}{2in}%
\listplot[plotstyle=dots,dotstyle=Bo,showpoints=true,
          nEnd=50,plotNo=1,plotNoMax=2]{\dataPC}
\listplot[plotstyle=dots,dotstyle=Btriangle,showpoints=true,
          nStart=51,plotNo=1,plotNoMax=2]{\dataPC}
\psset{dotscale=0.5,fillcolor=black}
\listplot[plotstyle=dots,dotstyle=Bo,showpoints=true]{\dataB}
\endpsgraph
}
\end{frame}

\bgroup{
\makeatletter
\long\def\@tablecaption#1#2{%
\fontsize{8}{11}\selectfont\raggedright%\sans%
\setbox\@tempboxa=\hbox{{#1}\ignorespaces{\hskip4\p@}\ignorespaces{#2}}%
\ifdim\wd\@tempboxa>\tempdime%
\centering{#1}\ignorespaces{\hskip4\p@}\ignorespaces{#2}%
\else%
\hbox to \tempdime{\hss{#1}\ignorespaces{\hskip4\p@}\ignorespaces#2\hss}\fi%
\vskip\belowcaptionskipa}
\makeatother

\begin{frame}[fragile]{Random Forest: Varying $K$}


We used the full Iris dataset
 which has four attributes ($d=4$), and has three classes ($k=3$), 
 $p=\sqrt{d} = 2$, $\eta=3$ and $\pi=1.0$.

\begin{center}
\begin{footnotesize}
  \begin{tabular}{|l|c|c|}
      \hline
      $K$ & $\epsilon_{oob}$ & $\epsilon$\\
      \hline
1 & 0.4333 &0.0267\\
2 & 0.2933 &0.0267\\
3 & 0.1867 &0.0267\\
4 & 0.1200 &0.0400\\
5 & 0.1133 &0.0333\\
6 & 0.1067 &0.0400\\
7 & 0.0733 &0.0333\\
8 & 0.0600 &0.0267\\
9 & 0.0467 &0.0267\\
10& 0.0467 &0.0267\\
\hline
  \end{tabular}
\end{footnotesize}
\end{center}

We can see that the
 out-of-bag error decreases rapidly as we increase the number of trees.
\end{frame}

\begin{frame}{Boosting}

  In {\em Boosting} the main idea is to carefully
select the samples to {\em boost} the performance on hard to classify
instances. 

\medskip
Starting from an initial training sample $\bD_1$,
we train the base
classif\/{i}er $M_1$, and obtain its training error rate. 

\medskip
To construct the
next sample $\bD_2$, we
select the misclassif\/{i}ed instances with higher probability, and after
training $M_2$, we obtain its training error rate. 

\medskip
To construct
$\bD_3$, those instances that are hard to classify by $M_1$ or $M_2$,
have a higher probability of being selected.
This process is repeated
for $K$ iterations. 

\medskip
F{i}nally, the combined classif\/{i}er
is obtained via weighted voting over the output of the $K$ base
classif\/{i}ers $M_1, M_2, \ldots, M_K$.
\end{frame}


\begin{frame}{Boosting}

  Boosting is most benef\/{i}cial when the base classif\/{i}ers are {\em
weak}, that is, have an error rate that is slightly less than that
for a random classif\/{i}er. 

\medskip
The idea is that whereas $M_1$ may not be
particularly good on all test instances, by design $M_2$ may help
classify some cases where $M_1$ fails, and $M_3$ may help classify
instances where
 $M_1$ and $M_2$ fail, and so on. Thus,
boosting has more of a bias reducing effect. 

\medskip
Each of the weak learners
is likely to have high bias (it is only slightly better than random
guessing), but the f\/{i}nal combined classif\/{i}er can have much lower bias,
since different weak learners learn to classify instances in
different regions of the input space.
\end{frame}



\begin{frame}{Adaptive Boosting: AdaBoost}

The boosting process is repeated $K$ times.
Let $t$ denote the iteration and let $\alpha_t$ denote the weight
for the $t$th classif\/{i}er $M_t$. 

\medskip
Let $w^t_i$ denote the weight for
$\bx_i$, with $\bw^t = (w^t_1, w^t_2, \ldots, w^t_n)^T$ being the
weight vector over all the points for the $t$th iteration. 

\medskip
$\bw$ is a probability vector, whose elements sum to one.
Initially all points have equal weights, that is,
\begin{align*}
  \bw^0 = \lB(\frac{1}{n}, \frac{1}{n}, \ldots, \frac{1}{n}\rB)^T =
  \frac{1}{n}\bone
\end{align*}


\medskip
During iteration $t$, the
training sample $\bD_{t}$ is obtained via weighted resampling
using the distribution $\bw^{t-1}$, that is, we draw a sample of
size $n$ with replacement, such that the $i$th point is chosen
according to its probability $w^{t-1}_i$. 

\medskip
Using $\bD_t$ we train the
classif\/{i}er $M_{t}$, and compute its weighted error
rate $\epsilon_{t}$ on the entire input dataset $\bD$:
\begin{align*}
  \epsilon_t = \sum_{i=1}^n w^{t-1}_i \cdot I\bigl(M_t(\bx_i) \ne y_i\bigr)
\end{align*}

\end{frame}



\begin{frame}{Adaptive Boosting: AdaBoost}

The weight for the $t$th classif\/{i}er is then set as
\begin{align*}
  \alpha_{t} =  \ln\lB( \frac{1-\epsilon_{t}}{\epsilon_{t}}\rB)
\end{align*}
The weight for each point $\bx_i \in \bD$ is updated as
\begin{align*}
  w^{t}_i = w^{t-1}_i \cdot \exp \Bigl\{ \alpha_{t} \cdot
  I\bigl(M_t(\bx_i) \ne y_i\bigr) \Bigr\}
\end{align*}

\medskip
If the predicted class matches the true class, that is, if
$M_{t}(\bx_i) = y_i$, then  the
weight for point $\bx_i$ remains unchanged. 

\medskip
If
the point is misclassif\/{i}ed, that is, $M_t(\bx_i) \ne y_i$, then \begin{align*}
  w^{t}_i = w^{t-1}_i \cdot \exp \bigl\{ \alpha_{t} \bigr\}  =
  w^{t-1}_i \exp \Biggl\{
    \ln \lB( \frac{1-\epsilon_{t}}{\epsilon_{t}}\rB)
  \Biggr\}
  = w^{t-1}_i \lB({1 \over \epsilon_{t}} -1\rB)
\end{align*}
Thus, if the error rate $\epsilon_{t}$ is small,
then there is a greater weight increment for $\bx_i$. 
The
intuition is that a point that is misclassif\/{i}ed by a good
classif\/{i}er (with a low error rate) should be more likely to be
selected for the next training dataset. 
\end{frame}



\begin{frame}{Adaptive Boosting: AdaBoost}
For boosting
we require that a base classif\/{i}er has an error
rate at least slightly better than random guessing, that is,
$\epsilon_{t} < 0.5$.  
If the error rate $\epsilon_t \ge 0.5$,
then the boosting method discards the classif\/{i}er, and tries 
another data sample.

\medskip
{\bf Combined Classifier:}
Given the set of boosted classif\/{i}ers, $M_1, M_2, \ldots, M_K$,
along with their weights $\alpha_1, \alpha_2, \ldots, \alpha_K$,
the class for a test case $\bx$ is
obtained via weighted majority voting.

\medskip
Let $v_{j}(\bx)$ denote the
weighted vote for class $c_{j}$ over the $K$ classif\/{i}ers, given as
\begin{align*}
  v_{j}(\bx) = \sum_{t=1}^K \alpha_t \cdot I\bigl(M_t(\bx)= c_{j}\bigr)
\end{align*}
Because $I(M_t(\bx) = c_{j})$ is $1$ only when $M_t(\bx) = c_{j}$, the
variable $v_{j}(\bx)$ simply obtains the tally for class $c_{j}$ among
the $K$ base classif\/{i}ers, taking into account the classif\/{i}er
weights. The combined classif\/{i}er, denoted $\bM^K$, then predicts
the class for $\bx$ as follows:
\begin{align*}
  \bM^K(\bx) = \mathop{\arg\max}_{c_{j}} \Bigl\{ v_{j}(\bx) \;\bigl| j=1,..,k \Bigr\}
\end{align*}
\end{frame}




\newcommand{\algAdaBoost}{\textsc{AdaBoost}}
\begin{frame}[fragile]{AdaBoost Algorithm}
%\begin{tightalgo}[!t]{\textwidth-18pt}
\begin{small}
\begin{tightalgo}[H]{\textwidth-18pt}
\SetKwInOut{Algorithm}{\algAdaBoost ($K$, $\bD$)}
\SetKw{KwBreak}{break}
\Algorithm{}
$\bw^0 \gets  \lB(\frac{1}{n}\rB)\cdot \bone \in \setR^n$\;
$t \gets 1$\;
\While{$t\le K$}{
\lnl{alg:class:eval:adaBoost:resample}
$\bD_t \assign$ weighted resampling with replacement from $\bD$ using $\bw^{t-1}$\;
  $M_t \assign $ train classif\/{i}er on $\bD_t$\;
  $\epsilon_t \gets \sum_{i=1}^n w^{t-1}_i \cdot I\bigl(M_t(\bx_i) \ne
  y_i\bigr)$ \tcp{weighted error rate on $\bD$}
  \lIf{$\epsilon_t = 0$}{\KwBreak}
  \ElseIf{$\epsilon_t < 0.5$}{
  $\alpha_{t} =  \ln\lB( \frac{1-\epsilon_{t}}{\epsilon_{t}}\rB)$ \tcp{classif\/{i}er weight}
  \ForEach{$i \in [1,n]$ }{\tcp{update point weights}
      $w^{t}_i =
        \begin{cases}
          w_i^{t-1} & \text{if } M_t(\bx_i) = y_i\\
          w^{t-1}_i \lB(\frac{1 - \epsilon_{t}}{\epsilon_t}\rB) &
          \text{if }  M_t(\bx_i) \ne y_i
        \end{cases}$\;
    }
    \lnl{alg:class:eval:adaBoost:normalize}
    $\bw^t = \frac{\bw^t}{\bone^T\bw^t}$\tcp{normalize weights}
    $t \gets t+1$\;
  }
}
\Return{$\{M_1, M_2, \ldots, M_K\}$}
\end{tightalgo}
\end{small}
\end{frame}





\begin{frame}[fragile]{Boosting SVMs: Linear Kernel ($C=1$)}
\setcounter{subfigure}{0}

\begin{figure}[!b]%\vspace*{14pt}
  \captionsetup[subfloat]{captionskip=0.25in}
  \def\pshlabel#1{ {\footnotesize {$#1$}}}
  \def\psvlabel#1{ {\footnotesize {$#1$}}}
  \centerline{
  \subfloat[$K=4$ Iterations]{
  \label{fig:class:eval:BoostSvm}
  \scalebox{0.75}{%
    \psset{dotscale=1.5,fillcolor=lightgray,
            arrowscale=2,PointName=none}
\psset{xAxisLabel=$\bu_1$, yAxisLabel= $\bu_2$}
  \psgraph[tickstyle=bottom,Ox=-4,Oy=-2,Dx=1,Dy=1]{->}(-4.0,-2)(4.0,
2){2.5in}{2in}%
    \listplot[plotstyle=dots,dotstyle=Bo,showpoints=true,
          nEnd=50,plotNo=1,plotNoMax=2]{\dataPC}
\listplot[plotstyle=dots,dotstyle=Btriangle,showpoints=true,
          nStart=51,plotNo=1,plotNoMax=2]{\dataPC}
\begin{psclip}{%
      \psline[](-4,-2)(-4,2)(4,2)(4,-2)(-4,-2)}
      %\psplotImp[algebraic](-4.1,-2.1)(4.1,2.1){%
      %-0.00*x-0.00*y-1.00}
      \psplotImp[algebraic](-4.1,-2.1)(4.1,2.1){%
      0.18*x+2.01*y+0.18}
      \psplotImp[algebraic](-4.1,-2.1)(4.1,2.1){%
      0.51*x+0.43*y+-0.48}
      \psplotImp[algebraic](-4.1,-2.1)(4.1,2.1){%
      1.03*x+0.89*y+1.60}
      %\psplotImp[algebraic](-4.1,-2.1)(4.1,2.1){%
      %-0.51*x+0.04*y+0.31}
      %the above line is the same as the one below
      \psline[](0.451,-2)(0.765,2)
    \end{psclip}
    \uput[r](4,-0.45){$h_1$}
    \uput[u](-0.75,2){$h_2$}
    \uput[u](-3.3,2){$h_3$}
    \uput[u](0.765,2){$h_4$}
    \endpsgraph
    }}
    \hspace{0.55in}
    \subfloat[Effect of $K$: Avg.\ error; 5-fold CV]{
\label{fig:class:eval:Boost5CV}
    \scalebox{0.75}{%
\readdata{\dataBoost}{CLASS/eval/figs/boosting_results_K.txt}
\psset{dotscale=1.5,fillcolor=lightgray}
\psset{xAxisLabel=$K$,yAxisLabel=}
\pslegend[rt]{\rule[1ex]{2em}{1.5pt}%
\psdot[dotstyle=Btriangle,dotscale=1.5](-10,0.0125) & Testing Error\\
\rule[1ex]{2em}{1.5pt}%
\psdot[dotstyle=Bo,dotscale=1.5](-10,0.0125) & Training Error}
\begin{psgraph}[Dy=0.05,Dx=50]{->}(0,0)(250,0.4){2.5in}{2in}
%\pstScalePoints(1,1){}{log 0.001 add }
\listplot[showpoints=true, dotstyle=Bo,plotNoMax=3,
plotNo=1,nEnd=27,nStep=1]{\dataBoost}
\listplot[showpoints=true,
dotstyle=Btriangle,plotNoMax=3,plotNo=2,nEnd=27,nStep=1]{\dataBoost}
\end{psgraph}
    }
    }
    }
\end{figure}

\small
Iris PC Data: Hyperplane learnt in $t$th iteration is $h_t$. 
We can observe that the first three hyperplanes $h_1$, $h_2$ and $h_3$
already capture the essential features of
  the nonlinear decision
  boundary.
  Further reduction in the training error is obtained by increasing the
  number of boosting steps $K$.
\end{frame}



\begin{frame}{Stacking}

Stacking or stacked generalization is an ensemble technique where we employ two
layers of classifiers. 

\medskip

The first layer is composed of $K$ base classifiers
which are trained independently on the entire training data $\bD$.

\medskip

The second layer comprises a combiner classifier $C$ that is trained on the
predicted classes from the base classifiers, so that it automatically 
learns how to combine the outputs of the base classifiers to make the
final prediction for a given input. 

\medskip

For example, the combiner classifier
may learn to ignore the output of a base classifiers for an input that
lies in a region of the input space where the base classifier has poor
performance. 
It can also learn to
correct the prediction in cases where most base classifiers do not
predict the outcome correctly. 

\medskip

Stacking is a strategy
for estimating and correcting the biases of the set of base classifiers.
\end{frame}

\begin{frame}[fragile]{Stacking}

 
\newcommand{\algStacking}{\textsc{Stacking}}
\begin{tightalgo}[H]{\textwidth-18pt} 
\SetKwInOut{Algorithm}{\algStacking ($K, \bM, C, \bD$)} 
\SetKw{KwBreak}{break} 
\Algorithm{} 
\tcp{Train base classifiers}
\For{$t \in [1,K]$ }{ 
   \label{alg:class:ensemble:stacking:trainbasestart} 
   $M_t \assign $ train $t$th base classifier on $\bD$\;
   \label{alg:class:ensemble:stacking:trainbaseend} 
} 
\tcp{Train combiner model $C$ on $\bZ$}
$\bZ \assign \emptyset$\; 
\label{alg:class:ensemble:stacking:traincombstart} 
\ForEach{$(\bx_i, y_i) \in \bD$ }{
    $\bz_i \assign \bigl(M_1(\bx_i), M_2(\bx_i), \ldots, M_K(\bx_i) \bigr)^T$\;
    $\bZ \assign \bZ \cup \{(\bz_i, y_i)\}$ \;
}
$C \assign $ train combiner classifier on $\bZ$\;
\label{alg:class:ensemble:stacking:traincombend} 
\Return($C, M_1, M_2, \ldots, M_K$)\; 
\end{tightalgo}
\end{frame}



\begin{frame}{Stacking}
    We apply stacking on the Iris principal components dataset.
    We use three base classifiers, namely SVM with a linear kernel
    (with regularization constant $C=1$), random forests (with number of
    trees $K=5$ and number of random attributes $p=1$), and naive Bayes. 

\medskip

    The combiner classifier is an SVM with a Gaussian kernel (with
        regularization constant $C=1$ and spread parameter
    $\sigma^2=0.2$).

\medskip

    We trained the data on a random subset of 100 points, and tested
    on the remaining 50 points. 

\begin{center}
\begin{tabular}{|l|c|}
\hline
Classifier & Test Accuracy \\ \hline
Linear SVM & 0.68 \\ \hline
Random Forest &  0.82 \\ \hline
Naive Bayes & 0.74 \\ \hline
Stacking & 0.92 \\ \hline
\end{tabular}
\end{center}
\end{frame}

\begin{frame}[fragile]{Stacking}
\readdata{\dataN}{CLASS/eval/figs/stacking_svc_trained.txt}
\readdata{\dataB}{CLASS/eval/figs/stacking_svc_boundary.txt}
\readdata{\dataBo}{CLASS/eval/figs/linearSVC_C1_stacking_boundary.txt}
\readdata{\dataBi}{CLASS/eval/figs/nb_s_stacking_boundary.txt}
\readdata{\dataBii}{CLASS/eval/figs/rf_mf1_stacking_boundary.txt}

\psset{dotscale=1.5,fillcolor=lightgray,arrowscale=2,PointName=none}
\centerline{
\psset{xAxisLabel=$X_1$, yAxisLabel= $X_2$}
\psgraph[tickstyle=bottom,Ox=-4,Oy=-1.4,Dx=1,Dy=1]{->}(-4.0,-1.4)(4.0,1.4){3.5in}{2in}%
\listplot[plotstyle=dots,dotstyle=Bo,showpoints=true,
          nEnd=50,plotNo=1,plotNoMax=3]{\dataN}
\listplot[plotstyle=dots,dotstyle=Btriangle,showpoints=true,
          nStart=51,plotNo=1,plotNoMax=3]{\dataN}
\psset{dotscale=0.5,fillcolor=lightgray}
\listplot[plotstyle=line,linecolor=lightgray]{\dataBo}
\listplot[plotstyle=dots,dotstyle=square,showpoints=true]{\dataBi}
\psset{dotscale=0.5,fillcolor=lightgray}
\listplot[plotstyle=dots,dotstyle=+,showpoints=true]{\dataBii}
  \psset{dotscale=0.5,fillcolor=black}
\listplot[plotstyle=dots,dotstyle=Bo,showpoints=true]{\dataB}
\endpsgraph
}
\end{frame}


