\lecture{logit}{logit}

\date{Chapter 24: Logistic Regression}

\begin{frame}
\titlepage
\end{frame}

%\chapter{Logistic Regression}
%\label{ch:reg:logit}
%
%Given a set of predictor attributes or independent 
%variables $X_1, X_2, \cdots, X_d$, 
%and given a {\em categorical} response or dependent variable $Y$, 
%the aim of {\em logistic regression} is to predict the probability of
%the response variable values based on the independent variables.
%Logistic regression is in fact a classification technique, that given a
%point $\bx_j \in \setR^{d}$ predicts $P(c_i | \bx_j)$ for each class
%$c_i$ in the domain of $Y$ (the set of possible classes or values for
%the response variable). In this chapter, we first consider the binary
%class problem, where the response variable takes on one of two classes
%($0$ and $1$, or positive and negative, and so on). Next, we consider
%the multiclass case, where there are $K$ classes for the response
%variable.
%
%\section{Binary Logistic Regression}
%\index{logistic regression}
%% \index{logistic regression!Bernoulli variable}
\begin{frame}{Binary Logistic Regression}
Given a set of $d$ 
predictor or independent variables $X_1, X_2, \cdots, X_d$, 
and a {\em binary} or {\em Bernoulli} response variable $Y$ 
that takes on only two values, namely, $0$ and $1$. 
%Thus, we are given a training dataset $\bD$ comprising $n$ points
%$\bx_i \in \setR^d$ and the corresponding observed values $y_i \in
%\{0, 1\}$.
%As done in \cref{chap:reg:linear}, we augment the data matrix $\bD$
%by adding a new attribute $X_0$ that is always fixed at the value $1$
%for each point, so that $\abx_i = (1, x_1, x_2,
%\cdots, x_d)^T \in \setR^{d+1}$ denotes the augmented point, 
%and the multivariate random vector
%$\abX$, comprising all the independent attributes
%is given as $\abX = (X_0, X_1, \cdots, X_d)^T$. The augmented training dataset
%is given as $\abD$ comprising the $n$ augmented points 
%$\abx_i$ along with the class labels $y_i$ for $i=1,2,\cdots,n$.

\medskip

Since there are only two outcomes for $Y$, its PMF
for $\abX=\abx$ is:
\begin{align*}
    P(Y=1 | \abX=\abx) & = \pi(\abx) & P(Y=0 | \abX=\abx) = 1 - \pi(\abx)
\end{align*}
where $\pi(\abx)$ denotes the probability of $Y=1$ given $\abX=\abx$. 

\medskip

Instead of directly 
predicting the response value, the goal is to learn
the probability, $P(Y=1|\abX=\abx)$, which is also the expected value of
$Y$ given $\abX=\abx$.

\medskip

Since $P(Y=1|\abX=\abx)$ is a probability, it is not
appropriate to directly use the linear regression model.

\end{frame}



%Further, note that the expected value of $Y$ given $\abX=\abx$
%is
%\begin{align*}
%    E[Y|\abX=\abx] & = 1 \cdot P(Y=1|\abX=\abx) + 0 \cdot
%    P(Y=0|\abX=\abx)\\
%    & = P(Y=1|\abX=\abx)
%    = \pi(\abx)
%\end{align*}
%Therefore, in logistic regression, 
%
%\begin{align*}
%    f(\abx) 
%    = \truew_0 \cdot x_0 + \truew_1 \cdot x_1 + \truew_2 \cdot x_2 +
%    \cdots + \truew_d \cdot x_d
%    = \trueabw^T\abx 
%\end{align*}
%where $\trueabw = (\truew_0, \truew_1, \cdots, \truew_d)^T \in \setR^{d+1}$ is the true
%augmented weight vector, with
%$\truew_0=\trueb$ the true unknown bias term, and $\truew_i$ the true unknown regression
%coefficient or weight for attribute $X_i$.

%\medskip

%The reason we cannot simply use $P(Y=1|\abX=\abx) = f(\abx)$ 
%is due to the fact that $f(\abx)$ can be arbitrarily large or arbitrarily
%small, whereas for logistic regression, we require that the output
%represents a probability value.
%, and thus we need a model that results in an
%output that lies in the interval $[0,1]$. 


\begin{frame}{Logistic Function}
\textit{Logistic regression} comes from the {\em logistic} function
(aka {\em
sigmoid} function) that %meets this requirement. 
%The logistic function 
``squashes'' the output to be between $0$ and $1$ for any scalar input
$z$. %(see \cref{fig:reg:logit:logit})
%It is defined as
%follows:
%\index{logistic function}
%\index{sigmoid function|see{logistic function}}
%\index{logistic regression!logistic function}
%\index{logistic regression!sigmoid function}
\begin{empheq}[box=\tcbhighmath]{align*}
    \theta(z) = \frac{1}{1+\exp\{-z\}} = \frac{\exp\{z\}}{1+\exp\{z\}}
    %\label{eq:reg:logit:logit}
\end{empheq}
%The logistic function ``squashes'' the output to be between $0$ and $1$ for any scalar input
%z$ %(see \cref{fig:reg:logit:logit})
%. The output $\theta(z)$ can therefore be interpreted as a probability.
%
%\begin{figure}[b!]
%    % \vspace{0.2in}
    \def\logit{1/(1 + EXP(-x))}
    \centering
    \scalebox{0.6}{%
    \psset{xAxisLabel=$z$,yAxisLabel=$\theta(z)$,%
    xAxisLabelPos={c,-0.15},yAxisLabelPos={-1.5,c} }
    \psset{labels=y,ticks=y}
    \psset{linewidth=1pt,axesstyle=frame,algebraic,plotstyle=curve}
    \psgraph[Dy=0.1,Ox=-7]{->}(-7,0)(7,1.0001){4in}{2.5in}%
    \psxTick(-7){-\infty}
    \psxTick(0){0}
    \psxTick(7){+\infty}
    \psline[linewidth=0.5pt,linecolor=gray](0,0)(0,1.0)
    \psline[linewidth=0.5pt,linecolor=gray](-7,0.5)(7,0.5)
       \psplot[linewidth=2pt]{-7}{7}{\logit}
    \endpsgraph
    }
%    \vspace{0.5in}
%    \caption{Logistic function.}
%    \label{fig:reg:logit:logit}
%    \vspace*{-0.2in}
%\end{figure}
\end{frame}
%
%% \enlargethispage{0.5\baselineskip}
\begin{frame}{Logistic Function}
\framesubtitle{Example}
%\begin{example}[Logistic Function]
%    \cref{fig:reg:logit:logit} 
    Consider what happens when $z$ is $-\infty$, $+\infty$ and $0$:
    \begin{align*}
        \theta(-\infty) & = \frac{1}{1+\exp\{\infty\}} = \frac{1}{\infty} =
        0\\
        \theta(+\infty) & = \frac{1}{1+\exp\{-\infty\}} =
        \frac{1}{1} = 1\\
        \theta(0) & = \frac{1}{1+\exp\{0\}} = \frac{1}{2} = 0.5
    \end{align*}
    $z=0$ acts as
    a threshold, since, for $z > 0$,
    $\theta(z) > 0.5$, and, for $z < 0$, $\theta(z) < 0.5$. 

	\medskip

    Interpreting $\theta(z)$ as a probability, the larger the $z$
    value, the higher the probability.

	\medskip
    
    Another interesting property of the logistic function is that
    \begin{align*}
        1 - \theta(z) & = 1 - \frac{\exp\{z\}}{1 + \exp\{z\}} =
        \frac{1+\exp\{z\} - \exp\{z\}}{1+\exp\{z\}} =
        \frac{1}{1+\exp\{z\}} = \theta(-z) 
        \label{eq:reg:logit:logit_neg}
    \end{align*}
\end{frame}
%\end{example}
%
\begin{frame}{Binary Logistic Regression}
%
We define the logistic regression
model as follows:
\begin{equation*}
    P(Y=1 | \; \abX = \abx) = \pi(\abx) = \theta(f(\abx)) =
    \theta(\trueabw^T\abx) = \frac{\exp\{\trueabw^T\abx\}}{1
    + \exp\{\trueabw^T\abx\}}
    \label{eq:reg:logit:f}
\end{equation*}
The probability that $Y=1$ is the output of the
logistic function for the input $\trueabw^T\abx$.
%which in turn is linear in
%$\abX$. 
The probability for $Y=0$ is given as
\begin{equation*}
    P(Y=0|\;\abX=\abx)  = 1-P(Y=1 | \; \abX=\abx) =
    \theta(-\trueabw^T\abx) 
    = \frac{1}{1 + \exp\{\trueabw^T\abx\}}
\end{equation*}
%where we used \cref{eq:reg:logit:logit_neg}, 
that is, $1-\theta(z) = \theta(-z)$ for $z= \trueabw^T\abx$.

\medskip

Combining these two cases the full logistic regression model is given as
\begin{empheq}[box=\tcbhighmath]{align*}
    P(Y | \;\abX=\abx) = \theta(\trueabw^T\abx)^Y \cdot
    \theta(-\trueabw^T\abx)^{1-Y}
     %\label{eq:reg:logit:logistic_regression}
\end{empheq}
Since $Y$ is a Bernoulli binary random variable,
$P(Y|\;\abX=\abx) = \theta(\trueabw^T\abx)$ when $Y=1$ and 
$P(Y|\;\abX=\abx) = \theta(-\trueabw^T\abx)$ when $Y=0$.
\end{frame}
%
%% %We will estimate the unknown weight vector $\abw$ using the method
%% %of maximum likelihood as discussed next.
%% Considering both the values of $Y$, the full
%% logistic regression model can be written as follows:
%% \begin{align*}
%%     P(Y | \abX) = 
%%     \begin{cases}
%%         \theta(\abw^T\abX) & \text{if } Y=1\\
%%         1-\theta(\abw^T\abX) = \theta(-\abw^T\abX) & \text{if } Y=0
%%     \end{cases}
%% \end{align*}
%% where we used \cref{eq:reg:logit:logit_neg}, to note 
%% that $1-\theta(z) = \theta(-z)$ for $z= \abw^T\abX$. Since $Y$ takes on only two
%% distinct values, we can write the combined logistic regression model as
%% follows:
%% \begin{align}
%%     P(Y | \;\abX) = \theta(Y\cdot \abw^T\abX)
%%     \label{eq:reg:logit:logistic_regression}
%% \end{align}
%% since $\theta(Y\cdot \abw^T\abX) = \theta(\abw^T\abX)$ when $Y=+1$ and 
%% $\theta(Y\cdot \abw^T\abX) = \theta(-\abw^T\abX)$ when $Y=-1$.
%
%
%\paragraph{Log-Odds Ratio}
%\label{par:reg:logit:log_odds_ratio}
%
%\index{log-odds ratio}
%\index{logistic regression!log-odds ratio}
%\index{logistic regression!logit function}
%\index{logit|see{log-odds ratio}}
\begin{frame}{Log-Odds Ratio}
%
Define the {\em odds ratio} for the occurrence of $Y=1$ as follows:
\begin{align*}
    \odds(Y=1|\abX=\abx) & = \frac{P(Y=1|\abX=\abx)}{P(Y=0|\abX=\abx)}
    = \frac{\theta(\trueabw^T\abx)}{\theta(-\trueabw^T\abx)}\notag\\
    & =  \frac{\exp\{\trueabw^T\abx\}}{1 + \exp\{\trueabw^T\abx\}} \cdot
    \bigl(1+\exp\{\trueabw^T\abx\}\bigr)\notag\\
    & = \tcbhighmath{\exp\{\trueabw^T\abx\}}
\end{align*}
%% This implies
%% \begin{align}
%%     \tcbhighmath{
%%     \odds(Y=1|\abX=\abx) = \exp\{\trueabw^T\abx\}}
%% \end{align}
The logarithm of the odds ratio, called the {\em log-odds ratio}, is
therefore given as:
\begin{align*}
    \ln\bigl(\odds(Y=1|\abX=\abx)\bigr) & = \ln
    \lB(\frac{P(Y=1|\abX=\abx)}{1-P(Y=1|\abX=\abx)}\rB) \notag   = \ln
    \lB(\exp\{\trueabw^T\abx\}\rB) 
    = \trueabw^T\abx\notag\\
    & = \truew_0 \cdot x_0 + \truew_1 \cdot x_1 + \cdots + \truew_d
    \cdot x_d
    %\label{eq:reg:logit:logodds}
\end{align*}
The log-odds ratio function is also called the {\em
logit} function, defined as 
\begin{align*}    
    \text{logit}(z) = \ln\lB(\frac{z}{1-z}\rB)
\end{align*}
It is the inverse of the logistic function.
\end{frame}

\begin{frame}{Log-Odds Ratio}

We can see that 
\begin{align*}
    \ln\bigl(\odds(Y=1|\abX=\abx)\bigr) =
    \text{logit}\bigl(P(Y=1|\abX=\abx)\bigr)
\end{align*}
%
The logistic regression model is therefore based on the assumption that
the log-odds ratio for $Y=1$ given  $\abX=\abx$ is a linear function (or a
weighted sum) of the independent attributes. 
\medskip

Consider the effect of attribute $X_i$ by fixing
the values for all other attributes:% in \cref{eq:reg:logit:logodds}; 
\begin{align*}
    & \ln(\odds(Y=1|\abX=\abx)) = \truew_i \cdot x_i + C\\
    \implies & \odds(Y=1|\abX=\abx) = \exp\lB\{\truew_i \cdot x_i + C\rB\} =
    \exp\lB\{\truew_i \cdot x_i\rB\} \cdot
    \exp\{C\} \; \propto \; \exp\{\truew_i \cdot x_i\}
\end{align*}
where $C$ is a constant comprising the fixed attributes.

\medskip

$\truew_i$ can be interpreted as the change in the log-odds ratio for $Y=1$ for
a unit change in $X_i$, or, equivalently, the odds ratio for $Y=1$ increases
exponentially per unit change in $X_i$.  
\end{frame}
%
%\subsection{Maximum Likelihood Estimation} \label{sec:reg:logit:MLE}
%
%\index{MLE|see{maximum likelihood estimation}} \index{maximum likelihood
%estimation} \index{logistic regression!maximum likelihood estimation}

\begin{frame}{Maximum Likelihood Estimation}
%
%Let $\abD$ be the augmented training dataset comprising the $n$
%augmented points $\abx_i$ along with their labels $y_i$.
%%be given as $\abD = \lB\{\abx_i^T, y_i \rB\}_{i=1,2,\cdots,n}$, 
%Let $\abw = (w_0, w_1, \cdots, w_d)^T$ be the augmented weight
%vector for estimating $\trueabw$. 
%Note that $w_0 = b$ denotes the estimated bias term, and $w_i$ the
%estimated weight for attribute $X_i$.
We will use the maximum likelihood approach to learn
the weight vector $\abw$.

\medskip

{\em Likelihood} is defined as the probability of the observed 
data given $\abw$.
%We assume that the binary response variables $y_i$ are all independent.
%Therefore, the likelihood of the observed responses
%is given as 
\begin{align*}
    L(\abw) & = P(Y|\abw) = \prod_{i=1}^n P(y_i |\; \abx_i)
 = \prod_{i=1}^n \theta(\abw^T\abx_i)^{y_i} \cdot
    \theta(-\abw^T\abx_i)^{1-y_i}
\end{align*}
Instead of maximizing the likelihood, we can maximize the
{\em log-likelihood}, to convert the product into a summation:
%\index{logistic regression!log-likelihood}
%\index{log-likelihood}
\begin{empheq}[box=\tcbhighmath]{align*}
    \ln\left( L(\abw) \right) = 
    \sum_{i=1}^n y_i \cdot \ln \lB( \theta(\abw^T\abx_i) \rB) +
    (1-y_i) \cdot \ln \lB( \theta(-\abw^T\abx_i) \rB)
    %\label{eq:reg:logit:loglikelihood}
\end{empheq}

\end{frame}


\begin{frame}{Maximum Likelihood Estimation}
The negative of the log-likelihood can also be considered as an error
function, the {\em cross-entropy error function}:
%\index{cross-entropy error}
%\index{log-likelihood!cross-entropy error}
%\index{logistic regression!cross-entropy error}
%\index{maximum likelihood estimation!cross-entropy error}
%\enlargethispage{1\baselineskip}
\begin{empheq}[box=\tcbhighmath]{align*}
    E(\abw) & = - \ln\left( L(\abw) \right) = 
    \sum_{i=1}^n y_i \cdot \ln \lB( \frac{1}{\theta(\abw^T\abx_i)} \rB) +
    (1-y_i) \cdot \ln \lB( \frac{1}{1-\theta(\abw^T\abx_i)} \rB)
    %\label{eq:reg:logit:crossentropy}
\end{empheq}
The task of maximizing the log-likelihood is therefore equivalent to
minimizing the cross-entropy error.
\end{frame}
%
%% However, we will directly maximize the log-likelihood
%% function.
%
\begin{frame}{Maximum Likelihood Estimation}
To obtain the optimal weight vector $\abw$,
we would differentiate the log-likelihood function
with respect to $\abw$, set the
result to $\bzero$, and then solve for $\abw$.

\medskip

However, for the log-likelihood formulation presented,%in
%\cref{eq:reg:logit:loglikelihood} 
there is no closed form solution to
compute the weight vector $\abw$.

\medskip

Instead, we use an iterative {\em gradient ascent} method to
compute the optimal value.
%, since \cref{eq:reg:logit:loglikelihood} is a 
%concave function, and has a unique global optimal.
%\index{maximum likelihood estimation!gradient ascent}


\medskip

The gradient ascent method relies on the gradient of the log-likelihood
function, which can be obtained by taking its partial derivative 
with respect to $\abw$, as follows:
\begin{align*}
    \grad(\abw) = 
    \frac{\partial}{\partial \abw} \Bigl\{\ln\left( L(\abw)
    \right)\Bigr\}
    = 
\frac{\partial}{\partial \abw} \Bigl\{\sum_{i=1}^n y_i \cdot \ln \lB( \theta(z_i) \rB) +
(1-y_i) \cdot \ln \lB( \theta(-z_i) \rB) \Bigr\}
    %\label{eq:reg:logit:derivLL}
\end{align*}
where $z_i = \abw^T\abx_i$.
%We use the chain rule to obtain the derivative of $\ln(\theta(z_i))$ with respect to $\abw$. 
%We note the following facts:
%\begin{align*}
%    \frac{\partial}{\partial \theta(z_i)} \Bigl\{\ln \left( \theta(z_i)
%    \right)\Bigr\} & =
%    \frac{1}{\theta(z_i)}\\ 
%    \frac{\partial}{\partial \theta(z_i)} \Bigl\{\ln \left( \theta(-z_i)
%    \right)\Bigr\} & =
%    \frac{\partial}{\partial \theta(z_i)}\Bigl\{ \ln \left(1- \theta(z_i)
%    \right)\Bigr\} = \frac{-1}{1-\theta(z_i)}\\ 
%    \frac{\partial \theta(z_i) }{\partial z_i} & = \theta(z_i) \cdot
%    (1-\theta(z_i)) =
%    \theta(z_i) \cdot \theta(-z_i)\\ 
%    \frac{\partial z_i}{\partial \abw} & = \frac{\partial \abw^T\abx_i}{\partial \abw}
%    = \abx_i 
%\end{align*}
%
%As per the chain rule, we have
%\begin{align}
%   \frac{\partial \ln \lB( \theta(z_i) \rB)}{\partial \abw}  & = 
%   \frac{\partial \ln \lB(\theta(z_i)\rB)}{\partial \theta(z_i)}  \cdot
%   \frac{\partial \theta(z_i)}{\partial z_i}  \cdot  
%\frac{\partial z_i}{\partial \abw} \notag\\
%& = \frac{1}{\theta(z_i)} \cdot \bigl(\theta(z_i) \cdot \theta(-z_i)\bigr) \cdot \abx_i  = \theta(-z_i) \cdot \abx_i
%\label{eq:reg:logit:derivLL_stochastic}
%\end{align}
%Likewise, using the chain rule, we have
%\begin{align*}
%   \frac{\partial \ln \lB( \theta(-z_i) \rB)}{\partial \abw} & = 
%   \frac{\partial \ln \lB(\theta(-z_i)\rB)}{\partial \theta(z_i)} \cdot
%   \frac{\partial \theta(z_i)}{\partial z_i} \cdot  
%\frac{\partial z_i}{\partial \abw}\\
% & = \frac{-1}{1-\theta(z_i)} \cdot \bigl(\theta(z_i) \cdot
%(1-\theta(z_i)\bigr) \cdot \abx_i = - \theta(z_i) \cdot \abx_i
%\end{align*}
%
%\enlargethispage{1\baselineskip}
%Substituting the above into \cref{eq:reg:logit:derivLL} we get

\medskip

After solving the derivative:
\begin{align*}
    \grad(\abw) & = 
%     \sum_{i=1}^n y_i \cdot \theta(-z_i) \cdot \abx_i
%     - (1-y_i) \cdot \theta(z_i) \cdot \abx_i \notag\\
%     & = \sum_{i=1}^n y_i \cdot \bigl( \theta(-z_i)+\theta(z_i) \bigr) \cdot \abx_i -
%     \theta(z_i) \cdot \abx_i \notag\\
%     & = \sum_{i=1}^n \bigl(y_i - \theta(z_i)\bigr) \cdot \abx_i, 
%     \qquad \text{since } \theta(-z_i)+\theta(z_i) = 1 \notag\\
     \sum_{i=1}^n \bigl(y_i - \theta(\abw^T\abx_i)\bigr) \cdot \abx_i
%    \label{eq:reg:logit:wLL}
\end{align*}
\end{frame}

\begin{frame}{Maximum Likelihood Estimation}
%
The gradient ascent method starts at 
$\abw^{\;0}$. 

\medskip

At each step $t$, the method moves in the direction of 
steepest ascent, which is given by the gradient vector. 

\medskip

Given the
current $\abw^{\;t}$, the next estimate is:
\begin{empheq}[box=\tcbhighmath]{align*}
    \abw^{\;t+1} = \abw^{\;t} + \eta \cdot \grad(\abw^{\;t})
    \label{eq:reg:logit:update_w} 
\end{empheq}
$\eta > 0$ is the {\em learning
rate}. It should not be too large, otherwise the estimates will vary
wildly from one iteration to the next, and it should not be too small,
otherwise it  will take a long time to converge. 

\medskip

	At the optimal value of
$\abw$, the gradient will be zero, i.e., $\grad(\abw) = \bzero$, as
desired. 
\end{frame}
%
%% The pseudo-code for the gradient ascent method for logistic
%% regression is shown in \cref{alg:reg:logit:logistic_regression}.
%
%% \begin{tightalgo}[!t]{\textwidth-18pt}
%%     \SetKwInOut{Algorithm}{\textsc{LogisticRegression-Batch}
%%     ($\bD,\eta, \epsilon$)}
%% \Algorithm{} 
%% \lForEach{$\abx_i \in \bD$}{ 
%%     $\abx_i \assign \matr{1\\\abx_i}$ \tcp*[h]{map to $\setR^{d+1}$} 
%% }
%% $t \assign 0$ \tcp*[h]{step/iteration counter}\; 
%% $\abw_0 \assign (0, \ldots, 0)^T$ \tcp*[h]{initial weight vector}\;
%% \Repeat{$\norm{\abw_t-\abw_{t-1}} \le \epsilon$}
%% {%
%%     $\grad(\abw_t) \assign \sum_{i=1}^n (y_i - \theta(\abw^T\abx_i)) \cdot
%%     \abx_i$ \tcp*[h]{compute gradient vector}\;
%%     $\abw_{t+1} \assign \abw_t + \eta \cdot \grad(\abw_t)$
%%     \tcp*[h]{update estimate for $\abw$}\;
%%     $t \assign t+1$\;
%% }%
%% \caption{Logistic Regression Algorithm: Batch Gradient Ascent}
%% \label{alg:reg:logit:logistic_regression}
%% \end{tightalgo}
%
%\subsubsection{Stochastic Gradient Ascent}
\begin{frame}{Stochastic Gradient Ascent}
The gradient ascent method computes the gradient by considering all the
data points, and it is therefore called {\em batch} gradient ascent. 

\medskip

For large datasets, it is typically much faster to compute
the gradient by considering only one (randomly chosen) point at a time, which is called {\em stochastic gradient ascent} (SGA).
%\index{stochastic gradient ascent}
%\index{SGA|see{stochastic gradient ascent}}
%\index{logistic regression!stochastic gradient ascent}
%
%The pseudo-code for SGA based logistic
%regression is shown in \cref{alg:reg:logit:logistic_regression_SGA}.
%Given a randomly chosen point $\abx_i$, the
%point-specific gradient %(see \cref{eq:reg:logit:wLL}) 
%is given as
\begin{align*}
    \tcbhighmath{
    \grad(\abw, \abx_i) = \bigl(y_i - \theta(\abw^T\abx_i)\bigr) \cdot
\abx_i}
\end{align*}
%% As in the batch approach, in stochastic gradient ascent we update the
%% weight vector as follows:
%% \begin{align}
%%     \abw^{t+1} = \abw^{t} + \eta \cdot \grad(\abw^{t}, \abx_i)
%%     \label{eq:reg:logit:update_w}
%% \end{align}
%Unlike batch gradient ascent that updates $\abw$ by considering all the
%points, in stochastic gradient ascent the weight vector is updated after
%observing each point, and the updated values are used immediately in the
%next update. Computing the full gradient in the batch approach can be
%very expensive. In contrast, computing the partial gradient at each
%point is very fast, and due to the stochastic updates to $\abw$,
%typically SGA is much faster than the batch
%approach for very large datasets.
Once the model has been trained, we can predict the response for any new
augmented test point $\abz$ as follows:
\begin{empheq}[box=\tcbhighmath]{align*}
    \hy = 
\begin{cases}
    1 & \text{if } \theta(\abw^T\abz) \ge 0.5\\
    0 & \text{if } \theta(\abw^T\abz) < 0.5
\end{cases}
\end{empheq}
\end{frame}
%
\begin{frame}{Logistic Regression: Stochastic Gradient Ascent}
\begin{tightalgo}[H]{\textwidth-18pt}
    \SetKwInOut{Algorithm}{\textsc{LogisticRegression-SGA} ($\bD, \eta, \epsilon$)}
\Algorithm{} 
\lForEach{$\bx_i \in \bD$}{ 
    $\abx_i^T \assign \matr{1 & \bx_i^T}$ \tcp*[h]{map to $\setR^{d+1}$} 
}
$t \assign 0$ \tcp*[h]{step/iteration counter}\; 
$\abw^{\;0} \assign (0, \ldots, 0)^T \in \setR^{d+1}$ \tcp*[h]{initial weight vector}\;
\Repeat{$\norm{\abw^{\;t}-\abw^{\;t-1}} \le \epsilon$}{
    $\abw \gets \abw^{\;t}$ \tcp*[h]{make a copy of $\abw^{\;t}$}\;
    \ForEach{$\abx_i \in \abD$ in random order}{%
        $\grad(\abw, \abx_i) \assign \bigl(y_i -
        \theta(\abw^T\abx_i)\bigr) \cdot \abx_i$
        \tcp*[h]{gradient at $\abx_i$}\;
        $\abw \assign \abw + \eta \cdot \grad(\abw, \abx_i)$
        \tcp*[h]{update estimate for $\abw$}
    }
    $\abw^{\;t+1} \gets \abw$ \tcp*[h]{update $\abw^{\;t+1}$}\;
    $t \assign t+1$\;
}%
\end{tightalgo}
\end{frame}
%\caption{Logistic Regression: Stochastic Gradient Ascent}
%\label{alg:reg:logit:logistic_regression_SGA}
%\end{tightalgo}
%
%Once the model has been trained, we can predict the response for any new
%augmented test point $\abz$ as follows:
%\begin{empheq}[box=\tcbhighmath]{align}
%    \hy = 
%\begin{cases}
%    1 & \text{if } \theta(\abw^T\abz) \ge 0.5\\
%    0 & \text{if } \theta(\abw^T\abz) < 0.5
%\end{cases}
%\end{empheq}
%
%
%% \begin{figure}[!t]
%% \captionsetup[subfloat]{captionskip=12pt} 
%% \readdata{\dataPC}{REG/logit/figs/iris-PC-virginica.txt}
%% \psset{stepFactor=0.3} \psset{dotscale=1.5,fillcolor=lightgray,
%%             arrowscale=2,PointName=none}
%% \psset{xAxisLabel=$\bu_1$, yAxisLabel= $\bu_2$}
%% \centerline{%
%% \subfloat[Linear Regression]{%
%% \label{fig:reg:logit:iris-PC-vir:linear}
%% \scalebox{0.55}{%
%% \psgraph[axesstyle=frame,tickstyle=bottom,Ox=-4,Oy=-1.5,Dx=1,Dy=0.5]{->}(-4.0,-1.5)(4.0,1.5){4in}{3in}%
%% \listplot[plotstyle=dots,dotstyle=Bo,showpoints=true,
%%     nEnd=50,plotNo=1,plotNoMax=2]{\dataPC}
%% \listplot[plotstyle=dots,dotstyle=Btriangle,showpoints=true,
%%     nStart=51,plotNo=1,plotNoMax=2]{\dataPC}
%% \input{REG/logit/figs/wrong_linear.tex}
%% %\psset{algebraic=true}
%% %\psclip{\psframe[linestyle=none](-4,-1.5)(4,1.5)}
%% %\psplot[plotstyle=line,linecolor=gray,linewidth=1pt]{-4}{4}{-5 + 2.26 * x}
%% %\endpsclip
%% \endpsgraph
%% }}
%% \hspace{0.3in}
%% \subfloat[Logistic Regression]{%
%% \label{fig:reg:logit:iris-PC-vir:logistic}
%% \scalebox{0.55}{%
%% \psgraph[axesstyle=frame,tickstyle=bottom,Ox=-4,Oy=-1.5,Dx=1,Dy=0.5]{->}(-4.0,-1.5)(4.0,1.5){4in}{3in}%
%% \listplot[plotstyle=dots,dotstyle=Bo,showpoints=true,
%%     nEnd=50,plotNo=1,plotNoMax=2]{\dataPC}
%% \listplot[plotstyle=dots,dotstyle=Btriangle,showpoints=true,
%%     nStart=51,plotNo=1,plotNoMax=2]{\dataPC}
%% \input{REG/logit/figs/wrong_logistic.tex}
%% %\psset{algebraic=true}
%% %\psclip{\psframe[linestyle=none](-4,-1.5)(4,1.5)}
%% %\psplot[plotstyle=line,linecolor=gray,linewidth=1pt]{-4}{4}{-5 + 2.26 * x}
%% %\endpsclip
%% \endpsgraph
%% }}}
%% \caption{Logistic versus Linear Regression: Iris principal components
%% data. Points in black are misclassified.} 
%% \label{fig:reg:logit:logistic_irisPC}
%% \end{figure}
%
%% %kernel ridge regression, linear, alpha=0.01
%% %weight [-0.16749884  0.07410432  0.33331111]
%% %accuracy, sse 0.8866666666666667 15.4732523363
%% % equation: -0.167x + 0.074y + 0.333 = 0
%% % BIAS = -4.5 (-0.333/0.074), SLOPE = 2.26 (0.167/0.074)
%
%% %logistic regression, 0.001 0.001
%% %-5.06698862 -3.28657331 -6.78945911
%% %accuracy 0.9666666666666667
%
%% \begin{figure}[t!]
%%     \vspace{0.2in}
%%     \def\logit{1/(1 + EXP(-x))}
%%     \centering
%% \psset{labelFontSize=\scriptstyle} 
%%     \psset{xAxisLabel=$z$,yAxisLabel=$\theta(z)$,%
%%     xAxisLabelPos={c,-0.15},yAxisLabelPos={-4,c} }
%%     \psset{linewidth=1pt,axesstyle=frame,algebraic,plotstyle=curve}
%%     \psgraph[Dy=0.1,Dx=5,Ox=-25]{->}(-25,0)(15,1.0001){4in}{2.5in}%
%%     \psline[linewidth=0.5pt,linecolor=gray](0,0)(0,1.0)
%%     \psline[linewidth=0.5pt,linecolor=gray](-25,0.5)(15,0.5)
%%     \psplot[linewidth=1pt,linecolor=gray]{-25}{15}{\logit}
%%     \input{REG/logit/figs/virginica-logistic-probs.tex}
%%     \endpsgraph
%%     \vspace{0.4in}
%%     \caption{Logistic function}
%%     \label{fig:reg:logit:virginica-logit}
%% \end{figure}
%

\begin{frame}{Logistic Regression}
\framesubtitle{Example}
Figure shows the Iris principal components data. 

\medskip

$X_1$ and $X_2$ are the independent attributes and represent the first two
principal components. 

\medskip

$Y$ is the binary response variable and represents the type of Iris flower; $Y=1$ corresponds to {\tt Iris-virginica}, whereas $Y=0$ corresponds to the two other Iris types, namely {\tt Iris-setosa} and {\tt Iris-versicolor}.
\vspace*{1.5cm}

%\begin{figure}[t!]
%    \vspace*{0.3in}
%\captionsetup[subfloat]{captionskip=5pt}
\psset{unit=0.25in}
\psset{lightsrc=viewpoint}
\psset{incolor=gray}
\psset{opacity=0.2}
\def\PsEuler{2.71828182846}
\psset{transform={1.0 1.0 1.5 scaleOpoint3d}}
\psset{axisnames={X_1, X_2, Y}}
\centerline{
    \scalebox{0.9}{%
\begin{pspicture}(-6,-2.5)(6,3)
\psset{viewpoint=30 95 12 rtp2xyz,Decran=60}
    \axesIIID[](-4,-2,0)(4.5,3,2.0)
\psset{linewidth=0.5pt, dotsize=0.3}
\psset{showAxes=false, fillcolor=gray}
\psSolid[object=parallelepiped,a=8,b=5,c=1.0,action=draw](0,-0.5,0.5)
    %\psSurface[ngrid=.2 .2,incolor=white,axesboxed=false,
        %linewidth=0.5\pslinewidth,
       %color1 = {[rgb]{0 0 0}},
        %color2 = {[rgb]{1 1 1}},
        %hue=(color1) (color2), lightintensity=5,
%algebraic,Zmin=0,Zmax=1](-4,-3)(4,2){%
        %(1/(1 + \PsEuler^((5.06652752*x)+(3.28608846*y)+6.78873402)))}
\psset{dotstyle=Bo,fillcolor=white}
\input{REG/logit/figs/iris-vir-3dC1.tex}
\psset{dotstyle=Btriangle,fillcolor=white}
\input{REG/logit/figs/iris-vir-3dC0.tex}
\psPoint(-0.519383245084936,-1.19135168905065,1.5){p17}
\psdot[dotstyle=Bo,fillcolor=white](p17)
\psPoint(-1.1688534694704,-0.16450249981509,1.5){p18}
\psdot[dotstyle=Bo,fillcolor=white](p18)
\psPoint(-1.5573962720707,0.267392584812765,0){p19}
\psdot[dotstyle=Btriangle,fillcolor=white](p19)
\psPoint(-1.29646885028553,-0.327561519794944,0){p20}
\psdot[dotstyle=Btriangle,fillcolor=white](p20)
\psPoint(-1.37873698277558,-0.421205138214628,0){p21}
\psdot[dotstyle=Btriangle,fillcolor=white](p21)

\end{pspicture}
}}%}
\end{frame}
%\vspace{0.5in}
%\centerline{
%\subfloat[Linear Regression]{ \label{fig:reg:logit:iris-vir:linear}

\begin{frame}{Linear Regression}
	\framesubtitle{Example}

The plane of best fit in linear regression has the weight vector:
    \begin{align*}
        \abw & = (0.333, -0.167, 0.074)^T\\
        \hy & = f(\abx) = 0.333 - 0.167\cdot x_1 + 0.074 \cdot x_2
    \end{align*}
Since $Y$ is binary, we predict $y=1$ if $f(\abx) \ge 0.5$, and $y=0$ otherwise.

\medskip

Linear regression misclassifies 17 points, achieving $88.7\%$ accuracy. 
    
\vspace*{1.5cm}

\psset{unit=0.25in}
\psset{lightsrc=viewpoint}
\psset{incolor=gray}
\psset{opacity=0.2}
\def\PsEuler{2.71828182846}
\psset{transform={1.0 1.0 1.5 scaleOpoint3d}}
\psset{axisnames={X_1, X_2, Y}}
\centerline{
    \scalebox{0.9}{%
        \begin{pspicture}(-6,-2.5)(6,3)
\psset{viewpoint=30 95 18 rtp2xyz,Decran=60}
            \axesIIID[](-4,-2,0)(4.5,3,2)
\psset{linewidth=0.5pt, dotsize=0.3}
\psset{showAxes=false, fillcolor=gray}
\psSolid[object=parallelepiped,a=8,b=5,c=1.0,action=draw](0,-0.5,0.5)
    \psSurface[ngrid=.2 .2,incolor=gray,axesboxed=false,
        linewidth=0.5\pslinewidth,
       color1 = {[rgb]{0 0 0}},
        color2 = {[rgb]{1 1 1}},
        hue=(color1) (color2), lightintensity=5,
    algebraic,Zmin=0,Zmax=1](-4,-3)(4,2){%
(0.074*y)-(0.167*x)+0.333 }
%%(0.148*y)-(0.335*x)-0.333 }
\psset{dotstyle=Bo,fillcolor=white}
\input{REG/logit/figs/iris-vir-3dC1.tex}
\psset{dotstyle=Btriangle,fillcolor=white}
\input{REG/logit/figs/iris-vir-3dC0.tex}
\input{REG/logit/figs/wrong_linear.tex}
\end{pspicture}
}}%}
\end{frame}
%\vspace{0.2in}
%\caption{Logistic versus linear regression: Iris principal components
%data. Misclassified point are shown in dark gray color. Circles denote
%{\tt Iris-virginica} and triangles denote the other two Iris types.}
%\label{fig:reg:logit:iris-vir-logistic}
%\vspace{-0.2in}
%\end{figure}
%
\begin{frame}{Logistic Regression}
    The fitted logistic model is given as
    \begin{align*}
        \abw & = (w_0, w_1, w_2)^T = (-6.79, -5.07, -3.29)^T\\
        P(Y=1|\abx) & = \theta(\abw^T\abx) = 
        \frac{1}{1 + \exp\{ 6.79 + 5.07\cdot x_1 + 3.29 \cdot x_2\}}
    \end{align*}
%Figure plots $P(Y=1|\abx)$ for various
    %values of $\abx$. 
%    
Given $\abx$, if $P(Y=1|\abx) \ge 0.5$, then we predict $\hy=1$,
    otherwise we predict $\hy=0$. 

\medskip

Logistic regression misclassifies only 5 points, achieving $96.7\%$ accuracy.
\vspace*{1.5cm}

%\begin{figure}[t!]
%    \vspace*{0.3in}
%\captionsetup[subfloat]{captionskip=5pt}
\psset{unit=0.25in}
\psset{lightsrc=viewpoint}
\psset{incolor=gray}
\psset{opacity=0.2}
\def\PsEuler{2.71828182846}
\psset{transform={1.0 1.0 1.5 scaleOpoint3d}}
\psset{axisnames={X_1, X_2, Y}}
%%\readdata{\dataVIR}{REG/logit/figs/iris-PC-virginica.txt}
\centerline{
%\subfloat[Logistic Regression]{ 
    \scalebox{0.9}{%
%\label{fig:reg:logit:iris-vir:logit}
\begin{pspicture}(-6,-2.5)(6,3)
%    %\gridIIID[Zmin=0,Zmax=1,stepZ=2](-4,4)(-2,2)
\psset{viewpoint=30 95 12 rtp2xyz,Decran=60}
    \axesIIID[](-4,-2,0)(4.5,3,2.0)
%%\pstThreeDCoor[Alpha=-10,Beta=10]
\psset{linewidth=0.5pt, dotsize=0.3}
%%\dataplotThreeD[fillcolor=black,plotstyle=dots,showpoints=true,dotscale=1.5]{\dataVIR}
%%\psset{color1=black!50,color2=black!20,hue=(color1) (color2)}
\psset{showAxes=false, fillcolor=gray}
\psSolid[object=parallelepiped,a=8,b=5,c=1.0,action=draw](0,-0.5,0.5)
    \psSurface[ngrid=.2 .2,incolor=white,axesboxed=false,
        linewidth=0.5\pslinewidth,
       color1 = {[rgb]{0 0 0}},
        color2 = {[rgb]{1 1 1}},
        hue=(color1) (color2), lightintensity=5,
algebraic,Zmin=0,Zmax=1](-4,-3)(4,2)
\end{frame}

%\begin{frame}{Logistic Regression}
%\framesubtitle{Example}
%
%Figure 
%    shows that five points (shown in dark gray) are misclassified. For
%    example, for $\abx = (1,-0.52,-1.19)^T$ we have:
%        \begin{align*}
%            P(Y=1|\abx) & = \theta(\abw^T\abx) = \theta(-0.24) = 0.44\\
%            P(Y=0|\abx) & = 1 - P(Y=1|\abx) = 0.54
%        \end{align*}
%    Thus, the predicted response for $\abx$ is $\hy = 0$, whereas the
%    true class is $y=1$.
%%
%%    \cref{fig:reg:logit:iris-vir-logistic} also contrasts logistic
%%    versus linear regression. 
%The plane of best fit in linear regression
%    is shown in the figure,
%%\cref{fig:reg:logit:iris-vir:linear}, 
%with the weight
%    vector:
%    \begin{align*}
%        \abw & = (0.333, -0.167, 0.074)^T\\
%        \hy & = f(\abx) = 0.333 - 0.167\cdot x_1 + 0.074 \cdot x_2
%    \end{align*}
%    Since the response vector $Y$ is binary, we predict the response
%    class as $y=1$ if $f(\abx) \ge 0.5$, and $y=0$ otherwise.
%    The linear regression model results in 17 points being misclassified
%    (dark gray points).%, as shown in
%%    \cref{fig:reg:logit:iris-vir:linear}. 
%    
%    Since there are $n=150$ points in total, this results in a training
%    set or in-sample accuracy of $88.7\%$ for linear regression,
% while logistic regression misclassifies only 5 points,
%    an accuracy of $96.7\%$, which is a much better fit.%, as
%%    is also apparent in \cref{fig:reg:logit:iris-vir-logistic}.
%\end{frame}
%\end{example}
%
%
%
%\section{Multiclass Logistic Regression}
%\label{sec:reg:logit:multiclass}
%\index{multiclass logistic regression}
%\index{logistic regression!multiclass}
%
%\index{one-hot encoding}
%\index{logistic regression!one-hot encoding}

\ifdefined\wox \begin{frame} \titlepage \end{frame} \fi


\begin{frame}{Multiclass Logistic Regression}

We now generalize logistic regression to the case when 
$Y$ can take on $K$ distinct nominal categorical values called
{\em classes}, i.e., $Y \in \{c_1, c_2, \cdots, c_K\}$. 

\medskip

We model $Y$ as a $K$-dimensional
multivariate Bernoulli random variable. %(see \cref{sec:eda:cat:multivarBernoulli}). 
Since $Y$ can assume only one of
the $K$ values, we use the {\em one-hot encoding} approach to map each
categorical value $c_i$ to the $K$-dimensional binary vector 
$$\be_i =
(\;\overbrace{0,\ldots,0}^{i-1},1,\overbrace{0,\ldots,0}^{K-i}\;)^T$$
whose $i$th element $e_{ii}=1$, and all other elements
$e_{ij} = 0$, so that $\sum_{j=1}^K e_{ij} = 1$.
%Henceforth, we assume that the categorical response variable $Y$ is a
%multivariate Bernoulli variable 
%$\bY \in \{\be_1, \be_2, \cdots, \be_K\}$, with $Y_{\!\!j}$ referring 
%to the $j$th component of $\bY$. 
%
\medskip

The probability mass function for $\bY$ given $\abX=\abx$ is
\begin{align*}
    P(\bY=\be_i|\abX = \abx) = \pi_i(\abx), \text{ for } i=1,2,\ldots,K
\end{align*}
%where $\pi_i(\abx)$ is the (unknown) probability of observing class $c_i$ 
%given $\abX = \abx$.
There are $K$ unknown parameters, which must satisfy the following
constraint:
\begin{align*}
    \sum^{K}_{i=1} \pi_i(\abx) = \sum^{K}_{i=1} P(\bY=\be_i|\abX=\abx) = 1
\end{align*}
%

\end{frame}

\begin{frame}{Multiclass Logistic Regression}

Given that only one element of $\bY$ is 1, the PMF of $\bY$ is:
\begin{align*}
    \tcbhighmath{
    P(\bY|\abX=\abx) = \prod_{j=1}^{K} \lB(\pi_{\!j}(\abx)
    \rB)^{Y_{\!\!j}}
}
\end{align*}
%Note that if $\bY=\be_i$, only $Y_{\!i}=1$ and the rest of the elements
%$Y_{\!\!j}=0$ for $j\ne i$.
%
We select %one of the values, say
$c_K$ as a reference or base class, and consider the log-odds ratio of
the
other classes w.r.t. $c_K$. 

	\medskip

We assume these log-odd
ratios are linear in $\abX$, but 
$\trueabw_i$ is specific to
for class $c_i$:
\begin{align*}
    \ln(\odds(\bY=\be_i|\abX=\abx)) & = \ln\left(
    \frac{P(\bY=\be_i|\abX=\abx)}{P(\bY=\be_K|\abX=\abx)} \right) = 
    \ln\left( \frac{\pi_i(\abx)}{\pi_K(\abx)} \right) =
    \trueabw_i^T\abx\notag\\
    & = \truew_{i0} \cdot x_0 + \truew_{i1} \cdot x_1 + \cdots +
    \truew_{id} \cdot x_d
    % \label{eq:reg:logit:nominalLOR}
\end{align*}
where $\truew_{i0} = \trueb_i$ is the true bias value for class $c_i$.
%
%We can rewrite the above set of equations as follows:
%\begin{align}
%    & \frac{\pi_i(\abx)}{\pi_K(\abx)} = \exp\{\trueabw_i^T\abx\} \notag\\
%    \implies & \pi_i(\abx) = \exp\{\trueabw_i^T\abx\} \cdot \pi_K(\abx)
%    \label{eq:reg:logit:nominal_ratio}
%\end{align}
%Given that $\sum^{K}_{j=1} \pi_{\!j}(\abx) = 1$, we have
%\begin{align*}
%    & \sum^{K}_{j=1} \pi_{\!j}(\abx) = 1\\ 
%    \implies & \Bigl(\sum_{j\ne K} \exp\{\trueabw_{\!j}^T\abx\} \cdot
%    \pi_K(\abx)\Bigr) + \pi_K(\abx) =
%   1\\
%   \implies & \pi_K(\abx) = \frac{1}{1 + \sum_{j\ne K}
%   \exp\{\trueabw_{\!j}^T\abx\} }
%\end{align*}
%Substituting the above into \cref{eq:reg:logit:nominal_ratio}, we have
%\begin{align*}
%    \pi_i(\abx) = \exp\{\trueabw_i^T\abx\} \cdot \pi_K(\abx) = 
%    \frac{\exp\{\trueabw_i^T\abx\}}{1 + \sum_{j\ne K}
%    \exp\{\trueabw_{\!j}^T\abx\} }
%\end{align*}
%Finally, 

\end{frame}

\begin{frame}{Multiclass Logistic Regression}

Setting $\trueabw_K = \bzero$, we have $\exp\{\trueabw_K^T\abx\} = 1$, and
thus we can write the full model for multiclass logistic regression as
follows:
\begin{empheq}[box=\tcbhighmath]{align*}
    \pi_i(\abx) = \frac{\exp\{\trueabw_i^T\abx\}}{\sum_{j=1}^K
    \exp\{\trueabw_{\!j}^T\abx\} }, \qquad
    \text{ for all } i = 1, 2, \cdots, K
    \label{eq:reg:logit:softmax}
\end{empheq}
%
%\index{softmax function}
%\index{logistic regression!softmax function}
This function is also called the {\em softmax} function.

\medskip

When $K=2$, this formulation yields exactly the same model as in 
binary logistic regression.

\medskip

Note that the choice of the reference class is
not important, since we can derive the log-odds ratio for any two
classes $c_i$ and $c_{\!j}$.
\end{frame}
%\begin{align*}
%    \ln\lB(\frac{\pi_i(\abx)}{\pi_{\!j}(\abx)}\rB) & = 
%    \ln\lB(\frac{\pi_i(\abx)}{\pi_K(\abx)} \cdot
%    \frac{\pi_K(\abx)}{\pi_{\!j}(\abx)} \rB)\\
%    & = \ln\lB(\frac{\pi_i(\abx)}{\pi_K(\abx)}\rB) +
%    \ln\lB(\frac{\pi_K(\abx)}{\pi_{\!j}(\abx)}
%    \rB)\\
%    & = \ln\lB(\frac{\pi_i(\abx)}{\pi_K(\abx)}\rB) -
%    \ln\lB(\frac{\pi_{\!j}(\abx)}{\pi_K(\abx)}
%    \rB)\\
%    & = \trueabw_i^T\abx - \trueabw_{\!j}^T\abx\\
%    & = (\trueabw_i - \trueabw_{\!j})^T\abx
%\end{align*}
%That is, the log-odds ratio between any two classes 
%can be computed from the difference of the corresponding weight vectors.
%
%\subsection{Maximum Likelihood Estimation}
%\index{maximum likelihood estimation}
\begin{frame}{Maximum Likelihood Estimation}
%Let $\abD$ be the augmented dataset
%comprising $n$ points $\abx_i$ and their labels $\by_i$. We assume that
%$\by_i$ is 
%a one-hot encoded (multivariate Bernoulli) response vector, 
%so that
%$y_{ij}$ denotes the $j$th element of $\by_i$.
%For example, if
%$\by_i = \be_a$, then $y_{ij}=1$ for $j=a$, and $y_{ij}=0$ for all $j\ne a$.
%We assume that all the $\by_i$'s are independent.
%Let $\abw_i \in \setR^{d+1}$ denote the estimated augmented weight
%vector for class $c_i$, with $w_{i0} = b_i$ denoting the bias term.
%
To find the $K$ sets of regression weight vectors $\abw_i$, for
$i=1,2,\cdots, K$, we use the gradient ascent approach to
maximize the log-likelihood function. 
The likelihood of the data is given as
%\index{maximum likelihood estimation!log-likelihood function}
\begin{align*}
    L(\abW) = P(\bY|\abW) =
    \prod_{i=1}^n P(\by_i | \abX=\abx_i) = 
    \prod_{i=1}^n \prod_{j=1}^{K} \lB(\pi_{\!j}(\abx_i)\rB)^{y_{ij}}
\end{align*}
where $\abW = \{\abw_1, \abw_2, \cdots, \abw_K\}$ is the set of $K$
weight vectors.

\medskip

The log-likelihood is then given as:
\begin{empheq}[box=\tcbhighmath]{align*}
    \ln\left( L(\abW) \right) = 
    \sum_{i=1}^n \sum_{j=1}^{K} y_{ij} \cdot \ln(\pi_{\!j}(\abx_i))
    =\sum_{i=1}^n \sum_{j=1}^{K} y_{ij} \cdot 
    \ln\left( \frac{\exp\{\abw_{\!j}^T\abx_i\}}{\sum_{a=1}^K
    \exp\{\abw_a^T\abx_i\}} \right)
    \label{eq:reg:logit:multi_likelihood}
\end{empheq}
Note that the negative of the log-likelihood function can be regarded as
an error function, commonly known as {\em cross-entropy error}.
%\index{maximum likelihood estimation!cross-entropy error}
%\index{cross-entropy error}
%\index{log-likelihood!cross-entropy error}
%\index{logistic regression!cross-entropy error}
%
%We note the following facts:
%\begin{align*}
%    \frac{\partial}{\partial \pi_{\!j}(\abx_i)} \ln(\pi_{\!j}(\abx_i)) & =
%    \frac{1}{\pi_{\!j}(\abx_i)}\\
%    \frac{\partial}{\partial \abw_a} \pi_{\!j}(\abx_i) = & 
%  \begin{cases}
%      \pi_a(\abx_i) \cdot (1-\pi_a(\abx_i)) \cdot \abx_i & \text{if } j=a\\
%   -\pi_a(\abx_i) \cdot \pi_{\!j}(\abx_i) \cdot
%   \abx_i & \text{if } j \ne a
%  \end{cases}
%\end{align*}
%Let us consider the gradient of the log-likelihood function 
%with respect to the weight vector $\abw_a$:
%\begin{align*}
%    \grad(\abw_a) & = \frac{\partial}{\partial \abw_a} \Bigl\{\ln
%    (L(\abW))\Bigr\}\\
%        & = \sum_{i=1}^n \sum_{j=1}^K y_{ij} \cdot
%        \frac{\partial \ln(\pi_{\!j}(\abx_i))}{\partial \pi_{\!j}(\abx_i)}
%        \cdot \frac{\partial \pi_{\!j}(\abx_i)}{\partial \abw_a}\\
%& = \sum_{i=1}^n \Biggl( 
%    y_{ia} \cdot \frac{1}{\pi_a(\abx_i)} \cdot \pi_a(\abx_i)
%\cdot (1-\pi_a(\abx_i)) \cdot \abx_i 
%    + \sum_{j \ne a} y_{ij} \cdot
%    \frac{1}{\pi_{\!j}(\abx_i)} \cdot (- \pi_a(\abx_i) \cdot
%    \pi_{\!j}(\abx_i)) \cdot \abx_i  
%\Biggr)\\
%& = \sum_{i=1}^n \Bigl(y_{ia} - y_{ia} \cdot \pi_a(\abx_i) - \sum_{j \ne a}
%y_{ij} \cdot \pi_a(\abx_i) \Bigr) \cdot \abx_i\\
%& = \sum_{i=1}^n \Bigl(y_{ia} - \sum_{j=1}^K y_{ij} \cdot \pi_a(\abx_i)
%    \Bigr) \cdot \abx_i\\
%& = \sum_{i=1}^n \lB(y_{ia} - \pi_a(\abx_i) \rB) \cdot \abx_i
%\end{align*}
%The last step follows from the fact that $\sum_{j=1}^K y_{ij} = 1$,
%since only one element of $\by_i$ can be $1$.
%
\medskip

For stochastic gradient ascent, we update the weight vectors by
considering only one point at a time. 

\end{frame}

\begin{frame}{Maximum Likelihood Estimation}

The gradient of the log-likelihood
function w.r.t. $\abw_{\!j}$ at a given point $\abx_i$ is:
\begin{empheq}[box=\tcbhighmath]{align*}
    \grad(\abw_{\!j}, \abx_i) = \bigl(y_{ij} - \pi_j(\abx_i)\bigr) \cdot \abx_i
\end{empheq}
which results in the following update rule for the $j$th weight vector:
\begin{empheq}[box=\tcbhighmath]{align*}
\abw_{\!j}^{\;t+1} = \abw_{\!j}^{\;t} + \eta \cdot
\grad(\abw_{\!j}^{\;t}, \; \abx_i)
\end{empheq}
where $\abw_{\!j}^{\;t}$ denotes the estimate of $\abw_{\!j}$ at step
$t$, and $\eta$ is the learning rate.
%The pseudo-code for the stochastic gradient ascent method for multiclass
%logistic regression is shown in \cref{alg:reg:logit:nominal_logistic_regression}.
%Notice that the weight vector for the base class $c_K$ is never updated;
%it remains $\abw_K = \bzero$, as required.
%\index{stochastic gradient ascent}
%\index{logistic regression!stochastic gradient ascent}
%\index{learning rate}
%\index{logistic regression!learning rate}
%
%

\medskip

Once the model has been trained, we can predict $\hy$ for any new
$\abz$ as:
\begin{empheq}[box=\tcbhighmath]{align*}
    \hy = \argmax_{c_i} \bigl\{ \pi_i(\abz) \bigr\} 
    = \argmax_{c_i} \lB\{
    \frac{\exp\{\abw_i^T\abz\}}{\sum_{j=1}^K \exp\{\abw_j^T\abz\} }
    \rB\}
\end{empheq}
We evalute the softmax function and the predicted 
$\hy$ has the highest probability.
\end{frame}
%
%
\begin{frame}{Multiclass Logistic Regression Algorithm}
\begin{tightalgo}[H]{\textwidth-18pt}
    \SetKwInOut{Algorithm}{\textsc{LogisticRegression-MultiClass}
    ($\bD,\eta, \epsilon$)}
\Algorithm{} 
\ForEach{$(\bx_i^T, y_i) \in \bD$}{ 
    $\abx_i^T \assign \matr{1 & \bx_i^T}$ \tcp*[h]{map to $\setR^{d+1}$}\;
    $\by_i \assign \be_{\!j} \text { if } y_i=c_{\!j}$ \tcp*[h]{map $y_i$ to
    $K$-dim Bernoulli vector}
}
$t \assign 0$ \tcp*[h]{step/iteration counter}\;
\lForEach{$j = 1, 2, \cdots, K$}{
    $\abw_{\!j}^{\;t} \assign (0, \ldots, 0)^T \in \setR^{d+1}$ %\tcp*[h]{initial weight vector}
}
\Repeat{$\sum_{j=1}^{K-1} \norm{\abw_{\!j}^{\;t}-\abw_{\!j}^{\;t-1}} \le \epsilon$}
{%
    \lForEach{$j = 1, 2, \cdots, K-1$}{
        $\abw_{\!j} \gets \abw^{\;t}_{\!j}$ \tcp*[h]{copy 
        $\abw^{\;t}_{\!j}$}
    }
    \ForEach{$\abx_i \in \abD$ in random order}{%
        \ForEach{$j = 1,2,\cdots,K-1$}{%
            $\pi_{\!j}(\abx_i) \assign 
            {\exp\lB\{\abw_{\!\!j}^T\abx_i\rB\}}/{\sum_{a=1}^K
            \exp\lB\{\abw_{a}^T\abx_i\rB\}}
            $\;
        $\grad(\abw_{\!j}, \abx_i) \assign \bigl(y_{ij} -
        \pi_{\!j}(\abx_i)\bigr) \cdot \abx_i$
            \tcp*[h]{gradient at $\abw_{\!j}$}\;
        $\abw_{\!j} \assign \abw_{\!j} + \eta \cdot
    \grad(\abw_{\!j}, \abx_i)$
            \tcp*[h]{update estimate for $\abw_{\!j}$}
        }
    }
    \lForEach{$j = 1, 2, \cdots, K-1$}{
        $\abw^{\;t+1}_{\!j} \gets \abw_{\!j}$ \tcp*[h]{update
        $\abw^{\;t+1}_{\!j}$}
    }
    $t \assign t+1$\;
}%
%\caption{Multiclass Logistic Regression Algorithm}
%\label{alg:reg:logit:nominal_logistic_regression}
\end{tightalgo}
\end{frame}
%
%Once the model has been trained, we can predict the class for any new
%augmented test point $\abz$ as follows:
%\begin{empheq}[box=\tcbhighmath]{align}
%    \hy = \argmax_{c_i} \bigl\{ \pi_i(\abz) \bigr\} 
%    = \argmax_{c_i} \lB\{
%    \frac{\exp\{\abw_i^T\abz\}}{\sum_{j=1}^K \exp\{\abw_j^T\abz\} }
%    \rB\}
%\end{empheq}
%That is, we evalute the softmax function, and then predict the
%class of $\abz$ as the one with the highest probability.
%
%%nominal iris-PC-3K.txt 0.001 0.001
%% [[ 3.61802926  2.60948418 -3.51898333]
%%  [-5.17847804 -3.39836667 -6.95000289]
%%  [ 0.          0.          0.        ]]
%% accuracy 0.9666666666666667
%% labels {'1': 1, '0': 0, '2': 2}
%
%% \begin{figure}[tb!]
%% \psset{unit=0.25in}
%% \psset{lightsrc=viewpoint}
%% \psset{incolor=gray}
%% \psset{opacity=0.2}
%% \def\PsEuler{2.71828182846}
%% \psset{viewpoint=30 120 10 rtp2xyz,Decran=60}
%% \psset{axisnames={X_1, X_2, Y}}
%% \centerline{
%% \scalebox{1}{%
%% \begin{pspicture}(-6,-4)(6,6)
%% \axesIIID[](-4,-2,0)(4.5,3,1.5)
%% \psset{linewidth=0.5pt, dotsize=0.3}
%% \psset{showAxes=false, fillcolor=gray}
%% \psset{transform={1.0 1.0 1.5 scaleOpoint3d}}
%% \psSurface[ngrid=.2 .2,incolor=white,axesboxed,
%%     linewidth=0.5\pslinewidth, 
%%     algebraic,Zmin=0,Zmax=1](-4,-3)(4,2){%
%%     (1/(1 + \PsEuler^((-3.618*x)-(2.609*y)+3.52)))}
%% \psset{transform={1.0 1.0 1.5 scaleOpoint3d}}
%% \psSurface[ngrid=.2 .2,incolor=white,axesboxed,
%%     linewidth=0.5\pslinewidth, 
%%     algebraic,Zmin=0,Zmax=1](-4,-3)(4,2){%
%%     (1/(1 + \PsEuler^((5.178*x)+(3.398*y)+6.95)))}
%% \psset{dotstyle=Bsquare,fillcolor=lightgray}
%% \input{REG/logit/figs/iris-3K-C0.tex}
%% \psset{dotstyle=Bo,fillcolor=lightgray}
%% \input{REG/logit/figs/iris-3K-C1.tex}
%% \psset{dotstyle=Btriangle,fillcolor=lightgray}
%% \input{REG/logit/figs/iris-3K-C2.tex}
%% \end{pspicture}
%% }}
%% \vspace{0.1in}
%% \caption{Multiclass Logistic Regression: Iris principal components
%% data. Misclassified point are shown in dark gray color.}
%% \label{fig:reg:logit:iris-3K-multiclass}
%% \end{figure}
\begin{frame}{Multiclass Logistic Regression}
\framesubtitle{Example}

Consider the 2D Iris PCA dataset, $n=150$.

\medskip

$Y$ takes on three values: $Y=c_1$: {\tt Iris-setosa} ($\square$),
$Y=c_2$: {\tt Iris-versicolor} ($\bigcirc$) and $Y=c_3$: 
{\tt Iris-virginica} ($\bigtriangleup$).

\medskip

$Y=c_1$ to $\be_1 = (1,0,0)^T$, $Y=c_2$ to $\be_2 = (0,1,0)^T$ and $Y=c_3$ to $\be_3 = (0,0,1)^T$. 

    \medskip

All the points actually lie in the $(X_1,X_2)$ plane, but $c_1$ and $c_2$ are shown displaced along $Y$ with respect to the base class $c_3$ purely for
illustration purposes.
%\begin{figure}[t!]
%\vspace{-0.1in}
\psset{unit=0.25in}
\psset{lightsrc=viewpoint}
\psset{incolor=gray}
\psset{opacity=0.2}
\def\PsEuler{2.71828182846}
\psset{viewpoint=30 120 10 rtp2xyz,Decran=60}
\psset{axisnames={X_1, X_2, Y}}
\centerline{
\scalebox{0.8}{%
\begin{pspicture}(-6,-2)(6,6)
    \axesIIID[](-4,-2,0)(4.5,2.5,2.5)
\psset{linewidth=0.5pt, dotsize=0.3}
\psset{showAxes=false, fillcolor=gray}
\psset{transform={1.0 1.0 2.0 scaleOpoint3d}}
\psSolid[object=parallelepiped,a=8,b=5,c=1.0,action=draw](0,-0.5,0.5)
\psset{dotstyle=Bsquare,fillcolor=white}
\input{REG/logit/figs/iris-3K-C0.tex}
\psset{dotstyle=Bo,fillcolor=white}
\input{REG/logit/figs/iris-3K-C1.tex}
\psset{dotstyle=Btriangle,fillcolor=white}
\input{REG/logit/figs/iris-3K-C2.tex}
%%misclassified points
\psset{dotstyle=Bo, fillcolor=white}
\psPoint(-0.52,-1.19,2.0){x0}
\psdot(x0)
\psPoint(-1.17,-0.16,2.0){x1}
\psdot(x1)
\psset{dotstyle=Btriangle, fillcolor=white}
\psPoint(-1.55,0.26,0){x2}
\psdot(x2)
\psPoint(-1.296,-0.33,0){x3}
\psdot(x3)
\psPoint(-1.38,-0.42,0){x4}
\psdot(x4)
\end{pspicture}
}}
\end{frame}
\begin{frame}{Multiclass Logistic Regression}
	\framesubtitle{Example}

    We use $Y=c_3$ as the reference or base
    class. The fitted model is:
    \begin{align*}
    \abw_1 & = (-3.52, 3.62, 2.61)^T &
    \abw_2 & = (-6.95, -5.18, -3.40)^T &
    \abw_3 & = (0, 0, 0)^T
    \end{align*}

    The decision surface corresponding to $c_1$ is:
    \begin{align*}
	    \pi_1(\abx) & = \frac{\exp\{ \abw_1^T\abx \}}{1 + \exp\{ \abw_1^T\abx \} + \exp\{\abw_2^T\abx \}} \\
    \end{align*}

%\begin{figure}[t!]
%\vspace{-0.1in}
\psset{unit=0.25in}
\psset{lightsrc=viewpoint}
\psset{incolor=gray}
\psset{opacity=0.2}
\def\PsEuler{2.71828182846}
\psset{viewpoint=30 120 10 rtp2xyz,Decran=60}
\psset{axisnames={X_1, X_2, Y}}
\centerline{
\scalebox{0.8}{%
\begin{pspicture}(-6,-2)(6,6)
    \axesIIID[](-4,-2,0)(4.5,2.5,2.5)
\psset{linewidth=0.5pt, dotsize=0.3}
\psset{showAxes=false, fillcolor=gray}
\psset{transform={1.0 1.0 2.0 scaleOpoint3d}}
\rput(-4,5.1){$\pi_1(\abx)$}
\psSolid[object=parallelepiped,a=8,b=5,c=1.0,action=draw](0,-0.5,0.5)
\psSurface[ngrid=.2 .2,fillcolor=white,axesboxed=false,
    linewidth=0.5\pslinewidth, 
       color1 = {[rgb]{0 0 0}},
        color2 = {[rgb]{1 1 1}},
        hue=(color1) (color2), lightintensity=5,
    algebraic,Zmin=0,Zmax=1](-4,-3)(4,2){%
        ((\PsEuler^((3.618*x)+(2.609*y)-3.52))/%
        (1 + \PsEuler^((3.618*x)+(2.609*y)-3.52) +%
            \PsEuler^((-5.178*x)-(3.398*y)-6.95)))}
            \psset{transform={1.0 1.0 2.0 scaleOpoint3d}}
\psset{dotstyle=Bsquare,fillcolor=white}
\input{REG/logit/figs/iris-3K-C0.tex}
\psset{dotstyle=Bo,fillcolor=white}
\input{REG/logit/figs/iris-3K-C1.tex}
\psset{dotstyle=Btriangle,fillcolor=white}
\input{REG/logit/figs/iris-3K-C2.tex}
%%misclassified points
\psset{dotstyle=Bo, fillcolor=white}
\psPoint(-0.52,-1.19,2.0){x0}
\psdot(x0)
\psPoint(-1.17,-0.16,2.0){x1}
\psdot(x1)
\psset{dotstyle=Btriangle, fillcolor=white}
\psPoint(-1.55,0.26,0){x2}
\psdot(x2)
\psPoint(-1.296,-0.33,0){x3}
\psdot(x3)
\psPoint(-1.38,-0.42,0){x4}
\psdot(x4)
\end{pspicture}
}}
\end{frame}
\begin{frame}{Multiclass Logistic Regression}
	\framesubtitle{Example}

    We use $Y=c_3$ as the reference or base
    class. The fitted model is:
    \begin{align*}
    \abw_1 & = (-3.52, 3.62, 2.61)^T &
    \abw_2 & = (-6.95, -5.18, -3.40)^T &
    \abw_3 & = (0, 0, 0)^T
    \end{align*}

    The decision surface corresponding to $c_2$ is:
    \begin{align*}
        \pi_2(\abx) & = \frac{\exp\{ \abw_2^T\abx \}}{1 + \exp\{ \abw_1^T\abx \} + 
            \exp\{\abw_2^T\abx \}}\\
    \end{align*}

%\begin{figure}[t!]
%\vspace{-0.1in}
\psset{unit=0.25in}
\psset{lightsrc=viewpoint}
\psset{incolor=gray}
\psset{opacity=0.2}
\def\PsEuler{2.71828182846}
\psset{viewpoint=30 120 10 rtp2xyz,Decran=60}
\psset{axisnames={X_1, X_2, Y}}
\centerline{
\scalebox{0.8}{%
\begin{pspicture}(-6,-2)(6,6)
    \axesIIID[](-4,-2,0)(4.5,2.5,2.5)
\psset{linewidth=0.5pt, dotsize=0.3}
\psset{showAxes=false, fillcolor=gray}
\psset{transform={1.0 1.0 2.0 scaleOpoint3d}}
\rput(5,4.75){$\pi_2(\abx)$}
\psSolid[object=parallelepiped,a=8,b=5,c=1.0,action=draw](0,-0.5,0.5)
\psSurface[ngrid=.2 .2,fillcolor=gray,axesboxed=false,
    linewidth=0.5\pslinewidth, 
       color1 = {[rgb]{0 0 0}},
        color2 = {[rgb]{1 1 1}},
        hue=(color1) (color2), lightintensity=5,
    algebraic,Zmin=0,Zmax=1](-4,-3)(4,2){%
    ((\PsEuler^((-5.178*x)-(3.398*y)-6.95))/%
    (1 + \PsEuler^((3.618*x)+(2.609*y)-3.52) +
        \PsEuler^((-5.178*x)-(3.398*y)-6.95)))}
        \psset{transform={1.0 1.0 2.0 scaleOpoint3d}}
\psset{dotstyle=Bsquare,fillcolor=white}
\input{REG/logit/figs/iris-3K-C0.tex}
\psset{dotstyle=Bo,fillcolor=white}
\input{REG/logit/figs/iris-3K-C1.tex}
\psset{dotstyle=Btriangle,fillcolor=white}
\input{REG/logit/figs/iris-3K-C2.tex}
%%misclassified points
\psset{dotstyle=Bo, fillcolor=darkgray}
\psPoint(-0.52,-1.19,2.0){x0}
\psdot(x0)
\psPoint(-1.17,-0.16,2.0){x1}
\psdot(x1)
\psset{dotstyle=Btriangle, fillcolor=white}
\psPoint(-1.55,0.26,0){x2}
\psdot(x2)
\psPoint(-1.296,-0.33,0){x3}
\psdot(x3)
\psPoint(-1.38,-0.42,0){x4}
\psdot(x4)
\end{pspicture}
}}
\end{frame}
\begin{frame}{Multiclass Logistic Regression}
	\framesubtitle{Example}

    We use $Y=c_3$ as the reference or base
    class. The fitted model is:
    \begin{align*}
    \abw_1 & = (-3.52, 3.62, 2.61)^T &
    \abw_2 & = (-6.95, -5.18, -3.40)^T &
    \abw_3 & = (0, 0, 0)^T
    \end{align*}

    The decision surface corresponding to $c_3$ is:
    \begin{align*}
        \pi_3(\abx) & = \frac{1}{1 + \exp\{ \abw_1^T\abx \} + 
            \exp\{\abw_2^T\abx \}}
    \end{align*}

%\begin{figure}[t!]
%\vspace{-0.1in}
\psset{unit=0.25in}
\psset{lightsrc=viewpoint}
\psset{incolor=gray}
\psset{opacity=0.2}
\def\PsEuler{2.71828182846}
\psset{viewpoint=30 120 10 rtp2xyz,Decran=60}
\psset{axisnames={X_1, X_2, Y}}
\centerline{
\scalebox{0.8}{%
\begin{pspicture}(-6,-2)(6,6)
    \axesIIID[](-4,-2,0)(4.5,2.5,2.5)
\psset{linewidth=0.5pt, dotsize=0.3}
\psset{showAxes=false, fillcolor=gray}
\psset{transform={1.0 1.0 2.0 scaleOpoint3d}}
\rput(-0.75,4.95){$\pi_3(\abx)$}
\psSolid[object=parallelepiped,a=8,b=5,c=1.0,action=draw](0,-0.5,0.5)
\psSurface[ngrid=.2 .2,fillcolor=darkgray,axesboxed=false,
    linewidth=0.5\pslinewidth, 
       color1 = {[rgb]{0 0 0}},
        color2 = {[rgb]{1 1 1}},
        hue=(color1) (color2), lightintensity=5,
    algebraic,Zmin=0,Zmax=1](-4,-3)(4,2){%
    (1/(1 + \PsEuler^((3.618*x)+(2.609*y)-3.52) +
        \PsEuler^((-5.178*x)-(3.398*y)-6.95)))}
\psset{dotstyle=Bsquare,fillcolor=white}
\input{REG/logit/figs/iris-3K-C0.tex}
\psset{dotstyle=Bo,fillcolor=white}
\input{REG/logit/figs/iris-3K-C1.tex}
\psset{dotstyle=Btriangle,fillcolor=white}
\input{REG/logit/figs/iris-3K-C2.tex}
%%misclassified points
\psset{dotstyle=Bo, fillcolor=white}
\psPoint(-0.52,-1.19,2.0){x0}
\psdot(x0)
\psPoint(-1.17,-0.16,2.0){x1}
\psdot(x1)
\psset{dotstyle=Btriangle, fillcolor=darkgray}
\psPoint(-1.55,0.26,0){x2}
\psdot(x2)
\psPoint(-1.296,-0.33,0){x3}
\psdot(x3)
\psPoint(-1.38,-0.42,0){x4}
\psdot(x4)
\end{pspicture}
}}
\end{frame}
\begin{frame}{Multiclass Logistic Regression}
	\framesubtitle{Example}


The training set accuracy is $96.7\%$, since it misclassifies only five points (shown in dark gray). 

\medskip

    For
    example, for the point $\abx = (1, -0.52, -1.19)^T$, we have:
    \begin{align*}
        \pi_1(\abx) & = 0 & \pi_2(\abx) & = 0.448 & \pi_3(\abx) & = 0.552
    \end{align*}
    $\hy = \arg\max_{c_i} \{ \pi_i(\abx) \} = c_3$, whereas the true class is
    $y=c_2$. 


%\begin{figure}[t!]
%\vspace{-0.1in}
\psset{unit=0.25in}
\psset{lightsrc=viewpoint}
\psset{incolor=gray}
\psset{opacity=0.2}
\def\PsEuler{2.71828182846}
\psset{viewpoint=30 120 10 rtp2xyz,Decran=60}
\psset{axisnames={X_1, X_2, Y}}
\centerline{
\scalebox{0.8}{%
\begin{pspicture}(-6,-2)(6,6)
    \axesIIID[](-4,-2,0)(4.5,2.5,2.5)
\psset{linewidth=0.5pt, dotsize=0.3}
\psset{showAxes=false, fillcolor=gray}
\psset{transform={1.0 1.0 2.0 scaleOpoint3d}}
\rput(-4,5.1){$\pi_1(\abx)$}
\rput(-0.75,4.95){$\pi_3(\abx)$}
\rput(5,4.75){$\pi_2(\abx)$}
\psSolid[object=parallelepiped,a=8,b=5,c=1.0,action=draw](0,-0.5,0.5)
\psSurface[ngrid=.2 .2,fillcolor=white,axesboxed=false,
    linewidth=0.5\pslinewidth, 
       color1 = {[rgb]{0 0 0}},
        color2 = {[rgb]{1 1 1}},
        hue=(color1) (color2), lightintensity=5,
    algebraic,Zmin=0,Zmax=1](-4,-3)(4,2){%
        ((\PsEuler^((3.618*x)+(2.609*y)-3.52))/%
        (1 + \PsEuler^((3.618*x)+(2.609*y)-3.52) +%
            \PsEuler^((-5.178*x)-(3.398*y)-6.95)))}
            \psset{transform={1.0 1.0 2.0 scaleOpoint3d}}
\psSurface[ngrid=.2 .2,fillcolor=gray,axesboxed=false,
    linewidth=0.5\pslinewidth, 
       color1 = {[rgb]{0 0 0}},
        color2 = {[rgb]{1 1 1}},
        hue=(color1) (color2), lightintensity=5,
    algebraic,Zmin=0,Zmax=1](-4,-3)(4,2){%
    ((\PsEuler^((-5.178*x)-(3.398*y)-6.95))/%
    (1 + \PsEuler^((3.618*x)+(2.609*y)-3.52) +
        \PsEuler^((-5.178*x)-(3.398*y)-6.95)))}
        \psset{transform={1.0 1.0 2.0 scaleOpoint3d}}
\psSurface[ngrid=.2 .2,fillcolor=darkgray,axesboxed=false,
    linewidth=0.5\pslinewidth, 
       color1 = {[rgb]{0 0 0}},
        color2 = {[rgb]{1 1 1}},
        hue=(color1) (color2), lightintensity=5,
    algebraic,Zmin=0,Zmax=1](-4,-3)(4,2){%
    (1/(1 + \PsEuler^((3.618*x)+(2.609*y)-3.52) +
        \PsEuler^((-5.178*x)-(3.398*y)-6.95)))}
\psset{dotstyle=Bsquare,fillcolor=white}
\input{REG/logit/figs/iris-3K-C0.tex}
\psset{dotstyle=Bo,fillcolor=white}
\input{REG/logit/figs/iris-3K-C1.tex}
\psset{dotstyle=Btriangle,fillcolor=white}
\input{REG/logit/figs/iris-3K-C2.tex}
%%misclassified points
\psset{dotstyle=Bo, fillcolor=darkgray}
\psPoint(-0.52,-1.19,2.0){x0}
\psdot(x0)
\psPoint(-1.17,-0.16,2.0){x1}
\psdot(x1)
\psset{dotstyle=Btriangle, fillcolor=darkgray}
\psPoint(-1.55,0.26,0){x2}
\psdot(x2)
\psPoint(-1.296,-0.33,0){x3}
\psdot(x3)
\psPoint(-1.38,-0.42,0){x4}
\psdot(x4)
\end{pspicture}
}}
\end{frame}
%\vspace{0.2in}
%\caption{Multiclass logistic regression: Iris principal components
%data. Misclassified point are shown in dark gray color. All the points
%actually lie in the $(X_1,X_2)$ plane, but $c_1$ and $c_2$ are shown
%displaced along $Y$ with respect to the base class $c_3$ purely for
%illustration purposes.}
%\label{fig:reg:logit:iris-3K-multiclass}
%\end{figure}
%% incorrect 58 [ 0.  0.  1.] [ 0.  1.  0.] [  1.11753856e-04   4.47148660e-01   5.52739586e-01] [-0.51938325 -1.19135169  1.        ]
%% incorrect 97 [ 0.  0.  1.] [ 0.  1.  0.] [  1.64071772e-04   4.16171365e-01   5.83664564e-01] [-1.16885347 -0.1645025   1.        ]
%% incorrect 126 [ 0.  1.  0.] [ 0.  0.  1.] [  9.54272464e-05   5.51283860e-01   4.48620712e-01] [-1.55739627  0.26739258  1.        ]
%% incorrect 140 [ 0.  1.  0.] [ 0.  0.  1.] [  3.40207855e-05   7.06094574e-01   2.93871405e-01] [-1.29646885 -0.32756152  1.        ]
%% incorrect 142 [ 0.  1.  0.] [ 0.  0.  1.] [  1.11160412e-05   8.34892528e-01   1.65096356e-01] [-1.37873698 -0.42120514  1.        ]
%
%
%\begin{example}
%    \label{ex:reg:logit:multi-logit}

%\end{example}
%
%
%\section{Further Reading}
%\label{sec:reg:logit:ref}
%\begin{refsection}
%
%For a description of the class of generalized linear models, of which
%logistic regression is a special case, see
%\citet{agresti2015foundations}.
%
%\printbibliography[heading=emptyheading]
%\end{refsection}
%
%\section{Exercises}
%\label{sec:reg:logit:exercise}
%
%\begin{exercises}[Q1.]
%
%\item Show that 
%$\frac{\partial \theta(z)}{\partial z} = \theta(z) \cdot
%\theta(-z)$, where $\theta(\cdot)$ is the logistic function.
%
%\item Show that the logit function is the inverse of the logistic
%    function.
%
%\item Given the softmax function:
%\begin{align*}
%    \pi_{\!j}(\abx) = \frac{\exp\{\abw_{\!j}^T\abx\}}{\sum_{i=1}^K
%    \exp\{\abw_{i}^T\abx\} }
%\end{align*}
%Show that 
%\begin{align*}
%    \frac{\partial \pi_{\!j}(\abx)}{\partial \abw_a} = 
%  \begin{cases}
%      \pi_a(\abx) \cdot (1-\pi_a(\abx)) \cdot \abx & \text{if } j=a\\
%      -\pi_a(\abx) \cdot \pi_{\!j}(\abx) \cdot \abx & \text{if } j \ne a    
%  \end{cases}
%\end{align*}
%
%% \item Show that the cross-entropy error function
%%     [\cref{eq:reg:logit:crossentropy}] is convex.
%
%\end{exercises}

