\lecture{numeric}{numeric}

\date{Chapter 2: Numeric Attributes}

\begin{frame}
\titlepage
\end{frame}


\begin{frame}
  \frametitle{Univariate Analysis}
Univariate analysis focuses on a single attribute at a time.
The data matrix $\bD$ is an $n \times 1$ matrix,
\begin{align*}
    \bD = \matr{X\\
        \hline x_1\\x_2\\\vdots\\ x_n}
\end{align*}
where $X$ is the numeric attribute of interest,
with $x_i \in \setR$.

\bigskip
$X$ is assumed to be a random variable, and 
the observed data a random
sample drawn from $X$, i.e., $x_i$'s  are
independent and identically distributed as $X$.

\bigskip
In the vector view, we treat the sample
as an $n$-dimensional vector, and write $X \in \setR^n$.
\end{frame}


\begin{frame}{Empirical Probability Mass Function}
The {\em empirical probability mass
function (PMF)} of $X$
is given as
\begin{align*}
    \hf(x) = P(X=x) = \frac{1}{n}\;\sum_{i=1}^n I(x_i=x)
\end{align*}
where the indicator variable $I$ takes on the value $1$ when its
argument is true, and $0$ otherwise.
The empirical PMF puts a probability mass of $\tfrac{1}{n}$
at each point $x_i$.

\bigskip
The {\em empirical cumulative distribution
function (CDF)}
of $X$ is given as
\begin{align*}
    \hF(x) = \frac{1}{n}\;\sum_{i=1}^n I(x_i \le x)
\end{align*}

\bigskip
The {\em inverse cumulative distribution function}
or
{\em quantile function}
for $X$ is defined as follows:
\begin{align*}
    F^{-1}(q) = \min\{x\;|\; \hF(x) \ge q\} \qquad \text{for } q \in
    [0,1]
\end{align*}
The inverse CDF gives the least value of $X$,
for which $q$ fraction of the values are higher, and $1-q$
fraction of the values are lower.
\end{frame}


\begin{frame}{Mean}
The {\em mean} or {\em expected value}
of a random variable
$X$ is the arithmetic average of the values of $X$. It
provides a one-number summary of the {\em location}
or {\em central tendency} for the distribution of $X$.

\bigskip
If $X$ is discrete, 
it is def\/{i}ned as
\begin{empheq}[box=\tcbhighmath]{align*}
    \mu = E[X] = \sum_x x \cdot f(x)
\end{empheq}
where $f(x)$ is the probability mass function of $X$.

\bigskip
If $X$ is continuous it is 
def\/{i}ned as
\begin{empheq}[box=\tcbhighmath]{align*}
    \mu = E[X] = \int_{-\infty}^{\infty} x \cdot f(x)\; dx
\end{empheq}
where $f(x)$ is the probability density function of $X$.
\end{frame}


\begin{frame}{Sample Mean}
The {\em sample mean}
 is a statistic, that is, a function $\hmu: \{x_1, x_2, \ldots,
 x_n\} \to \setR$, def\/{i}ned as the average value of $x_i$'s:
\begin{empheq}[box=\tcbhighmath]{align*}
    \hmu = \frac{1}{n}\;\sum_{i=1}^n x_i
\end{empheq}
It serves as an
estimator for the unknown mean value $\mu$ of $X$.


\bigskip
An estimator $\htheta$ is called an {\em unbiased
estimator} for
parameter $\theta$ if $E[\htheta] = \theta$ for every possible
value of $\theta$.
The sample mean $\hmu$ is an unbiased estimator for the population
mean $\mu$, as
\begin{align*}
    E[\hmu] = E\lB[ \frac{1}{n}\;\sum_{i=1}^n x_i \rB] = {1 \over n}
    \sum_{i=1}^n E[x_i] = {1\over n}\sum_{i=1}^n \mu = \mu
\end{align*}

\bigskip
We say that a statistic is {\em
robust} if it is not affected by
extreme values (such as outliers) in the data. The sample mean is
not robust because a single large value
can skew the average. 
\end{frame}

\readdata{\dataSL}{EDA/numeric/figs/iris-sl.dat}
\begin{frame}{Sample Mean: Iris {\tt sepal length}}
    \centering
    \psset{dotstyle=Bo,dotscale=1.5,fillcolor=lightgray,arrowscale=2,PointName=none}
    \psset{xAxisLabel=$X_1$,yAxisLabel=Frequency}
    \psgraph[tickstyle=bottom,Ox=4,Dx=0.5,dx=0.5]{->}(4.0,0)(8.5,0.5){4in}{1in}%
    \dataplot[plotstyle=dots,showpoints=true]{\dataSL}
    \psdots[dotstyle=*,dotscale=2](5.843,0)
    \pcline[arrowscale=1.5]{->}(5.843,-0.2)(5.843,-0.02)
    \uput[d](5.843,-0.2){$\hmu=5.843$}
    \endpsgraph
\end{frame}


\begin{frame}{Median}
The {\em median}
of a random variable is def\/{i}ned as the value $m$ such that
\begin{align*}
    P(X \le m) \ge \frac{1}{2} \mbox{  and  } P(X \ge m) \ge \frac{1}{2}
\end{align*}
The median $m$ is the ``middle-most'' value; half
of the values of $X$ are less and half of the values of $X$
are more than $m$. 

\bigskip
In terms of the (inverse) cumulative
distribution function, the median is the
value $m$ for which
\begin{align*}
    F(m) = 0.5 \text{  or  } m = F^{-1}(0.5)
\end{align*}

\bigskip
The {\em sample median} is given as
\begin{align*}
    \hF(m) = 0.5 \text{ or } m = \hF^{-1}(0.5)
\end{align*}
Median is robust, as it is not affected very
much by extreme values. 
\end{frame}

\begin{frame}{Mode}
The {\em mode} of a
random variable $X$ is the value at which the probability mass
function or the probability density function attains its maximum
value, depending on whether $X$ is discrete or continuous,
respectively.

\bigskip
The {\em sample mode} is a
value for
which the empirical probability mass function
attains its maximum, given as
\begin{align*}
    \mbox{mode}(X) = \arg \max_x\,\hf(x)
\end{align*}
\end{frame}


\readdata{\dCDF}{EDA/numeric/figs/sl-CDF.dat}
\readdata{\diCDF}{EDA/numeric/figs/sl-iCDF.dat}
\begin{frame}{Empirical CDF: {\tt sepal length}}
\centering
    \psset{xAxisLabel=$x$,yAxisLabel=$\hF(x)$}
    \psset{xAxisLabelPos={c,-0.2},yAxisLabelPos={-0.55,c}}
    \psgraph[tickstyle=bottom,axesstyle=frame,%
    Dy=0.25,dy=0.25,Ox=4,Dx=0.5,dx=0.5]{->}(4,0)(8,1){3.5in}{2in}%
    \dataplot[linewidth=1.5pt]{\dCDF}
    \endpsgraph
\end{frame}


\begin{frame}{Empirical Inverse CDF: {\tt sepal length}}
    \centerline{
    \psset{xAxisLabel=$q$,yAxisLabel=$\hF^{-1}(q)$}
    \psset{xAxisLabelPos={c,-0.75},yAxisLabelPos={-0.125,c}}
    \psgraph[tickstyle=bottom,axesstyle=frame,%
    Dx=0.25,dx=0.25,Oy=4,Dy=0.5,dy=0.5]{->}(0,4)(1,8){3.5in}{2in}%
    \dataplot[linewidth=1.5pt]{\diCDF}
    \psline[linestyle=dotted](0.25,4)(0.25,5.1)
    \psline[linestyle=dotted](0.25,5.1)(0,5.1)
    \psline[linestyle=dotted](0.5,4)(0.5,5.8)
    \psline[linestyle=dotted](0.5,5.8)(0,5.8)
    \psline[linestyle=dotted](0.75,4)(0.75,6.4)
    \psline[linestyle=dotted](0.75,6.4)(0,6.4)
    \endpsgraph
	}

	~\\~\\~\\
	The median is $5.8$, since
	$$\hF(5.8)
    = 0.5 \text{ or } 5.8 = \hF^{-1}(0.5)$$
\end{frame}


\begin{frame}{Range}
The {\em value range} or simply {\em range}  
of a random variable $X$ is the difference
between the maximum and minimum values of $X$, given as
\begin{align*}
    r = \max\{X\} - \min\{X\}
\end{align*}

\smallskip
The {\em sample range} is a
statistic, given as
\begin{align*}
    \hat{r} = \max_{i=1}^n \{x_i\} - \min_{i=1}^n \{x_i\}
\end{align*}
Range is sensitive to extreme values,
and thus is not robust.

\bigskip
A more robust measure of the dispersion of $X$ is the {\em
interquartile range (IQR)},
def\/{i}ned as
\begin{align*}
    \hbox{\textit{IQR}} = F^{-1}(0.75) - F^{-1}(0.25)
\end{align*}

\smallskip
The {\em sample IQR} is given as
\begin{align*}
    \widehat{\hbox{\textit{IQR}}} &
    = \hF^{-1}(0.75) - \hF^{-1}(0.25)
\end{align*}
\end{frame}



\begin{frame}{Variance and Standard Deviation}
The {\em variance}
of a random variable $X$ provides a measure of how
much the values of $X$ deviate from the mean or expected value of
$X$
\begin{align*}
\tcbhighmath{\sigma^2 = \var(X) = E\bigl[(X-\mu)^2\bigr]} =
    \begin{cases}
     \displaystyle \sum_{x} (x - \mu)^2 \; f(x) & \text{if $X$
     is discrete} \\[2em]
    \displaystyle \int_{-\infty}^{\infty} (x - \mu)^2 \;
    f(x)\; dx & \text{if $X$ is continuous}
\end{cases}
\end{align*}
The {\em standard deviation} $\sigma$,
is the positive square root of the variance, $\sigma^2$.

\bigskip
The {\em sample variance}  is
def\/{i}ned as
\begin{empheq}[box=\tcbhighmath]{align*}
    \hsigma^2 & = \frac{1}{n}\;\sum_{i=1}^n (x_i - \hmu)^2
\end{empheq}
and the {\em sample standard deviation} is
\begin{align*}
    \hsigma = \sqrt{\frac{1}{n}\;\sum_{i=1}^n
  (x_i - \hmu)^2}
\end{align*}
\end{frame}


\begin{frame}{Geometric Interpretation of Sample Variance}
The sample values for $X$ comprise a vector in
$n$-dimensional space, where $n$ is the sample size. 
Let $Z$ denote the centered sample 
$$Z = X- \bone \cdot \hmu = \matr{x_1 -\hmu\\x_2-\hmu\\ \vdots \\ x_n-\hmu}$$
where 
$\bone \in \setR^n$ is the vector of ones.

Sample variance is 
squared
magnitude of the centered attribute vector, normalized by the sample
size:
\begin{align*}
    \hsigma^2 = \frac{1}{n}\;\norm{Z}^2 =
    \frac{1}{n}\;Z^T Z = \frac{1}{n}\;\sum_{i=1}^n
    (x_i-\hmu)^2
\end{align*}
\end{frame}


\begin{frame}{Variance of the Sample Mean and Bias}
Sample mean $\hmu$ is itself a statistic. We can 
compute its mean value and
variance
\begin{align*}
  E[\hmu] & = \mu\\
  var(\hmu) & = E[(\hmu -\mu)^2] =  {\sigma^2 \over n}
\end{align*}
The sample mean $\hmu$ varies or deviates from the
 mean $\mu$ in proportion to the population
variance $\sigma^2$. However, the deviation can be made smaller by
considering larger sample size $n$.

\bigskip
The sample variance
is a {\em biased estimator}
for the true population variance, since
\begin{align*}
    E[\hsigma^2]    & = \lB({n-1\over n} \rB) \sigma^2
\end{align*}
But it is asymptotically unbiased, since
$$E[\hsigma^2] \to \sigma^2 \qquad \text{as } n \to \infty$$

\end{frame}

\ifdefined\wox \begin{frame} \titlepage \end{frame} \fi

\begin{frame}{Bivariate Analysis}
In bivariate analysis\index{bivariate analysis!numeric}, we consider two attributes at the same
time. 
The data $\bD$ comprises an $n \times 2$ matrix:
\begin{align*}
    \bD = &
    \matr{%
        X_1 & X_2\\
        \hline
        x_{11} & x_{12}\\
        x_{21} & x_{22}\\
        \vdots & \vdots\\
        x_{n1} & x_{n2}%
        }
\end{align*}

Geometrically, $\bD$ comprises
$n$ points or vectors in 2-dimensional space
$$\bx_i = (x_{i1}, x_{i2})^T \in \setR^2$$ 
$\bD$ can also be viewed as two points or vectors in an
$n$-dimensional space:
\begin{align*}
X_1 & = (x_{11}, x_{21}, \ldots, x_{n1})^T\\
X_2 & = (x_{12}, x_{22}, \ldots, x_{n2})^T
\end{align*}

\bigskip
In the probabilistic view,
$\bX = (X_1,X_2)^T$ is a bivariate
vector random variable, and the points $\bx_i$ ($1
\le i \le n$) are a random sample drawn from $\bX$,
that is, $\bx_i$'s IID with $\bX$.
\end{frame}


\begin{frame}{Bivariate Mean and Variance}
The bivariate mean is def\/{i}ned as the expected value of the vector
random variable $\bX$:
\begin{align*}
    \bmu = E[\bX] = E\lB[ \matr{X_1\\X_2} \rB] =
    \matr{E[X_1] \\[0.2em] E[X_2]} = \matr{\mu_{1}\\ \mu_{2}}
\end{align*}
%
The sample mean vector is given as
\begin{empheq}[box=\tcbhighmath]{align*}
    \hbmu =
    \sum_\bx \bx \hf(\bx) =
    \sum_\bx \bx \lB(
    \frac{1}{n}\;\sum_{i=1}^n I(\bx_i=\bx)\rB)
        = {1 \over n} \sum_{i=1}^n \bx_i
\end{empheq}
\end{frame}


\begin{frame}{Covariance}
The {\em covariance}\index{covariance}\index{random
variable!covariance} between two attributes $X_1$ and $X_2$ provides a
measure of the association or linear dependence between them, and
is def\/{i}ned as
\begin{empheq}[box=\tcbhighmath]{align*}
    \sigma_{12} & = E[(X_1-\mu_{1})(X_2-\mu_{2})]\\
     & = E[X_1X_2] - E[X_1] E[X_2]
\end{empheq}

If $X_1$ and $X_2$ are independent, then
$$E[X_1X_2] = E[X_1] \cdot E[X_2]$$
which implies that $\sigma_{12} = 0$.

\bigskip
The {\em sample covariance} between $X_1$ and $X_2$ is given as
\begin{empheq}[box=\tcbhighmath]{align*}
    \hsigma_{12} =
    \frac{1}{n}\;\sum_{i=1}^n (x_{i1} - \hmu_{1})(x_{i2} - \hmu_{2})
\end{empheq}
\end{frame}


\begin{frame}{Correlation} 
The {\em correlation}
between variables $X_1$
and $X_2$ is the {\em standardized covariance}, obtained by normalizing the
covariance with the standard deviation of each
variable, given as
\begin{align*}
  \rho_{12} =
  \frac{\sigma_{12}}{\sigma_{1}\sigma_{2}} =
  \frac{\sigma_{12}}{\sqrt{\sigma^2_{1}\sigma^2_{2}}}
\end{align*}

\bigskip
The {\em sample correlation}
for attributes $X_1$ and $X_2$ is given as
\begin{empheq}[box=\tcbhighmath]{align*}
  \hrho_{12} =
  \frac{\hsigma_{12}}{\hsigma_{1}\hsigma_{2}} =
  \frac{\sum_{i=1}^n (x_{i1} - \hmu_{1})(x_{i2} - \hmu_{2})}
  {\sqrt{\sum_{i=1}^n (x_{i1} - \hmu_{1})^2
  \sum_{i=1}^n (x_{i2} - \hmu_{2})^2}}
\end{empheq}
\end{frame}


\begin{frame}{Geometric Interpretation of Sample Covariance and
Correlation} 
Let $\mX_1$ and $\mX_2$ denote the centered
attribute vectors in $\setR^n$:
\begin{align*}
  \mX_1 & = X_1-\bone \cdot \hmu_{1} = \matr{
  x_{11} - \hmu_{1}\\
  x_{21} - \hmu_{1}\\
      \vdots\\
      x_{n1} - \hmu_{1}\\
    } &
  \mX_2 & = X_2-\bone \cdot \hmu_{2} = \matr{
  x_{12} - \hmu_{2}\\
  x_{22} - \hmu_{2}\\
      \vdots\\
      x_{n2} - \hmu_{2}\\
  }
\end{align*}

The sample covariance 
and the sample correlation 
are given as
\begin{empheq}[box=\tcbhighmath]{align*}
    \hsigma_{12} = {\mX_1^T \mX_2 \over n}
\end{empheq}
\begin{empheq}[box=\tcbhighmath]{align*}
  \hrho_{12} = \frac{\mX_1^T \mX_2}{\sqrt{\mX_1^T \mX_1} \sqrt{\mX_2^T \mX_2}}
  = \frac{\mX_1^T \mX_2}{\norm{\mX_1} \; \norm{\mX_2}}
  = \lB(\frac{\mX_1}{\norm{\mX_1}}\rB)^T
  \lB(\frac{\mX_2}{\norm{\mX_2}}\rB) = \cos \theta
\end{empheq}
The correlation coeff\/{i}cient is simply the cosine  of the
angle between the two centered
attribute vectors.
\end{frame}

\begin{frame}{Geometric Interpretation of Covariance and Correlation}
\begin{center}
\begin{pspicture}(-1,-1)(4.5,4.5)
    \psset{nameX=$\bx_n$,nameY=$\bx_2$,nameZ=$\bx_1$,
        Alpha=70,Beta=15,arrowscale=1.5}
    \pstThreeDCoor[linecolor=black,xMin=0,xMax=5,
        yMin=0,yMax=4,zMin=0,zMax=4]
    \pstThreeDLine[linewidth=2pt,arrows=->](0,0,0)(1,3,4)
    \pstThreeDLine[linewidth=2pt,arrows=->](0,0,0)(3,0.5,3)
    \pstThreeDDot[drawCoor=true,linestyle=dotted,linecolor=gray](1,3,4)
    \pstThreeDDot[drawCoor=true,linestyle=dotted,linecolor=gray](3,0.5,3)
\psPoint(0,0,0){O}
\psPoint(1,3,4.4){B}
\psPoint(1.5,0.5,3.4){A}
\pstMarkAngle[MarkAngleRadius=0.6,LabelSep=0.85]{B}{O}{A}{$\theta$}
    \pstThreeDPut[](1,3,4.4){$Z_2$}
    \pstThreeDPut[](3,0.5,3.4){$Z_1$}
\end{pspicture}
\end{center}
\end{frame}




\begin{frame}{Covariance Matrix}
The variance--covariance information for the two attributes $X_1$
and $X_2$ can be summarized in the
square $2 \times 2$ {\em covariance matrix}
\begin{align*}
    \cov & = E[(\bX-\bmu)(\bX-\bmu)^T] \notag\\
  & = \matr{ \sigma^2_{1} & \sigma_{12}\\
      \sigma_{21} & \sigma^2_{2}\\
  }
\end{align*}
Because $\sigma_{12} = \sigma_{21}$, $\cov$ is {\em symmetric}.

The {\em total variance}
 is given as 
\begin{empheq}[box=\tcbhighmath]{align*}
    var(\bD) = tr(\cov) = \sigma_1^2 + \sigma_2^2
\end{empheq}
We immediately have $tr(\cov) \ge 0$.

The {\em generalized variance} is
\begin{align*}
    |\cov| = \det(\cov) = \sigma^2_{1}\sigma^2_2 - \sigma_{12}^2 =
\sigma^2_{1}\sigma^2_2 - \rho^2_{12}\sigma^2_{1}\sigma^2_2
= (1-\rho^2_{12}) \sigma^2_{1}\sigma^2_2
\end{align*}
Note that $|\rho_{12}| \le 1$ implies that $\det(\cov) \ge 0$.
\end{frame}


\readdata{\dataSLW}{EDA/numeric/figs/iris-slw.dat}
\begin{frame}{Correlation: {\tt sepal length} and {\tt sepal width}}
  \begin{columns}
	\column{0.6\textwidth}
        \centering
        \scalebox{0.55}{
        \psset{dotstyle=Bo,dotscale=1.5,fillcolor=lightgray,
                arrowscale=2,PointName=none}
        \psset{xAxisLabel=$X_1$: sepal length,
                yAxisLabel= $X_2$: sepal width,
                xAxisLabelPos={c,-0.4in},
                yAxisLabelPos={-0.5in,c}}
        \psgraph[tickstyle=bottom,Dx=0.5,Dy=0.5,
                Ox=4,Oy=2,subticks=2]{->}(4.0,2.0)(8.5,4.5){4in}{3in}%
        \dataplot[plotstyle=dots,showpoints=true]{\dataSLW}
        \psline[linestyle=dashed](4,3.15)(8.5,2.9)
        \endpsgraph
        }
	\column{0.4\textwidth}
	\small
The sample mean is
$$\hbmu = \matr{5.843\\3.054}$$

The sample covariance matrix is
$$\hcov = \amatr{r}{0.681 & -0.039\\ -0.039 & 0.187}$$

The sample correlation is
$$\hrho_{12} = {-0.039 \over \sqrt{0.681 \cdot 0.187}} = -0.109$$

  \end{columns}
\end{frame}

\ifdefined\wox \begin{frame} \titlepage \end{frame} \fi

\begin{frame}{Multivariate Analysis}
In multivariate analysis we consider all the
$d$ numeric attributes $X_1, X_2,
\ldots, X_d$.
\begin{align*}
    \bD = &
    \matr{%
        X_1 & X_2 & \cdots & X_d\\
        \hline
        x_{11} & x_{12} & \cdots & x_{1d}\\
        x_{21} & x_{22} & \cdots & x_{2d}\\
        \vdots & \vdots & \ddots & \vdots\\
        x_{n1} & x_{n2} & \cdots & x_{nd}
        }
\end{align*}

In the row view, the data is a set of
$n$ points or vectors in the \hbox{$d$-dimensional} attribute space
$${\bx_i} = (x_{i1}, x_{i2}, \ldots, x_{id})^T \in \setR^d$$

In the column view, the data is a set of
$d$ points or vectors in the $n$-dimensional space spanned by the
data points
$$X_{j} = (x_{1j}, x_{2j}, \ldots, x_{nj})^T \in \setR^n$$

\end{frame}


\begin{frame}{Mean and Covariance}
In the probabilistic view, the $d$ attributes are modeled as a
vector random variable, $\bX = (X_1, X_2,\ldots, X_d)^T$, and
the points $\bx_i$ are considered to be a random sample drawn
from $\bX$, i.e., IID with $\bX$.

The {\em multivariate mean vector}
is 
\begin{align*}
    \bmu & = E[\bX] =
  \matr{\mu_{1} & \mu_{2} & \cdots & \mu_{d}}^T
\end{align*}

The {\em sample mean} is
\begin{empheq}[box=\tcbhighmath]{align*}
    \hbmu = {1 \over n} \sum_{i=1}^n \bx_i
\end{empheq}


The (sample) covariance matrix is a $d\times d$ (square) symmetric matrix
\begin{align*}
  \cov &  =
  \matr{
      \sigma^2_{1} & \sigma_{12} & \cdots & \sigma_{1d}\\[0.2em]
      \sigma_{21} & \sigma^2_{2} & \cdots & \sigma_{2d}\\[0.2em]
      \cdots & \cdots & \cdots & \cdots\\
      \sigma_{d1} & \sigma_{d2} & \cdots & \sigma^2_{d}\\
  }
  &
  \hcov & =
  \matr{
      \hsigma^2_{1} & \hsigma_{12} & \cdots & \hsigma_{1d}\\[0.2em]
      \hsigma_{21} & \hsigma^2_{2} & \cdots & \hsigma_{2d}\\[0.2em]
      \cdots & \cdots & \cdots & \cdots\\
      \hsigma_{d1} & \hsigma_{d2} & \cdots & \hsigma^2_{d}\\
  }
\end{align*}
\end{frame}


\begin{frame}{Covariance Matrix is Positive Semidef\/{i}nite}
$\cov$ is a {\em positive semidef\/{i}nite}
matrix\index{covariance matrix!positive semi-definite}, that is,
\begin{align*}
    \ba^T \cov \ba \ge 0
    \text{ for any $d$-dimensional vector } \ba
\end{align*}
To see this, observe that
\begin{align*}
  \ba^T \cov \ba & = \ba^T E\bigl[(\bX-\bmu) (\bX-\bmu)^T\bigr] \ba\\
    & = E\bigl[\ba^T (\bX-\bmu) (\bX-\bmu)^T \ba\bigr]\\
    & = E\bigl[Y^2\bigr]\\
    & \ge 0
  \end{align*}

\medskip
Because
$\cov$ is also symmetric, this implies that
all the eigenvalues of $\cov$ are real and non-negative,
and they can be arranged from the
largest to the smallest as follows: 
$\lambda_1 \ge \lambda_2 \ge \cdots
\ge \lambda_d \ge 0$.

\medskip
The total variance is given as: $var(\bD) = \prod_{i=1}^d \sigma^2_i$

The generalized variance is
$\det(\cov) = \prod_{i=1}^d \lambda_i \ge 0$

\end{frame}


\begin{frame}{Sample Covariance Matrix: Inner and Outer Product}
  \small
Let $\mbD$ represent the centered data matrix
\begin{align*}
    \mbD = \bD - \bone \cdot \hbmu^T
    = \matr{\bx_1^T - \hbmu^T\\[0.2em] \bx_2^T - \hbmu^T \\[0.2em] \vdots \\
    \bx_n^T-\hbmu^T}
    = \matr{\mbox{---} & \mbx_1^T & \mbox{---} \\[0.2em]
    \mbox{---} & \mbx_2^T& \mbox{---} \\[0.2em]
    & \vdots & \\
    \mbox{---} & \mbx_n^T & \mbox{---}}
\end{align*}

\medskip
Inner product and outer product 
form for sample covariance matrix:
\begin{empheq}[box=\tcbhighmath]{align*}
    \hcov = {1\over n} \lB(\mbD^T\; \mbD\rB)
   = {1 \over n} \matr{\mX_1^T\; \mX_1 & \mX_1^T\; \mX_2 & \cdots &
       \mX_1^T\;
    \mX_d\\[0.5em]
        \mX_2^T\; \mX_1 & \mX_2^T\; \mX_2 & \cdots & \mX_2^T\; \mX_d\\[0.5em]
        \vdots & \vdots & \ddots & \vdots\\[0.5em]
        \mX_d^T\; \mX_1 & \mX_d^T\; \mX_2 & \cdots & \mX_d^T\; \mX_d\\
}
	& 
    \hspace*{0.3cm} \hcov  = {1\over n} \sum_{i=1}^n \mbx_i \cdot \mbx_i^T
\end{empheq}
i.e., $\hcov$ is given as
the pairwise {\em inner or dot products} 
of the centered attribute vectors,
normalized by the sample size, or as a 
sum of rank-one matrices obtained as the
{\em outer product} of each centered point.
\begin{align*}
\end{align*}
\end{frame}

\ifdefined\wox \begin{frame} \titlepage \end{frame} \fi

\begin{frame}{Data Normalization}
  If the attribute values are in vastly different scales, then it is
necessary to
normalize them.

\medskip
{\bf Range Normalization:}
Let $X$ be an attribute and let $x_1, x_2, \dots, x_n$ be a random
sample drawn from $X$.
In {\em range normalization} each value is scaled by the sample range
$\hat{r}$ of $X$:
\begin{align*}
    x_i' = \frac{x_i - \min_i\{x_i\}}{\hat{r}} =
    \frac{x_i - \min_i \{x_i\}}{\max_i\{x_i\} - \min_i\{x_i\}}
\end{align*}
After transformation the new attribute takes on values in the
range $[0,1]$.

\medskip
{\bf Standard Score Normalization: }
Also called $z$-normalization;
each value is replaced by its $z$-score:
\begin{equation*}
x_i' = \frac{x_i-\hmu}{\hsigma}
\end{equation*}
where $\hmu$ is the sample mean and $\hsigma^2$ is the sample
variance of $X$.
After transformation, the new attribute has mean $\hmu' = 0$,
and standard deviation $\hsigma' = 1$.
\end{frame}


\begin{frame}{Normalization Example}
  \begin{columns}
	\column{0.4\textwidth}
	\small
\begin{center}
\begin{tabular}{|l||c|c|}
\hline
$\bx_i$ & Age ($X_1$) & Income ($X_2$) \\
\hline
$\bx_1$ & 12 & 300 \\
\hline
$\bx_2$ & 14 & 500 \\
\hline
$\bx_3$ & 18 & 1000 \\
\hline
$\bx_4$ & 23 & 2000 \\
\hline
$\bx_5$ & 27 & 3500 \\
\hline
$\bx_6$ & 28 & 4000 \\
\hline
$\bx_7$ & 34 & 4300 \\
\hline
$\bx_8$ & 37 & 6000 \\
\hline
$\bx_9$ & 39 & 2500 \\
\hline
$\bx_{10}$ & 40 & 2700 \\
\hline
\end{tabular}
\end{center}

\column{0.55\textwidth}
\small
Since {\tt Income} is much larger, it dominates {\tt Age}.

The sample range for {\tt Age} is $\hat{r} = 40-12 = 28$, whereas for
{\tt Income} it is $6000-300=5700$. For range normalization,
the point $\bx_2 = (14,500)$ is scaled to 
$$\bx'_2 = \lB( {14-12 \over 28}, {500-300 \over 5700}\rB) = (0.071,0.035)$$

For $z$-normalization, we have 
\begin{center}
{\renewcommand{\arraystretch}{1.1}\begin{tabular}{|r|c|c|}
\hline
 & Age & Income \\
\hline
$\hmu$ & 27.2 & 2680 \\
$\hsigma$ & 9.77 & 1726.15 \\
\hline
\end{tabular}}
\end{center}
Thus, $\bx_2 = (14,500)$ is scaled to 
$$\bx'_2 = \lB( {14-27.2 \over 9.77}, {500-2680 \over 1726.15}\rB) =
(-1.35,-1.26)$$
\end{columns}

\end{frame}


\begin{frame}{Univariate Normal Distribution}
The normal distribution plays
an important role as the parametric distribution of choice
in clustering, density estimation, and classif\/{i}cation.

\bigskip
A random variable $X$ has a normal
distribution, with
the parameters mean $\mu$ and variance $\sigma^2$, if the
probability density function of $X$ is given as follows:
\begin{empheq}[box=\tcbhighmath]{align*}
    f(x|\mu, \sigma^2) = \frac{1}{\sqrt{2\pi\sigma^2}}
  \;\;\exp\lB\{-\frac{(x-\mu)^2}{2\sigma^2}\rB\}
\end{empheq}
The term $(x-\mu)^2$ measures the distance of a value $x$ from
the mean $\mu$ of the distribution, and thus the probability
density decreases exponentially as a function of the distance
from the mean. 

\medskip
The maximum value of the density occurs at the
mean value $x =
\mu$, given as $f(\mu) = \tfrac{1}{\sqrt{2\pi\sigma^2}}$, which
is inversely proportional to the standard deviation $\sigma$ of the distribution.
\end{frame}




\def\mpsGauss#1#2#3#4{
    \psset{plotpoints=200}
    \psplot[]{#3}{#4}{%
    Euler x #1 sub dup mul 2 div #2 dup mul div neg exp
    1.0 #2 div TwoPi sqrt div mul%
    }%
}
\begin{frame}{Normal Distribution: $\mu=0$, and Different Variances}
    \centering
    \psset{xAxisLabel=$x$,yAxisLabel=$f(x)$}
    \psgraph[Dy=0.1,Ox=-6]{->}(-6,0)(6,0.85){4in}{2.5in}%
    \psline[linestyle=dotted](0,0)(0,0.798)
    \psset{linewidth=2pt}
    \mpsGauss{0}{1}{-6}{6}
    \psset{linewidth=1pt,linestyle=dashed}
    \mpsGauss{0}{2}{-6}{6}
    \psset{linewidth=1pt,linestyle=solid}
    \mpsGauss{0}{0.5}{-6}{6}
    \uput[ur](1.2,0.25){$\sigma=1$}
    \uput[ur](3.5,0.07){$\sigma=2$}
    \uput[ur](0.5,0.6){$\sigma=0.5$}
    \endpsgraph
\end{frame}


\begin{frame}{Multivariate Normal Distribution}
Given the $d$-dimensional vector random
variable $\bX = (X_1, X_2, \ldots,
X_d)^T$, it has
a multivariate
normal distribution, with the parameters
mean $\bmu$ and covariance matrix $\cov$,
if its joint multivariate probability density function
is given as follows:
\begin{empheq}[box=\tcbhighmath]{align*}
  f(\bx | \bmu, \cov) = \frac{1}{(\sqrt{2\pi})^d\;\sqrt{|\cov|}}
  \;\;\exp\lB\{-\frac{(\bx-\bmu)^T\;\cov^{-1}\;(\bx-\bmu)}{2}\rB\}
\end{empheq}
where $|\cov|$ is the determinant of the covariance matrix.

\medskip
The term
\begin{align*}
(\bx_i-\bmu)^T\;\cov^{-1}\;(\bx_i-\bmu)
\end{align*}
measures the distance, called the {\em Mahalanobis
distance}
of the point $\bx$
from the mean $\bmu$ of the distribution, taking into account all of the
variance--covariance information between the attributes.
\end{frame}


\def\mye{2.7183}
        \def\mypi{3.1416}
        \def\msigma#1{(sqrt(#1))}%
        \def\rho#1#2#3{((#3)/(\msigma{#1}*\msigma{#2}))}%sigma1, sigma2, sigma12
        \def\ccA#1#2#3{((1.0)/(2*\mypi*\msigma{#1}*\msigma{#2}*sqrt(1-\rho{#1}{#2
}{#3}^2)))} %sigma1, sigma2,rho
        \def\ccB#1#2#3{((-1.0)/(2*(1-(\rho{#1}{#2}{#3})^2)))}%rho
        \def\fA#1#2#3#4#5{(\ccA{#3}{#4}{#5})*\mye^(
        (\ccB{#3}{#4}{#5})*(((x-(#1))/(\msigma{#3}))^2 +
        ((y-(#2))/(\msigma{#4}))^2 -
        (2*(\rho{#3}{#4}{#5})*(x-(#1))*(y-(#2)))/(
        (\msigma{#3})*(\msigma{#4})) ))}%
\begin{frame}{Standard Bivariate Normal Density}
\psset{unit=0.35in}
\psset{viewpoint=50 30 40 rtp2xyz,Decran=60}
\psset{lightsrc=viewpoint,opacity=0.8,incolor=white}
\begin{pspicture}(-5,-14)(4,6)
\scalebox{0.5}{%
\psSurface[ngrid=0.25 0.25, linewidth=0.1pt,
linecolor=lightgray,
intersectionplan ={[0 0 1 -0.01] [0 0 1 -0.1] [0 0 1 -0.75] [0 0 1 -2]},
intersectionlinewidth =1 1 1 1,
intersectioncolor=(Black) (Black) (Black) (Black),
intersectiontype=0,
transform={1 1 15 scaleOpoint3d},algebraic](-4,-4)(4,4){
  \fA{0}{0}{1}{1}{0}
  }
\axesIIID[axisnames={X_1,X_2,f(\bx)}](-4.1,-4.1,0)(5,5,4)
\multido{\ix=-4+1}{9}{%
        \psPoint(\ix\space,4,0){X1}
        \psPoint(\ix\space,4.2,0){X2}
        \psline(X1)(X2)\uput[dr](X1){$\ix$}}
\multido{\iy=-4+1}{9}{%
        \psPoint(4,\iy\space,0){Y1}
        \psPoint(4.2,\iy\space,0){Y2}
        \psline(Y1)(Y2)\uput[dl](Y1){$\iy$}}
\multido{\nz=0+1,\nl=0+0.07}{4}{%
        \psPoint(0,0,\nz\space){Z1}
        \psPoint(0,-0.2,\nz\space){Z2}
        \psline(Z1)(Z2)\uput[l](Z1){\nl}}
  \psdot[dotsize=0.2](0,0,0)
  }%

\rput[tr](1.0,4.5){
Parameters: $\bmu = \matr{0\\0}$, $\cov=\matr{1 & 0\\0& 1}$
}

\rput[tr](5.0,3.5){%
\scalebox{0.4}{%
\psset{unit=0.3in}
\psset{viewpoint=50 30 90 rtp2xyz,Decran=60}
\psset{lightsrc=viewpoint,opacity=0.9,incolor=white}
%%\begin{pspicture}(-9,-5)(7,6)
\psSolid[object=plan,definition=equation, args={[0 0 1 0]},
  base = -4 4 -4 4]
\psSurface[ngrid=0.2 0.2, linewidth=0.0pt,
linecolor=lightgray,
intersectionplan ={[0 0 1 -0.01] [0 0 1 -0.1] [0 0 1 -0.75] [0 0 1 -2]},
intersectionlinewidth =1 1 1 1,
intersectioncolor=(Black) (Black) (Black) (Black),
intersectiontype=0,
transform={1 1 15 scaleOpoint3d},algebraic](-4,-4)(4,4){
  \fA{0}{0}{1}{1}{0}
  }
\axesIIID[arrows=-,axisnames={X_1,X_2}](-4,-4,0)(4,4,0)
\multido{\ix=-4+1}{9}{%
        \psPoint(\ix\space,-4,0){X1}
        \psPoint(\ix\space,-4.1,0){X2}
        \psline(X1)(X2)\uput[l](X1){$\ix$}
        }
\multido{\iy=-4+1}{9}{%
        \psPoint(-4,\iy\space,0){Y1}
        \psPoint(-4.1,\iy\space,0){Y2}
        \psline(Y1)(Y2)\uput[u](Y1){$\iy$}
        }
  \psdot[dotsize=0.2](0,0,0)
  \rput[tr]{-45}(1.25,0.5){$0.13$}
  \rput[tr]{-45}(2,1.25){$0.05$}
  \rput[tr]{-45}(3,2){$0.007$}
  \rput[tr]{-45}(3.75,2.5){$0.0007$}
}}
\end{pspicture}
\end{frame}



\begin{frame}{Geometry of the Multivariate Normal}
  \small
  Compared to the standard multivariate normal,
  the mean $\bmu$ translates the center of the distribution, whereas
  the covariance matrix $\cov$ scales and rotates the distribution.
The 
eigen-decomposition of $\cov$ is given as
\begin{align*}
    \cov \bu_i = \lambda_i \bu_i
\end{align*}
where $\lambda_1 \ge \lambda_2 \ge \dots \lambda_d \ge 0$ are the
eigenvalues and $\bu_i$ the corresponding eigenvectors.
This can be expressed compactly
as follows:
\begin{align*}
    \cov = \bU \bLambda \bU^T
\end{align*}
where 
\begin{align*}
    \bLambda & =
\matr{
      \lambda_1 & 0 & \cdots & 0 \\
      0 & \lambda_2 & \cdots & 0\\
      \vdots & \vdots & \ddots & \vdots\\
      0 & 0 & \cdots & \lambda_d \\
  } &
\bU & = \matr{
    | & | &  & | \\
    \bu_1 & \bu_2 & \cdots & \bu_d\\
    | & | &  & | \\
}
\end{align*}
The eigenvectors represent the new basis vectors, with the covariance
matrix given by $\bLambda$ (all covariances
become zero).
Since the trace of a square matrix is invariant to
similarity transformation, such as a change of basis, we have
\begin{align*}
  var(\bD) = tr(\cov) =
  \sum_{i=1}^d \sigma^2_i = \sum_{i=1}^d
  \lambda_i = tr(\bLambda)
\end{align*}

\end{frame}


\def\mye{2.7183}
        \def\mypi{3.1416}
        \def\msigma#1{(sqrt(#1))}%
        \def\rho#1#2#3{((#3)/(\msigma{#1}*\msigma{#2}))}%sigma1, sigma2, sigma12
        \def\ccA#1#2#3{((1.0)/(2*\mypi*\msigma{#1}*\msigma{#2}*sqrt(1-\rho{#1}{#2
}{#3}^2)))} %sigma1, sigma2,rho
        \def\ccB#1#2#3{((-1.0)/(2*(1-(\rho{#1}{#2}{#3})^2)))}%rho
        \def\fA#1#2#3#4#5{(\ccA{#3}{#4}{#5})*\mye^(
        (\ccB{#3}{#4}{#5})*(((x-(#1))/(\msigma{#3}))^2 +
        ((y-(#2))/(\msigma{#4}))^2 -
        (2*(\rho{#3}{#4}{#5})*(x-(#1))*(y-(#2)))/(
        (\msigma{#3})*(\msigma{#4})) ))}%

\begin{frame}{Bivariate Normal for Iris: {\tt sepal length} and {\tt sepal width}}
  \small
  \begin{columns}
	\column{0.3\textwidth}
    \begin{align*}
	  \hbmu & = \matr{5.843\\3.054}\\ 
	  \hcov & = \amatr{r}{0.681 & -0.039\\ -0.039 & 0.187}
    \end{align*}

We have
	\begin{align*}
    \hcov & = \bU\bLambda\bU^T\\
\bU & = 
	\matr{-0.997 & -0.078\\
		  0.078 & -0.997}\\
  \bLambda & = 
	\matr{0.684 & 0\\0 & 0.184}
\end{align*}
Angle of rotation is:\\
    $\cos \theta = \be_1^T\bu_1 = -0.997$\\
or $\theta = 175.5^\circ$

\column{0.65\textwidth}
\centerline{
\scalebox{0.9}{%
\psset{unit=0.5in}
%%\psset{viewpoint=5.843 3.054 50,Decran=50}
\psset{viewpoint=50 -50 50,Decran=70}
\psset{lightsrc=viewpoint}
\begin{pspicture}(2,-3.25)(7,1.75)
\psPoint(2,1,0){O2}
\psPoint(2,5.5,0){X2}
\psPoint(9.5,1,0){Y2}
\psPoint(2,1,2){Z2}
\psline[arrows=->,arrowscale=2](O2)(X2)
\psline[arrows=->,arrowscale=2](O2)(Y2)
\psline[arrows=->,arrowscale=2](O2)(Z2)
\uput[dr](Y2){$X_1$}
\uput[ur](X2){$X_2$}
\uput[u](Z2){$f(\bx)$}
\psset{dotsize=0.1,dotstyle=Bo,fillcolor=gray}
\input{./EDA/numeric/figs/iris-slw3d}
\psset{fillcolor=white, incolor=white, opacity=0.5}
\psSurface[ngrid=.25 .25, linewidth=0.01pt,
linecolor=lightgray,
intersectionplan ={[0 0 1 -0.01] [0 0 1 -0.1] [0 0 1 -0.75] [0 0 1 -1.5]},
intersectionlinewidth =1 1 1 1,
intersectioncolor=(Black) (Black) (Black) (Black),
intersectiontype=0,QZ=4,
transform={1 1 5 scaleOpoint3d},algebraic](2,1)(9,5){
  \fA{5.843}{3.054}{0.681}{0.187}{-0.039}
  }
%%\axesIIID[arrows=->,axisnames={X_1,X_2,f(\bx)}](2,1,0)(10,6,2)
\multido{\ix=2+1}{8}{%
        \psPoint(\ix\space,0.8,0){X1}
        \psPoint(\ix\space,1.0,0){X2}
        \psline(X1)(X2)\uput[dl](X1){\small \ix}}
\multido{\iy=1+1}{5}{%
        \psPoint(1.8,\iy\space,0){Y1}
        \psPoint(2,\iy\space,0){Y2}
        \psline(Y1)(Y2)\uput[ul](Y1){\small \iy}}
  \psPoint(5.843,3.054,0){O1}
  \psdot[dotsize=0.2,dotstyle=Bo,fillcolor=gray](O1)
  \psPoint(2,3.355,0){u1a}
  \psPoint(9,2.806,0){u1b}
  \psset{arrowscale=1.5}
  \psline[linewidth=1.5pt,arrows=->](u1a)(u1b)
  \psPoint(5.68,1,0){u2a}
  \psPoint(6,5,0){u2b}
  \psline[linewidth=1.5pt,arrows=->](u2a)(u2b)
  \uput[dr](u1b){$\bu_1$}
  \uput[ur](u2b){$\bu_2$}
\end{pspicture}
}}
\end{columns}
\end{frame}
