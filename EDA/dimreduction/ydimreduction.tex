\lecture{dimreduction}{dimreduction}

\date{Chapter 7: Dimensionality Reduction}

\begin{frame}
\titlepage
\end{frame}


\begin{frame}{Dimensionality Reduction}
  The goal of dimensionality reduction is to find a lower dimensional
  representation of the data matrix $\bD$ to avoid the curse
  of dimensionality.

\medskip
Given $n \times d$ data matrix, 
each
point $\bx_i =
(x_{i1}, x_{i2}, \ldots, x_{id})^T$ is a vector in the
ambient
$d$-dimensional vector space spanned by the $d$ standard basis
vectors $\be_1, \be_2, \ldots, \be_d$.

\medskip
Given any other set of $d$
orthonormal vectors $\bu_1, \bu_2, \ldots, \bu_d$
we can
re-express each point $\bx$ as
\begin{align*}
\tcbhighmath{
    \bx = a_1 \bu_1 + a_2 \bu_2 + \cdots + a_d \bu_d
}
\end{align*}
where $\ba = (a_1, a_2, \ldots, a_d)^T$ represents the
coordinates of $\bx$ in the new basis. 
More compactly:
\begin{align*}
\tcbhighmath{
    \bx = \bU \ba
}
\end{align*}
where $\bU$ is the $d \times d$ orthogonal matrix, 
whose $i$th column comprises the $i$th basis vector $\bu_i$.
Thus $\bU^{-1} = \bU^T$, and we have
\begin{empheq}[box=\tcbhighmath]{align*}
\begin{split}
    \ba &= \bU^T \bx
\end{split}
\end{empheq}
\end{frame}
%


\begin{frame}{Optimal Basis: Projection in Lower Dimensional Space}

There are potentially inf\/{i}nite choices for the
 orthonormal
basis vectors. Our goal is to choose an 
{\em optimal} basis that preserves essential information about $\bD$.

\medskip
We are interested in f\/{i}nding the
optimal $r$-dimensional representation
of $\bD$, with $r \ll d$.
Projection of $\bx$ onto the f\/{i}rst $r$
basis vectors is given as
\begin{align*}
    \bx' & = a_1 \bu_1 + a_2 \bu_2 + \cdots + a_r \bu_r =
    \sum_{i=1}^r a_i \bu_i
	= \bU_r \ba_r
\end{align*}
where $\bU_r$ and $\ba_r$ comprises the $r$ basis vectors and
coordinates, respv. 
Also, restricting $\ba = \bU^T \bx$ to $r$ terms, we have
\begin{align*}	
\ba_r & = \bU_r^T \bx
\end{align*}


\medskip
The $r$-dimensional projection of $\bx$ is thus given as:
\begin{align*}
\tcbhighmath{
    \bx' = \bU_r\bU_r^T \bx = \bP_r \bx
}
\end{align*}
where $\bP_r = \bU_r \bU_r^T = \sum_{i=1}^r \bu_i \bu_i^T$ is the {\em
orthogonal projection
matrix} for the subspace spanned by the f\/{i}rst $r$ basis vectors.

\end{frame}


\begin{frame}{Optimal Basis: Error Vector}

Given the projected vector $\bx' = \bP_r \bx$, the corresponding
{\em error vector}, is the projection onto the remaining $d-r$ basis
vectors
\begin{align*}
\tcbhighmath{
    \bepsilon = \sum_{i=r+1}^d a_i \bu_i = \bx - \bx'
}
\end{align*}
The error vector $\bepsilon$ is orthogonal to $\bx'$.


\medskip
The goal of dimensionality reduction is to
seek an $r$-dimensional basis that gives the best
possible approximation $\bx_i'$
over all the points $\bx_i \in \bD$.
Alternatively, we seek to minimize the error $\bepsilon_i =
\bx_i - \bx_i'$ over all the points.
\end{frame}



\readdata{\dataSLWPL}{EDA/dimreduction/figs/iris-slwpl-c.dat}

\begin{frame}{Iris Data: Optimal One-dimensional Basis}

  \begin{center}
  \begin{tabular}{ccc}
\psset{unit=0.5in}
\psset{arrowscale=2}
\psset{Alpha=60,Beta=-30}
%\psset{Alpha=-190,Beta=-45}
\psset{nameX=$~$, nameY=$~$, nameZ=$~$}
%\subfloat[Original Basis]{
%\label{fig:eda:dimr:3dirisOrig}
\scalebox{0.55}{
\begin{pspicture}(-2,-4.5)(2,4.5)
\pstThreeDCoor[xMin=-2, xMax= 2, yMin=-2,
        yMax=2, zMin=-3, zMax=4, Dx=0.5, Dy=0.5, Dz=1,
        linewidth=2pt,linecolor=black]
\pstThreeDPut(2.3,0,0){$X_1$}
\pstThreeDPut(0,2.5,0){$X_2$}
\pstThreeDPut(0,0,4.3){$X_3$}
\psset{dotstyle=Bo,dotscale=1.75,fillcolor=lightgray}
\dataplotThreeD[plotstyle=dots,showpoints=true]{\dataSLWPL}
\pstThreeDBox[linecolor=gray](-2,-2,-3)(4,0,0)(0,4,0)(0,0,7)
\end{pspicture}
}%}
&
\hspace{1in}
&
\psset{unit=0.5in}
\psset{arrowscale=2}
%\psset{Alpha=-190,Beta=-45}
\psset{Alpha=60,Beta=-30}
\psset{nameX=$~$, nameY=$~$, nameZ=$~$}
\scalebox{0.55}{%
\begin{pspicture}(-2,-4.5)(2,4.5)
\pstThreeDCoor[xMin=-2, xMax= 2, yMin=-2,
        yMax=2, zMin=-3, zMax=4, Dx=0.5, Dy=0.5, Dz=1,
        linewidth=1pt, linecolor=black]
\pstThreeDPut(2.3,0,0){$X_1$}
\pstThreeDPut(0,2.5,0){$X_2$}
\pstThreeDPut(0,0,4.3){$X_3$}
\input{EDA/dimreduction/figs/iris-1dmap.tex}
\psset{dotstyle=Bo,dotscale=1.75,fillcolor=lightgray}
\dataplotThreeD[plotstyle=dots,showpoints=true]{\dataSLWPL}
\pstThreeDLine[linewidth=2pt,arrows=->](1.43,-0.32,3.36)(-1.43,0.32,-3.36)
\pstThreeDPut(-1.6,0.35,-3.7){$\bu_1$}
%\psset{dotstyle=Bo,dotscale=1.75,fillcolor=white}
%\dataplotThreeD[plotstyle=dots,showpoints=true]{\dataIrisOneD}
\pstThreeDBox[linecolor=gray](-2,-2,-3)(4,0,0)(0,4,0)(0,0,7)
\end{pspicture}
}\\
Iris Data: 3D & & Optimal 1D Basis\\
\end{tabular}
\end{center}
\end{frame}



\readdata{\dataIrisDL}{EDA/dimreduction/figs/iris-2dL.dat}
\readdata{\dataIrisDR}{EDA/dimreduction/figs/iris-2dR.dat}
\begin{frame}{Iris Data: Optimal 2D Basis}
\psset{unit=0.5in}
\psset{arrowscale=2}
\psset{Alpha=60,Beta=-30}
  \begin{center}
  \begin{tabular}{ccc}
%\psset{Alpha=-190,Beta=-45}
\psset{nameX=$~$, nameY=$~$, nameZ=$~$}
%\subfloat[Original Basis]{
%\label{fig:eda:dimr:3dirisOrig}
\scalebox{0.6}{
\begin{pspicture}(-2,-4.5)(2,4.5)
\pstThreeDCoor[xMin=-2, xMax= 2, yMin=-2,
        yMax=2, zMin=-3, zMax=4, Dx=0.5, Dy=0.5, Dz=1,
        linewidth=2pt,linecolor=black]
\pstThreeDPut(2.3,0,0){$X_1$}
\pstThreeDPut(0,2.5,0){$X_2$}
\pstThreeDPut(0,0,4.3){$X_3$}
\psset{dotstyle=Bo,dotscale=1.75,fillcolor=lightgray}
\dataplotThreeD[plotstyle=dots,showpoints=true]{\dataSLWPL}
\pstThreeDBox[linecolor=gray](-2,-2,-3)(4,0,0)(0,4,0)(0,0,7)
\end{pspicture}
}%}
&
\hspace{1in}
&
\psset{unit=0.5in}
\psset{arrowscale=2,dotscale=1.75}
\psset{Alpha=60,Beta=-30}
\psset{nameX=$~$, nameY=$~$, nameZ=$~$}
%\subfloat[Optimal basis]{
%\label{fig:eda:dimr:2dirisA}
\scalebox{0.55}{%
\begin{pspicture}(-2,-4.5)(2,4.5)
%\pstThreeDSquare[fillstyle=solid, fillcolor=lightgray]
\pstThreeDSquare[fillstyle=vlines,
    hatchcolor=lightgray, hatchwidth=0.1\pslinewidth,
    hatchsep=3\pslinewidth]
(3.99,2.64,2.56)(-2.86,0.64,-6.72)(-5.12,-5.92, 1.6)
\psset{dotstyle=Bo,fillcolor=gray}
\dataplotThreeD[plotstyle=dots,showpoints=true]{\dataIrisDR}
%\pstThreeDSquare[
%fillstyle=solid, fillcolor=lightgray]
%(2.71,1.16,2.96)(-2.86,0.64,-6.72)(-2.56,-2.96, 0.8)
\pstThreeDCoor[xMin=-2, xMax= 2, yMin=-2,
        yMax=2, zMin=-3, zMax=4, Dx=0.5, Dy=0.5, Dz=1,
        linewidth=1pt,linecolor=black,arrowscale=2]
\pstThreeDPut(2.3,0,0){$X_1$}
\pstThreeDPut(0,2.5,0){$X_2$}
\pstThreeDPut(0,0,4.3){$X_3$}
\input{EDA/dimreduction/figs/iris-2dmap.tex}
\pstThreeDLine[linewidth=2pt,arrows=->](1.43,-0.32,3.36)(-1.43,0.32,-3.36)
\pstThreeDLine[linewidth=2pt,arrows=->](2.56,2.96,-0.8)(-2.56,-2.96,0.8)
\pstThreeDPut(-1.6,0.35,-3.7){$\bu_1$}
\pstThreeDPut(-2.88,-3.33,0.9){$\bu_2$}
\psset{dotstyle=Bo,fillcolor=white}
\dataplotThreeD[plotstyle=dots,showpoints=true]{\dataIrisDL}
\pstThreeDBox[linecolor=gray](-2,-2,-3)(4,0,0)(0,4,0)(0,0,7)
%\pstThreeDBox[](-3.5,-2.5,-3)(5,0,0)(0,4,0)(0,0,7.5)
\end{pspicture}
}\\
Iris Data (3D) & & Optimal 2D Basis\\
\end{tabular}
\end{center}
\end{frame}

\ifdefined\wox \begin{frame} \titlepage \end{frame} \fi

\begin{frame}{Principal Component Analysis}

Principal Component Analysis (PCA) is a technique that
seeks a $r$-dimensional basis that best captures the
variance in the data.

\medskip
The direction with the largest projected variance is called
the f\/{i}rst principal component.

\medskip
The orthogonal direction that captures
the second largest projected variance is called the second principal component,
and so on. 

\medskip
The direction that maximizes the
variance is also the one that minimizes the mean squared
error.
\end{frame}





\begin{frame}{Principal Component: Direction of Most Variance}
We seek to find the unit vector $\bu$ that maximizes the projected
variance of the points. Let $\bD$ be centered, and let $\cov$ be its
covariance matrix.

\medskip
The projection of $\bx_i$ on $\bu$ is given as
\begin{align*}
  \bx_i' =
  \lB(\frac{\bu^T \bx_i}{\bu^T\bu}\rB) \bu = (\bu^T \bx_i) \bu =
  a_i \bu
\end{align*}

\medskip
Across all the points, the 
projected variance along $\bu$ is
\begin{align*}
\tcbhighmath{
  \sigma^2_\bu  
  = \frac{1}{n}\sum_{i=1}^n (a_i-\mu_\bu)^2 
  = \frac{1}{n} \sum_{i=1}^n \bu^T \lB(\bx_{i}\bx_i^T\rB) \bu
   =  \bu^T \lB(\frac{1}{n}\sum_{i=1}^n \bx_i\bx_i^T\rB) \bu
  =  \bu^T \cov \bu
}
\end{align*}


\medskip
We have to find the
optimal basis vector $\bu$ that maximizes the projected variance 
$\sigma^2_\bu = \bu^T \cov \bu$, subject to the 
constraint that $\bu^T\bu=1$. The maximization objective is given as
\begin{align*}
  \max_\bu J(\bu) = 
  \bu^T \cov \bu - \alpha (\bu^T\bu-1)
\end{align*}

\end{frame}


\begin{frame}{Principal Component: Direction of Most Variance}

  Given the objective $\max_\bu J(\bu) =  
  \bu^T \cov \bu - \alpha (\bu^T\bu-1)$, we solve it by
setting the derivative of $J(\bu)$
with respect to $\bu$ to the zero vector, to obtain
\begin{align*}
  \frac{\partial}{\partial \bu} \lB(\bu^T \cov \bu - \alpha
  (\bu^T\bu-1)\rB) & = \bzero\\
  \text{that is, }2 \cov \bu - 2 \alpha \bu  &= \bzero \\
  \text{which implies } 
\tcbhighmath{
\cov \bu  = \alpha \bu
}
\end{align*}
Thus $\alpha$ is an eigenvalue of the covariance matrix
$\cov$, with the associated eigenvector $\bu$.

\medskip
Taking the dot product with $\bu$ on both sides, we have
\begin{align*}
\sigma^2_\bu  = \bu^T\cov\bu  \bu^T \alpha \bu = 
\alpha \bu^T\bu =  \alpha
\end{align*}

\medskip
To maximize the projected variance $\sigma_\bu^2$,
we thus choose the largest eigenvalue $\lambda_1$ of $\cov$, and
the dominant eigenvector $\bu_1$ specif\/{i}es the
direction of most variance, also called the
{\em f\/{i}rst principal
component}.
\end{frame}


\begin{frame}{Iris Data: First Principal Component}
  \centerline{
\psset{unit=0.5in}
\psset{arrowscale=2}
%\psset{Alpha=-190,Beta=-45}
\psset{Alpha=60,Beta=-30}
\psset{nameX=$~$, nameY=$~$, nameZ=$~$}
\scalebox{0.65}{%
\begin{pspicture}(-2,-4.5)(2,4.5)
\pstThreeDCoor[xMin=-2, xMax= 2, yMin=-2,
        yMax=2, zMin=-3, zMax=4, Dx=0.5, Dy=0.5, Dz=1,
        linewidth=1pt, linecolor=black]
\pstThreeDPut(2.3,0,0){$X_1$}
\pstThreeDPut(0,2.5,0){$X_2$}
\pstThreeDPut(0,0,4.3){$X_3$}
\input{EDA/dimreduction/figs/iris-1dmap.tex}
\psset{dotstyle=Bo,dotscale=1.75,fillcolor=lightgray}
\dataplotThreeD[plotstyle=dots,showpoints=true]{\dataSLWPL}
\pstThreeDLine[linewidth=2pt,arrows=->](1.43,-0.32,3.36)(-1.43,0.32,-3.36)
\pstThreeDPut(-1.6,0.35,-3.7){$\bu_1$}
%\psset{dotstyle=Bo,dotscale=1.75,fillcolor=white}
%\dataplotThreeD[plotstyle=dots,showpoints=true]{\dataIrisOneD}
\pstThreeDBox[linecolor=gray](-2,-2,-3)(4,0,0)(0,4,0)(0,0,7)
\end{pspicture}
}}
\end{frame}



\begin{frame}{Minimum Squared Error Approach}
The direction that
maximizes the projected variance is also the
one that minimizes the average squared error.
The mean squared error ({\it MSE}) optimization condition is 
\begin{empheq}[box=\tcbhighmath]{align*}
\begin{split}
    MSE(\bu) & =  {1 \over n} \sum_{i=1}^n \norm{\bepsilon_i}^2
     =  {1 \over n} \sum_{i=1}^n \| \bx_i - \bx'_i \|^2 
	 =\sum_{i=1}^n \frac{\| \bx_i\|^2}{n}-\bu^T \cov \bu
\end{split}
\end{empheq}
Since the first term is fixed for a dataset $\bD$, we see that the
direction $\bu_1$ that maximizes the
variance is also the one that minimizes the MSE.
Further,
\begin{empheq}[box=\tcbhighmath]{align*}
\begin{split}
    \sum_{i=1}^n \frac{\| \bx_i\|^2}{n}-\bu^T \cov \bu = var(\bD) = 
	tr(\cov) = \sum_{i=1}^d \sigma_i^2
\end{split}
\end{empheq}

\medskip
Thus, for the direction $\bu_1$ that minimizes MSE, we have
\begin{empheq}[box=\tcbhighmath]{align*}
\begin{split}
    MSE(\bu_1) = var(\bD) - \bu_1^T\cov \bu_1 = var(\bD) - \lambda_1 
\end{split}
\end{empheq}
\end{frame}


\begin{frame}{Best 2-dimensional Approximation}

The best 2D subspace that captures the most variance in $\bD$ comprises
the eigenvectors $\bu_1$ and $\bu_2$ corresponding to the largest and
second largest eigenvalues $\lambda_1$ and $\lambda_2$, respv.

\medskip
Let $\bU_2 = \matr{\bu_1 & \bu_2}$ 
be the matrix whose columns correspond to the two principal components.
Given the point $\bx_i \in \bD$ its projected coordinates are computed as follows:
\begin{align*}
    \ba_i = \bU_2^T \bx_i
\end{align*}
Let $\bA$ denote the projected 2D dataset.
The total projected variance for $\bA$ is given as
\begin{align*}
    var(\bA)  = \bu_1^T\cov\bu_1  + \bu_2^T\cov \bu_2 =
	\bu_1^T\lambda_1\bu_1 + \bu_2^T\lambda_2\bu_2 =
    \lambda_1 + \lambda_2
\end{align*}

The first two principal components also
minimize the mean square error objective, since
\begin{align*}
    MSE & =  {1 \over n} \sum_{i=1}^n
    \norm{\bx_i-\bx_i'}^2
	= var(\bD) - {1\over n} \sum_{i=1}^n \lB(\bx_i^T \bP_2 \bx_i\rB)
	= var(\bD) - var(\bA)
\end{align*}
\end{frame}


\readdata{\dataIrisDLno}{EDA/dimreduction/figs/iris-2dLnonopt.dat}
\readdata{\dataIrisDRno}{EDA/dimreduction/figs/iris-2dRnonopt.dat}
\begin{frame}{Optimal and Non-optimal 2D Approximations}

The optimal subspace
    maximizes the variance, and minimizes the squared error,
    whereas the nonoptimal subspace captures less variance, and
     has a high mean squared error value, as seen from the 
	 lengths of the error vectors (line  segments).

\psset{unit=0.5in}
\psset{arrowscale=2,dotscale=1.75}
\psset{Alpha=60,Beta=-30}
\psset{nameX=$~$, nameY=$~$, nameZ=$~$}
\centerline{
%\subfloat[Optimal basis]{
%\label{fig:eda:dimr:2dirisA}
\scalebox{0.55}{%
\begin{pspicture}(-2,-4.5)(2,4.5)
%\pstThreeDSquare[fillstyle=solid, fillcolor=lightgray]
\pstThreeDSquare[fillstyle=vlines,
    hatchcolor=lightgray, hatchwidth=0.1\pslinewidth,
    hatchsep=3\pslinewidth]
(3.99,2.64,2.56)(-2.86,0.64,-6.72)(-5.12,-5.92, 1.6)
\psset{dotstyle=Bo,fillcolor=gray}
\dataplotThreeD[plotstyle=dots,showpoints=true]{\dataIrisDR}
%\pstThreeDSquare[
%fillstyle=solid, fillcolor=lightgray]
%(2.71,1.16,2.96)(-2.86,0.64,-6.72)(-2.56,-2.96, 0.8)
\pstThreeDCoor[xMin=-2, xMax= 2, yMin=-2,
        yMax=2, zMin=-3, zMax=4, Dx=0.5, Dy=0.5, Dz=1,
        linewidth=1pt,linecolor=black,arrowscale=2]
\pstThreeDPut(2.3,0,0){$X_1$}
\pstThreeDPut(0,2.5,0){$X_2$}
\pstThreeDPut(0,0,4.3){$X_3$}
\input{EDA/dimreduction/figs/iris-2dmap.tex}
\pstThreeDLine[linewidth=2pt,arrows=->](1.43,-0.32,3.36)(-1.43,0.32,-3.36)
\pstThreeDLine[linewidth=2pt,arrows=->](2.56,2.96,-0.8)(-2.56,-2.96,0.8)
\pstThreeDPut(-1.6,0.35,-3.7){$\bu_1$}
\pstThreeDPut(-2.88,-3.33,0.9){$\bu_2$}
\psset{dotstyle=Bo,fillcolor=white}
\dataplotThreeD[plotstyle=dots,showpoints=true]{\dataIrisDL}
\pstThreeDBox[linecolor=gray](-2,-2,-3)(4,0,0)(0,4,0)(0,0,7)
%\pstThreeDBox[](-3.5,-2.5,-3)(5,0,0)(0,4,0)(0,0,7.5)
\end{pspicture}
}%}
\hspace{1in}
%\subfloat[Nonoptimal basis]{
%\label{fig:eda:dimr:2dirisB}
\scalebox{0.55}{%
\begin{pspicture}(-2,-4.5)(2,4.5)
\input{EDA/dimreduction/figs/iris-2dmapnonopt.tex}
\psset{dotstyle=Bo,fillcolor=gray}
\dataplotThreeD[plotstyle=dots,showpoints=true]{\dataIrisDLno}
%\pstThreeDSquare[fillstyle=solid, fillcolor=lightgray]
\pstThreeDSquare[fillstyle=vlines,
hatchcolor=lightgray,hatchwidth=0.1\pslinewidth,hatchsep=3\pslinewidth]
(3.55,1.97,-1.325)(-5.12,-5.92,1.6)(-1.98,1.98,1.05)
\pstThreeDCoor[xMin=-2, xMax= 2, yMin=-2,
        yMax=2, zMin=-3, zMax=4, Dx=0.5, Dy=0.5, Dz=1,
        linewidth=1pt,linecolor=black,arrowscale=2]
\pstThreeDPut(2.3,0,0){$X_1$}
\pstThreeDPut(0,2.5,0){$X_2$}
\pstThreeDPut(0,0,4.3){$X_3$}
\pstThreeDLine[linewidth=2pt,arrows=->](2.56,2.96,-0.8)(-2.56,-2.96,0.8)
\pstThreeDLine[linewidth=2pt,arrows=->](0.99,-0.99,-0.525)(-0.99,0.99,0.525)
\psset{dotstyle=Bo,fillcolor=white}
\dataplotThreeD[plotstyle=dots,showpoints=true]{\dataIrisDRno}
\pstThreeDBox[linecolor=gray](-2,-2,-3)(4,0,0)(0,4,0)(0,0,7)
%\pstThreeDBox[](-3.5,-2.5,-3)(5,0,0)(0,4,0)(0,0,7.5)
\end{pspicture}
}%}
}
\end{frame}



\begin{frame}{Best $\textbf{\textit{r}}$-dimensional Approximation}
  \small

To f\/{i}nd the best $r$-dimensional
approximation to $\bD$,
we compute the eigenvalues of $\cov$.
Because $\cov$ is positive
  semidef\/{i}nite, its eigenvalues are non-negative
\begin{align*}
  \lambda_{1} \geq \lambda_{2} \geq \cdots \lambda_{r}\geq
  \lambda_{r+1} \cdots \geq \lambda_{d} \ge 0
\end{align*}
We select the $r$ largest eigenvalues, and their
corresponding eigenvectors to form the best $r$-dimensional
approximation.

\medskip
{\bf Total Projected Variance:}
Let $\bU_r = \matr{\bu_1 & \cdots & \bu_r }$ be the $r$-dimensional
basis vector matrix, withe the 
projection matrix given as
$\bP_r = \bU_r\bU_r^T = \sum_{i=1}^r \bu_i \bu_i^T$.

\medskip
Let $\bA$ denote the dataset formed by the coordinates of the
projected points in the \hbox{$r$-dimensional} subspace.
The projected variance is given as
\begin{align*}
    var(\bA) = {1\over n} \sum_{i=1}^n \bx_i^T\bP_r\bx_i =
    \sum_{i=1}^r \bu_i^T \cov \bu_i = \sum_{i=1}^r \lambda_i
\end{align*}

\medskip
{\bf Mean Squared Error:}
The mean squared error in $r$~dimensions is
\begin{align*}
    MSE &= {1\over n} \sum_{i=1}^n \norm{\bx_i -
    \bx_i'}^2 = 
    var(\bD) - \sum_{i=1}^r \lambda_i = 
	\sum_{i=1}^d \lambda_i - \sum_{i=1}^r \lambda_i
\end{align*}
\end{frame}



\begin{frame}{Choosing the Dimensionality}

  One criteria for
choosing $r$ is to compute the fraction of the total variance
captured by the
f\/{i}rst $r$ principal components, computed as
\begin{align*}
\tcbhighmath{
  f(r) =
  \frac{\lambda_1+\lambda_2+\cdots+\lambda_r}
  {\lambda_1+\lambda_2+\cdots+\lambda_d}
  = \frac{\sum_{i=1}^r \lambda_i}{\sum_{i=1}^d \lambda_i}
  = \frac{\sum_{i=1}^r \lambda_i}{var(\bD)}
}
  \label{eq:eda:dimr:fracvar}
\end{align*}

Given a certain desired variance threshold, say $\alpha$,
starting from
the f\/{i}rst principal component, we keep on adding additional
components, and stop at the smallest value $r$,
for which $f(r) \geq \alpha$. In other words, we select the
fewest number of dimensions such that the subspace spanned by
those $r$ dimensions captures at least $\alpha$ fraction (say 0.9) 
of the
total variance.
\end{frame}


\newcommand{\PCA}{\textsc{PCA}}
\begin{frame}{Principal Component Analysis: Algorithm}
\begin{tightalgo}[H]{\textwidth-18pt}
\SetKwInOut{Algorithm}{\PCA\ ($\bD, \alpha$)}
\Algorithm{}
$\bmu = {1\over n}\sum_{i=1}^n \bx_i$ \tcp{compute mean}
$\bZ = \bD - \bone \cdot \bmu^T$ \tcp{center the data}
$\cov = {1\over n} \lB(\bZ^T \bZ\rB)$ \tcp{compute covariance matrix}
$(\lambda_1, \lambda_2, \ldots, \lambda_d) = \text{eigenvalues}(\cov)$ \tcp{compute eigenvalues}
$\bU = \matr{\bu_1 & \bu_2 & \cdots & \bu_d} = \text{eigenvectors}(\cov)$ \tcp{compute eigenvectors}
$f(r) = \frac{\sum_{i=1}^r \lambda_i}{\sum_{i=1}^d \lambda_i},\;\; 
\text{for all } r=1, 2, \ldots, d$ \tcp{fraction of total variance}
Choose smallest $r$ \ so that $f(r) \ge \alpha$ \tcp{choose dimensionality}
$\bU_r = \matr{\bu_1 & \bu_2 & \cdots & \bu_r}$ \tcp{reduced basis}
$\bA = \{\ba_i \mid \ba_i = \bU_r^T\bx_i, \text{for } i=1, \ldots, n\}$ \tcp{reduced dimensionality data}
\end{tightalgo}
\end{frame}

\begin{frame}{Iris Principal Components}
Covariance matrix:
    \begin{align*}
        \cov = \amatr{r}{
        0.681 &-0.039 &1.265\\
        -0.039& 0.187 &-0.320\\
        1.265 &-0.32 &3.092\\
        }
    \end{align*}
	
	\medskip
    The eigenvalues and eigenvectors of $\cov$
    \begin{alignat*}{3}
    \lambda_1 & = 3.662 &\qquad\lambda_2 &= 0.239 &\qquad\lambda_3 &= 0.059\\
        \bu_1 &= \amatr{r}{-0.390\\0.089\\-0.916} &\bu_2 &= \amatr{r}{-0.639\\-0.742\\0.200} &\bu_3 &= \amatr{r}{-0.663\\0.664\\0.346}
    \end{alignat*}

    The total variance is therefore $\lambda_1 + \lambda_2 +
    \lambda_3  = 3.662  +0.239 + 0.059 = 3.96$.

	The fraction of total variance for different values of $r$ is
    given as
    \begin{center}
    \vspace*{12pt plus2pt minus1pt}
    {\renewcommand{\arraystretch}{1.1}\begin{tabular}{|c||c|c|c|}
        \hline
        $r$ & 1 & 2 & 3\\
        \hline
        $f(r)$ & 0.925 & 0.985 & 1.0\\
        \hline
    \end{tabular}}
        \vspace*{12pt plus2pt minus1pt}
    \end{center}
	This $r=2$ PCs are need to capture $\alpha=0.95$ fraction of
	variance.

\end{frame}



\begin{frame}{Iris Data: Optimal 3D PC Basis}
\psset{unit=0.5in}
\psset{arrowscale=2}
\psset{Alpha=60,Beta=-30}
  \begin{center}
  \begin{tabular}{ccc}
%\psset{Alpha=-190,Beta=-45}
\psset{nameX=$~$, nameY=$~$, nameZ=$~$}
%\subfloat[Original Basis]{
%\label{fig:eda:dimr:3dirisOrig}
\scalebox{0.6}{
\begin{pspicture}(-2,-4.5)(2,4.5)
\pstThreeDCoor[xMin=-2, xMax= 2, yMin=-2,
        yMax=2, zMin=-3, zMax=4, Dx=0.5, Dy=0.5, Dz=1,
        linewidth=2pt,linecolor=black]
\pstThreeDPut(2.3,0,0){$X_1$}
\pstThreeDPut(0,2.5,0){$X_2$}
\pstThreeDPut(0,0,4.3){$X_3$}
\psset{dotstyle=Bo,dotscale=1.75,fillcolor=lightgray}
\dataplotThreeD[plotstyle=dots,showpoints=true]{\dataSLWPL}
\pstThreeDBox[linecolor=gray](-2,-2,-3)(4,0,0)(0,4,0)(0,0,7)
\end{pspicture}
}%}
&
\hspace{1in}
&
%\subfloat[Optimal Basis]{
%\label{fig:eda:dimr:3dirisOpt}
\scalebox{0.55}{
\begin{pspicture}(-2,-4.5)(2,4.5)
%\pstThreeDSquare[fillstyle=hlines, hatchangle=64,
%hatchcolor=gray,hatchwidth=0.1pt,hatchsep=3pt]
%(3.37,0.5,2.61)(-2.86,0.64,-6.72)(-2.56,-2.96, 0.8)
%\pstThreeDSquare[fillstyle=hlines, hatchangle=32,
%hatchcolor=gray,hatchwidth=0.1pt,hatchsep=3pt]
%(3.37,0.5,2.61)(-2.86,0.64,-6.72)(-1.32,1.32,0.7)
%\pstThreeDSquare[fillstyle=hlines, hatchangle=32,
%hatchcolor=gray,hatchwidth=0.1pt,hatchsep=3pt]
%(3.37,0.5,2.61)(-2.56,-2.96,0.8)(-1.32,1.32,0.7)
%\pstThreeDCoor[xMin=-2, xMax= 2, yMin=-2,
        %yMax=2, zMin=-3, zMax=4, Dx=0.5, Dy=0.5, Dz=1,
        %linewidth=1pt, linecolor=gray]
%\pstThreeDPut(2.3,0,0){$X_1$}
%\pstThreeDPut(0,2.5,0){$X_2$}
%\pstThreeDPut(0,0,4.3){$X_3$}
\psset{dotstyle=Bo,dotscale=1.75,fillcolor=lightgray}
\dataplotThreeD[plotstyle=dots,showpoints=true]{\dataSLWPL}
%\pstThreeDBox[](3.37,0.5,2.61)(-2.86,0.64,-6.72)(-2.56,-2.96,0.8)(-1.32,1.32,0.7)
\pstThreeDBox[](4.01,1.24,2.41)(-2.86,0.64,-6.72)(-3.84,-4.44,1.2)(-1.32,1.32,0.7)
\pstThreeDLine[linewidth=2pt,arrows=->](1.43,-0.32,3.36)(-1.43,0.32,-3.36)
%\pstThreeDLine[linewidth=2pt,arrows=->](1.28,1.48,-0.4)(-1.28,-1.48,0.4)
\pstThreeDLine[linewidth=2pt,arrows=->](0.66,-0.66,-0.35)(-0.66,0.66,0.35)
\pstThreeDPut(-1.6,0.35,-3.6){$\bu_1$}
%\pstThreeDPut(-1.4,-1.63,0.44){$\bu_2$}
\pstThreeDPut(-0.79,0.79,0.37){$\bu_3$}
%\pstThreeDLine[linewidth=2pt,arrows=->](2.56,2.96,-0.8)(-2.56,-2.96,0.8)
\pstThreeDLine[linewidth=2pt,arrows=->](1.92,2.22,-0.6)(-1.92,-2.22,0.6)
\pstThreeDPut(-2.05,-2.37,0.64){$\bu_2$}
\pstThreeDBox[linecolor=gray](-2,-2,-3)(4,0,0)(0,4,0)(0,0,7)
\end{pspicture}
}\\
Iris Data (3D) & & Optimal 3D Basis\\
\end{tabular}
\end{center}
\end{frame}



\readdata{\dataA}{EDA/dimreduction/figs/iris-2dproj-dimr.dat}
\begin{frame}{Iris Principal Components: Projected Data (2D)}

\centering
%\vspace{0.1in}
\scalebox{0.9}{
\psset{dotstyle=Bo,dotscale=1.5,fillcolor=lightgray,arrowscale=2,PointName=none}
\psset{xAxisLabel=$\bu_1$,yAxisLabel= $\bu_2$}
\psgraph[Dx=1,Dy=0.5,Ox=-4,Oy=-1.5]{->}(-4,-1.5)(4.1,1.6){4in}{2.5in}%
\dataplot[plotstyle=dots,showpoints=true]{\dataA}
\endpsgraph
}

\end{frame}


\begin{frame}{Geometry of PCA}
  \small
Geometrically, when $r=d$,
PCA corresponds to a orthogonal change of basis, 
so that the total variance
is captured by the sum of the variances along each of the
principal directions $\bu_1, \bu_2, \ldots, \bu_d$, and
further, all covariances are zero.

\smallskip
Let $\bU$ be the $d
\times d$ orthogonal matrix
$\bU = \matr{\bu_1 & \bu_2 & \cdots & \bu_d}$,
with $\bU^{-1} = \bU^T$.
Let $\bLambda = diag(\lambda_1, \cdots, \lambda_d)$ be
the diagonal matrix of eigenvalues.
Each principal component $\bu_i$ corresponds
to an eigenvector of the covariance matrix $\cov$
\begin{align*}
  \cov \bu_i = \lambda_i \bu_i \mbox{ for all } 1 \le i \le d
\end{align*}
which can be written compactly in matrix notation:
\begin{align*}
  \cov\bU =  \bU \bLambda \text{ which implies }
  \cov =  \bU \bLambda \bU^T
\end{align*}
Thus, $\bLambda$ represents the covariance matrix in the new PC basis.


\smallskip
In the new PC basis, the equation
\begin{align*}
    \bx^T \cov^{-1} \bx = 1
\end{align*}
def\/{i}nes a $d$-dimensional ellipsoid (or hyper-ellipse). The
eigenvectors $\bu_i$ of $\cov$, that is, the principal components,
are the directions for the principal axes of the ellipsoid. The
square roots of the eigenvalues, that is, $\sqrt{\lambda_i}$, give
the lengths of the semi-axes.

\end{frame}



\readdata{\dataSLWPP}{EDA/dimreduction/figs/iris-slwpl-p.dat}
\begin{frame}{Iris: Elliptic Contours in Standard Basis}
\psset{unit=0.5in}
\psset{arrowscale=2}
\centerline{
%\subfloat[Elliptic contours in standard basis]{
%\label{fig:eda:dimr:3dellipsoidSB}
\scalebox{0.6}{
\psset{Alpha=60,Beta=-30}
\psset{nameX=$~$, nameY=$~$, nameZ=$~$}
\begin{pspicture}(-4,-4.5)(4,4.5)
\psset{dotstyle=Bo,dotscale=1.75,fillcolor=lightgray}
\dataplotThreeD[plotstyle=dots,showpoints=true]{\dataSLWPL}
\pstThreeDBox[](4.01,1.24,2.41)(-2.86,0.64,-6.72)(-3.84,-4.44,1.2)(-1.32,1.32,0.7)
\pstThreeDLine[linewidth=2pt,arrows=->](1.43,-0.32,3.36)(-1.43,0.32,-3.36)
\pstThreeDLine[linewidth=2pt,arrows=->](0.66,-0.66,-0.35)(-0.66,0.66,0.35)
\pstThreeDPut(-1.6,0.35,-3.6){$\bu_1$}
\pstThreeDPut(-0.79,0.79,0.37){$\bu_3$}
\pstThreeDLine[linewidth=2pt,arrows=->](1.92,2.22,-0.6)(-1.92,-2.22,0.6)
\pstThreeDPut(-2.05,-2.37,0.64){$\bu_2$}
\pstThreeDBox[linecolor=gray](-2,-2,-3)(4,0,0)(0,4,0)(0,0,7)
\pstThreeDEllipse(0,0,0)(-1.43,0.32,-3.36)(-1.92,-2.22,0.6)
\pstThreeDEllipse[linestyle=dashed](0,0,0)(-1.43,0.32,-3.36)(-0.66,0.66,0.35)
\pstThreeDEllipse[linestyle=solid](0,0,0)(-1.92,-2.22,0.6)(-0.66,0.66,0.35)
\end{pspicture}
}}%}
\end{frame}

\begin{frame}[fragile]{Iris: Axis-Parallel Ellipsoid in PC Basis}
\centerline{
\scalebox{0.8}{
\centering
\psset{unit=1}
\psset{arrowscale=2}
\begin{pspicture}(-5,-3.3)(5,3)
\psset{viewpoint=30 60 30 rtp2xyz,Decran=50}
\psset{lightsrc=viewpoint,opacity=0.75,incolor=white}
%\psSolid[object=sphere, hollow, r=3.66, ngrid=20 20,
%transform={1 0.4 0.2 scaleOpoint3d}, linewidth=0.5pt]%
%\psSolid[object=parallelepiped,a=7.29,b=2.72,c=1.27,action=draw](0,0,0)
\psSolid[object=sphere, hollow, r=1.91, ngrid=20 20,
transform={1 0.4 0.2 scaleOpoint3d}, linewidth=0.5pt]%
\psSolid[object=parallelepiped,a=7.29,b=2.72,c=1.27,action=draw](0,0,0)
\psset{dotstyle=Bo,dotscale=1.75,fillcolor=lightgray}
\input{EDA/dimreduction/figs/iris-slwpl-p-dots.tex}
\axesIIID[axisnames={\bu_1,\bu_2,\bu_3},linewidth=2pt](-3.65,-1.36,-0.64)(3.75,1.8,1.2)
\end{pspicture}
}}%}
\end{frame}

\ifdefined\wox \begin{frame} \titlepage \end{frame} \fi

\begin{frame}{Kernel Principal Component Analysis}

Principal component analysis can be extended to f\/{i}nd nonlinear
``directions'' in the data using kernel methods.
Kernel PCA f\/{i}nds the directions of most
variance in the feature space instead of the input space.  
Using the {\em kernel trick}, all PCA operations 
can be carried out in terms
of the kernel function in input space, without having to transform the
data into feature space.


\bigskip
Let $\phi$ be a function that maps a point
$\bx$ in input space to its image $\phi(\bx_i)$ in feature space. 
Let the points in
feature space be centered and let $\cov_\phi$ be the covariance matrix. 
The first PC in feature space correspond to the dominant eigenvector
\begin{align*}
  \cov_\phi \bu_1 = \lambda_1 \bu_1
\end{align*}
  where
\begin{align*}
  \cov_\phi = {1\over n}\dsum_{i=1}^n \phi(\bx_i) \phi(\bx_i)^T
\end{align*}

\end{frame}

\begin{frame}{Kernel Principal Component Analysis}
It can be shown that $\bu_1 = \dsum_{i=1}^n c_{i} \phi(\bx_i)$.
That is, the PC direction in 
feature space is a linear combination
of the transformed points. 

\medskip
The coefficients are captured in the weight vector
$$\bc = \matr{c_1, c_2, \cdots, c_n}^T$$

\medskip
Substituting into the eigen-decomposition of $\cov_\phi$ and
simplifying, we get:
\begin{align*}
  \bK\bc = n \lambda_1 \bc = \eta_1 \bc
\end{align*}
Thus,
the weight vector $\bc$ is the eigenvector
corresponding to the
largest eigenvalue $\eta_1$ of the kernel matrix
$\bK$.
\end{frame}


\begin{frame}{Kernel Principal Component Analysis}
  \small
  The weight vector $\bc$ can be used to then find $\bu_1$ via
$\bu_1 = \dsum_{i=1}^n c_{i} \phi(\bx_i)$.

\medskip
The only constraint we impose is that $\bu_1$
should be
normalized to be a unit vector, which implies
  $\norm{\bc}^2 = {1 \over \eta_1}$.


\medskip
We cannot compute directly the
principal direction, but we can
project any point $\phi(\bx)$ onto the principal
direction $\bu_1$, as follows:
\begin{align*}
  \bu_1^T\phi(\bx) = \dsum_{i=1}^n c_i\phi(\bx_i)^T\phi(\bx) =
  \dsum_{i=1}^n c_i K(\bx_i,\bx)
\end{align*}
which requires only kernel operations. 

\medskip
We can obtain the additional principal
components by solving for the other eigenvalues and eigenvectors of
$$\bK\bc_j = n \lambda_j \bc_j = \eta_j \bc_j$$
If we sort the
eigenvalues of $\bK$ in decreasing order
$\eta_1 \ge \eta_2 \ge \cdots
\ge \eta_n \ge 0$, we can obtain the $j$th principal component as
the corresponding eigenvector $\bc_{j}$.
The variance
along the $j$th principal component is given as 
$\lambda_{j} = {\eta_{j} \over n}$. 
\end{frame}


\begin{frame}{Kernel PCA Algorithm}
\begin{tightalgo}[H]{\textwidth-18pt}
\newcommand{\KPCA}{\textsc{KernelPCA}}
\SetKwInOut{Algorithm}{\KPCA\ ($\bD, K, \alpha$)}
\Algorithm{}
$\bK = \bigl\{K(\bx_i, \bx_{j}) \bigr\}_{i,j=1,\ldots,n}$ \tcp{compute $n \times n$ kernel matrix}
$\bK  = (\bI - {1\over n}\bone_{n\times n}) \bK (\bI - {1\over
n}\bone_{n \times n})$ \tcp{center the kernel matrix}
$(\eta_1, \eta_2, \ldots, \eta_d) =
\text{eigenvalues}(\bK)$ \tcp{compute eigenvalues}
$\matr{\bc_1 & \bc_2 & \cdots & \bc_n} =
\text{eigenvectors}(\bK)$ \tcp{compute eigenvectors}
$\lambda_i = {\eta_i \over n} \text{ for all } i = 1, \ldots,
n$ \tcp{compute variance for each component}
$\bc_i = \sqrt{1 \over \eta_i} \cdot \bc_i \text{ for all } i =
1, \ldots, n$ \tcp{ensure that $\bu_i^T\bu_i=1$}
$f(r) = \frac{\sum_{i=1}^r \lambda_i}{\sum_{i=1}^d \lambda_i},
\;\; \text{for all } r=1, 2, \ldots, d$ \tcp{fraction of total
variance}
Choose smallest $r$ so that $f(r) \ge \alpha$ \tcp{choose
dimensionality}
$\bC_r = \matr{\bc_1 & \bc_2 & \cdots & \bc_r}$ \tcp{reduced
basis}
$\bA = \{\ba_i \mid \ba_i = \bC_r^T\bK_i, \text{for } i=1, \ldots,
n\}$ \tcp{reduced dimensionality data}
\end{tightalgo}
\end{frame}


\readdata{\dataI}{EDA/dimreduction/figs/iris-2d-nonlinear.dat}
\begin{frame}{Nonlinear Iris Data: PCA in Input Space}
\psset{stepFactor=0.4}
\psset{dotstyle=Bo,dotscale=1.5,fillcolor=lightgray,arrowscale=2,PointName=none}
\psset{xAxisLabel=$X_1$,yAxisLabel= $X_2$}
\psset{xAxisLabelPos={c,-0.4in},yAxisLabelPos={-0.4in,c}}
\centerline{
\scalebox{0.90}{\hspace{0.25in}
\psgraph[axesstyle=frame,Dx=0.5,Dy=0.5,Ox=-0.5,Oy=-1]{->}(-0.5,-1)(1.5,1.5){2in}{2.5in}%
\dataplot[plotstyle=dots,showpoints=true]{\dataI}
\psline[linewidth=2pt]{->}(-0.316,-1)(0.474,1.5)
\uput[r](0.4,1.55){$\bu_1$}
\psplotImp[algebraic](-0.5,-1)(1.5,1.5){%
        0.3015*x+0.9535*y}
\psplotImp[algebraic](-0.5,-1)(1.5,1.5){%
        -0.25+0.3015*x+0.9535*y}
\psplotImp[algebraic](-0.5,-1)(1.5,1.5){%
        0.25+0.3015*x+0.9535*y}
\psplotImp[algebraic](-0.5,-1)(1.5,1.5){%
        -0.5+0.3015*x+0.9535*y}
\psplotImp[algebraic](-0.5,-1)(1.5,1.5){%
        0.5+0.3015*x+0.9535*y}
\psplotImp[algebraic](-0.5,-1)(1.5,1.5){%
        -0.75+0.3015*x+0.9535*y}
\psplotImp[algebraic](-0.5,-1)(1.5,1.5){%
        0.75+0.3015*x+0.9535*y}
\psplotImp[algebraic](-0.5,-1)(1.5,1.5){%
        -1+0.3015*x+0.9535*y}
\psplotImp[algebraic](-0.5,-1)(1.5,1.5)
\hspace{0.75in}
%\subfloat[$\lambda_2=0.087$]{
%\label{fig:eda:dimr:2dnonlinearL2}
\psgraph[axesstyle=frame,Dx=0.5,Dy=0.5,Ox=-0.5,Oy=-1]{->}(-0.5,-1)(1.5,1.5){2in}{2.5in}%
\dataplot[plotstyle=dots,showpoints=true]{\dataI}
\psline[linewidth=2pt]{->}(-0.5,0.158)(1.5,-0.474)
\uput[r](1.5,-0.5){$\bu_2$}
\psplotImp[algebraic](-0.5,-1)(1.5,1.5){%
        -0.9535*x+0.3015*y}
\psplotImp[algebraic](-0.5,-1)(1.5,1.5){%
        -0.25-0.9535*x+0.3015*y}
\psplotImp[algebraic](-0.5,-1)(1.5,1.5){%
        0.25-0.9535*x+0.3015*y}
\psplotImp[algebraic](-0.5,-1)(1.5,1.5){%
        -0.5-0.9535*x+0.3015*y}
\psplotImp[algebraic](-0.5,-1)(1.5,1.5){%
        0.5-0.9535*x+0.3015*y}
\psplotImp[algebraic](-0.5,-1)(1.5,1.5){%
        -0.75-0.9535*x+0.3015*y}
\psplotImp[algebraic](-0.5,-1)(1.5,1.5){%
        0.75-0.9535*x+0.3015*y}
\psplotImp[algebraic](-0.5,-1)(1.5,1.5){%
        1-0.9535*x+0.3015*y}
\psplotImp[algebraic](-0.5,-1)(1.5,1.5)
\end{frame}


\readdata{\dataLIN}{EDA/dimreduction/figs/iris-nonlinear-PCA.dat}
\begin{frame}{Nonlinear Iris Data: Projection onto PCs}
\scalebox{0.9}{
\psset{dotstyle=Bo,dotscale=1.5,fillcolor=lightgray,arrowscale=2,PointName=none}
\psset{xAxisLabel=$\bu_1$,yAxisLabel= $\bu_2$}
%\psset{xAxisLabelPos={c,-0.4in},yAxisLabelPos={-0.4in,c}}
\centerline{
\psgraph[Dx=0.5,Dy=0.5,Ox=-0.75,Oy=-1.5]{->}(-0.75,-1.5)(1.75,0.5){3.5in}{2.5in}%
\dataplot[plotstyle=dots,showpoints=true]{\dataLIN}
\endpsgraph
}
}
\end{frame}


\def\mye{2.7183}
\begin{frame}[fragile]{Kernel PCA: 3 PCs (Contours of Constant Projection)} 
  \framesubtitle{Homogeneous Quadratic Kernel: $K(\bx_i, \bx_j) =
  (\bx_i^T\bx_j)^2$}
\centering 
\begin{figure}[!t]
%\vspace{0.1in} 
\psset{stepFactor=0.4}
\psset{dotstyle=Bo,dotscale=1.5,fillcolor=lightgray,arrowscale=2,PointName=none}
\psset{xAxisLabel=$X_1$,yAxisLabel= $X_2$}
\psset{xAxisLabelPos={c,-0.4in},yAxisLabelPos={-0.4in,c}}
\setcounter{subfigure}{0}
\captionsetup[subfloat]{captionskip=35pt} 
\centerline{
\hspace{0.25in} 
\subfloat[$\lambda_1=0.2067$]{
\scalebox{0.6}{
\psgraph[axesstyle=frame,Dx=0.5,Dy=0.5,Ox=-0.5,Oy=-1]{->}(-0.5,-1)(1.5,1.5){2in}{2.5in}%
\dataplot[plotstyle=dots,showpoints=true]{\dataI}
\psplotImp[algebraic](-0.5,-1)(1.5,1.5){%
        -0.01+1.0426*x*y+0.995*x^2+0.914*y^2}
\psplotImp[algebraic](-0.5,-1)(1.5,1.5){%
        -0.1+1.0426*x*y+0.995*x^2+0.914*y^2}
\psplotImp[algebraic](-0.5,-1)(1.5,1.5){%
        -0.25+1.0426*x*y+0.995*x^2+0.914*y^2}
\psplotImp[algebraic](-0.5,-1)(1.5,1.5){%
        -0.5+1.0426*x*y+0.995*x^2+0.914*y^2}
\psplotImp[algebraic](-0.5,-1)(1.5,1.5){%
        -1+1.0426*x*y+0.995*x^2+0.914*y^2}
\psplotImp[algebraic](-0.5,-1)(1.5,1.5){%
        -2+1.0426*x*y+0.995*x^2+0.914*y^2}
\psplotImp[algebraic](-0.5,-1)(1.5,1.5){%
        -4+1.0426*x*y+0.995*x^2+0.914*y^2}
\endpsgraph
} }
\hspace{0.25in} 
\subfloat[$\lambda_2=0.0596$]{
\scalebox{0.6}{
\psgraph[axesstyle=frame,Dx=0.5,Dy=0.5,Ox=-0.5,Oy=-1]{->}(-0.5,-1)(1.5,1.5){2in}{2.5in}%
\dataplot[plotstyle=dots,showpoints=true]{\dataI}
\psplotImp[algebraic](-0.5,-1)(1.5,1.5){%
        0.01+0.9077*x*y-0.6088*x^2-1.4077*y^2}
\psplotImp[algebraic](-0.5,-1)(1.5,1.5){%
        -0.1+0.9077*x*y-0.6088*x^2-1.4077*y^2}
\psplotImp[algebraic](-0.5,-1)(1.5,1.5){%
        0.1+0.9077*x*y-0.6088*x^2-1.4077*y^2}
\psplotImp[algebraic](-0.5,-1)(1.5,1.5){%
        -0.25+0.9077*x*y-0.6088*x^2-1.4077*y^2}
\psplotImp[algebraic](-0.5,-1)(1.5,1.5){%
        0.25+0.9077*x*y-0.6088*x^2-1.4077*y^2}
\psplotImp[algebraic](-0.5,-1)(1.5,1.5){%
        0.5+0.9077*x*y-0.6088*x^2-1.4077*y^2}
\psplotImp[algebraic](-0.5,-1)(1.5,1.5){%
        1+0.9077*x*y-0.6088*x^2-1.4077*y^2}
\psplotImp[algebraic](-0.5,-1)(1.5,1.5){%
        1.5+0.9077*x*y-0.6088*x^2-1.4077*y^2}
\psplotImp[algebraic](-0.5,-1)(1.5,1.5){%
        2+0.9077*x*y-0.6088*x^2-1.4077*y^2}
\endpsgraph
}} 
\hspace{0.25in} 
%\vspace{0.2in} \centerline{ \hspace{0.25in}
\subfloat[$\lambda_3=0.0184$]{
\scalebox{0.6}{
\psgraph[axesstyle=frame,Dx=0.5,Dy=0.5,Ox=-0.5,Oy=-1]{->}(-0.5,-1)(1.5,1.5){2in}{2.5in}%
\dataplot[plotstyle=dots,showpoints=true]{\dataI}
\psplotImp[algebraic](-0.5,-1)(1.5,1.5){%
        1-0.298*x*y+1.625*x^2-1.087*y^2}
\psplotImp[algebraic](-0.5,-1)(1.5,1.5){%
        0.75-0.298*x*y+1.625*x^2-1.087*y^2}
\psplotImp[algebraic](-0.5,-1)(1.5,1.5){%
        0.5-0.298*x*y+1.625*x^2-1.087*y^2}
\psplotImp[algebraic](-0.5,-1)(1.5,1.5){%
        0.3-0.298*x*y+1.625*x^2-1.087*y^2}
\psplotImp[algebraic](-0.5,-1)(1.5,1.5){%
        0.1-0.298*x*y+1.625*x^2-1.087*y^2}
\psplotImp[algebraic](-0.5,-1)(1.5,1.5){%
        -0.1-0.298*x*y+1.625*x^2-1.087*y^2}
\psplotImp[algebraic](-0.5,-1)(1.5,1.5){%
        -0.3-0.298*x*y+1.625*x^2-1.087*y^2}
\psplotImp[algebraic](-0.5,-1)(1.5,1.5){%
        -0.6-0.298*x*y+1.625*x^2-1.087*y^2}
\psplotImp[algebraic](-0.5,-1)(1.5,1.5){%
        -1-0.298*x*y+1.625*x^2-1.087*y^2}
\psplotImp[algebraic](-0.5,-1)(1.5,1.5){%
        -1.5-0.298*x*y+1.625*x^2-1.087*y^2}
\psplotImp[algebraic](-0.5,-1)(1.5,1.5){%
        -2-0.298*x*y+1.625*x^2-1.087*y^2}
%%\psplotImp[linewidth=2pt, algebraic](-0.5,-1)(1.5,1.5){%
%%0.298*\mye^(-1*x^2)+0.625*\mye^(-1*y^2)-0.5}
\endpsgraph
}}
}
\end{figure}
\end{frame}


\readdata{\dataKQH}{EDA/dimreduction/figs/iris-2d-kPC-quadraticH.dat}
\begin{frame}{Kernel PCA: Projected Points onto 2 PCs}
  \framesubtitle{Homogeneous Quadratic Kernel: $K(\bx_i, \bx_j) =
  (\bx_i^T\bx_j)^2$}
\hspace*{0.5in}
\scalebox{0.9}{
\centering
%\vspace{0.1in}
\psset{dotstyle=Bo,dotscale=1.5,fillcolor=lightgray,arrowscale=2,PointName=none}
\psset{xAxisLabel=$\bu_1$,yAxisLabel= $\bu_2$}
\psgraph[Dx=0.5,Dy=0.5,Ox=-0.5,Oy=-2]{->}(-0.5,-2)(4,0.5){4in}{2.5in}%
\dataplot[plotstyle=dots,showpoints=true]{\dataKQH}
\endpsgraph
}
\end{frame}


\ifdefined\wox \begin{frame} \titlepage \end{frame} \fi


\begin{frame}{Singular Value Decomposition}
  \small

  Principal components analysis is a special case of a more general
matrix decomposition method called {\em Singular Value
Decomposition (SVD)}.  
PCA yields the following decomposition of the covariance
matrix:
\begin{align*}
   \cov & = \bU \bLambda \bU^T
\end{align*}
where the covariance matrix has been factorized into
the orthogonal matrix $\bU$ containing its eigenvectors, and a
diagonal matrix $\bLambda$ containing its eigenvalues (sorted in
decreasing order).

\medskip
SVD generalizes the above factorization for any matrix. In particular
for an $n \times d$ data matrix $\bD$ with $n$ points and $d$ columns, SVD factorizes $\bD$ as follows:
\begin{align*}
  \bD &= \bL \bDelta \bR^T
\end{align*}

where $\bL$ is a orthogonal $n \times n$ matrix,
$\bR$ is an
orthogonal $d \times d$ matrix, and $\bDelta$ is an 
$n \times d$ ``diagonal'' matrix, 
defined as $\bDelta(i,i) = \delta_i$, and $0$
otherwise.
The columns of $\bL$ are called the {\em left singular
  and the columns of $\bR$ (or rows of $\bR^T$) are called the
\em right singular vectors}.
The entries $\delta_{i}$ are called the 
{\em singular values} of $\bD$, and they are all non-negative.
\end{frame}

\begin{frame}{Reduced SVD}
  \small
If the rank of $\bD$ is $r \le \min(n,d)$, then
there are only $r$ nonzero
singular values, ordered as follows:
$\delta_1 \ge
\delta_2 \ge \cdots \ge \delta_r > 0$.

\medskip
We
discard the left and right singular vectors that correspond to zero
singular values, to obtain the {\em reduced SVD} as
\begin{align*}
\bD = \bL_r \bDelta_r \bR_r^T
\end{align*}
where $\bL_r$ is the $n \times r$ matrix of the left
singular vectors,
$\bR_r$ is the $d \times r$ matrix of the right
singular vectors, and $\bDelta_r$ is the $r \times r$ diagonal
matrix containing the positive singular vectors.

\medskip
The reduced SVD leads directly to
the {\em spectral decomposition} of
$\bD$ given as
\begin{align*}
\bD   = & \sum_{i=1}^r \delta_i \bl_i \br^T_i
\end{align*}

The best rank $q$ approximation to the
original data $\bD$ is the matrix
$\bD_q = \sum_{i=1}^q \delta_i \bl_i \br^T_i$
that
minimizes the expression $\|\bD - \bD_q\|_F$,
where $\|\bA\|_F = \sqrt{\sum_{i=1}^n \sum_{j=1}^d \bA(i,j)^2}$
is called the {\em Frobenius Norm} of $\bA$.
\end{frame}



\begin{frame}{Connection Between SVD and PCA}
  \small 
Assume $\bD$ has been centered, and let $\bD = \bL \bDelta \bR^T$ via
SVD. 
Consider the {\em scatter matrix} for $\bD$, given as $\bD^T\bD$.
We have
\begin{align*}
    \bD^T\bD  = \lB(\bL \bDelta \bR^T\rB)^T
         \lB(\bL \bDelta \bR^T \rB)
         = \bR \bDelta^T \bL^T \bL \bDelta \bR^T
         = \bR (\bDelta^T \bDelta) \bR^T
         = \bR \bDelta^2_d \bR^T
\end{align*}
where $\bDelta^2_d$ is
the $d \times d$ diagonal matrix def\/{i}ned as
$\bDelta^2_d(i,i) = \delta_i^2$, for
$i=1,\ldots,d$.

\medskip
The covariance matrix of centered $\bD$ is given as
$\cov = {1\over n} \bD^T\bD$; we get
\begin{align*}
    \bD^T\bD & = n \cov\nonumber\\
        & = n \bU \bLambda \bU^T\nonumber\\
        & = \bU (n \bLambda) \bU^T
\end{align*}

\medskip
The right singular vectors $\bR$ are the same as
the eigenvectors of $\cov$.
The singular values of $\bD$ are
related to the eigenvalues of $\cov$ as
\begin{align*}
n \lambda_i = \delta_i^2\nonumber
\text{, which implies } \lambda_i = {\delta_i^2 \over n},
\text{ for } i=1, \ldots, d
\end{align*}

\medskip
Likewise the left
singular vectors in $\bL$ are the eigenvectors of the matrix $n
\times n$ matrix $\bD\bD^T$, and the corresponding eigenvalues
are given as $\delta^2_i$.
\end{frame}

